% +---------------------------------------------------------------------------+
% | chap/Hentaigana                                                           |
% |                                                                           |
% | This chapter introduces Japanese hentaigana. It summarizes shortly        |
% | history, contemporary usage and examples. It gives a table of             |
% | hentaigana.                                                               |
% |                                                                           |
% | Version: 0.1.0                                                            |
% |                                                                           |
% | Status: Not in use                                                        |
% |                                                                           |
% | I18n: english                                                             |
% |                                                                           |
% | Changes:                                                                  |
% |                                                                           |
% | 0.1.0 2022-08-30  Christian Külker <c@c8i.org>                            |
% |     - Initial release                                                     |
% |                                                                           |
% +---------------------------------------------------------------------------+
%
% TODO:
% [ ] Make all code points display
% [ ] Write history
% [ ] Write contemporary usage
% [ ] Write examples
% [o] Write hentaigana table
%
\chapter{Hentaigana}\jchap{変体仮名}\label{chap:Hentaigana}
\ithree{Hentaigana tables}{変体仮名図}{Hentaigana Tabellen}

\section{History}\jsec{歴史}\label{sec:HentaiganaHistory}
\section{Usage}\jsec{使い方}\label{sec:HentaiganaUsage}
\section{Some Examples}\jsec{少数例}\label{sec:HentaiganaSomeExamples}
\section{Hentaigana Table}\jsec{変体仮名のレファレンス}\label{sec:HentaiganaTable}
\JapaneseFontN %Not all code points are displayed
% +---------------------------------------------------------------------------+
% | table/Hentaigana.tex                                                      |
% |                                                                           |
% | Table of some Hentaigana                                                  |
% |                                                                           |
% | Version: 0.1.0                                                            |
% |                                                                           |
% | Changes:                                                                  |
% |                                                                           |
% | 0.1.0 2020-07-10 Christian Külker <c@c8i.org>                             |
% |     - initial release                                                     |
% |                                                                           |
% +---------------------------------------------------------------------------+
% HINT:
% use gvim, plumar to edit this file (not vi or vim)

\ide{Basistabelle Hentaigana}
\ide{Hentaigana}
\ija{変体仮名の五十音図}

\bigskip
\begin{center}
\Huge

% Hentaigana
\begin{tabular}{m{1.0cm}||m{2.5cm}|m{2.5cm}|m{2.5cm}|m{2.5cm}|m{2.5cm}|}
& \textbf{a}& \textbf{i}& \textbf{u}& \textbf{e}& \textbf{o}\\
& \textbf{あ(安)}& \textbf{い(以)}& \textbf{う(宇)}& \textbf{え(衣)}& \textbf{お(於)}\\ \hline \hline
\textbf{-}&\smallskip 𛀂(安) 𛀅(惡) 𛀃(愛) 𛀄(阿)
          &\smallskip 𛀆(以) 𛀇(伊) 𛀈(意) 𛀉(移)
          &\smallskip 𛀊(宇) 𛀋(宇) 𛀌(憂) 𛀍(有) 𛀎(雲)
          &\smallskip 𛀁(江) 𛀏(盈) 𛀐(縁) 𛀑(衣) 𛀒(衣) 𛀓(要)
          &\smallskip 𛀔(於) 𛀕(於) 𛀖(隱) \\ \hline
\end{tabular}

\begin{tabular}{m{1.0cm}||m{2.5cm}|m{2.5cm}|m{2.5cm}|m{2.5cm}|m{2.5cm}|}
& \textbf{a}& \textbf{i}& \textbf{u}& \textbf{e}& \textbf{o}\\
& \textbf{か(加)}& \textbf{き(幾)}& \textbf{く(久)}& \textbf{け(計)}& \textbf{こ(己)}\\ \hline \hline
\textbf{k}&\smallskip 𛀗(佳) 𛀘(加) 𛀙(可) 𛀚(可) 𛀛(嘉) 𛀢(家) 𛀜(我) 𛀝(歟) 𛀞(賀) 𛀟(閑) 𛀠(香) 𛀡(駕)
          &\smallskip 𛀣(喜) 𛀤(幾) 𛀥(幾) 𛀦(支) 𛀻(期) 𛀧(木) 𛀨(祈) 𛀩(貴) 𛀪(起)
          &\smallskip 𛀫(久) 𛀬(久) 𛀭(九) 𛀮(供) 𛀯(倶) 𛀰(具) 𛀱(求)
          &\smallskip 𛀳(介) 𛀲(介) 𛀢(家) 𛀴(希) 𛀵(氣) 𛀶(計) 𛀷(遣)
          &\smallskip 𛀸(古) 𛂘(子) 𛀹(故) 𛀻(期) 𛀺(許) \\ \hline
\end{tabular}

\begin{tabular}{m{1.0cm}||m{2.5cm}|m{2.5cm}|m{2.5cm}|m{2.5cm}|m{2.5cm}|}
& \textbf{a}& \textbf{i}& \textbf{u}& \textbf{e}& \textbf{o}\\
& \textbf{さ(左)}& \textbf{し(之}& \textbf{す(寸)}& \textbf{せ(世)}& \textbf{そ(曾)}\\ \hline \hline
\textbf{s}&\smallskip 𛀼(乍) 𛀽(佐) 𛀾(佐) 𛀿(左) 𛁀(差) 𛁁(散) 𛁂(斜) 𛁃(沙)
          &\smallskip 𛁄(之) 𛁅(之) 𛁆(事) 𛁇(四) 𛁈(志) 𛁉(新)
          &\smallskip 𛁊(受) 𛁋(壽) 𛁌(數) 𛁍(數) 𛁎(春) 𛁏(春) 𛁐(須) 𛁑(須)
          &\smallskip 𛁒(世) 𛁓(世) 𛁔(世) 𛁕(勢) 𛁖(聲)
          &\smallskip 𛁗(所) 𛁘(所) 𛁙(曾) 𛁚(曾) 𛁛(楚) 𛁜(蘇) 𛁝(處) \\ \hline
\end{tabular}

\begin{tabular}{m{1.0cm}||m{2.5cm}|m{2.5cm}|m{2.5cm}|m{2.5cm}|m{2.5cm}|}
& \textbf{a}& \textbf{i}& \textbf{u}& \textbf{e}& \textbf{o}\\
& \textbf{た(太)}& \textbf{ち(知)}& \textbf{つ(州)}& \textbf{て(天)}& \textbf{と(止)}\\ \hline \hline
\textbf{t}&\smallskip 𛁞(堂) 𛁟(多) 𛁠(多) 𛁡(當)
          &\smallskip 𛁢(千) 𛁣(地) 𛁤(智) 𛁥(知) 𛁦(知) 𛁧(致) 𛁨(遲)
          &\smallskip 𛁩(川) 𛁪(川) 𛁫(津) 𛁬(都) 𛁭(徒)
          &\smallskip 𛁮(亭) 𛁯(低) 𛁰(傳) 𛁱(天) 𛁲(天) 𛁳(天) 𛁴(帝) 𛁵(弖) 𛁶(轉) 𛂎(而)
          &\smallskip 𛁷(土) 𛁸(度) 𛁹(東) 𛁺(登) 𛁻(登) 𛁼(砥) 𛁽(等) 𛁭(徒) \\ \hline
\end{tabular}

\begin{tabular}{m{1.0cm}||m{2.5cm}|m{2.5cm}|m{2.5cm}|m{2.5cm}|m{2.5cm}|}
& \textbf{a}& \textbf{i}& \textbf{u}& \textbf{e}& \textbf{o}\\
& \textbf{な(奈)}& \textbf{に(仁)}& \textbf{ぬ(奴)}& \textbf{ね(祢)}& \textbf{の(乃)}\\ \hline \hline
\textbf{n}&\smallskip 𛁾(南) 𛁿(名) 𛂀(奈) 𛂁(奈) 𛂂(奈) 𛂃(菜) 𛂄(那) 𛂅(那) 𛂆(難)
          &\smallskip 𛂇(丹) 𛂈(二) 𛂉(仁) 𛂊(兒) 𛂋(爾) 𛂌(爾) 𛂍(耳) 𛂎(而)
          &\smallskip 𛂏(努) 𛂐(奴) 𛂑(怒)
          &\smallskip 𛂒(年) 𛂓(年) 𛂔(年) 𛂕(根) 𛂖(熱) 𛂗(禰) 𛂘(子)
          &\smallskip 𛂙(乃) 𛂚(濃) 𛂛(能) 𛂜(能) 𛂝(農) \\ \hline
\end{tabular}

\begin{tabular}{m{1.0cm}||m{2.5cm}|m{2.5cm}|m{2.5cm}|m{2.5cm}|m{2.5cm}|}
& \textbf{a}& \textbf{i}& \textbf{u}& \textbf{e}& \textbf{o}\\
& \textbf{は(波)}& \textbf{ひ(比)}& \textbf{ふ(不)}& \textbf{へ(部)}& \textbf{ほ(保)}\\ \hline \hline
\textbf{h}&\smallskip 𛂞(八) 𛂟(半) 𛂠(婆) 𛂡(波) 𛂢(盤) 𛂣(盤) 𛂤(破) 𛂥(者) 𛂦(者) 𛂧(葉) 𛂨(頗)
          &\smallskip 𛂩(悲) 𛂪(日) 𛂫(比) 𛂬(避) 𛂭(非) 𛂮(飛) 𛂯(飛)
          &\smallskip 𛂰(不) 𛂱(婦) 𛂲(布)
          &\smallskip 𛂳(倍) 𛂴(弊) 𛂵(弊) 𛂶(遍) 𛂷(邊) 𛂸(邊) 𛂹(部)
          &\smallskip 𛂺(保) 𛂻(保) 𛂼(報) 𛂽(奉) 𛂾(寶) 𛂿(本) 𛃀(本) 𛃁(豊) \\ \hline
\end{tabular}

% XeLaTeX or fonts-hanazono bug?
% code points U+1B11D HEINTAIGANA LETTER N-MU-MO-1 𛄝
%             U+1B11E HEINTAIGANA LETTER N-MU-MO-2 𛄞
% are visible with HanMinA (Hanazono Mincho Reguar) with font-manager, pluma, 
% gvim on Debian Buster, but are not visible in the PDF

\begin{tabular}{m{1.0cm}||m{2.5cm}|m{2.5cm}|m{2.5cm}|m{2.5cm}|m{2.5cm}|}
& \textbf{a}& \textbf{i}& \textbf{u}& \textbf{e}& \textbf{o}\\
& \textbf{ま(末)}& \textbf{み(美)}& \textbf{む(武)}& \textbf{め(女)}& \textbf{も(毛)}\\ \hline \hline
\textbf{m}&\smallskip 𛃂(万) 𛃃(末) 𛃄(末) 𛃅(滿) 𛃆(滿) 𛃇(萬) 𛃈(麻) 𛃖(馬)
          &\smallskip 𛃉(三) 𛃊(微) 𛃋(美) 𛃌(美) 𛃍(美) 𛃎(見) 𛃏(身)
          &\smallskip 𛃐(武) 𛃑(無) 𛃒(牟) 𛃓(舞) 𛄝(无) 𛄞(无)
          &\smallskip 𛃔(免) 𛃕(面) 𛃖(馬)
          &\smallskip 𛃗(母) 𛃘(毛) 𛃙(毛) 𛃚(毛) 𛃛(茂) 𛃜(裳) 𛄝(无) 𛄞(无) \\ \hline
\end{tabular}
%

\begin{tabular}{m{1.0cm}||m{2.5cm}|m{2.5cm}|m{2.5cm}|m{2.5cm}|m{2.5cm}|}
& \textbf{a}& \textbf{i}& \textbf{u}& \textbf{e}& \textbf{o}\\
& \textbf{や(也)}& \textbf{𛀆(以)}& \textbf{ゆ(由)}& \textbf{𛀁(江)}& \textbf{よ(与)}\\ \hline \hline
\textbf{y}&\smallskip 𛃝(也) 𛃞(也) 𛃟(屋) 𛃠(耶) 𛃡(耶) 𛃢(夜)
          &\smallskip 𛀆(以)
          &\smallskip 𛃣(游) 𛃤(由) 𛃥(由) 𛃦(遊)
          &\smallskip 𛀁(江)
          &\smallskip 𛃧(代) 𛃨(余) 𛃩(與) 𛃪(與) 𛃫(與) 𛃬(餘) 𛃢(夜) \\ \hline
\end{tabular}

\begin{tabular}{m{1.0cm}||m{2.5cm}|m{2.5cm}|m{2.5cm}|m{2.5cm}|m{2.5cm}|}
& \textbf{a}& \textbf{i}& \textbf{u}& \textbf{e}& \textbf{o}\\
& \textbf{ら(良)}& \textbf{り(利)}& \textbf{る(留)}& \textbf{れ(礼)}& \textbf{ろ(呂)}\\ \hline \hline
\textbf{r}&\smallskip 𛃭(羅) 𛃮(良) 𛃯(良) 𛃰(良) 𛁽(等)
          &\smallskip 𛃱(利) 𛃲(利) 𛃳(李) 𛃴(梨) 𛃵(理) 𛃶(里) 𛃷(離)
          &\smallskip 𛃸(流) 𛃹(留) 𛃺(留) 𛃻(留) 𛃼(累) 𛃽(類)
          &\smallskip 𛃾(禮) 𛃿(禮) 𛄀(連) 𛄁(麗)
          &\smallskip 𛄂(呂) 𛄃(呂) 𛄄(婁) 𛄅(樓) 𛄆(路) 𛄇(露) \\ \hline
\end{tabular}

\begin{tabular}{m{1.0cm}||m{2.5cm}|m{2.5cm}|m{2.5cm}|m{2.5cm}|m{2.5cm}|}
& \textbf{a}& \textbf{i}& \textbf{u}& \textbf{e}& \textbf{o}\\
& \textbf{わ(和)}& \textbf{ゐ(為)}& \textbf{(汙)}& \textbf{ゑ(恵)}& \textbf{を(遠)}\\ \hline \hline
\textbf{w}&\smallskip 𛄈(倭) 𛄉(和) 𛄊(和) 𛄋(王) 𛄌(王)
          &\smallskip 𛄍(井) 𛄎(井) 𛄏(居) 𛄐(爲) 𛄑(遺)
          &\smallskip
          &\smallskip 𛄒(惠) 𛄓(衞) 𛄔(衞) 𛄕(衞)
          &\smallskip 𛄖(乎) 𛄗(乎) 𛄘(尾) 𛄙(緒) 𛄚(越) 𛄛(遠) 𛄜(遠) 𛀅(惡) \\ \hline
\end{tabular}

% Not printable /wu/ 汙  https://kobunworld.blog.fc2.com/blog-entry-5.html

\begin{tabular}{m{1.0cm}||m{2.5cm}|m{2.5cm}|m{2.5cm}|m{2.5cm}|m{2.5cm}|}
& \textbf{a}& \textbf{i}& \textbf{u}& \textbf{e}& \textbf{o}\\
& \textbf{ん(无)}& \textbf{}& \textbf{}& \textbf{}& \textbf{}\\ \hline \hline
\textbf{*}&\smallskip 𛄝(无) 𛄞(无)
          &\smallskip
          &\smallskip
          &\smallskip
          &\smallskip   \\ \hline
\end{tabular}
\end{center}


