\chapter*{Introduction}\jchap{前書き}
% FOREWORD:     Vorwort        (someone else: why you should read?)
% INTRODUCTION: Einfuehrung    (author: what is in it, what can be expected?)
% PREFACE:      Vorbemerkungen (author: why the book exists?)
% [o] LABEL
\label{chap:Introduction}
% [o] INDEX DESTINATION (DEF) \ifor only
% [o] INDEX TARGET
\ithree{introduction}{前書き}{Einleitung}
\ifor{hiragana}{平仮名}{ひらがな}{Hiragana}
\ifor{katakana}{片仮名}{かたかな}{Katakana}

\textbf{The Japanese Script - Hiragana} was decided to be volume ①  and
\textbf{The Japanese Script - Katakana} was decided to be volume ②.
\ifthenelse{\equal{hiragana}{\jtopic}}{%
 This book is the \textbf{first} volume of \textbf{The Japanese Script} series with the
focus to teach \textbf{hiragana}.
}{}%
\ifthenelse{\equal{katakana}{\jtopic}}{%
 This book is the \textbf{second} volume of \textbf{The Japanese Script} series with the
focus to teach \textbf{katakana}.
}{}%

While there is no specific reason to start with \textbf{hiragana} when it comes
to learn the first Japanese writing script, historically \textbf{hiragana} is
the first choice. It might be the assumption that \textbf{hiragana} are more
common in Japanese texts than \textbf{katakana} and also when Japanese Chinese
characters (kanji) are transcribed (for example as station names)
\textbf{hiragana} is used. However, since a beginner in learning Japanese also
have almost no vocabulary knowledge, the usability is overestimated by Japanese
native people. Actually, a foreigner from an English speaking background might
consider to learn \textbf{katakana} first, because foreign words in Japanese
are written in \textbf{katakana} and after mastering the \textbf{katakana} such
learner would be able to read and write thousands of words with a little
imagination, compared to almost none in case of \textbf{hiragana}.

With whatever script to start with, it is recommended to finish this first
script and then learn the second script. It is also recommended to finish
\textbf{hiragana} and \textbf{katakana} before learning Japanese Chinese
characters (kanji).

Common chapters are repeated or are slightly modified in each volume, so that a
learner can choose to start with volume ①  or ②  and obtain all the references
in each volume. Skip the passages you already know.

\section*{Self Learning}\jsec{独修}
\addcontentsline{toc}{section}{Self Learning}

% 独習者   どくしゅう・しゃ
% 独学者   どくがく・しゃ
% 独学の   どくがくの
% 独学で   どくがくで
% 自学する じ│がくする
% 独学する どくがくする
% 自修     じ・しゅう
% 独学     どく│がく
% 自学自習 じがく・じしゅう autodidaktisches Studium; Selbststudium
% 独修     どく・しゅう     schriftspr. Selbststudiumn; autodidaktisches Studium
% 独習書   どくしゅう・しょ Buch für Selbstunterricht; autodidaktisches Lehrbuch
% [o] LABEL
\label{sec:SelfLearning}
% [o] INDEX DESTINATION (DEF) \ifor only
% [o] INDEX TARGET
\ithree{self learning}{独学}{Selbststudium}
\ithree{self learning}{独修}{Selbstudium}
\ifor{hiragana}{平仮名}{ひらがな}{Hiragana}
\ifor{katakana}{片仮名}{かたかな}{Katakana}

Being able to read and write Japanese is a core skill when learning Japanese.
And \textbf{\jtopic} is one of the two very basic scripts of Japanese to be
learned. This book is written with the aim to help in that, based on self
experience as a learner of \jtopic{} as well as from teaching experience
and with feedback of many students. This edition of the book targets a self
leaning approach. So please report suggestions or problems.

Even though this book gives many information how to write and more important
how not to write \jtopic{}, it is astonishing how the human creativity can
generate version of Japanese \jtopic{} characters that are slightly wrong.
Therefore passages describing the good or bad shape of characters should be
red carefully.

The \hyperref[chap:JapaneseWritingSystem]{first chapter} will introduce the
\nameref{chap:JapaneseWritingSystem} and different alphabets. If you are
already familiar with it, you can safely skip this chapter. In any case all
terms are explained in the \hyperref[chap:Terminology]{last chapter}.

The %
%
\jhiraganaonly{\hyperref[chap:TheWayToWriteHiragana]{second chapter} %
(\nameref{chap:TheWayToWriteHiragana})}
\jkatakanaonly{\hyperref[chap:TheWayToWriteKatakana]{second chapter} %
(\nameref{chap:TheWayToWriteKatakana})} %
%
starts with the introduction of writing and reading single \textbf{\jtopic}
letters. The chapter ends with special \textbf{\jtopic} letters. It is advised
to read this chapter before starting the training.

\jhiraganaonly{

The \hyperref[chap:HiraganaTraining]{third chapter}
(\nameref{chap:HiraganaTraining}) goes right into the action by offering row
based training sessions for each character as well as simple training for
writing some Japanese \textbf{\jtopic} words. This is the \textbf{main} part of
the book and should be studied extensively.

}
\jkatakanaonly{

The \hyperref[chap:KatakanaTraining]{third chapter}
(\nameref{chap:KatakanaTraining}) goes right into the action by offering row
based training sessions for each character as well as simple training for
writing some Japanese \textbf{\jtopic} words. This is the \textbf{main} part of
the book and should be studied extensively.

}

The \hyperref[chap:Terminology]{last chapter} (\nameref{chap:Terminology})
provides an Latin alphabetically ordered \textbf{glossary} about the most
important keywords and concepts. It is recommended to read \textbf{one} article
at a time to deepen the understanding of the Japanese language in general and
the way of writing Japanese in particular. The order do not matter. It is not
mandatory to read this chapter to learn \jtopic.

The \textbf{appendix} contain tables of all important \jtopic{} written in
different fonts in the \nameref{chap:KanaTables} part starting at page
\pageref{chap:KanaTables}. Even though this is not explicit mentioned in the
following chapters it is important to have a look at these tables from time to
time when learning \jtopic{} to understand the margin (how much can be diverted
from the standard shape but the character is still recognized) of the character
to learn. The second part includes \hyperref[chap:RomajiTables]{two tables with
Latin letters} to memorize the pronunciation. In the forth part a list of used
\hyperref[chap:ListOfJapaneseTechnicalTerms]{technical terms in Japanese}
(partly translated to English and German) can be found with references to the
text where they are explained.  The last part of the appendix offer three
indices: in \hyperref[chap:EnglishIndex]{English} (Index) and
\hyperref[chap:GermanIndex]{German} (Fachbegriffe) for the learner and in
\hyperref[chap:JapaneseIndex]{Japanese} (索引) for the teacher.


