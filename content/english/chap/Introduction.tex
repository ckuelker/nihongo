\chapter*{Introduction}\jchap{前書き}
\addcontentsline{toc}{chapter}{Introduction}
% FOREWORD:     Vorwort        (someone else: why you should read?)
% INTRODUCTION: Einfuehrung    (author: what is in it, what can be expected?)
% PREFACE:      Vorbemerkungen (author: why the book exists?)
% [o] LABEL
\label{chap:Introduction}
% [o] INDEX DESTINATION (DEF) \ifor only
% [o] INDEX TARGET
\ithree{introduction}{前書き}{Einleitung}
\ifor{hiragana}{平仮名}{ひらがな}{Hiragana}
\ifor{katakana}{片仮名}{かたかな}{Katakana}

\textbf{The Japanese Language - Nihongo} series has two volumes with the aim to
teach the two Japanese syllable scripts: Volume \jvolumes{①} \textbf{hiragana}
and volume \jvolumes{②} \textbf{katakana}.%
%
\jhiraganaonly{

This book is the \textbf{first} volume \jvolumes{①} of \textit{The Japanese
Language - Nihongo series} that can be used to learn the Japanese script called
\textbf{hiragana}.

}
%
\jkatakanaonly{

This book is the \textbf{second} volume \jvolumes{②} of \textit{The Japanese
Language - Nihongo} series that can be used to learn the Japanese script called
\textbf{katakana}.

}

While there is no specific reason to start with \jtopic or \jtopicopposite when
it comes to learn the first Japanese writing script, historically
\textbf{hiragana} is the first choice. It might be the assumption that hiragana
are more common in Japanese texts than katakana and also when Japanese Chinese
characters (kanji) are transcribed (for example as station names) hiragana is
used. However, since a beginner in learning Japanese also have almost no
vocabulary knowledge, the usability is overestimated by native Japanese people.
Actually, a foreigner from an English speaking background might consider to
learn katakana first, because foreign words in Japanese are written in katakana
and after mastering katakana such learner would be able to read and write
thousands of words with a little imagination, compared to almost none in case
of hiragana. \jhiraganaonly{This book assumes that you learn \textbf{hiragana}
either as first or second script.} \jkatakanaonly{This book assumes that you
learn \textbf{katakana} as first or second script.}

With whatever script to start with, it is recommended to finish the first
script and only then learn the second script. It is also recommended to finish
hiragana \textbf{and} katakana before learning Japanese Chinese characters
(kanji).

Common chapters are repeated or are slightly modified in each volume, so that a
learner can choose to start with volume {\jvolumes{①}} or {\jvolumes{②}} and
obtain all the references in each volume. Skip the passages you already know.

\section*{Self Learning}\jsec{独修}
\addcontentsline{toc}{section}{Self Learning}

% 独習者   どくしゅう・しゃ
% 独学者   どくがく・しゃ
% 独学の   どくがくの
% 独学で   どくがくで
% 自学する じ│がくする
% 独学する どくがくする
% 自修     じ・しゅう
% 独学     どく│がく
% 自学自習 じがく・じしゅう autodidaktisches Studium; Selbststudium
% 独修     どく・しゅう     schriftspr. Selbststudiumn; autodidaktisches Studium
% 独習書   どくしゅう・しょ Buch für Selbstunterricht; autodidaktisches Lehrbuch
% [o] LABEL
\label{sec:SelfLearning}
% [o] INDEX DESTINATION (DEF) \ifor only
% [o] INDEX TARGET
\ithree{self learning}{独学}{Selbststudium}
\ithree{self learning}{独修}{Selbstudium}
\ifor{hiragana}{平仮名}{ひらがな}{Hiragana}
\ifor{katakana}{片仮名}{かたかな}{Katakana}

Being able to read and write Japanese is a core skill when learning Japanese.
And \textbf{\jtopic} is one of the two very basic scripts of Japanese to be
learned. This book is written with the aim to help in that, based on self
experience as a learner of \jtopic{} as well as from teaching experience
and with feedback of many students. This edition of the book targets a self
leaning approach. So please report suggestions or problems.

Even though this book gives many information how to write and more important
how not to write \jtopic{}, it is astonishing how the human creativity can
generate version of Japanese \jtopic{} characters that are slightly wrong.
Therefore passages describing the good or bad shape of characters should be
red carefully.

The \hyperref[chap:JapaneseWritingSystem]{first chapter} will introduce the
\nameref{chap:JapaneseWritingSystem} and different alphabets. If you are
already familiar with it, you can safely skip this chapter. In any case all
terms are explained in the \hyperref[chap:Terminology]{last chapter}.

The %
%
\jhiraganaonly{\hyperref[chap:TheWayToWriteHiragana]{second chapter} %
(\nameref{chap:TheWayToWriteHiragana})}
\jkatakanaonly{\hyperref[chap:TheWayToWriteKatakana]{second chapter} %
(\nameref{chap:TheWayToWriteKatakana})} %
%
starts with the introduction of writing and reading single \textbf{\jtopic}
letters. The chapter ends with special \textbf{\jtopic} letters. It is advised
to read this chapter before starting the training.

\jhiraganaonly{

The \hyperref[chap:HiraganaTraining]{third chapter}
(\nameref{chap:HiraganaTraining}) goes right into the action by offering row
based training sessions for each character as well as simple training for
writing some Japanese \textbf{\jtopic} words. This is the \textbf{main} part of
the book and should be studied extensively.

}
\jkatakanaonly{

The \hyperref[chap:KatakanaTraining]{third chapter}
(\nameref{chap:KatakanaTraining}) goes right into the action by offering row
based training sessions for each character as well as simple training for
writing some Japanese \textbf{\jtopic} words. This is the \textbf{main} part of
the book and should be studied extensively.

}

The \hyperref[chap:Terminology]{last chapter} (\nameref{chap:Terminology})
provides an Latin alphabetically ordered \textbf{glossary} about the most
important keywords and concepts. It is recommended to read \textbf{one} article
at a time to deepen the understanding of the Japanese language in general and
the way of writing Japanese in particular. The order do not matter. It is not
mandatory to read this chapter to learn \jtopic.

The \textbf{appendix} contain tables of all important \jtopic{} written in
different fonts in the \nameref{chap:KanaTables} part starting at page
\pageref{chap:KanaTables}. Even though this is not explicit mentioned in the
following chapters it is important to have a look at these tables from time to
time when learning \jtopic{} to understand the margin (how much can be diverted
from the standard shape but the character is still recognized) of the character
to learn. The second part includes \hyperref[chap:RomajiTables]{two tables with
Latin letters} to memorize the pronunciation. In the forth part a list of used
\hyperref[chap:ListOfJapaneseTechnicalTerms]{technical terms in Japanese}
(partly translated to English and German) can be found with references to the
text where they are explained.  The last part of the appendix offer three
indices: in \hyperref[chap:EnglishIndex]{English} (Index) and
\hyperref[chap:GermanIndex]{German} (Fachbegriffe) for the learner and in
\hyperref[chap:JapaneseIndex]{Japanese} (索引) for the teacher.



\section*{Conventions Used in this Book}\jsec{使用方法}\label{sec:ConventionsUsesInThisBook}
\addcontentsline{toc}{section}{Conventions}

\ifor{kanji}{漢字}{かんじ}{Kanji}
\ifor{hiragana}{平仮名}{ひらがな}{Hiragana}
\ifor{katakana}{片仮名}{かたかな}{Katakana}
\ifor{rōmaji}{ローマ字}{ろーまじ}{Rōmaji}
\ifor{Hepburn System}{ヘボン式}{へぼんしき}{Hepburn System}

\begin{itemize}

    \item[\Link]

        External hyperlinks are marked with a blue arrow.

        \Info{Example:}{
        \begin{center}

            Please look at the download page for this document, if there is a
            new version\\ \Link
            \href{https://christian.kuelker.info/nihongo/}{https://christian.kuelker.info/nihongo/}

        \end{center}
        }{true}

    \item[{【}\ldots{】}]

        The reading of Japanese characters (\hyperref[sec:Kanji]{kanji}) are
        \textbf{not} given in the section or chapter heading but as soon as
        possible. If the reading is given it will be given in
        \hyperref[sec:Hiragana]{hiragana} script. To mark this reading it will
        start with a Japanese bracket '{【}' and end with a Japanese bracket
        '{】}'.

        \Info{Example:}{
        \begin{center} \Large Kanji {漢字} {【かんじ】} \end{center}
        }{true}

        \medskip

    \item[\jtl{\ldots}]

         If readings, \textbf{transliteration} of Japanese are also given in
         \hyperref[sec:Romaji]{rōmaji} according to the
         \hyperref[sec:Hepburn]{Hepburn system}, this is indicated by an
         \textit{angle bracket} '\textlangle{}' at the beginning of the reading
         (transliteration) and an \textit{angle bracket} at the end
         '\textrangle{}'.
%        This follow the International Phonetic Alphabet (IPA)
%        rules of \Link
%        \href{https://en.wikipedia.org/wiki/International_Phonetic_Alphabet#Brackets_and_transcription_delimiters}{brackets
%        and transcription delimiters}.

        \Info{Example:}{
        \begin{center} First hiragana letter \Large {あ} {\jtl{a}} \end{center}
        \begin{center} First katakana letter \Large {ア} {\jtl{a}} \end{center}
        }{true}

% Remark: The problem of using phonemic and phonetic transcription is that it
%         uses the IPA alphabet and is not easy to understand
%         While \jphonetic{ɕimbɯn} seems correct, \jphonemic{simbun} is
%         questionable: correct si, m and n, but bu?
%
%   \item[\jphonemic{\ldots}]
%
%        If the \textbf{phonemic} pronunciation of Japanese is also given in
%        the International Phonetic Alphabet (IPA), this is indicated by a
%        \textit{slash} '/' at the beginning of the reading and a
%        \textit{slash} at the end '/'. This indicates an abstract phonemic
%        notation by expressing only features that are distinctive.
%
%        In simple terms, the phonemic notation has more details regarding
%        sound in comparison to the transliteration.
%
%        Example:
%
%        \begin{center}
%          \begin{tabular}{llllll}
%            \textbf{Kanji}&\textbf{Hiragana}       &\textbf{Transliteration}&\textbf{Phonemic}    &\textbf{Phonetic}    &\textbf{English}\\
%            {新聞}        &ん: \jhiragana{しんぶん}&n: \jtl{shinbun}        &m: \jphonemic{simbun}&m: \jphonetic{ɕimbɯn}&News paper\\
%            {進軍}        &ん: \jhiragana{しんぐん}&n: \jtl{shingun}        &n: \jphonemic{singun}&ŋ: \jphonetic{ɕiŋɣɯn}&March, advance\\
%          \end{tabular}
%        \end{center}
%
%   \item[\jphonetic{\ldots}]
%
%       If the (phonetic) prounciation of Japanese is given, it includes
%       usually details beyond the phonemic pronunciation and is according to
%       \Link
%       \href{https://en.wikipedia.org/wiki/International_Phonetic_Alphabet#Brackets_and_transcription_delimiters}{IPA}
%       indicated by \textit{square brackets}.
%
%       \medskip \textit{Example:} \medskip
%
%       The \textit{kana} letter \jquotesingleja{ん} or \jquotesingleja{ン} is phonetically pronounced differently, depending on the
%       letter context of the word.
%
%       \begin{tabular}{ll}
%       \jphonetic{n} &before n, t, d, r, ts, z, ch and j\\
%       \jphonetic{m} &before m, p and b\\
%       \jphonetic{ŋ} &before k and g\\
%       \jphonetic{ɴ} &at the end of utterances\\
%       \jphonetic{ũ͍} &before vowels, palatal approximants (y), consonants h, f, s, sh and w\\
%       \jphonetic{ĩ} &after the vowel i if another vowel, palatal approximant or consonant f, s, sh, h or w follows\\
%       \end{tabular}
%       {\tiny\Link\textit{Source: \url{https://en.wikipedia.org/wiki/N_(kana)}}}

    \item[\jquotesingleja{\ldots}]

            The graphemes '\textbf{「}' and '\textbf{」}' are Japanese
            quotation marks. They will be used to quote something written in
            Japanese.  This can be kanji, hiragana or katakana.

            \Info{Example of Japanese quotation marks:}{

              The Japanese hiragana \jquotesingleja{あ}is written
              \jquotesingle{a} in the Hepburn system.

            }{true}

    \item[{\jpitch[br]{{ }}\jpitch[lt]{{ }}}]

            A red line is used to indicate the Japanese pronunciation
            intonation pitch. A line above a letter, character or grapheme
            indicate a high intonation. A line below indicate a low intonation
            and a line in-between indicate a shift in intonation.

            \Info{Red intonation line example:}{

              High and then low katakana {\jpitch[tr]{{ア}}\jpitch[lb]{メ}},
              low and then high hiragana {\jpitch[br]{あ}\jpitch[lt]{め}} and
              as a transcription high and low
              {\jtl{\jpitch[tr]{a}\jpitch[lb]{me}}}.

            }{true}


\end{itemize}

