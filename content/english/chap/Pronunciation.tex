% +---------------------------------------------------------------------------+
% | english/chap/Pronunciation.tex                                            |
% |                                                                           |
% | Pronounciation of hiragana and katakana.                                  |
% |                                                                           |
% | Version: 0.1.0                                                            |
% |                                                                           |
% | Changes:                                                                  |
% |                                                                           |
% | 0.1.1 2022-09-10 Christian Külker <c@c8i.org>                             |
% |     - Rename to english/chap/Pronunciation.tex                            |
% | 0.1.0 2020-07-10 Christian Külker <c@c8i.org>                             |
% |     - Initial release (as english/sec/KanaPronunciation.tex)              |
% |                                                                           |
% +---------------------------------------------------------------------------+
\chapter{Pronunciation}
\label{chap:Pronunciation}\label{sec:Pronunciation}

This chapter explains how to pronounce written Japanese with the focus on
\jkanavoc. After the layout of the pronunciation structure it gives
a  detailed view of the intonation.

The \textbf{pronunciation} of \jkanavoc is exactly the same as for
\hyperref[sec:\jscript]{\jtopic} and most sounds are very close to the Latin
\textbf{pronunciation} but in general are \textbf{pronounced} a little shorter
without any stress. Only the \jtl{ra} sounds, like in \jtl{ra}, \jtl{ri},
\jtl{ru}, \jtl{re} and \jtl{ro} have no similarity in European languages.


The sound of the Japanese /r/ is  neither a central nor a lateral flap, but may
vary between the two. To an English speaker, its pronunciation varies between a
flapped 'd' (as in American English buddy) and a flapped 'l'.
\href{http://en.wikipedia.org/wiki/Japanese_phonology}{(Wikipedia Japanese
Phonology)}.


The following table displays the \textbf{pronunciation} in the
\hyperref[sec:Gojuonzu]{gojūonzu}.

\ien{Rōmaji Gojūonzu}
\ien{Rōmaji}
\ien{Gojūon}
\ija{ローマ字五十音図}
\ija{ローマ字}
\bigskip
\begin{center}
%\LARGE
%\Huge
\Padding
\begin{tabular}{c||c|c|c|c|c|}
&\textbf{a}&\textbf{i}&\textbf{u}&\textbf{e}&\textbf{o}\\\hline\hline
\textbf{-}&a&i&u&e&o\\\hline
\textbf{k}&ka&ki&ku&ke&ko\\\hline
\textbf{s}&sa&shi&su&se&so\\\hline
\textbf{t}&ta&chi&tsu&te&to\\\hline
\textbf{n}&na&ni&nu&ne&no\\\hline
\textbf{h}&ha&hi&fu&he&ho\\\hline
\textbf{m}&ma&mi&mu&me&mo\\\hline
\textbf{y}&ya&&yu&&yo\\\hline
\textbf{r}&ra&ri&ru&re&ro\\\hline
\textbf{w}&wa&&&&o\\\hline
\textbf{{*}}&n&&&&\\\hline
\end{tabular}
\end{center}


\section{Structure}

% Sound structure:
% - mora (haku,moora)
% - syllable

A \textbf{mora}, known Japanese as \ivoc{haku}{拍}{はく}{Haku} or
\ivoc{モーラ}{モーラ}もおら{Mōra}, is similar to a
\hyperref[sec:Syllable]{syllable} and can be found in all languages. The plural
form of mora is \hyperref[sec:Mora]{morae}. A mora is a timing unit of fixed
length that is equal or shorter than a syllable. The spoken Japanese language
structure is based on morae, rather than syllables.

In Japanese each normaly sized kana is usually one mora. Also so called
constructed sounds that consists of a normally sized kana and a small kana
are one mora. And there some special mora that can not be pronounced by
it self, but sill occupy one time slot.

 That means that unlike English, the timing of the
pronounciation of Japanese is equal and exact.

Also unlike most European languages the writing
and the morae have fixed ralation.

The two
Japanese scripts \textbf{hiragana} and \textbf{katakana} are based on morae.
Therefore, instaed of using the word letter the book will use the word mora,
when it references a Japanese sound entity and its written representation.

It is possible to write Japanese with our letters. We refer to our lettes as
Latin letters, while in Japan our letters are named Roman letters. Japanese
sentences look very similar to European languages when writing them with Latin
letters. Reading Japanese sentences with Latin letters has one major problem
in regard to the Japanese pronunciation. While English, German, French and
other European languages are based on syllables and syllables have different
length, Japanese morae have the same length.

For example is the English \jtl{i} and \jtl{shi} of different length.
Transcribed to Japanese \jtl{i} becomes \jquotesingleja{\jkanaletteri} and
\jtl{shi} becomes \jquotesingleja{\jkanalettershi}. Both are just a single
letter in the Japanese script \jtopic. An English speaker usually assumes
\jtl{i} and \jtl{shi} have different length in pronunciation. In contrary it is
natural to assume for Japanese speaker that \jquotesingleja{\jkanaletteri} and
\jquotesingleja{\jkanalettershi} have the same lenghth.

\section{Intonation}\jsec{イントネーション}

It is tempting to call the section about intonation "Eating Candy in the Rain"
or "Using Chopsticks on the Bridge"\footnote{The author spoke about chopsticks
in Japanese with the correct intonation and got confusing results while
crossing a bridge with a Japanese native in the Chiba prefecture in 1994.}. Two
things wich should be avoided during a Japanese converstion.

\ifor{pronunciation}{発音}{はつおん}{Aussprache}
\ifor{intonation}{イントネーション}{いんとねーしょん}{Betonung}
\ifor{katakana}{片仮名}{かたかな}{Katakana}
\ifor{hiragana}{平仮名}{ひらがな}{Hiragana}
\ifor{mora}{モーラ}{もーら}{Mora}
\ifor{gojūonzu}{五十音図}{ごじゅうおんず}{50@50 Laute Tafel}

The \textbf{pronunciation} of \hyperref[sec:Hiragana]{hiragana} is the same as
for \hyperref[sec:Katakana]{katakana}. Therefore every
\hyperref[sec:Syllable]{syllable}, more precise every \hyperref[sec:Mora]{mora}
corresponds to a \hyperref[sec:\jscript]{\jtopic} character and is constructed
as 'consonant' + 'vowel' with the exception of \jtl{n}. This system of letter for
each \hyperref[sec:Mora]{mora} makes \textbf{pronunciation} absolutely clear
with no ambiguities. However the simplicity of \hyperref[sec:\jscript]{\jtopic}
does not mean that \textbf{pronunciation} in Japanese is simple for English
speakers as it is for Germans. The rigid structure of the fixed
\hyperref[sec:Mora]{mora} sound in Japanese creates the challenge of learning
the proper intonation and duration of Japanese \textbf{pronunciation}.

Almost each Japanese word can be chunked into \hyperref[sec:Mora]{morae} of
high and low pitch witch is a crucial aspect of the spoken language. Compared
to Chinese, Japanese luckily have only two pitches: hi and low. Sometimes this
difference can be even important for the lexis. Words with exactly the same
kana writings but different kanji can have
a difference in pitch which make them distinguishable. The intonation of high
and low pitches is a crucial aspect of the spoken language.

{
\begin{table}[H]
  \begin{center}
    \setlength{\fboxsep}{.2ex}
    \bgroup
      \def\arraystretch{1.2}%  1 is the default
      \begin{tabular}{llllll}
        \textbf{English}&\textbf{Japanese}&
        \multicolumn{1}{p{2cm}}{\textbf{Intonation rōmaji}}&
        \multicolumn{1}{p{2cm}}{\textbf{Intonation hiragana}}&
        \multicolumn{1}{p{2cm}}{\textbf{Intonation katakana}}&\textbf{Remark}\\
        Rain    &雨      &\jtl{\jpitch[tr]{a}\jpitch[lb]{me}}&\jpitch[tr]{あ}\jpitch[lb]{め}&\jpitch[tr]{ア}\jpitch[lb]{メ}&\\
        Sky     &天      &\jtl{\jpitch[tr]{a}\jpitch[lb]{me}}&\jpitch[tr]{あ}\jpitch[lb]{め}&\jpitch[tr]{ア}\jpitch[lb]{メ}&\\
        Candy   &飴/あめ&\jtl{\jpitch[br]{a}\jpitch[lt]{me}}&\jpitch[br]{あ}\jpitch[lt]{め}&\jpitch[br]{ア}\jpitch[lt]{メ}&Usually kanji is not used\\
        American&アメ&\jtl{ame}&あ{ }め&ア{ }メ&Slang\\
      \end{tabular}
    \egroup
    \caption{Intonation of \jtl{ame}}
    \label{tab:IntonationOfAme}
  \end{center}
\end{table}
}

The \hyperref[tab:IntonationOfAme]{above} example of \jtl{ame} looks simple. A
given Japanese word of two morae can have one intonation or the other or in
some cases none (either heigh or low). And it seems that one kanji can only
have one intonation per pronunciation. However that is not true. A given kanji
like \jquotesingleja{橋} \hyperref[tab:IntonationOfHashi]{for example} can have
more than one intonation, depending on its meaning and context.

{
\begin{table}[H]
  \begin{center}
    \setlength{\fboxsep}{.2ex}
    \bgroup
      \def\arraystretch{1.2}%  1 is the default
      \begin{tabular}{llllll}
        \textbf{English}&\textbf{Japanese}&
        \multicolumn{1}{p{2cm}}{\textbf{Intonation rōmaji}}&
        \multicolumn{1}{p{2cm}}{\textbf{Intonation hiragana}}&
        \multicolumn{1}{p{2cm}}{\textbf{Intonation katakana}}&\textbf{Remark}\\
        Edge     &端    &\jtl{\jpitch[br]{ha}\jpitch[lt]{shi}}&\jpitch[br]{は}\jpitch[lt]{し}&\jpitch[br]{ハ}\jpitch[lt]{シ}&\\
        Bridge   &橋    &\jtl{\jpitch[br]{ha}\jpitch[lt]{shi}}&\jpitch[br]{は}\jpitch[lt]{し}&\jpitch[br]{ハ}\jpitch[lt]{シ}&\\
        Hashi    &橋    &\jtl{\jpitch[tr]{ha}\jpitch[lb]{shi}}&\jpitch[tr]{は}\jpitch[lb]{し}&\jpitch[tr]{ハ}\jpitch[lb]{シ}&Family name\\
        Chopstick&箸    &\jtl{\jpitch[tr]{ha}\jpitch[lb]{shi}}&\jpitch[tr]{は}\jpitch[lb]{し}&\jpitch[tr]{ハ}\jpitch[lb]{シ}&Usually \jquotesingleja{お箸}\\
        Stairs   &階;梯&\jtl{\jpitch[tr]{ha}\jpitch[lb]{shi}}&\jpitch[tr]{は}\jpitch[lb]{し}&\jpitch[tr]{ハ}\jpitch[lb]{シ}&\\
        Beak     &嘴    &\jtl{\jpitch[tr]{ha}\jpitch[lb]{shi}}&\jpitch[tr]{は}\jpitch[lb]{し}&\jpitch[tr]{ハ}\jpitch[lb]{シ}&\\
        Persia   &波斯  &\jtl{\jpitch[tr]{ha}\jpitch[lb]{shi}}&\jpitch[tr]{は}\jpitch[lb]{し}&\jpitch[tr]{ハ}\jpitch[lb]{シ}&Old Chinese name of Iran\\
      \end{tabular}
    \egroup
    \caption{Intonation of \jtl{hashi}}
    \label{tab:IntonationOfHashi}
  \end{center}
\end{table}
}


As Japanese is a context sensitive language make sure you do not talk about
candy in the rain or about chopsticks while crossing a bridge.

It is also true that the intonation varies among areas in Japan. The Tokyo area
is considered standard when it comes to intonation. Other areas can have
different intonations or the complete opposite intonation. When it comes to
intonation training, it is advised to train intonation with a native speaker
from Kanto region or a person who speaks the standard intonation. Nonetheless,
if this is not possible, the next best thing to do is to learn only
\textbf{one} other intonation and not a mixture of different intonations. If a
consistency in the intonation origin can not obtained, it might be better to
skip the training altogether.

One of the biggest
problems for obtaining a natural sounding \textbf{pronunciation} is the
incorrect intonation. Many European or American learners speak without paying
attention to the correct pitch. That makes the speech sound non-natural for
Japanese. In some language course try to let the learner memorize the natural
pitch of a word or even teach rules for memorization. While there is clearly a
possibility for linguistic rules, they are hard to remember and master. It is
still possible to learn the correct intonation by resorting to language
learning techniques used by infants or small children: mimicking native
Japanese speakers. Therefore it is highly advised to expose oneself to as many
Japanese spoken language as possible and to mimic it. Radio, podcasts, drama
and television to name a few. However, it is not advised to listen too much
artificial sources like anime or commercials.

\bigskip
\begin{tabular}{rl}
-&every (yes \textbf{every}) \hyperref[sec:Mora]{mora} is \textbf{pronounced}
  with the same length\\
-&there is no short and long \hyperref[sec:Mora]{mora} or letters\\
-&every \hyperref[sec:Mora]{mora} has a pitch: high or low\\
-&every pitch matters\\
-&the pitch can change  sometimes with its context\\
-&the pitch can change with a dialect - however standard Japanese has well
  defined pitches\\
\end{tabular}

\bigskip

\section{Pronunctiation}\jsec{発音}







