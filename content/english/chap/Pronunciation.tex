% +---------------------------------------------------------------------------+
% | english/chap/Pronunciation.tex                                            |
% |                                                                           |
% | Pronunciation of hiragana and katakana.                                   |
% |                                                                           |
% | Version: 0.1.3                                                            |
% |                                                                           |
% | Changes:                                                                  |
% |                                                                           |
% | 0.1.3 2024-03-28 Christian Külker <c@c8i.org>                             |
% |     - Fix typo                                                            |
% | 0.1.2 2023-06-05 Christian Külker <c@c8i.org>                             |
% |     - Improve English writing and paragraph about lexicon                 |
% | 0.1.1 2022-09-10 Christian Külker <c@c8i.org>                             |
% |     - Rename to english/chap/Pronunciation.tex                            |
% | 0.1.0 2020-07-10 Christian Külker <c@c8i.org>                             |
% |     - Initial release (as english/sec/KanaPronunciation.tex)              |
% |                                                                           |
% +---------------------------------------------------------------------------+
\chapter{Pronunciation}
\label{chap:Pronunciation}\label{sec:Pronunciation}

This chapter explains how to pronounce written Japanese with a focus on
\jkanavoc. After explaining the structure of pronunciation, it gives a detailed
view of intonation.

The \textbf{pronunciation} of \jkanavoc is exactly the same as for
\hyperref[sec:\jscript]{\jtopic}, and most sounds are very close to the Latin
\textbf{pronunciation}, but generally \textbf{pronounced} a little shorter
without any stress. Only the \jtl{ra} sounds, as in \jtl{ra}, \jtl{ri},
\jtl{ru}, \jtl{re} and \jtl{ro}, have no similarity in European languages.


The sound of the Japanese /r/ is  neither a central nor a lateral flap, but may
vary between the two. To an English speaker, its pronunciation varies between a
flapped 'd' (as in American English buddy) and a flapped 'l'.
\href{http://en.wikipedia.org/wiki/Japanese_phonology}{(Wikipedia Japanese
Phonology)}.


The following table shows the \textbf{pronunciation} in the
\hyperref[sec:Gojuonzu]{gojūonzu}.

\ien{Rōmaji Gojūonzu}
\ien{Rōmaji}
\ien{Gojūon}
\ija{ローマ字五十音図}
\ija{ローマ字}
\bigskip
\begin{center}
%\LARGE
%\Huge
\Padding
\begin{tabular}{c||c|c|c|c|c|}
&\textbf{a}&\textbf{i}&\textbf{u}&\textbf{e}&\textbf{o}\\\hline\hline
\textbf{-}&a&i&u&e&o\\\hline
\textbf{k}&ka&ki&ku&ke&ko\\\hline
\textbf{s}&sa&shi&su&se&so\\\hline
\textbf{t}&ta&chi&tsu&te&to\\\hline
\textbf{n}&na&ni&nu&ne&no\\\hline
\textbf{h}&ha&hi&fu&he&ho\\\hline
\textbf{m}&ma&mi&mu&me&mo\\\hline
\textbf{y}&ya&&yu&&yo\\\hline
\textbf{r}&ra&ri&ru&re&ro\\\hline
\textbf{w}&wa&&&&o\\\hline
\textbf{{*}}&n&&&&\\\hline
\end{tabular}
\end{center}


\section{Structure}

% Sound structure:
% - mora (haku,moora)
% - syllable

\textbf{Mora}, known in Japanese as \ivoc{haku}{拍}{はく}{Haku} or
\ivoc{モーラ}{モーラ}もおら{Mōra}, is similar to a
\hyperref[sec:Syllable]{syllable} and can be found in all languages. The plural
form of mora is \hyperref[sec:Mora]{morae}. A mora is a unit of time of fixed
length, equal to or shorter than a syllable. The structure of the spoken
Japanese language is based on morae rather than syllables.

In Japanese, each normal-sized kana is usually one mora. So-called constructed
sounds, which consist of a normal-sized kana and a small kana, are also one
mora. And there are some special moras that cannot be pronounced by themselves,
but still occupy a time slot. This means that, unlike English, the timing of
Japanese is equal and exact. Also unlike most European languages, writing and
mora have a fixed relation.

The two Japanese scripts \textbf{hiragana} and \textbf{katakana} are based on
morae. Therefore, instead of using the word letter, this book will use the
word mora when referring to a Japanese sound unit and its written
representation.

It is possible to write Japanese with our characters. We call our letters Latin
letters, while in Japan our letters are called Roman letters. Japanese
sentences look very similar to European languages when written with Latin
letters. Reading Japanese sentences with Roman characters has a major problem
with Japanese pronunciation. While English, German, French, and other European
languages are based on syllables, and syllables have different lengths,
Japanese morae have the same length.

For example, the English \jtl{i} and \jtl{shi} are of different lengths.
Transcribed into Japanese, \jtl{i} becomes \jquotesingleja{\jkanaletteri} and
\jtl{shi} becomes \jquotesingleja{\jkanalettershi}. Both are just a single
letter in the Japanese script \jtopic. An English speaker usually assumes that
\jtl{i} and \jtl{shi} have different lengths in pronunciation. In contrast, it
is natural for a Japanese speaker to assume that \jquotesingleja{\jkanaletteri}
and \jquotesingleja{\jkanalettershi} have the same length.

\section{Intonation}\jsec{イントネーション}

It is tempting to call the section on intonation "Eating Candy in the Rain" or
"Using Chopsticks on the Bridge"\footnote{The author talked about chopsticks in
Japanese with the correct intonation and got confused results while crossing a
bridge with a Japanese native in Chiba Prefecture in 1994.}. Two things to
avoid during a Japanese conversation.

\ifor{pronunciation}{発音}{はつおん}{Aussprache}
\ifor{intonation}{イントネーション}{いんとねーしょん}{Betonung}
\ifor{katakana}{片仮名}{かたかな}{Katakana}
\ifor{hiragana}{平仮名}{ひらがな}{Hiragana}
\ifor{mora}{モーラ}{もーら}{Mora}
\ifor{gojūonzu}{五十音図}{ごじゅうおんず}{50@50 Laute Tafel}

The \textbf{pronunciation} of \hyperref[sec:Hiragana]{hiragana} is the same as
for \hyperref[sec:Katakana]{katakana}. Therefore, each
\hyperref[sec:Syllable]{syllable}, more precisely each
\hyperref[sec:Mora]{mora}, corresponds to a \hyperref[sec:\jscript]{\jtopic}
character and is constructed as 'consonant' + 'vowel' with the exception of
\jtl{n}. This system of letters for each \hyperref[sec:Mora]{mora} makes
\textbf{pronunciation} absolutely clear without any ambiguity. However, the
simplicity of \hyperref[sec:\jscript]{\jtopic} does not mean that
\textbf{pronunciation} in Japanese is as easy for English speakers as it is for
Germans. The rigid structure of the fixed \hyperref[sec:Mora]{mora} sound in
Japanese creates the challenge of learning the correct intonation and duration
of Japanese \textbf{pronunciation}.

Almost every Japanese word can be broken down into \hyperref[sec:Mora]{morae}
of high and low pitch, which is a crucial aspect of the spoken language.
Compared to Chinese, Japanese fortunately has only two pitches: high and low.
This distinction in pitch is so significant in the Japanese language that even
words spelled identically in hiragana or katakana can have different meanings
based on their pitch patterns, warranting separate entries in the lexicon.
Further complicating this, words that are spelled the same way in kana can be
sometimes differentiated by their kanji writing. Even if they have the same or
different pitch patterns, different kanji representations give the words
distinct meanings, adding to the richness and complexity of the Japanese
lexicon. Therefore, understanding the nuances of pitch and word writing,
including the high and low pitch intonation and kanji usage, becomes an
essential aspect of mastering the spoken language.

{
\begin{table}[H]
  \begin{center}
    \setlength{\fboxsep}{.2ex}
    \bgroup
      \def\arraystretch{1.2}%  1 is the default
      \begin{tabular}{llllll}
        \textbf{English}&\textbf{Japanese}&
        \multicolumn{1}{p{2cm}}{\textbf{Intonation rōmaji}}&
        \multicolumn{1}{p{2cm}}{\textbf{Intonation hiragana}}&
        \multicolumn{1}{p{2cm}}{\textbf{Intonation katakana}}&\textbf{Remark}\\
        Rain    &雨      &\jtl{\jpitch[tr]{a}\jpitch[lb]{me}}&\jpitch[tr]{あ}\jpitch[lb]{め}&\jpitch[tr]{ア}\jpitch[lb]{メ}&\\
        Sky     &天      &\jtl{\jpitch[tr]{a}\jpitch[lb]{me}}&\jpitch[tr]{あ}\jpitch[lb]{め}&\jpitch[tr]{ア}\jpitch[lb]{メ}&\\
        Candy   &飴/あめ&\jtl{\jpitch[br]{a}\jpitch[lt]{me}}&\jpitch[br]{あ}\jpitch[lt]{め}&\jpitch[br]{ア}\jpitch[lt]{メ}&Usually kanji is not used\\
        American&アメ&\jtl{ame}&あ{ }め&ア{ }メ&Slang\\
      \end{tabular}
    \egroup
    \caption{Intonation of \jtl{ame}}
    \label{tab:IntonationOfAme}
  \end{center}
\end{table}
}

The \hyperref[tab:IntonationOfAme]{above} example of \jtl{ame} looks simple. A
given Japanese word with two morae can have one or the other intonation, or in
some cases none (either high or low). And it seems that a kanji can have only
one intonation per pronunciation. However, this is not true. A given kanji such
as \jquotesingleja{橋} \hyperref[tab:IntonationOfHashi]{for example} can have
more than one intonation, depending on its meaning and context.

{
\begin{table}[H]
  \begin{center}
    \setlength{\fboxsep}{.2ex}
    \bgroup
      \def\arraystretch{1.2}%  1 is the default
      \begin{tabular}{llllll}
        \textbf{English}&\textbf{Japanese}&
        \multicolumn{1}{p{2cm}}{\textbf{Intonation rōmaji}}&
        \multicolumn{1}{p{2cm}}{\textbf{Intonation hiragana}}&
        \multicolumn{1}{p{2cm}}{\textbf{Intonation katakana}}&\textbf{Remark}\\
        Edge     &端    &\jtl{\jpitch[br]{ha}\jpitch[lt]{shi}}&\jpitch[br]{は}\jpitch[lt]{し}&\jpitch[br]{ハ}\jpitch[lt]{シ}&\\
        Bridge   &橋    &\jtl{\jpitch[br]{ha}\jpitch[lt]{shi}}&\jpitch[br]{は}\jpitch[lt]{し}&\jpitch[br]{ハ}\jpitch[lt]{シ}&\\
        Hashi    &橋    &\jtl{\jpitch[tr]{ha}\jpitch[lb]{shi}}&\jpitch[tr]{は}\jpitch[lb]{し}&\jpitch[tr]{ハ}\jpitch[lb]{シ}&Family name\\
        Chopstick&箸    &\jtl{\jpitch[tr]{ha}\jpitch[lb]{shi}}&\jpitch[tr]{は}\jpitch[lb]{し}&\jpitch[tr]{ハ}\jpitch[lb]{シ}&Usually \jquotesingleja{お箸}\\
        Stairs   &階;梯&\jtl{\jpitch[tr]{ha}\jpitch[lb]{shi}}&\jpitch[tr]{は}\jpitch[lb]{し}&\jpitch[tr]{ハ}\jpitch[lb]{シ}&\\
        Beak     &嘴    &\jtl{\jpitch[tr]{ha}\jpitch[lb]{shi}}&\jpitch[tr]{は}\jpitch[lb]{し}&\jpitch[tr]{ハ}\jpitch[lb]{シ}&\\
        Persia   &波斯  &\jtl{\jpitch[tr]{ha}\jpitch[lb]{shi}}&\jpitch[tr]{は}\jpitch[lb]{し}&\jpitch[tr]{ハ}\jpitch[lb]{シ}&Old Chinese name of Iran\\
      \end{tabular}
    \egroup
    \caption{Intonation of \jtl{hashi}}
    \label{tab:IntonationOfHashi}
  \end{center}
\end{table}
}


Since Japanese is a context-sensitive language, be sure not to talk about candy
in the rain or chopsticks while crossing a bridge.

It is also true that intonation varies from area to area in Japan. The Tokyo
area is considered standard when it comes to intonation. Other areas may have
different intonation or the opposite intonation. When it comes to intonation
training, it is advisable to practice intonation with a native speaker from the
Kanto area or a person who speaks the standard intonation. However, if this is
not possible, the next best thing is to learn only \textbf{one} other
intonation and not a mixture of different intonations. If consistency in
intonation origin cannot be achieved, it may be better to skip the training
altogether.

One of the biggest problems in achieving a natural sounding
\textbf{pronunciation} is incorrect intonation. Many European or American
learners speak without paying attention to pitch. This makes the speech sound
unnatural to Japanese. Some language courses try to have the learner memorize
the natural pitch of a word or even teach memorization rules. While there is
clearly a way for linguistic rules to work, they are hard to remember and
master. It is still possible to learn the correct intonation by resorting to a
language learning technique used by infants and toddlers: imitating native
speakers. Therefore, it is highly recommended to listen to and imitate as much
spoken Japanese as possible. Radio, podcasts, plays, and television, to name a
few. However, it is not advisable to listen to too many artificial sources such
as anime or commercials.

\bigskip
\begin{tabular}{rl}
-&Every (yes \textbf{every}) \hyperref[sec:Mora]{mora} is \textbf{pronounced}
  with the same length\\
-&There are no short and long \hyperref[sec:Mora]{mora} or letters\\
-&Every \hyperref[sec:Mora]{mora} has a pitch: high or low\\
-&Every pitch matters\\
-&Pitch can sometimes change with its context\\
-&Pitch can change with a dialect - but standard Japanese has well-defined
  pitches\\
\end{tabular}

\bigskip

\section{Pronunciation}\jsec{発音}

