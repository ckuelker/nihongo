% \input can not handle \jscript{}
\newcommand{\jTableKana}[2]{\jptab/#1#2}
%\newcommand{\jTablKanaJapaneseDefault}[1]{\jptab/#1JapaneseDefault}

%/\chapter{Katakana Tables}\jchap{{片仮名図} \label{chap:KatakanaTables}
%\chapter{\jscript{} Tables}\jchap{{\jsubtitlejapanese}図} \label{chap:{\jscript{}}Tables}
\chapter{\jscript{} Tables}\jchap{\jsubtitlejapanese{}図}\label{chap:KanaTables}
\ithree{\jtopic{}!tables}{平仮名図}{\jscript{}!Tabellen}

\section{\jscript{} Reference}\jsec{片仮名のレファレンス}
%\input{\jptab/\jscript{}Reference}

\section{\jscript{} Writing Reference}\jsec{片仮名の書くレファレンス}
%\input{\jppath{\jptab}{\jscript}{WritingReference}}
\begin{center}
\Padding
\begin{tabular}{|c||c|c|c|c|c|}\hline
       &\Huge    a &\Huge    i & \Huge   u &\Huge    e &\Huge    o\\\hline\hline
\Huge *&\Jletter{as}&\Jletter{is}&\Jletter{us}&\Jletter{es}&\Jletter{os}\\\hline
\Huge k&\Jletter{kas}&\Jletter{kis}&\Jletter{kus}&\Jletter{kes}&\Jletter{kos}\\\hline
\Huge s&\Jletter{sas}&\Jletter{shis}&\Jletter{sus}&\Jletter{ses}&\Jletter{sos}\\\hline
\Huge t&\Jletter{tas}&\Jletter{chis}&\Jletter{tsus}&\Jletter{tes}&\Jletter{tos}\\\hline
\Huge n&\Jletter{nas}&\Jletter{nis}&\Jletter{nus}&\Jletter{nes}&\Jletter{nos}\\\hline
\Huge h&\Jletter{has}&\Jletter{his}&\Jletter{fus}&\Jletter{hes}&\Jletter{hos}\\\hline
\Huge m&\Jletter{mas}&\Jletter{mis}&\Jletter{mus}&\Jletter{mes}&\Jletter{mos}\\\hline
\Huge y&\Jletter{yas}&\Jletter{s}&\Jletter{yus}&\Jletter{s}&\Jletter{yos}\\\hline
\Huge r&\Jletter{ras}&\Jletter{ris}&\Jletter{rus}&\Jletter{res}&\Jletter{ros}\\\hline
\Huge w&\Jletter{was}&\Jletter{s}&\Jletter{s}&\Jletter{s}&\Jletter{wos}\\\hline
\Huge n&\Jletter{ns}&\Jletter{s}&\Jletter{s}&\Jletter{s}&\Jletter{s}\\\hline
\end{tabular}
\end{center}


%\section{\jscript{} Hand Writing Reference}\jsec{片仮名の手で書くレファレンス}
%\input{../share/\jscript{}HandWritingReference}

\newpage

% ---------------------------------------------------------------------------
\section{Empty Gojūonzu for Training}\jsec{練習のため五十音図}
\ithree{empty Gojūonzu}{練習のため五十音図}{50@50 Laute Tafel!leer}
\ithree{Gojūonzu training}{練習のため五十音図}{50@50 Laute Tafel!trainieren}
\label{app:Leere50LauteTafel} Please fill out this table (as fast as possible)
10 - 20 times a day in the active learning phase.
\begin{center}
\Padding
\begin{tabular}{|c||c|c|c|c|c|}\hline
                   &\Huge    a &\Huge    i & \Huge   u &\Huge    e &\Huge    o\\\hline\hline
\Huge *&\Kletter{s}&\Kletter{s}&\Kletter{s}&\Kletter{s}&\Kletter{s}\\\hline
\Huge k&\Kletter{s}&\Kletter{s}&\Kletter{s}&\Kletter{s}&\Kletter{s}\\\hline
\Huge s&\Kletter{s}&\Kletter{s}&\Kletter{s}&\Kletter{s}&\Kletter{s}\\\hline
\Huge t&\Kletter{s}&\Kletter{s}&\Kletter{s}&\Kletter{s}&\Kletter{s}\\\hline
\Huge n&\Kletter{s}&\Kletter{s}&\Kletter{s}&\Kletter{s}&\Kletter{s}\\\hline
\Huge h&\Kletter{s}&\Kletter{s}&\Kletter{s}&\Kletter{s}&\Kletter{s}\\\hline
\Huge m&\Kletter{s}&\Kletter{s}&\Kletter{s}&\Kletter{s}&\Kletter{s}\\\hline
\Huge y&\Kletter{s}&\Kletter{s}&\Kletter{s}&\Kletter{s}&\Kletter{s}\\\hline
\Huge r&\Kletter{s}&\Kletter{s}&\Kletter{s}&\Kletter{s}&\Kletter{s}\\\hline
\Huge w&\Kletter{s}&\Kletter{s}&\Kletter{s}&\Kletter{s}&\Kletter{s}\\\hline
\Huge n&\Kletter{s}&\Kletter{s}&\Kletter{s}&\Kletter{s}&\Kletter{s}\\\hline
\end{tabular}
\end{center}


% ---------------------------------------------------------------------------
\newpage\section{\jscript{} Gojūonzu}\jsec{片仮名五十音図}
\ithree{\jtopic{}!Gojūonzu}{片仮名五十音図}{50@50 Laute Tafel!\jscript{}}
\input{\jptab/\jscript}

% ---------------------------------------------------------------------------
%\newpage\section{\jscript{} (bold) Gojūonzu}
%\input{\jptab/\jscript{}Bold.tex}

% ---------------------------------------------------------------------------
\newpage\JapaneseDejima\section{\jscript{} Font Dejima}\jsec{出島フォント片仮名}
\ithree{\jtopic{}!Font Dejima}{出島フォント片仮名}{\jscript{}!Font Dejima }
\input{\jptab/\jscript}\JapaneseDefault

% ---------------------------------------------------------------------------
\newpage\JapaneseFontA\section{\jscript{} YOzAb}\jsec{\jquotesingleja{YOzAb}の片仮名}
\ithree{\jtopic{}!YOzAb}{\jquotesingleja{YOzAb}の片仮名}{\jscript{}!YOzAb}
\input{\jptab/\jscript}\JapaneseDefault

% ---------------------------------------------------------------------------
\newpage\JapaneseFontB\section{\jscript{} YOzC90b}\jsec{\jquotesingleja{YOzC90b}の片仮名}
\ithree{\jtopic{}!YOzC90b}{\jquotesingleja{YOzC90b}の片仮名}{\jscript{}!YOzC90b}
\input{\jptab/\jscript}\JapaneseDefault

% ---------------------------------------------------------------------------
\newpage\JapaneseFontC\section{\jscript{} YOzE90b}\jsec{\jquotesingleja{YOzE90b}の片仮名}
\ithree{\jtopic{}!YOzE90b}{\jquotesingleja{YOzE90b}の片仮名}{\jscript{}!YOzE90b}
\input{\jptab/\jscript}\JapaneseDefault

% ---------------------------------------------------------------------------
\newpage\JapaneseFontD\section{\jscript{} AoyagiSosekiFont2}\jsec{\jquotesingleja{AoyagiSosekiFont2}の片仮名}
\ithree{\jtopic{}!AoyagiSosekiFont2}{\jquotesingleja{AoyagiSosekiFont2}の片仮名}{\jscript{}!AoyagiSosekiFont2}
\input{\jptab/\jscript}\JapaneseDefault

% ---------------------------------------------------------------------------
\newpage\JapaneseFontE\section{\jscript{} IPAGothic}\jsec{\jquotesingleja{IPAGothic}の片仮名}
\ithree{\jtopic{}!IPAGothic}{\jquotesingleja{IPAGothic}の片仮名}{\jscript{}!IPAGothic}
\input{\jptab/\jscript}\JapaneseDefault

% ---------------------------------------------------------------------------
\newpage\JapaneseFontF\section{\jscript{} IPAMincho}\jsec{\jquotesingleja{IPAMincho}の片仮名}
\ithree{\jtopic{}!IPAMincho}{\jquotesingleja{IPAMincho}の片仮名}{\jscript{}!IPAMincho}
\input{\jptab/\jscript}\JapaneseDefault

% ---------------------------------------------------------------------------
\newpage\JapaneseFontG\section{\jscript{} KanjiStrokeOrders}\jsec{\jquotesingleja{KanjiStrokeOrders}の片仮名}
\ithree{\jtopic{}!KanjiStrokeOrders}{\jquotesingleja{KanjiStrokeOrders}の片仮名}{\jscript{}!KanjiStrokeOrders}
\input{\jptab/\jscript}\JapaneseDefault

% ---------------------------------------------------------------------------
\newpage\JapaneseFontH\section{\jscript{} kiloji - B}\jsec{\jquotesingleja{kiloji - B}の片仮名}
\ithree{\jtopic{}!kiloji - B}{\jquotesingleja{kiloji - B}の片仮名}{\jscript{}!kiloji - B}
\input{\jptab/\jscript}\JapaneseDefault

% ---------------------------------------------------------------------------
\newpage\JapaneseFontI\section{\jscript{} KouzanBrushFontGyousyo}\jsec{\jquotesingleja{KouzanBrushFontGyousyo}の片仮名}
\ithree{\jtopic{}!KouzanBrushFontGyousyo}{\jquotesingleja{KouzanBrushFontGyousyo}の片仮名}{\jscript{}!KouzanBrushFontGyousyo}
\input{\jptab/\jscript}\JapaneseDefault

% ---------------------------------------------------------------------------
\newpage\JapaneseFontJ\section{\jscript{} MotoyaLMaru}\jsec{\jquotesingleja{MotoyaLMaru}の片仮名}
\ithree{\jtopic{}!MotoyaLMaru}{\jquotesingleja{MotoyaLMaru}の片仮名}{\jscript{}!MotoyaLMaru}
\input{\jptab/\jscript}\JapaneseDefault

% ---------------------------------------------------------------------------
\newpage\JapaneseFontK\section{\jscript{} SetoFont}\jsec{\jquotesingleja{SetoFont}の片仮名}
\ithree{\jtopic{}!SetoFont}{\jquotesingleja{SetoFont}の片仮名}{\jscript{}!SetoFont}
\input{\jptab/\jscript}\JapaneseDefault

% ---------------------------------------------------------------------------
\newpage\JapaneseFontL\section{\jscript{} TakaoMincho}\jsec{\jquotesingleja{TakaoMincho}の片仮名}
\ithree{\jtopic{}!TakaoMincho}{\jquotesingleja{TakaoMincho}の片仮名}{\jscript{}!TakaoMincho}
\input{\jptab/\jscript}\JapaneseDefault

% ---------------------------------------------------------------------------
\newpage\JapaneseFontM\section{\jscript{} VL Gothic}\jsec{\jquotesingleja{VL Gothic}の片仮名}
\ithree{\jtopic{}!VL Gothic}{\jquotesingleja{VL Gothic}の片仮名}{\jscript{}!VL Gothic}
\input{\jptab/\jscript}\JapaneseDefault

% ---------------------------------------------------------------------------
\newpage
\JapaneseMikachanPB\section{\jscript{} MikachanPB}\jsec{\jquotesingleja{MikachanPB}の片仮名}
\label{sec:\jscript{}MikachanPB}
\ithree{\jtopic{}!MikachanPB}{\jquotesingleja{MikachanPB}の片仮名}{\jscript{}!MikachanPB}
\input{\jptab/\jscript}\JapaneseDefault

% ---------------------------------------------------------------------------
\newpage\section{\jscript{} Total Table}\jsec{全部片仮名}
\ithree{\jtopic{}!total table}{全部片仮名}{\jscript{}!Tabelle komplett}
\input{\jTableKana{\jscript}{All}}
