% ===========================================================================
\chapter{Terminology}\jchap{専門用語}
% [o] LABEL
\label{chap:Terminology}
\label{sec:Terminology}
% [o] INDEX DESTINATIOn (DEF)
\ifor{terminology}{専門用語}{せんもんようご}{Fachbegriffe}
% [o] INDEX TARGET

The following sections (ordered Latin alphabetically) can be used by itself to
understand some key concepts of the Japanese language by explaining keywords
or \ivoc{technical terms}{専門用語}{せんもんようご}{Fachbegriffe}.

\begingroup
\setlength{\cftbeforejtermtitleskip}{-2em}
\listofjterminology
\endgroup

% A
% B
% C
% D
% ---------------------------------------------------------------------------
\section{Dakuten}\jsec{濁点}
%
\label{sec:Dakuten}
\label{sec:Tenten}
% INDEX
%\ithree{diacritic sign}{濁点}{Diakritisches Zeichen}% Nigorierungszeichen
%\ithree{diacritic sign}{だくてん}{Diakritisches Zeichen}%Nigorierungszeichen
\ithree{diacritic sign, colloquial}{点々}{Diakritisches Zeichen, umgangsspr.}% Nigorierungszeichen
\ithree{diacritic sign, colloquial}{てんてん}{Diakritisches Zeichen, umgangsspr.}% Nigorierungszeichen
\ithree{umlaut}{ウムラウト}{Umlaut}
\ithree{syllable}{音節}{Silbe}
\ithree{kana}{仮名}{Kana}

\ifor{diacritic sign}{濁点}{だくてん}{diakritisches Zeichen}% Nigorierungszeichen

\newcommand{\ldakuten}{\ivoc{dakuten}{濁点}{だくてん}{diakritisches Zeichen}}
\newcommand{\lkana}{\ivoc{kana}{仮名}{かな}{Kana}}
\newcommand{\ltenten}{\ivoc{tenten}{点々}{てんてん}{tenten}}

The \ldakuten{} is written with two strokes \jquotesingleja{゙} and can be
attached not to all but certain \hyperref[sec:Kana]{kana} letters to mark them.
As a consequence the pronunciation of the kana letter changes in a similar way
as a German umlaut. The dakuten is a  diacritic sign and referenced colloquial
as \ltenten{}. For other dakuten, please see \textit{\nameref{sec:Iteration}}
on page \pageref{sec:Iteration}.

\begin{table}[H]
  \begin{center}
    \begin{tabular}{lllll}
      \textbf{Dakuten:}&\textbf{Kana without}&\textbf{Kana with}&\textbf{Reading without}&\textbf{Reading with}\\
      \jHiragana:      &か                    &が                & \jtl{ka}               & \jtl{ga}            \\
      \jKatakana:      &カ                   &ガ                & \jtl{ka}               & \jtl{ga}            \\
    \end{tabular}
  \end{center}
  \caption{Examples for dakuten}
  \label{tab:ExamplesForDakuten}
\end{table}

    % label sec:Dakuten
% ---------------------------------------------------------------------------
\section{Diphthong}\jsec{二重母音} \label{sec:Diphthong}
\ithree{diphthong}{二重母音}{Diphthong}
\ithree{diphthong}{にじゅうぼいん}{Diphthong}
\ithree{syllable}{音節}{Silbe}
\ija{「アエ」}
\ija{「アイ」}
\ija{「アウ」}
\ija{「アオ」}
\ija{「ウエ」}
\ija{「ウイ」}
\ija{「オエ」}
\ija{「オイ」}
\ija{「オウ」}

A \textbf{diphthong} {二重母音} {【にじゅうぼいん】} is a sound that is
constructed from two different sounds that glide into each other while
pronouncing and form a \hyperref[sec:Syllable]{syllable}. A \textbf{diphthong}
is made out of vocals.  Examples for a \textbf{diphthong} in Japanese are {姪}
|me.i| and {甥} |o.i|.  Also  {「アエ」}, {「アイ」}, {「アウ」},
{「アオ」}、{「ウエ」}, {「ウイ」}, {「オエ」}, {「オイ」} or {「オウ」} are
likely to appear as a \textbf{diphthong} in normal conversation in Japanese.
However, they becomes vowel connections when it is pronounced slowly and it is
treated as two vowels in the consciousness of the Japanese speaker.
  % label sec:Diphthong
% E
% F
% ---------------------------------------------------------------------------
\section{Furigana}\jsec{振り仮名} \label{sec:Furigana}\label{sec:Rubi}
\label{sec:Yomigana}
\ifor{furigana}{振り仮名}{ふりがな}{Furigana}
\ifor{kanji}{漢字}{かんじ}{Kanji}
\ifor{katakana}{片仮名}{かたかな}{Katakana}
\ifor{rōmaji}{ローマ字}{ろーまじ}{Rōmaji}

\newcommand{\lfurigana}{\ivoc{furigana}{振り仮名}{ふりがな}{Furigana}}

The Japanese language has the pronunciation hint build in on demand. It is
called \lfurigana{} and it is an aid for reading \hyperref[sec:Kanji]{kanji}.
Mostly \textbf{furigana} are \hyperref[sec:Kana]{kana}, so basically
\hyperref[sec:Hiragana]{hiragana} or \hyperref[sec:Katakana]{katakana}.
Japanese \textbf{furigana} are written next to the character (mostly
\hyperref[sec:Kanji]{kanji}) which reading can not be expected to be known,
mostly as annotative glosses. At first unknown or difficult
\hyperref[sec:Kanji]{kanji} are candidates for \textbf{furigana} but also in
books for children some if not all \hyperref[sec:Kanji]{kanji} have
\textbf{furigana}. But even in books, for learning English for example,
\textbf{furigana} can be found next to words written in
\hyperref[sec:Romaji]{rōmaji}.

\ifor{hiragana}{平仮名}{ひらがな}{Hiragana}
\ifor{space character}{空白文字}{くうはく・もじ}{Leerzeichen}

When text is written horizontally \textbf{furigana} are written mostly above
the referenced character. In vertically written text \textbf{furigana} are
written on the right site next to the character. \textbf{Good}
\textbf{furigana} tries to place the reading distinguishable to each character
separately. So the first example (kanji+hiragana) is \textbf{not} good. While
the second (kanji+hiragana) is a \textbf{good} usage of \textbf{furigana}. As a
matter of fact \textbf{furigana} is one rare case of using the
\hyperref[sec:SpaceCharacter]{space character}.

\begin{table}[H]
\begin{center}
\begin{tabular}{rl}
 \normalsize over:&\Huge \ruby{東京}{とうきょう} 
 \ruby{東}{とう}\ruby{京}{きょう} 
 \ruby{東}{トー}\ruby{京}{キョー} 
 \ruby{東}{tō}\ruby{京}{kyō} \\
 \normalsize behind:& \Huge 東京(とうきょう)  東京【とうきょう】\\
 \end{tabular}
\end{center}
\caption{Different types of furigana}
\label{tab:DifferentTypesOfFurigana}
\end{table}

\begin{tabular}{ll}
\raisebox{10\height}{
 \framebox[20mm][r]{
 \rotatebox{-90}{
  \begin{minipage}{2.0cm}
\setCJKfamilyfont{cjk-vert}[Script=CJK,RawFeature=vertical]{IPAPMincho}
\renewcommand{\rubysep}{-0.5ex}
  \CJKfamily{cjk-vert}
   \Huge \ruby{東}{とう}\ruby{京}{ きょう}
  \end{minipage}
 }
}
}
&\begin{minipage}{14cm}
Vertically written Tōkyō, as it also can be seen on many signs.\smallskip

\newcommand{\lrubi}{\ivoc{rubi, ruby}{ルビ}{るび}{Rubi}}
\newcommand{\lyomigana}{\ivoc{yomigana}{読み仮名}{よみがな}{Yomigana}}

Other names for \textbf{furigana} are \lrubi{} or \lyomigana{}. The name Ruby
(Japanese {ルビ} \jtl{rubi}) is also known to be a annotation system that can be
used in \LaTeX or HTML. Also in China, Taiwan and Korea \textbf{rubi} are
common.

\end{minipage} \\
\end{tabular}
\medskip

\begin{tabular}{ll}
\begin{minipage}{13cm}

A common example for using \textbf{furigana} for adults would be to rename
(better re-read) single words to give them a specific connotation. In science
fictions some astronaut could use the Japanese word {ふるさと} \jtl{furusato} with
the meaning of "my hometown" to refer to the planet Earth {( = {地球}
{【ちきゅう】})}.

\end{minipage}&
\hspace{2em}\begin{minipage}{3cm}
\Huge \ruby{地球}{ふるさと} 
\end{minipage}\\
\end{tabular}\medskip

\begin{tabular}{ll}
\begin{minipage}{13cm}

Or to make it more fancy and international example (maybe also with the
connotation that Japan has no space in the future): Here {アース} refers to
'earth', but {地球} is better understandable by the Japanese audience, if the
Japanese \hyperref[sec:Kanji]{kanji} and the \textbf{furigana} is seen
together.

\end{minipage}&
\hspace{2em}\begin{minipage}{3cm}
\Huge\ruby{地球}{アース}
\end{minipage}\\
\end{tabular}


   % label sec:Furigana
% G
% +---------------------------------------------------------------------------+
% | content/tab/Gojuonzu.tex                                                  |
% |                                                                           |
% | 50 sound table in hiragana or katakana                                    |
% |                                                                           |
% | Version: 0.1.0                                                            |
% |                                                                           |
% | Changes:                                                                  |
% |                                                                           |
% | 0.1.0 2020-07-10 Christian Külker <c@c8i.org>                             |
% |     - Initial release                                                     |
% |                                                                           |
% +---------------------------------------------------------------------------+
\ifthenelse{\equal{hiragana}{\jtopic}}{%
% あいうえお
% かきくけこ
% さしすせそ
% たちつてと
% なにぬねの
% はひふへほ
% まみむめも
% やゆよ
% らりるれろ
% わを
% ん
\ien{hiragana gojūonzu}
\ien{hiragana}
\ien{gojūon}
\ija{平仮名五十音図}
\ifor{gojūonzu}{五十音図}{ごじゅうおんず}{50 Laute Tafel}
\bigskip
\begin{center}
%\Huge
\Padding
%\begin{tabular}{m{1.0cm}||m{1.0cm}|m{1.0cm}|m{1.0cm}|m{1.0cm}|m{1.0cm}|}
\begin{tabular}{r||c|c|c|c|c|}
          & \textbf{a}& \textbf{i}& \textbf{u}& \textbf{e}& \textbf{o}\\ \hline \hline
\textbf{-}&あ&い&う&え&お\\\hline
\textbf{k}&か&き&く&け&こ\\\hline
\textbf{s}&さ&し&す&せ&そ\\\hline
\textbf{t}&た&ち&つ&て&と\\\hline
\textbf{n}&な&に&ぬ&ね&ノ\\\hline
\textbf{h}&は&ひ&ふ&へ&ほ\\\hline
\textbf{m}&ま&み&む&め&も\\\hline
\textbf{y}&や&  &ゆ&  &よ\\\hline
\textbf{r}&ら&り&る&れ&ろ\\\hline
\textbf{w}&わ&  &  &  &を\\\hline
\textbf{*}&ん&  &  &  &  \\\hline
\end{tabular}
\end{center}
}{}
\ifthenelse{\equal{katakana}{\jtopic}}{%
% アイウエオ
% カキクケコ
% サシスセソ
% タチツテト
% ナニヌネノ
% ハヒフヘホ
% マミムメモ
% ヤユヨ
% ラリルレロ
% ワヲ
% ン
\ien{katakana!gojūonzu}
\ien{katakana}
\ien{gojūon}
\ija{片仮名五十音図}
\ifor{gojūonzu}{五十音図}{ごじゅうおんず}{50 Laute Tafel}
\bigskip
\begin{center}
%\Huge
\Padding
%\begin{tabular}{m{1.0cm}||m{1.0cm}|m{1.0cm}|m{1.0cm}|m{1.0cm}|m{1.0cm}|}
\begin{tabular}{r||c|c|c|c|c|}
             & \textbf{a}& \textbf{i}& \textbf{u}& \textbf{e}& \textbf{o}\\ \hline \hline
\textbf{-}&ア&イ&ウ&エ&オ\\\hline
\textbf{k}&カ&キ&ク&ケ&コ\\\hline
\textbf{s}&サ&シ&ス&セ&ソ\\\hline
\textbf{t}&タ&チ&ツ&テ&ト\\\hline
\textbf{n}&ナ&ニ&ヌ&ネ&ノ\\\hline
\textbf{h}&ハ&ヒ&フ&ヘ&ホ\\\hline
\textbf{m}&マ&ミ&ム&メ&モ\\\hline
\textbf{y}&ヤ&  &ユ&  &ヨ\\\hline
\textbf{r}&ラ&リ&ル&レ&ロ\\\hline
\textbf{w}&ワ&  &  &  &ヲ\\\hline
\textbf{*}&ン&  &  &  &  \\\hline
\end{tabular}
\end{center}
}{}
   % label sec:Gojuonzu
% H
% ---------------------------------------------------------------------------
\section{Handakuten}\jsec{半濁点} \label{sec:Handakuten}
\ifor{Handakuten}{半濁点}{はんだくてん}{Handakuten}
\ifor{Dakuten}{濁点}{だくてん}{Dakuten}
\ifor{circle}{丸}{まる}{Kreis}
\ithree{"゚"}{「゚」}{"゚"}
\ien{|h|} \ide{|h|}
\ien{|p|} \ide{|p|}
\ien{pronunciation shift} \ide{Ausprache Verschiebung}

In Japanese two different {濁点} {【だくてん】} are used. The {濁点}  and  the
{半濁点} {【はんだくてん】} has the marker of a little circle {「゚」} and is
therefore colloquially described as {丸} {【まる】} and indicates when the
pronunciation shifts from |h| to |p|.

 % label sec:Handakuten
% +---------------------------------------------------------------------------+
% | chap/Hentaigana                                                           |
% |                                                                           |
% | This chapter introduces Japanese hentaigana. It summarizes shortly        |
% | history, contemporary usage and examples. It gives a table of             |
% | hentaigana.                                                               |
% |                                                                           |
% | Version: 0.1.0                                                            |
% |                                                                           |
% | Status: Not in use                                                        |
% |                                                                           |
% | I18n: english                                                             |
% |                                                                           |
% | Changes:                                                                  |
% |                                                                           |
% | 0.1.0 2022-08-30  Christian Külker <c@c8i.org>                            |
% |     - Initial release                                                     |
% |                                                                           |
% +---------------------------------------------------------------------------+
%
% TODO:
% [ ] Make all code points display
% [ ] Write history
% [ ] Write contemporary usage
% [ ] Write examples
% [o] Write hentaigana table
%
\chapter{Hentaigana}\jchap{変体仮名}\label{chap:Hentaigana}
\ithree{Hentaigana tables}{変体仮名図}{Hentaigana Tabellen}

\section{History}\jsec{歴史}\label{sec:HentaiganaHistory}
\section{Usage}\jsec{使い方}\label{sec:HentaiganaUsage}
\section{Some Examples}\jsec{少数例}\label{sec:HentaiganaSomeExamples}
\section{Hentaigana Table}\jsec{変体仮名のレファレンス}\label{sec:HentaiganaTable}
\JapaneseFontN %Not all code points are displayed
% +---------------------------------------------------------------------------+
% | table/Hentaigana.tex                                                      |
% |                                                                           |
% | Table of some Hentaigana                                                  |
% |                                                                           |
% | Version: 0.1.0                                                            |
% |                                                                           |
% | Changes:                                                                  |
% |                                                                           |
% | 0.1.0 2020-07-10 Christian Külker <c@c8i.org>                             |
% |     - initial release                                                     |
% |                                                                           |
% +---------------------------------------------------------------------------+
% HINT:
% use gvim, plumar to edit this file (not vi or vim)

\ide{Basistabelle Hentaigana}
\ide{Hentaigana}
\ija{変体仮名の五十音図}

\bigskip
\begin{center}
\Huge

% Hentaigana
\begin{tabular}{m{1.0cm}||m{2.5cm}|m{2.5cm}|m{2.5cm}|m{2.5cm}|m{2.5cm}|}
& \textbf{a}& \textbf{i}& \textbf{u}& \textbf{e}& \textbf{o}\\
& \textbf{あ(安)}& \textbf{い(以)}& \textbf{う(宇)}& \textbf{え(衣)}& \textbf{お(於)}\\ \hline \hline
\textbf{-}&\smallskip 𛀂(安) 𛀅(惡) 𛀃(愛) 𛀄(阿)
          &\smallskip 𛀆(以) 𛀇(伊) 𛀈(意) 𛀉(移)
          &\smallskip 𛀊(宇) 𛀋(宇) 𛀌(憂) 𛀍(有) 𛀎(雲)
          &\smallskip 𛀁(江) 𛀏(盈) 𛀐(縁) 𛀑(衣) 𛀒(衣) 𛀓(要)
          &\smallskip 𛀔(於) 𛀕(於) 𛀖(隱) \\ \hline
\end{tabular}

\begin{tabular}{m{1.0cm}||m{2.5cm}|m{2.5cm}|m{2.5cm}|m{2.5cm}|m{2.5cm}|}
& \textbf{a}& \textbf{i}& \textbf{u}& \textbf{e}& \textbf{o}\\
& \textbf{か(加)}& \textbf{き(幾)}& \textbf{く(久)}& \textbf{け(計)}& \textbf{こ(己)}\\ \hline \hline
\textbf{k}&\smallskip 𛀗(佳) 𛀘(加) 𛀙(可) 𛀚(可) 𛀛(嘉) 𛀢(家) 𛀜(我) 𛀝(歟) 𛀞(賀) 𛀟(閑) 𛀠(香) 𛀡(駕)
          &\smallskip 𛀣(喜) 𛀤(幾) 𛀥(幾) 𛀦(支) 𛀻(期) 𛀧(木) 𛀨(祈) 𛀩(貴) 𛀪(起)
          &\smallskip 𛀫(久) 𛀬(久) 𛀭(九) 𛀮(供) 𛀯(倶) 𛀰(具) 𛀱(求)
          &\smallskip 𛀳(介) 𛀲(介) 𛀢(家) 𛀴(希) 𛀵(氣) 𛀶(計) 𛀷(遣)
          &\smallskip 𛀸(古) 𛂘(子) 𛀹(故) 𛀻(期) 𛀺(許) \\ \hline
\end{tabular}

\begin{tabular}{m{1.0cm}||m{2.5cm}|m{2.5cm}|m{2.5cm}|m{2.5cm}|m{2.5cm}|}
& \textbf{a}& \textbf{i}& \textbf{u}& \textbf{e}& \textbf{o}\\
& \textbf{さ(左)}& \textbf{し(之}& \textbf{す(寸)}& \textbf{せ(世)}& \textbf{そ(曾)}\\ \hline \hline
\textbf{s}&\smallskip 𛀼(乍) 𛀽(佐) 𛀾(佐) 𛀿(左) 𛁀(差) 𛁁(散) 𛁂(斜) 𛁃(沙)
          &\smallskip 𛁄(之) 𛁅(之) 𛁆(事) 𛁇(四) 𛁈(志) 𛁉(新)
          &\smallskip 𛁊(受) 𛁋(壽) 𛁌(數) 𛁍(數) 𛁎(春) 𛁏(春) 𛁐(須) 𛁑(須)
          &\smallskip 𛁒(世) 𛁓(世) 𛁔(世) 𛁕(勢) 𛁖(聲)
          &\smallskip 𛁗(所) 𛁘(所) 𛁙(曾) 𛁚(曾) 𛁛(楚) 𛁜(蘇) 𛁝(處) \\ \hline
\end{tabular}

\begin{tabular}{m{1.0cm}||m{2.5cm}|m{2.5cm}|m{2.5cm}|m{2.5cm}|m{2.5cm}|}
& \textbf{a}& \textbf{i}& \textbf{u}& \textbf{e}& \textbf{o}\\
& \textbf{た(太)}& \textbf{ち(知)}& \textbf{つ(州)}& \textbf{て(天)}& \textbf{と(止)}\\ \hline \hline
\textbf{t}&\smallskip 𛁞(堂) 𛁟(多) 𛁠(多) 𛁡(當)
          &\smallskip 𛁢(千) 𛁣(地) 𛁤(智) 𛁥(知) 𛁦(知) 𛁧(致) 𛁨(遲)
          &\smallskip 𛁩(川) 𛁪(川) 𛁫(津) 𛁬(都) 𛁭(徒)
          &\smallskip 𛁮(亭) 𛁯(低) 𛁰(傳) 𛁱(天) 𛁲(天) 𛁳(天) 𛁴(帝) 𛁵(弖) 𛁶(轉) 𛂎(而)
          &\smallskip 𛁷(土) 𛁸(度) 𛁹(東) 𛁺(登) 𛁻(登) 𛁼(砥) 𛁽(等) 𛁭(徒) \\ \hline
\end{tabular}

\begin{tabular}{m{1.0cm}||m{2.5cm}|m{2.5cm}|m{2.5cm}|m{2.5cm}|m{2.5cm}|}
& \textbf{a}& \textbf{i}& \textbf{u}& \textbf{e}& \textbf{o}\\
& \textbf{な(奈)}& \textbf{に(仁)}& \textbf{ぬ(奴)}& \textbf{ね(祢)}& \textbf{の(乃)}\\ \hline \hline
\textbf{n}&\smallskip 𛁾(南) 𛁿(名) 𛂀(奈) 𛂁(奈) 𛂂(奈) 𛂃(菜) 𛂄(那) 𛂅(那) 𛂆(難)
          &\smallskip 𛂇(丹) 𛂈(二) 𛂉(仁) 𛂊(兒) 𛂋(爾) 𛂌(爾) 𛂍(耳) 𛂎(而)
          &\smallskip 𛂏(努) 𛂐(奴) 𛂑(怒)
          &\smallskip 𛂒(年) 𛂓(年) 𛂔(年) 𛂕(根) 𛂖(熱) 𛂗(禰) 𛂘(子)
          &\smallskip 𛂙(乃) 𛂚(濃) 𛂛(能) 𛂜(能) 𛂝(農) \\ \hline
\end{tabular}

\begin{tabular}{m{1.0cm}||m{2.5cm}|m{2.5cm}|m{2.5cm}|m{2.5cm}|m{2.5cm}|}
& \textbf{a}& \textbf{i}& \textbf{u}& \textbf{e}& \textbf{o}\\
& \textbf{は(波)}& \textbf{ひ(比)}& \textbf{ふ(不)}& \textbf{へ(部)}& \textbf{ほ(保)}\\ \hline \hline
\textbf{h}&\smallskip 𛂞(八) 𛂟(半) 𛂠(婆) 𛂡(波) 𛂢(盤) 𛂣(盤) 𛂤(破) 𛂥(者) 𛂦(者) 𛂧(葉) 𛂨(頗)
          &\smallskip 𛂩(悲) 𛂪(日) 𛂫(比) 𛂬(避) 𛂭(非) 𛂮(飛) 𛂯(飛)
          &\smallskip 𛂰(不) 𛂱(婦) 𛂲(布)
          &\smallskip 𛂳(倍) 𛂴(弊) 𛂵(弊) 𛂶(遍) 𛂷(邊) 𛂸(邊) 𛂹(部)
          &\smallskip 𛂺(保) 𛂻(保) 𛂼(報) 𛂽(奉) 𛂾(寶) 𛂿(本) 𛃀(本) 𛃁(豊) \\ \hline
\end{tabular}

% XeLaTeX or fonts-hanazono bug?
% code points U+1B11D HEINTAIGANA LETTER N-MU-MO-1 𛄝
%             U+1B11E HEINTAIGANA LETTER N-MU-MO-2 𛄞
% are visible with HanMinA (Hanazono Mincho Reguar) with font-manager, pluma, 
% gvim on Debian Buster, but are not visible in the PDF

\begin{tabular}{m{1.0cm}||m{2.5cm}|m{2.5cm}|m{2.5cm}|m{2.5cm}|m{2.5cm}|}
& \textbf{a}& \textbf{i}& \textbf{u}& \textbf{e}& \textbf{o}\\
& \textbf{ま(末)}& \textbf{み(美)}& \textbf{む(武)}& \textbf{め(女)}& \textbf{も(毛)}\\ \hline \hline
\textbf{m}&\smallskip 𛃂(万) 𛃃(末) 𛃄(末) 𛃅(滿) 𛃆(滿) 𛃇(萬) 𛃈(麻) 𛃖(馬)
          &\smallskip 𛃉(三) 𛃊(微) 𛃋(美) 𛃌(美) 𛃍(美) 𛃎(見) 𛃏(身)
          &\smallskip 𛃐(武) 𛃑(無) 𛃒(牟) 𛃓(舞) 𛄝(无) 𛄞(无)
          &\smallskip 𛃔(免) 𛃕(面) 𛃖(馬)
          &\smallskip 𛃗(母) 𛃘(毛) 𛃙(毛) 𛃚(毛) 𛃛(茂) 𛃜(裳) 𛄝(无) 𛄞(无) \\ \hline
\end{tabular}
%

\begin{tabular}{m{1.0cm}||m{2.5cm}|m{2.5cm}|m{2.5cm}|m{2.5cm}|m{2.5cm}|}
& \textbf{a}& \textbf{i}& \textbf{u}& \textbf{e}& \textbf{o}\\
& \textbf{や(也)}& \textbf{𛀆(以)}& \textbf{ゆ(由)}& \textbf{𛀁(江)}& \textbf{よ(与)}\\ \hline \hline
\textbf{y}&\smallskip 𛃝(也) 𛃞(也) 𛃟(屋) 𛃠(耶) 𛃡(耶) 𛃢(夜)
          &\smallskip 𛀆(以)
          &\smallskip 𛃣(游) 𛃤(由) 𛃥(由) 𛃦(遊)
          &\smallskip 𛀁(江)
          &\smallskip 𛃧(代) 𛃨(余) 𛃩(與) 𛃪(與) 𛃫(與) 𛃬(餘) 𛃢(夜) \\ \hline
\end{tabular}

\begin{tabular}{m{1.0cm}||m{2.5cm}|m{2.5cm}|m{2.5cm}|m{2.5cm}|m{2.5cm}|}
& \textbf{a}& \textbf{i}& \textbf{u}& \textbf{e}& \textbf{o}\\
& \textbf{ら(良)}& \textbf{り(利)}& \textbf{る(留)}& \textbf{れ(礼)}& \textbf{ろ(呂)}\\ \hline \hline
\textbf{r}&\smallskip 𛃭(羅) 𛃮(良) 𛃯(良) 𛃰(良) 𛁽(等)
          &\smallskip 𛃱(利) 𛃲(利) 𛃳(李) 𛃴(梨) 𛃵(理) 𛃶(里) 𛃷(離)
          &\smallskip 𛃸(流) 𛃹(留) 𛃺(留) 𛃻(留) 𛃼(累) 𛃽(類)
          &\smallskip 𛃾(禮) 𛃿(禮) 𛄀(連) 𛄁(麗)
          &\smallskip 𛄂(呂) 𛄃(呂) 𛄄(婁) 𛄅(樓) 𛄆(路) 𛄇(露) \\ \hline
\end{tabular}

\begin{tabular}{m{1.0cm}||m{2.5cm}|m{2.5cm}|m{2.5cm}|m{2.5cm}|m{2.5cm}|}
& \textbf{a}& \textbf{i}& \textbf{u}& \textbf{e}& \textbf{o}\\
& \textbf{わ(和)}& \textbf{ゐ(為)}& \textbf{(汙)}& \textbf{ゑ(恵)}& \textbf{を(遠)}\\ \hline \hline
\textbf{w}&\smallskip 𛄈(倭) 𛄉(和) 𛄊(和) 𛄋(王) 𛄌(王)
          &\smallskip 𛄍(井) 𛄎(井) 𛄏(居) 𛄐(爲) 𛄑(遺)
          &\smallskip
          &\smallskip 𛄒(惠) 𛄓(衞) 𛄔(衞) 𛄕(衞)
          &\smallskip 𛄖(乎) 𛄗(乎) 𛄘(尾) 𛄙(緒) 𛄚(越) 𛄛(遠) 𛄜(遠) 𛀅(惡) \\ \hline
\end{tabular}

% Not printable /wu/ 汙  https://kobunworld.blog.fc2.com/blog-entry-5.html

\begin{tabular}{m{1.0cm}||m{2.5cm}|m{2.5cm}|m{2.5cm}|m{2.5cm}|m{2.5cm}|}
& \textbf{a}& \textbf{i}& \textbf{u}& \textbf{e}& \textbf{o}\\
& \textbf{ん(无)}& \textbf{}& \textbf{}& \textbf{}& \textbf{}\\ \hline \hline
\textbf{*}&\smallskip 𛄝(无) 𛄞(无)
          &\smallskip
          &\smallskip
          &\smallskip
          &\smallskip   \\ \hline
\end{tabular}
\end{center}


 % label sec:Hentaigana
% ---------------------------------------------------------------------------
\section{Hepburn System}\jsec{ヘボン式}
%[o] LABEL
\label{sec:Hepburn}
\label{sec:HepburnSystem}
\label{sec:OlderHepburnSystem}
\label{sec:NewerHepburnSystem}
% [o] INDEX
\ifor{Hepburn system}{ヘボン式}{へぼんしき}{Hepburn System}
\ifor{Hepburn system!older}{標準ヘボン式ローマ字}{ひょうじゅん・へぼん・ろまあじ}{Hepburn System!altes}
\ifor{Hepburn system!newer}{修正ヘボン式ローマ字}{しゅうせい・へぼんしき・ろうまじ}{Hepburn System!neueres}
\ithree{James Curtis Hepburn}{James Curtis Hepburn}{James Curtis Hepburn}

\newcommand{\lhepburnsystem}{\ivoc{Hepburn system}{ヘボン式}{へぼんしき}{Hepburn System}}
\newcommand{\loldhepburnsystem}{\ivoc{old Hepburn system}{標準ヘボン式ローマ字}{ひょうじゅん・へぼん・ろまあじ}{altes Hepburn System}}
\newcommand{\lnewhepburnsystem}{\ivoc{new Hepburn system}{修正ヘボン式ローマ字}{しゅうせい・へぼんしき・ろうまじ}{neues Hepburn System}}

\begin{tabular}{lr}
\begin{minipage}{10.5cm}

The \lhepburnsystem{} is one of the two most important transcription systems
for the Japanese written \hyperref[sec:Mora]{morae} based language. The
\textbf{Hepburn system} is the most used system worldwide and in Japan.

The word {ヘボン} /hebon/ is an old writing of the name \textbf{Hepburn}, a US
American physician, translator, educator and a Christian missionary, who used
the transcription system in his first Japanese English Dictionary (3rd ed.) in
1867.

There are mainly two different variants: The \loldhepburnsystem{} variant,
which is used for signs at train stations. And the newer variant the
\lnewhepburnsystem{} which is used as a revised system since 1954 in Kenkyusha
dictionaries. Most western scientists are using this system. This system is
also used in this book.

\Link \href{https://en.wikipedia.org/wiki/James_Curtis_Hepburn}{Hepburn}

\end{minipage}
&
\raisebox{-.47\height}{
\includegraphics[scale=0.5,trim= 00 00 00 00]{../share/ei/James_Curtis_Hepburn.jpg}}
\\
\end{tabular}


         % label sec:Hepburn
% +---------------------------------------------------------------------------+
% | content/english/para/Hiragana.tex                                         |
% |                                                                           |
% | Brief paragraph about hiragana suited for the introduction                |
% |                                                                           |
% | Version: 0.1.0                                                            |
% |                                                                           |
% | Changes:                                                                  |
% |                                                                           |
% | 0.1.0 2022-09-14 Christian Külker <c@c8i.org>                             |
% |     - Initial versioned release (fixes in time, and others)               |
% |                                                                           |
% +---------------------------------------------------------------------------+

\ifor{kanji}{漢字}{かんじ}{Kanji}
\ifor{hiragana}{平仮名}{ひらがな}{Hiragana}
\ifor{katakana}{片仮名}{かたかな}{Katakana}
\ifor{okurigana}{送り仮名}{おくりがな}{Okurigana}

Approximately in the 9th century the \lhiragana{} script was developed by
simplifying Chinese characters used for pronunciation. The number of
contemporary \hyperref[sec:Hiragana]{hiragana} where reduced and today 46 are
in use. It is a \hyperref[sec:Mora]{morae} alphabet which is mostly constructed
out of syllables. In the modern Japanese language \textbf{hiragana} is used for
\hyperref[sec:Okurigana]{okurigana} like verb endings, other endings as well as
for phonetic transcription and for all other words which can or should not be
written with \hyperref[sec:Kanji]{kanji}, except words which are written in
\hyperref[sec:Katakana]{katakana}. A simple rule of thumb: if it is not known
whether the word should be written in kanji or katakana, it should probably be
written in \textbf{hiragana}.
   % label sec:Hiragana
\section{Homophone}\jsec{同音異語}
% [o] LABEL
\label{sec:Homophone}
% [o] INDEX DESTINATION (DEF)
\ifor{homophone}{同音異語}{どうおん・いご}{Homofon}
% [o] INDEX TARGET
\ifor{Kanji}{漢字}{かんじ}{Kanji}

% Used:
% - 同音異語     どうおん・いご   1. homophone
% Others:
% - 同音異義語   どうおんいぎご  1. homophone; homonym 2. Homonym [ling.]
% - 同音異字     どうおんいじ    1. homophony 2 Homophone
% - 同音語       どうおんご      1. homophone
% - 同訓         どうくん        1. kun homophone

\newcommand{\lhomophone}{\ivoc{homophone}{同音異語}{どうおん・いご}{Homofon}}

The linguistic term \lhomophone{} references the fact that some words in a
language are pronounced equal but posses' a different meaning. The spelling of
a \textbf{homophone} may be equal or different.

\begin{center}\begin{tabular}{lllll}
\textbf{Language}&\textbf{word 1}&\textbf{meaning 1}&\textbf{word 2}&\textbf{meaning 2}\\\hline
German (same writing)      &Fliege&the insect  &Fliege &the bow tie \\
German (different writing) &aß    &ate (to eat)&Aas    &carrion     \\
English (same writing)     &does  &to do       &does   &plural of doe\\
English (different writing)&eight &8           &ate    &to eat       \\
\end{tabular}\end{center}

In general the meaning of a \textbf{homophone} can be deducted from the
context.  The is especially true if the spelling is different and if the
\textbf{homophone} occurs while reading. It is more difficult, but generally in
most cases possible, to deduct the meaning also in the spoken language.

Homophones are rare in European languages like English or German. In Japanese
homophones are extraordinarily often. One reason\footnote{Except the one that
people accept it and may even like it do nothing to reduce them.} is the mass
import of Chinese words centuries ago by 'neglecting' the pronunciation. While
some Chinese words can be distinguished by pitch, they become a true
\textbf{homophone} by flattening all pitches to only two.

To give an extreme case, the following 22 \hyperref[sec:Kanji]{kanji} words
(two \hyperref[sec:Kanji]{kanji} each) are all pronounced /kikō/.

\begin{center}
{機構} {紀行} {稀覯} {騎行} {貴校} {奇功} {貴公} {起稿} {奇行} {機巧} {寄港}\\
{帰校} {気功} {寄稿} {機甲} {帰航} {奇効} {季候} {気孔} {起工} {気候} {帰港}
\end{center}

Even though they sound the same, in written language they can be
differentiated.

% TODO: what if the meaning is equal? Are they still homophones?
  % label sec:Homophone
% I
\section{Iroha}\jsec{伊呂波}
% [o] LABEL
\label{sec:Iroha}
% [o] INDEX DESTINATION (DEF)
\ifor{Iroha}{伊呂波}{いろは}{Iroha}
% [o] INDEX TARGET
\ifor{Gojūonzu}{五十音図}{ごじゅうおんず}{50@50 Laute Tafel}
\ifor{Hiragana}{平仮名}{ひらがな}{Hiragana}

\newcommand{\liroha}{\ivoc{iroha}{伊呂波}{いろは}{Iroha}}

The word \liroha{} stands for /iroha uta/ (iroha song) and is a Japanese poem
of the Heian era that contains all \hyperref[sec:Kana]{kana} characters. It was
used to order and memorize \hyperref[sec:Kana]{kana}. In contrast to today it
also contains more or less unused letters, like /we/ or /wi/ and it do not
contain the newer /n/. Usual the poem is written in
\hyperref[sec:Hiragana]{hiragana} from top to down.

\begin{center}
%\raisebox{10\height}{
%\framebox[20mm][r]{
\rotatebox{-90}{
\begin{minipage}{2.0cm}
    \setCJKfamilyfont{cjk-vert}[Script=CJK,RawFeature=vertical]{IPAPMincho}
    \renewcommand{\rubysep}{-0.5ex}
    \CJKfamily{cjk-vert}
いろはにほへとちりぬるをわかよたれそつねならむうゐのおくやまけふこえてあさきゆめみしゑひもせす
\end{minipage}
}
%}
%}
\end{center}

In this book the modern \hyperref[sec:Gojuonzu]{gojūonzu} is used.

      % label sec:Iroha
% ---------------------------------------------------------------------------
\section{Katakana Iteration Marks - ??? } \label{sec:Iteration}

As with {漢字} {【かんじ】} also {片仮名} has a iteration mark.  「ヽ」 and its
{濁点} {【だくてん】} form {「ヾ」}. This can only be
found in rare cases. For example the personal name Misuzu 【みすゞ】might
contain this character. And since the difference between the second last
and the last mora is only a change in pronunciation the {濁点} is added.

In vertical writing exist another iteration marker {くの字点} {【くのじてん】}
which consist out of two characters {「〳」+「〵」} and the {濁点} form
is {「〴」+「〵」}

  % label sec:Iteration
% J
% K
% ---------------------------------------------------------------------------
\section{Kana}\jsec{仮名}
% [o] LABEL
\label{sec:Kana}
% [o] INDEX
\ifor{kana}{仮名}{かな}{Kana}
\ifor{mora}{モーラ}{もーら}{Mora}
\ifor{kanji}{漢字}{かんじ}{Kanji}
\ifor{okurigana}{送り仮名}{おくりがな}{Okurigana}
\ifor{hentaigana}{変体仮名}{へんたいがな}{Hentaigana}
\ifor{hiragana}{平仮名}{ひらがな}{Hiragana}
\ifor{katakana}{片仮名}{かたかな}{Katakana}
\ifor{man'yōgana}{万葉仮名}{まんようがな}{Man'yōgana}

%\newcommand{\lkana}{\ivoc{kana}{仮名}{かな}{Kana}}

The Japanese script category \lkana{} is a subordinate concept of Japanese
\hyperref[sec:Mora]{mōra} scripts ending with \textit{-kana} or \textit{-gana}.
Some Japanese scripts like \hyperref[sec:Kanji]{kanji} or
\hyperref[sec:Romaji]{rōmaji}, that are ending in \textit{-ji}, are excluded
and \textbf{kana} is often used in contrast to \hyperref[sec:Kanji]{kanji} and
\hyperref[sec:Romaji]{rōmaji} as these are not \hyperref[sec:Mora]{mōra} based
scripts. The concept of \textbf{kana} is also used sometimes to contrast
\hyperref[sec:Kanji]{kanji}, because \hyperref[sec:Kanji]{kanji} possess a
meaning while \textbf{kana} exhibit none.

It is easy to see that in contemporary Japanese scripts like
\hyperref[sec:Hiragana]{hiragana} and \hyperref[sec:Katakana]{katakana} are in
fact \textbf{kana} scripts. However some script categories, like
\hyperref[sec:Okurigana]{okuriagana} or \hyperref[sec:Furigana]{furigana}, are
scripts usually written in \hyperref[sec:Hiragana]{hiragana}, sometimes
\hyperref[sec:Katakana]{katakana}, and are therefore not \textbf{kana} in the
sense of a distinct script. \hyperref[sec:Okurigana]{Okuriagana} or
\hyperref[sec:Furigana]{furigana} are \textbf{kana} used for a certain purpose
and point to a specific function. \hyperref[sec:Furigana]{Furigana} are for
example \textbf{kana} that are used to add endings to words.

Other \textbf{kana} like \hyperref[sec:Hentaigana]{hentaigana} are obsolete and
depreciated versions of \hyperref[sec:Hiragana]{hiragana}.

Historically for very \hyperref[sec:Mora]{mōra} there can often be more than
one \hyperref[sec:Hiragana]{hiragana} found that where stylistic variants or
distinct alternatives.

And finally \hyperref[sec:Manyogana]{man'yōgana} are Chinese characters that
have been used as phonetic characters around mid 7th century. The name suggest
that this \textbf{kana} are \textbf{kana} in relation to \textit{man'yō} which
points to an old text, the \textit{Man'yōshū}. This name is somewhat misleading
since this Chinese characters where not only used in the \textit{Man'yōshū} in
this fashion as phonetic characters. The same characters have been used over a
long time and the number of characters weren't been constant and also this
characters had a dual use as phonetic characters and ordinary
\hyperref[sec:Kanji]{kanji}.

       % label sec:Kana
1300 years ago the first endeavours where undertaken to display the Japanese
language with the only known alphabet in the region, the Chinese writing
system. While the Japanese language where hardly suited for the writing system
it was an  economical choice since the Chinese characters where well developed
at that time and introduced many new ideas in lexis. The 'borrowing' of Chinese
characters was not a one shot operation it took centuries and more then one
attempt. This long winded process led to the fact that some characters where
imported more then once from China from different times and different regions.
And because of this one Chinese character can have more then one pronunciation.
We hope that this will consolidate over the next centuries.  Today this
imported characters are known as \textbf{Kanji} in Japan.  \textbf{Kanji} is
written \textit{Hanzi} in Chinese and referencing the character from the Han
period of China. Even though today all Chinese based characters (and even some
self invented) are referenced nowadays as \textbf{Kanji}, it does not strictly
mean that they only from the Han period.

A standard Japanese text do contain \textbf{Kanji}. To master the Japanese
language over a certain level and to be over come the problem of personal
illiteracy in Japan it is highly encouraged to learn at least 600 to 800
characters. To become fully literate member of the Japanese society 2000 to
2300 \textbf{Kanji} should be learned.

Today  \textbf{Kanji} in written Japanese language are used for substantives/
nouns, verbs, adjectives and names.
                  % label sec:Kanji
% +---------------------------------------------------------------------------+
% | content/english/para/Katakana.tex                                         |
% |                                                                           |
% | Brief paragraph about katakana suited for the introduction                |
% |                                                                           |
% | Version: 0.1.0                                                            |
% |                                                                           |
% | Changes:                                                                  |
% |                                                                           |
% | 0.1.0 2022-09-15 Christian Külker <c@c8i.org>                             |
% |     - Initial versioned release (fixes in time, and others)               |
% |                                                                           |
% +---------------------------------------------------------------------------+

\ifor{hiragana}{平仮名}{ひらがな}{Hiragana}
\ifor{katakana}{片仮名}{かたかな}{Katakana}
\ifor{manga}{漫画}{まんが}{manga, Comic}

At roughly the same time as \hyperref[sec:Hiragana]{hiragana}, also
\ivoc{katakana}{片仮名}{かたかな}{Katakana} letters were invented by
simplifying Chinese characters used for pronunciation. However the look and
feel of \textbf{katakana} is more 'square' not so 'rounded' as hiragana.  The
word katakana \jquotesingleja{片仮名} means \jquotedouble{fragmentary kana}.
Katakana characters are derived from \textbf{components}, so called radical's,
of kanji constructed from multiple components.

\textbf{Katakana} is used today for writing words of foreign origin (gairaigo)
and for emphasizing (in commercials or \hyperref[sec:Manga]{manga} for
example), scientific terms, minerals as well as words in the fauna or flora and
onomatopoeia (words that mimic sound). In its emphasizing function it resembled
English text written in \textit{italics}. Some Japanese companies are written
in katakana.

   % label sec:Katakana
\newpage
\section{Kunrei System}\jsec{訓令式ローマ字}
% [o] LABEL
\label{sec:Kunrei}
\label{sec:KunreiSystem}
\label{sec:JapanSystemLatinLetters}
% [o] INDEX
\ifor{kunrei system}{訓令式ローマ字}{くんれいろうまじ}{Kunrei System}
\ifor{Japan system Latin letters}{日本式ローマ字}{にほんしきろうまじ}{Lateinische Buchstaben des Japanischen Systems}
\ifor{katakana}{片仮名}{かたかな}{Katakana}
\ifor{hiragana}{平仮名}{ひらがな}{Hiragana}
\ifor{gojūonzu}{五十音図}{ごじゅうおんず}{50@50 Laute Tafel}


% l, r, c, i or o = left, right, center, inner or outer
\begin{wrapfigure}{r}{0.4\textwidth}
        \raisebox{-.5\height}{
        \includegraphics[scale=0.25,trim= 00 00 00 00]{../share/ei/Aikitsu_Tanakadate_r.jpg}}
        \caption{Aikitsu Tanakadate}
        \label{fig:AikitsuTanakadate}
\end{wrapfigure}

The modern \ivoc{kunrei system}{訓令式ローマ字}{くんれいろうまじ}{Kunrei
System} (訓令 = directive; instructions) is the official government-authorized
writing system of Japan, sometimes called \textit{Monbushō system}. It maps
\hyperref[sec:Kana]{kana} (\hyperref[sec:Hiragana]{hiragana} and
\hyperref[sec:Katakana]{katakana}) to \ivoc{rōmaji}{ローマ字}{ろうまじ}{Rōmaji}
(Latin letters). The system was introduced in 1937 and confirmed in 1994 by the
cabinet and is available as ISO 3602:1989.

The \textbf{kunrei system's} predecessor was introduced in 1885 by \Link
\href{https://en.wikipedia.org/wiki/Tanakadate_Aikitsu}{Dr. Aikitsu Tanakadate}
({田中舘愛橘}) as
\ivoc{nihon-/nipponshikiromaji}{日本式ローマ字}{にほんしきろうまじ}{Nihon-/Nipponshikiromaji}
(Japan style Latin letters) known as \textit{Nihonshiki} and tried a more
systematical approach to map \hyperref[sec:Hiragana]{hiragana} and
\hyperref[sec:Katakana]{katakana} to equal Latin letters
(\hyperref[sec:Romaji]{rōmaji}) compared to the \nameref{sec:Hepburn}.

The difference between the \textbf{kunrei system} and the \textit{Nihonshiki}
is that the \textbf{kunrei system} merges the syllable pairs ぢ/じ /di/zi/,
づ/ず /du/zu/, ぢゃ/じゃ /dya/zya/, ぢゅ/じゅ /dyu/zyu/, ぢょ/じょ /dyo/zyo/,
ゐ/い /wi/i/, ゑ/え /we/e/, くゎ/か /kwa/ka/, and ぐゎ/が /gwa/ga/ as the
pronounciation is now identical in modern Japanese.

However, the modern \hyperref[sec:Gojuonzu]{gojūonzu} in the \textbf{kunrei
system} is as follows:

\Info{訓令式ローマ字 - Kunrei System}{
\begin{center}
\begin{tabular}{|c|c|c|c|c|}\hline
   a & i& u& e& o\\\hline
   ka&ki&ku&ke&ko\\\hline
   sa&si&su&se&so\\\hline
   ta&ti&tu&te&to\\\hline
   na&ni&nu&ne&no\\\hline
   ha&hi&hu&he&ho\\\hline
   ma&mi&mu&me&mo\\\hline
   ya&  &yu&  &yo\\\hline
   ra&ri&ru&re&ro\\\hline
   wa&  &  &  & o\\\hline
     &  &  &  & n\\\hline
\end{tabular}
\end{center}
}{true}

Even tough the system is official, many entities (like the train system) or
even the Japanese state is not using it always. Sometimes they use the Hepburn
system. The Japanese state uses the Hepburn system for passports and road
signs, the train system use the Hapburn system for station names for example.

The \textbf{kunrei system} is not used in this book and therefore not part of
this book. Please see \nameref{sec:Hepburn} (on page \pageref{sec:Hepburn}) for
the system used in this book.
          % label sec:Kunrei
% L
% M
% ---------------------------------------------------------------------------
\section{Manga - マンが} \label{sec:Manga
}

TODO
      % label sec:Manga
% ---------------------------------------------------------------------------
\section{Man'yōgana}\jsec{万葉仮名} \label{sec:Manyogana}

The development of distinct Japanese writing begun 600 AD by writers and
scholars reducing some Chinese characters to its bare phonetic value. The
meaning of this characters where ignored. Around 760 a collection of Japanese
poetry was published, the \Link
\href{http://en.wikipedia.org/wiki/Man%27y%C5%8Dsh%C5%AB}{万葉集
【まんようしゅう】}, in which Chinese characters where uses as phonetic
letters. In regard to {万葉集} {【まんようしゅう】} the characters are named
{万葉仮名} {【まんようがな】}

The origin of the \textbf{Man'yōgana} script in poetry and art lead to some
problems in the understanding for the reader. Since the usage of phonetic
Chinese characters where mixed with regular Chinese characters and the
reasoning about which character to use was more form and shape aesthetic then
pragmatic, the meaning was difficult to grasp.

However the royal household or other scholars did not see a necessity to change
the status quo, because the high aim was to write poetry and other texts in
Chinese and \textbf{Man'yōgana} was considered appropriate only for notes,
diaries and love letters.

\Note{Note}{\footnotesize By the end of the 8th Century 970
\hyperref[sec:Kanji]{{漢字} {【かんじ】}} where used to pronounce the 90
\hyperref[sec:Mora]{morae}. This directly shows that there was no bijective map
between sound and character. For |ka| for example the following
\textbf{Man'yōgana} can be used {「可」}, {「何」}, {「加」}, {「架」},
{「香」}, {「蚊」}, {「迦」}. }

The number of \textbf{Man'yōgana} from which \hyperref[sec:Katakana]{Katakana}
likely derived is smaller.  

\Hint{Man'yōgana used for creation of {片仮名} {【かたかな】}}{
\begin{center}
\begin{tabular}{|c||c|c|c|c|c|}\hline
 & a& i  & u  & e& o\\\hline\hline
-&阿&伊  &宇  &江&於\\\hline
k&加&機幾&久  &介&己\\\hline
s&散&之  &須  &世&曽\\\hline
t&多&千  &州川&天&止\\\hline
n&奈&仁  &奴  &祢&乃\\\hline
h&八&比  &不  &部&保\\\hline
m&末&三  &牟  &女&毛\\\hline
y&也&    &由  &  &與\\\hline
r&良&利  &流  &礼&呂\\\hline
w&和&井  &    &恵&乎\\\hline
*&尓&    &    &  &  \\\hline
\end{tabular}
\end{center}
}

The scientific term \textbf{Man'yōgana} is used by Western and Japanese
scientists. However it is not without critique. The term \textbf{Man'yōgana}
might lead to the illusion that it was a defined set of characters in use for
transcribing Chinese or writing Japanese texts or the second illusion that one
sound is represented by only  one \textbf{Man'yōgana}. Both is not true. First,
all Chinese Characters could in principle be used as \textbf{Man'yōgana} (and
therefore the term is basically useless). Actually the reason to chose one
character was sometimes just because out of aesthetic reasons, the shape or
some additional meaning. And second, normally many different
\textbf{Man'yōgana} (Chinese characters) where used for the same pronunciation
in the same text.  Making it efficient or easy was not the target of the
scholars using this kind of \hyperref[sec:PhoneticCharacter]{phonetic
characters} at that time.


\Link \href{http://en.wikipedia.org/wiki/Manyogana}{Man'yōgana}
\Link \href{http://en.wikipedia.org/wiki/Man%27y%C5%8Dsh%C5%AB}{万葉集}

  % label sec:Manyogana
% ---------------------------------------------------------------------------
\section{Mora - モーラ} \label{sec:Mora}

The concept of \textbf{mora}  (plural morae or moras; often symbolized μ) is
used in the science of linguistics. It describes a joint unit in pronunciation
(phonology) that constructs a syllable. The definition of a \textbf{mora} can
vary.  In Japanese the detection of \textbf{morae} is comparably simple. The
world {「チョコレート」} for example consist out of the following 5
\textbf{morae} {「チョ」},{「コ」},{「レ」},{「ー」} and {「ト」} while it
consist only out of four \hyperref[sec:Syllable]{syllables}
{(\hyperref[sec:Syllable]{音節} 【おんせつ】)} {「チョ」},{「コ」},{「レー」}
and {「ト」}.

       % label sec:Mora
% N
% O
% ---------------------------------------------------------------------------
\section{Okurigana}\jsec{送り仮名}
% [o] LABEL
\label{sec:Okurigana}
\label{sec:Nokurigana}
% [o] INDEX
\ifor{okurigana}{送り仮名}{おくりがな}{Okurigana}
\ifor{katakana}{片仮名}{かたかな}{Katakana}
\ifor{kana}{仮名}{かな}{Kana}
\ifor{hiragana}{平仮名}{ひらがな}{Hiragana}
\ifor{kanji}{漢字}{かんじ}{Kanji}
\ifor{nokurigana}{ノくり仮名}{のくりがな}{Nokurigana}

The term \ivoc{okurigana}{送り仮名}{おくりがな}{Okurigana} in Japanese is
\textit{not} a script by its own as the name \hyperref[sec:Kana]{kana} suggest.
\textbf{Okurigana} are \hyperref[sec:Kana]{kana} but either
\hyperref[sec:Hiragana]{hiragana} or \hyperref[sec:Katakana]{katakana} that are
used to write the ending of words in most cases verbs. More precise
\textbf{okuriagna} are suffixes of \hyperref[sec:Kanji]{kanji}. After 1945 only
\hyperref[sec:Hiragana]{hiragana} are used to write \textbf{okurigana} while
before \hyperref[sec:Katakana]{katakana} was used.

\textbf{Okurigana} are the mandatory compromise using static Chinese letters to
write the Japanese language. Next to make \hyperref[sec:Kanji]{kanji} flexible
the other function is to mark the beginning are ending of words in sentences.

\textbf{Okurigana} have two purposes. (1) conjugate (a) verbs and (b)
adjectives. With very few exceptions\footnote{ {皮肉る} {【ひにくる】},
{牛耳る}  {【ぎゅうじる】} and {退治る} {【たいじる】}.}  \textbf{okurigana}
will only inflect \hyperref[sec:Kanji]{kanji} as \textit{kun'yomi}.  (2) Change
the meaning or reading of a \hyperref[sec:Kanji]{kanji} by different
\textbf{okurigana}.

\textit{Example: Okuriagana change the meaning (tense):}

\medskip
\begin{tabular}{ll}
\hspace{2cm}(1) {見る} {【みる】} & see \\
\hspace{2cm}(2) {見た} {【みた】} & saw \\
\end{tabular}

\medskip
In the above example the \textbf{okurigana} of (1) is {「る」} and the
\textbf{okurigana} of (2) is {「た」}.

\textit{Example: Okuriagana change the reading:}

\medskip
\begin{tabular}{ll}
\hspace{2cm}(1) {下さる} {【くださる】} & to give \\
\hspace{2cm}(2) {下りる} {【おりる】} &  to get off (a train for example)/ to descend \\
\hspace{2cm}(3) {下がる} {【さがる】} &  to dangle (intransitive)\\
\end{tabular}

\medskip
So in many cases the \textbf{okurigana} directly after the
\hyperref[sec:Kanji]{kanji} changes the meaning.

\textit{Example: Okuriagana change the meaning (transitivity) :}

\medskip
\begin{tabular}{ll}
\hspace{2cm}(1) {下がる} {【さがる】} &  to dangle (intransitive)\\
\hspace{2cm}(2){下げる} {【さげる】} &  to let off (transitive)\\
\end{tabular}

\medskip
As in the above case many Japanese verbs come in transitive and intransitive
pairs. The reading of the \hyperref[sec:Kanji]{kanji} is often shared.

\subsection*{Okurigana in the Middle}

\textbf{Okurigana} can also be found in the middle of Japanese words.

\textit{Example:}

\begin{center}\begin{tabular}{ll}
(1) {繰り返し} {【くりかえし】} &  to repeat\\
\end{tabular}\end{center}

\subsection*{Invisible Okuriagna - ノくり仮名}

The term \ivoc{nokurigana}{ノくり仮名}{のくりがな}{Nokurigana} was inspired by
the site \texttt{https://kanjidamage.com} but the writing was changed from
\hyperref[sec:Romaji]{rōmaji} to katakana+okurigana+kanji (The
        \hyperref[sec:Katakana]{katakana} {「ノ」} derives from the English
        'no', and the word as such is a violation of the Japanese
        \textbf{okurigana}\footnote{Because \hyperref[sec:Katakana]{katakana}
                do not have \textbf{okurigana}. But also in case there would be
no violation the /o/ of /okuri/ would be vilify to a honorific prefix and then
to be ripped out by the 'no' in a very non polite way.} which describes a
violation of \textbf{okurigana}) Of course the term  is not official, but quite
funny in this case, that basically one should be very angry with the fact that
there are some Japanese words witch do have \textbf{okurigana} but are not
written (but of course pronounced!). The not so funny part with those words is
that if one knows the reading of the \hyperref[sec:Kanji]{kanji} it is
impossible to look them up in a dictionary. So lets strike back and spread the
word of the \textbf{nokurigana} - {ノくり仮名}.

\begin{center}\begin{tabular}{ll}
(1) {取引} = {取り引き} {【と(り)ひ(き)】} &  Transaction\\
(2) {受付} = {受け付け} {【う(け)つ(け)】} &  Reception\\
\end{tabular}\end{center}


\Link \href{https://kanjidamage.com/tags/43}{https://kanjidamage.com/tags/43}








  % label sec:Okurigana
% P
\section{Phonetic Character - } \label{sec:PhoneticCharacter}

In this document the term \textbf{Phonetic Character} refers genetically to a
Chinese characters reading and the usage of this character just for this
purpose and \textit{not} for its meaning. This common set expression has been
used in avoidance of the term \hyperref[sec:Manyogana]{Manyogana}. See the
section \nameref{sec:Manyogana} on page \pageref{sec:Manyogana} to understand
the critique.

 % label sec:PoneticCharacter
% Q
% R
% ---------------------------------------------------------------------------
\section{部首- Radical} \label{sec:Radical}

A radical {部首} {【ぶしゅ】} is a root particle or character of a Sino-Japanese 
character {漢字} {【かんじ】}. It is the most significant part of a Sino-Japanese
character. The concept was developed in China for Chinese characters and is
today known under the same name {部首} (pinyin: bùshǒu).

There is no general definition what a radical is or how many are existing and it 
can vary a lot. The author of a dictionary has the power to defined what a radical
is and how much there will be in that dictionary.

In more traditional Chinese or Japanese dictionaries a number of 214 or 244
radicals is quite common. However some modern approaches like the
\href{http://www.hadamitzky.de/english/works_books.htm#KD}{\textit{The Kanji
Dictionary} of Marc Spahn and Wolfgang Hadamitzky from 1996} a totally
different number of 79 can be found.

\Note{Note}{\footnotesize Before buying a {漢字} dictionary, make sure that the
radical system used suits your taste. Sometimes it can be observed that
Japanese dictionaries are stricter in the definition of a radical because a
given {漢字} can only be retrieved via exactly \textbf{one} radical. While in
many Chinese dictionaries \textbf{every} radical of a Chinese character can be
used to find it. The Japanese approach is of course good in terms of systematic
and didactic for learners, however it can take significant longer to look up a
character by radical.  }

    % label sec:Radical
% ---------------------------------------------------------------------------
\section{Rōmaji  - ローマ字} \label{sec:Romaji}


% ---------------------------------------------------------------------------
\ifor{Rōmaji}{ローマ字}{ろーまじ}{Rōmaji}

In temporary Japan words written in western letters become more popular and
some parts of the written language is already westernized, like (Indian/
Arabic) numbers written in horizontal text almost per default. This western
Latin letters are called \textbf{Rōmaji} and are written in Japanese as
{ローマ字} {【ろおまじ】}, even though some of them are from different origin
like Indian numbers for example.



The western characters are mainly used for writing numbers in the horizontal
writing. Also for abbreviations capital and small letters are used. Sometimes
they are modified. For example the measurement of distance in the metric entity
"km" occupies to places in western scripts "k" + "m" while it only hold one
place in Japanese {「㎞」} or even one place in
\hyperref[sec:Katakana]{Katakana}  {「㌔」}. While the latter is ambiguous to
us, because colloquial kilogram is referenced as only "kilo".

\Note{One Space Rōmaji}{\begin{center}\small
\begin{tabular}{ll}
\textit{Western Multiple Space Letters}&\textit{One Space Rōmaji}\\
mg&㎎\\
mm&㎜\\
kg&㎏\\
cm&㎝\\
km&㎞\\
qm&㎡\\
qcc&㏄\\
\end{tabular}
\end{center}
}

There are other shapes of Rōmaji for numbers or letters:

%\fontspec{IPAPMincho}
\fontspec{IPAPGothic}

\begin{center}
\begin{tabular}{ll}
Roman       &ⅠⅡⅢⅣⅤⅥⅦⅧⅨⅩⅪⅫ...\\
Blac circle & ❶❷❸❹❺❻❼❽❾❿...\\
Withe circle &①②③④⑤⑥⑦⑧⑨⑩...\\
Withe double circle & ⓵⓶⓷⓸⓹⓺⓻⓼⓽⓾...\\
Letters             &ⓐⓑⓒ...\\
\end{tabular}
\end{center}
\newpage
\fontspec{FreeSans}

In a number of incidents in typography multiple Katakana are condensed
into one space, where normally only one Katakana would exist. In some cases the
direction of writing is even diagonal. This part of exception are not part of
this document and should be viewed under the peculiar aesthetic of Japanese
printing.

\Note{One Space Katakana}{\begin{center}\small
\begin{tabular}{ll}
\textit{Western Meaning}&\textit{One Space Katakana}\\
&㌃\\
calorie&㌍\\
kilo&㌔\\
gram&㌘\\
centi-&㌢\\
cent&㌣\\
\$&㌦\\
t&㌧\\
\%&㌫\\
ha&㌶\\
pages&㌻\\
milli-&㍉\\
mbar (millibar)&㍊\\
m (meter)&㍍\\
l (liter)&㍑\\
&㍗\\
\end{tabular}
\end{center}
}

Citation of foreign books are also done in western letters an can pop up
without warning the middle of the text.
     % label sec:Romaji
% S
\section{Space Character}\jsec{空白文字} 
% [o] LABEL
\label{sec:SpaceCharacter}
% [o] INDEX DESTINATION (DEF)
\ifor{space character}{空白文字}{くうはく ・ もじ}{Leerzeichen}
% [o] INDEX TARGET
\ifor{Katakana}{片仮名}{かたかな}{Katakana}
\ifor{Kanji}{漢字}{かんじ}{Kanji}
\ifor{Hiragana}{平仮名}{ひらがな}{Hiragana}
\ifor{Rōmaji}{ローマ字}{ろーまじ}{Rōmaji}

The \textit{space character} in Western (Latin letter based) languages is used
to separate words. In antique texts a separation of words was \textbf{not}
common and those where difficult to read.  In the 7th century AD the word
separation was introduced. In the beginning of printed books the space wide was
fixed and to archive this the width of the letters where not fixed which
produced an easy to read text body.

The invention of typewriters and computers destroyed this approach of
aesthetically advanced typography. The typewriters had still a fixed (too
large) space width but also fixed letters. While the computer on screen behave
not better as a typewriter in the beginning and in printing, the spaces are
variable and the letters are fixed, the opposite of the elegant book printing
of the 15th century AD.

With the invention of Unicode the \textit{space character} is not longer a
singularity.  The Unicode fonts have now many\footnote{To give an example:
U+2008 Punctuation Space, U+2009 Thin Space, ..., U+FEFF Zero Width No-Break
Space, to just name a few.} \textit{space characters}. 

The Japanese computer fonts do have a \textit{space character}. Traditionally
more then one.  The most important \textit{space character} is the double wide
\textit{space character} which is exactly as wide as a
\hyperref[sec:Kanji]{Kanji} character. And the single wide space character that
is as wide as \hyperref[sec:Romaji]{Rōmaji} or half wide
\hyperref[sec:Hiragana]{Hiragana} or \hyperref[sec:Katakana]{Katakana}. 

However even though there is a \textit{space character} nowadays in Japanese
fonts it is \textbf{not} used to separate words from each other. Because of
this the word border can only be detected by heuristics and changes in scripts,
for example: \hyperref[sec:Katakana]{Katakana} to \hyperref[sec:Kanji]{Kanji},
\hyperref[sec:Hiragana]{Hiragana} to \hyperref[sec:Kanji]{Kanji},
\hyperref[sec:Katakana]{Katakana} to \hyperref[sec:Hiragana]{Hiragana} and so
on. Detecting words is a major task in learning Japanese.

The \textit{space character} in Japan is used to indent text to mark
paragraphs. To separate functional entities in the text like author from
heading. 

As a matter of fact the \textit{space character} in modern Japanese plays a
very unimportant role. 

This was not always so. In old Japanese there where an additional usage of
\textit{space characters} as {闕字} {【けつじ】} to leave space in front of
names of important persons or verbs to honor them.

\smallskip
\textbf{Example:}

\begin{center}\Padding\begin{tabular}{p{4cm}p{3cm}p{4cm}}
 {「  上様」} & {【うえさま】}      & Mister Ue \\
 {「登  城」} &  {【とう じょう】} & registered castle\\
\end{tabular}\end{center}
\smallskip

However this usage was abandoned in the Meiji era. 

\Link \href{http://ja.wikipedia.org/wiki/%E9%97%95%E5%AD%97}{http://ja.wikipedia.org/wiki/闕字}

 % sec:SpaceCharacter
% ---------------------------------------------------------------------------
\section{Syllable}\jsec{音節}
% [o] LABEL
\label{sec:Syllable}
% [o] INDEX
\ifor{syllable}{音節}{おんせつ}{Silbe} % DESTINATION
\ifor{mora}{モーラ}{もーら}{Mora}      % TARGET

A \textbf{syllable} {音節}  {【おんせつ】}  is a phonetic building block for
words. It influences the rhythm of a spoken language. In Western languages a
\textbf{syllable} is made out of one or more letters. In Japanese it is often
one character (of \hyperref[sec:Kana]{Kana}), but not always. For a better
understanding of the Japanese it is important to understand the concept of
\hyperref[sec:Mora]{mora}.

\Link \href{http://en.wikipedia.org/wiki/Syllable}{Syllable}
\Link \href{http://ja.wikipedia.org/wiki/%E9%9F%B3%E7%AF%80}{音節}

   % label sec:Syllable
% T
% U
% V
% W
% X
% Y
% Z

