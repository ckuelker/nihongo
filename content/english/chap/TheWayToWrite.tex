% +---------------------------------------------------------------------------+
% | english/chap/TheWayToWrite.tex                                            |
% |                                                                           |
% | Katakana or hiragana specific information                                 |
% |                                                                           |
% | Version: 0.1.0                                                            |
% |                                                                           |
% | Changes:                                                                  |
% |                                                                           |
% | 0.1.0 2022-09-10 Christian Külker <c@c8i.org>                             |
% |     - Initial version (disable input KanaPronunciation)                   |
% |                                                                           |
% +---------------------------------------------------------------------------+

\ifthenelse{\equal{hiragana}{\jtopic}}{%
\chapter{The Way to Write Hiragana}\jchap{平仮名の書き方}
% [o] LABEL
\label{chap:TheWayToWriteHiragana}
% [o] DESTINATIONS
% [o] INDEX
\ien{The Way to Write Hiragana}
\ien{hiragana!writing}
\ide{Hiragana!schreiben}
}{}%
\ifthenelse{\equal{katakana}{\jtopic}}{%
\chapter{The Way to Write Katakana}\jchap{片仮名の書き方}
% [o] LABEL
\label{chap:TheWayToWriteKatakana}
% [o] DESTINATIONS
% [o] INDEX
\ien{The Way to Write Katakana}
\ien{katakana!writing}
\ide{Katakana!schreiben}
}{}%
% [o] DESTINATIONS
\ifor{gojūonzu}{五十音図}{ごじゅうおんず}{50@50 Laute Tafel}
\ifor{mora}{モーラ}{もーら}{Mora}
\ifor{space character}{空白文字}{くうはく・もじ}{Leerzeichen}
\ifor{kanji}{漢字}{かんじ}{Kanji}

This chapter explains how to read and write \jkanavoc. After introducing the
sound structure, the traditional way to display and order \jtopic is
presented. A short part lists functions of \jtopic, what \jtopic
is used for. The following section introduces very briefly the pronounciation.
As this book focus' on writing the major part of the sections teaches
about how to write \jtopic and inform about special \jtopic as well as general
punctuation. The last part is about seldomly used \jtopic.


\jscript{} derived from \textbf{Chinese} characters, called
\hyperref[sec:Kanji]{kanji}. All \jtopic{} together form a complete phonetic
script with less than 50 letters (or morae). With this script all Japanese
words can be written.

\ifor{gojūonzu}{五十音図}{ごじゅうおんず}{50@50 Laute Tafel}

The collection of \textbf{\jtopic} is usually displayed in the
\hyperref[sec:Gojuonzu]{gojūonzu} (lit. table of fifty sounds), a grid of 10x5
fields in which the characters are displayed. Even though nominally the
\hyperref[sec:Gojuonzu]{gojūonzu} is containing 50 characters the grid is not
completely occupied. Additionally there is also one character added to the end.
So with five columns and one extra letter, the current number of
\textbf{\jtopic} is 46. %
\ifthenelse{\equal{hiragana}{\jtopic}}{
The script has no character for doubling a vowel (which would not be displayed
in the \hyperref[sec:Gojuonzu]{gojūonzu}) anyways. Unlike \textbf{katakaka}
with 47, \textbf{hiragana} has 46 distinct characters, still below 50.
}{}%
\ifthenelse{\equal{katakana}{\jtopic}}{
If we would count also the character for doubling a vowel (which is sadly not
displayed in the \hyperref[sec:Gojuonzu]{gojūonzu}). Unlike \textbf{hiragana}
with 46, \textbf{katakana} has 47 distinct characters, still below 50.
}{}%

% +---------------------------------------------------------------------------+
% | content/tab/Gojuonzu.tex                                                  |
% |                                                                           |
% | 50 sound table in hiragana or katakana                                    |
% |                                                                           |
% | Version: 0.1.0                                                            |
% |                                                                           |
% | Changes:                                                                  |
% |                                                                           |
% | 0.1.0 2020-07-10 Christian Külker <c@c8i.org>                             |
% |     - Initial release                                                     |
% |                                                                           |
% +---------------------------------------------------------------------------+
\ifthenelse{\equal{hiragana}{\jtopic}}{%
% あいうえお
% かきくけこ
% さしすせそ
% たちつてと
% なにぬねの
% はひふへほ
% まみむめも
% やゆよ
% らりるれろ
% わを
% ん
\ien{hiragana gojūonzu}
\ien{hiragana}
\ien{gojūon}
\ija{平仮名五十音図}
\ifor{gojūonzu}{五十音図}{ごじゅうおんず}{50 Laute Tafel}
\bigskip
\begin{center}
%\Huge
\Padding
%\begin{tabular}{m{1.0cm}||m{1.0cm}|m{1.0cm}|m{1.0cm}|m{1.0cm}|m{1.0cm}|}
\begin{tabular}{r||c|c|c|c|c|}
          & \textbf{a}& \textbf{i}& \textbf{u}& \textbf{e}& \textbf{o}\\ \hline \hline
\textbf{-}&あ&い&う&え&お\\\hline
\textbf{k}&か&き&く&け&こ\\\hline
\textbf{s}&さ&し&す&せ&そ\\\hline
\textbf{t}&た&ち&つ&て&と\\\hline
\textbf{n}&な&に&ぬ&ね&ノ\\\hline
\textbf{h}&は&ひ&ふ&へ&ほ\\\hline
\textbf{m}&ま&み&む&め&も\\\hline
\textbf{y}&や&  &ゆ&  &よ\\\hline
\textbf{r}&ら&り&る&れ&ろ\\\hline
\textbf{w}&わ&  &  &  &を\\\hline
\textbf{*}&ん&  &  &  &  \\\hline
\end{tabular}
\end{center}
}{}
\ifthenelse{\equal{katakana}{\jtopic}}{%
% アイウエオ
% カキクケコ
% サシスセソ
% タチツテト
% ナニヌネノ
% ハヒフヘホ
% マミムメモ
% ヤユヨ
% ラリルレロ
% ワヲ
% ン
\ien{katakana!gojūonzu}
\ien{katakana}
\ien{gojūon}
\ija{片仮名五十音図}
\ifor{gojūonzu}{五十音図}{ごじゅうおんず}{50 Laute Tafel}
\bigskip
\begin{center}
%\Huge
\Padding
%\begin{tabular}{m{1.0cm}||m{1.0cm}|m{1.0cm}|m{1.0cm}|m{1.0cm}|m{1.0cm}|}
\begin{tabular}{r||c|c|c|c|c|}
             & \textbf{a}& \textbf{i}& \textbf{u}& \textbf{e}& \textbf{o}\\ \hline \hline
\textbf{-}&ア&イ&ウ&エ&オ\\\hline
\textbf{k}&カ&キ&ク&ケ&コ\\\hline
\textbf{s}&サ&シ&ス&セ&ソ\\\hline
\textbf{t}&タ&チ&ツ&テ&ト\\\hline
\textbf{n}&ナ&ニ&ヌ&ネ&ノ\\\hline
\textbf{h}&ハ&ヒ&フ&ヘ&ホ\\\hline
\textbf{m}&マ&ミ&ム&メ&モ\\\hline
\textbf{y}&ヤ&  &ユ&  &ヨ\\\hline
\textbf{r}&ラ&リ&ル&レ&ロ\\\hline
\textbf{w}&ワ&  &  &  &ヲ\\\hline
\textbf{*}&ン&  &  &  &  \\\hline
\end{tabular}
\end{center}
}{}


This document is structured according to the \hyperref[sec:Gojuonzu]{gojūonzu},
five \textbf{\jtopic} will be introduced in one section to be learned together.

\ifor{space character}{空白文字}{くうはく・もじ}{Leerzeichen}
\ifor{homophone}{同音異語}{どうおん・いご}{Homophon}
\ija{同音語}

Even though \textbf{\jtopic} can be used by its own to express the complete
content of the Japanese language it is seldom used as such. Using all Japanese
scripts \textbf{hiragana}, \textbf{katakana} and \hyperref[sec:Kanji]{kanji}
has the advantage to see word borders easily and render the
\hyperref[sec:SpaceCharacter]{space character} obsolete. So a Japanese
\textbf{\jtopic} sentence with \textbf{\jtopic} only and without a
\hyperref[sec:SpaceCharacter]{space character} is hardly understandable,
because it is impossible to distinguish words. However even if the
\hyperref[sec:SpaceCharacter]{space character} would be introduced as in Roman
languages, Japanese has so many \hyperref[sec:Homophone]{homophones} that it
would be still very difficult to impossible to understand sentences.
Esspeccially \hyperref[sec:Kanji]{kanji} give context and meaning. Therefore
the booundaries of script types (\hyperref[sec:Hiragana]{hiragana},
\hyperref[sec:Katakana]{katakana} and \hyperref[sec:Kanji]{kanji}) are
the most significant indicator for word boundaries.

\ifor{hiragana}{平仮名}{ひらがな}{Hiragana}
\ifor{katakana}{片仮名}{かたかな}{Katakana}
\ifor{okurigana}{送り仮名}{おくりがな}{Okurigana}
\ien{role}\ija{機能}\ija{きのう}\ide{Funktion}
\ide{Rolle}

\phantomsection
\label{sec:role}

In the Japanese written language each script has a role. The role of
\textbf{\jtopic} changed over the time. The last big change was after the end
of World War II and \textbf{\jtopic} got the roles it still has up to date and
they are fixed for the time being.

\bigskip

% +---------------------------------------------------------------------------+
% | content/english/tab/Function.tex                                          |
% |                                                                           |
% | Table of hiragana and katakana functions                                  |
% |                                                                           |
% | Version: 0.1.0                                                            |
% |                                                                           |
% | Changes:                                                                  |
% |                                                                           |
% | 0.1.0 2022-09-10 Christian Külker <c@c8i.org>                             |
% |     - Initial release (moved from JapaneseWritingSystem.tex)              |
% |                                                                           |
% +---------------------------------------------------------------------------+
\ifthenelse{\equal{hiragana}{\jtopic}}{%
\begin{table}[h!]
\centering
\begin{tabular}{rp{15cm}}
 1. & Verb endings (furigana)\\
 2. & Adjective endings (furigana) \\
 3. & Words that have no kanji and are not written in katakana \\
 4. & Childrens books   \\
 5. & Educational material \\
 6. & Okurigana \\
 7. & Particles \\
 8. & \hyperref[sec:Manga]{Manga}\\
 9. & Words that have kanji, but the kanji is difficult to understand\\
10. & Onomatopoetica\\
11. & Transcription of kanji (for example in formulars, passports) \\
12. & Advertising might replace kanji or katakana for estetic reasons \\
13. & Sounds in the middle of a kanji word that are not covered by the kanji\\
14. & Honoric prefixes\\
15. & When one would like to be considered childish\\
16. & Some personal names (especially female) to make them cute \\

\end{tabular}
\caption{List of hiragana roles and functions}
\label{table:HiraganaFunction}
\end{table}
}{}

\ifthenelse{\equal{katakana}{\jtopic}}{%
\begin{table}[h!]
\centering
\begin{tabular}{rp{15cm}}
 1. & writing words of foreign origin\\
 2. & words that need to be emphasized\\
 3. & often indicate on-yomi in dictionaries\\
 4. & names of minerals \\
 5. & geological names \\
 6. & names of fauna (animals)\\
 7. & names of flora (plants)\\
 8. & partly onomatopoeias in \hyperref[sec:Manga]{manga}\\
 9. & sounds, like animal sounds or sounds made by humans\\
10. & telegrams (before 1988)\\
11. & banking system account names\\
12. & In literature (eg. \hyperref[sec:Manga]{manga}) words being spoken in a
(foreign) accent or "robotic" speech\\
13. & sometimes used as Furigana\\
14. & uncommon \hyperref[sec:Kanji]{Kanji}, eg. {皮膚科} {【ひふか】}
"dermatologist" written as {皮フ科}\\
15. & computer output (in 80s, before introduction of multi byte characters)\\
16. &some personal names (especially female) (common in the past: eg.
セツ (setsu))\\

\end{tabular}
\caption{List of katakana roles and functions}
\label{table:KatakanaFunction}
\end{table}
}{}


\medskip

\ifor{manga}{漫画}{まんが}{Manga}

Therefore in commercials, \hyperref[sec:Manga]{manga} and literature describing
foreign concepts \textbf{katakana} has a over proportional usage.

%% +---------------------------------------------------------------------------+
% | KanaPronunciation.tex                                                     |
% |                                                                           |
% | Pronounciation of hiragana and katakana.                                  |
% |                                                                           |
% | Version: 0.1.0                                                            |
% |                                                                           |
% | Changes:                                                                  |
% |                                                                           |
% | 0.1.0 2020-07-10 Christian Külker <c@c8i.org>                             |
% |     - Initial release                                                     |
% |                                                                           |
% +---------------------------------------------------------------------------+
\section{Pronunciation and Intonation}\jsec{発音とイントネーション}
% [o] LABEL
\label{sec:PronunciationAndIntonation}
\label{sec:Pronuciation}
\label{sec:Intonation}
% [o] INDEX
\ifor{pronuciation}{発音}{はつおん}{Aussprache}
\ifor{intonation}{イントネーション}{いんとねーしょん}{Betonung}
\ifor{katakana}{片仮名}{かたかな}{Katakana}
\ifor{hiragana}{平仮名}{ひらがな}{Hiragana}
\ifor{mora}{モーラ}{もーら}{Mora}
\ifor{gojūonzu}{五十音図}{ごじゅうおんず}{50@50 Laute Tafel}

The \textbf{pronunciation} of \hyperref[sec:Hiragana]{hiragana} is the same as
for \hyperref[sec:Katakana]{katakana}. Therefore every
\hyperref[sec:Syllable]{syllable}, more precise every \hyperref[sec:Mora]{mora}
corresponds to a \hyperref[sec:\jscript]{\jtopic} character and is constructed
as 'consonant' + 'vowel' with the exception of |n|. This system of letter for
each \hyperref[sec:Mora]{mora} makes \textbf{pronunciation} absolutely clear
with no ambiguities. However the simplicity of \hyperref[sec:\jscript]{\jtopic}
does not mean that \textbf{pronunciation} in Japanese is simple for English
speakers as it is for Germans. The rigid structure of the fixed
\hyperref[sec:Mora]{mora} sound in Japanese creates the challenge of learning
the proper intonation and duration of Japanese \textbf{pronunciation}.

Almost each Japanese word can be chunked into \hyperref[sec:Mora]{morae} of
high and low pitch witch is a crucial aspect of the spoken language. Compared
to Chinese, Japanese luckily have only two pitches: hi and low. Sometimes this
difference can be even important for the lexis. Homophones can have for example
a difference in pitch which make them distinguishable.  The intonation of high
and low pitches is a crucial aspect of the spoken language. One of the biggest
problems for obtaining a natural sounding \textbf{pronunciation} is the
incorrect intonation. Many European or American learners speak without paying
attention to the correct pitch. That makes the speech sound non-natural for
Japanese. In some language course try to let the learner memorize the natural
pitch of a word or even teach rules for memorization. While there is clearly a
possibility for linguistic rules, they are hard to remember and master.  It is
still possible to learn the correct intonation by resorting to language
learning techniques used by infants or small children: mimicking native
Japanese speakers. Therefore it is highly advised to expose oneself to as many
Japanese spoken language as possible and to mimic it. Radio, podcasts, drama
and television to name a few. However, it is not advised to listen too much
artificial sources like anime or commercials.

\bigskip
\begin{tabular}{rl}
-&every (yes \textbf{every}) \hyperref[sec:Mora]{mora} is \textbf{pronounced}
  with the same length\\
-&there is no short and long \hyperref[sec:Mora]{mora} or letters\\
-&every \hyperref[sec:Mora]{mora} has a pitch: high or low\\
-&every pitch matters\\
-&the pitch can change  sometimes with its context\\
-&the pitch can change with a dialect - however standard Japanese has well
  defined pitches\\
\end{tabular}

\bigskip

The \textbf{pronunciation} of \hyperref[sec:Katakana]{katakana} is exactly the
same as for \hyperref[sec:\jscript]{\jtopic} and most sounds are very close to
the Latin \textbf{pronunciation} but in general are \textbf{pronounced} a
little shorter without any stress. Only the /ra/ sounds, like in /ra/, /ri/,
/ru/, /re/ and /ro/ have no similarity in European languages.


The sound of the Japanese /r/ is  neither a central nor a lateral flap, but may
vary between the two. To an English speaker, its pronunciation varies between a
flapped 'd' (as in American English buddy) and a flapped 'l'.
\href{https://en.wikipedia.org/wiki/Japanese_phonology}{(Wikipedia Japanese
Phonology)}.


The following table displays the \textbf{pronunciation} in the
\hyperref[sec:Gojuonzu]{gojūonzu}.

\ien{rōmaji gojūonzu}
\ien{rōmaji}
\ien{gojūon}
\ija{ローマ字五十音図}
\ija{ローマ字}
\bigskip
\begin{center}
%\LARGE
%\Huge
\Padding
\begin{tabular}{c||c|c|c|c|c|}
&\textbf{a}&\textbf{i}&\textbf{u}&\textbf{e}&\textbf{o}\\\hline\hline
\textbf{-}&a&i&u&e&o\\\hline
\textbf{k}&ka&ki&ku&ke&ko\\\hline
\textbf{s}&sa&shi&su&se&so\\\hline
\textbf{t}&ta&chi&tsu&te&to\\\hline
\textbf{n}&na&ni&nu&ne&no\\\hline
\textbf{h}&ha&hi&fu&he&ho\\\hline
\textbf{m}&ma&mi&mu&me&mo\\\hline
\textbf{y}&ya&&yu&&yo\\\hline
\textbf{r}&ra&ri&ru&re&ro\\\hline
\textbf{w}&wa&&&&o\\\hline
\textbf{{*}}&n&&&&\\\hline
\end{tabular}
\end{center}



\input{\jpsecl/WritingLetters}
%\input{../share/sec/WritingKatakanaSentences}
% +---------------------------------------------------------------------------+
% | SpecialKanaCharacters.tex                                                 |
% |                                                                           |
% | Collect special hiragana and katakana characters                          |
% |                                                                           |
% | Version: 0.1.3                                                            |
% |                                                                           |
% | Changes:                                                                  |
% |                                                                           |
% | 0.1.3 2024-04-02 Christian Külker <c@c8i.org>                             |
% |     - Improve clarity and grammar for katakana                            |
% | 0.1.2 2024-04-02 Christian Külker <c@c8i.org>                             |
% |     - Fix doubling hiragana (oo vs. ou)                                   |
% | 0.1.1 2024-03-29 Christian Külker <c@c8i.org>                             |
% |     - Fix typos                                                           |
% | 0.1.0 2022-08-02 Christian Külker <c@c8i.org>                             |
% |     - Merge katakana and hiragana                                         |
% |                                                                           |
% +---------------------------------------------------------------------------+

\ifthenelse{\equal{hiragana}{\jtopic}}{%
\section{Special Hiragana Characters}\jsec{特別ひらがな}
% LABEL
\label{sec:SpecialHiraganaCharacters}
\label{sec:SpecialKanaCharacters}
% INDEX
\ifor{special hiragana characters}{特別ひらがな}{とくべつひらがな}{Spezielle Hiragana Zeichen}
}{}
\ifthenelse{\equal{katakana}{\jtopic}}{%
\section{Special Katakana Characters}\jsec{特別カタカナ}
% LABEL
\label{sec:SpecialKatakanaCharacters}
\label{sec:SpecialKanaCharacters}
% INDEX
\ifor{special katakana characters}{特別カタカナ}{とくべつかたかな}{Spezielle Katakana Zeichen}
}{}
% INDEX
\ifor{hiragana}{平仮名}{ひらがな}{Hiragana}
\ifor{gojūonzu}{五十音図}{ごじゅうおんず}{50@50 Laute Tafel}

As mentioned previously, both \textbf{kana} syllables are almost identical,
except for their shapes. This is particularly evident in the
\hyperref[sec:Gojuonzu]{gojūonzu (50 sound table)}. This section will highlight
the special characters, some of which differ from the standard \jtopic{} set.

%TODO check if point changes orientation and alignment in case of changing
%writing direction.

\ifthenelse{\equal{hiragana}{\jtopic}}{%
\subsection{Doubling Vowels in Hiragana}\jsubsec{ひらがなでの倍増母音}
}{}
\ifthenelse{\equal{katakana}{\jtopic}}{%
\subsection{Doubling Vowels in Katakana}\jsubsec{カタカナでの倍増母音}
}{}
% [o] LABEL
\label{subsec:DoublingVowelsIn\jscript}
\label{subsec:DoublingVowels}
\label{sec:DoublingVowelsIn\jscript}
\label{sec:DoublingVowels}
% [o] INDEX
\ifor{doubling vowels}{倍増母音}{ばいぞうぼいん}{Vokalverdopplung}
\ifor{repetition mark}{繰り返し記号}{くりかえしきごう}{Wiederholungszeichen}
\ifor{hiragana}{平仮名}{ひらがな}{Hiragana}
\ifor{katakana}{片仮名}{かたかな}{Katakana}

% ひらがなでの倍増母音 【ばいぞうぼいん】
% カタカナでの倍増母音 【ばいぞうぼいん】

\ifthenelse{\equal{hiragana}{\jtopic}}{%

The usual way to double a vowel in \textbf{hiragana} is to write that hiragana
again. In the hiragana \jquotesingleja{お} character can be doubled with either
\jquotesingleja{う} or \jquotesingleja{お} . So \jtl{ō} becomes either
\jquotesingleja{おう} or \jquotesingleja{おお}. There is no rule to it. This
has to be rememberd. However in most cases \jquotesingleja{お} is doubled as
\jquotesingleja{おう}.

}{}
\ifthenelse{\equal{katakana}{\jtopic}}{%

Special \hyperref[sec:Katakana]{katakana} characters also exist. The most
important is \ivoc{chōon}{長音}{ちょうおん}{Chōon}, the plain iteration
character \jquotesingleja{ー}, represented as a stroke. It is one of the few
characters that change orientation according to the writing direction. When
writing katakana horizontally (left to right), the iteration character is
horizontal; when writing vertically (top to bottom), it is vertical. This
character's function is to double the duration of the preceding mora, which is
different from \hyperref[sec:Hiragana]{hiragana}. (For more on doubling using
other katakana characters, refer to section \nameref{sec:Iteration} on page
\pageref{sec:Iteration}.)

\bigskip

\CharacterExplanation{k-iteration-s}{In standard gothic fonts, the
\hyperref[sec:Katakana]{katakana} iteration character appears as a simple
straight line, making it ambiguous in terms of writing direction.}

\bigskip

\CharacterExplanation{k-iteration-sm}{However, when written in different fonts
or with a brush, it is apparent that the character should be written from left
to right in horizontal writing.}

}{}
\bigskip

%\definecolor{orange}{rgb}{1,0.5,0}
%\definecolor{mygreen}{rgb}{.2,1,.2}

\setCJKfamilyfont{cjk-horiz-m}[Script=CJK,RawFeature=horizontal]{IPAMincho}
\setCJKfamilyfont{cjk-horiz-g}[Script=CJK,RawFeature=horizontal]{IPAPGothic}
%\setCJKfamilyfont{cjk-vert}[Script=CJK,RawFeature=vertical]{Kozuka Gothic Pro M}
\setCJKfamilyfont{cjk-vert-m}[Script=CJK,RawFeature=vertical]{IPAMincho}
\setCJKfamilyfont{cjk-vert-g}[Script=CJK,RawFeature=vertical]{IPAPGothic}

\bigskip
\textit{Example:}

\bigskip

\begin{figure}[H]
\begin{center}
\begin{tabular}{p{7cm}p{7cm}}
Katakana:&Hiragana:\\
\CJKfamily{cjk-horiz-h}
\Huge カ\textbf{\color{magenta}ー}ド \jtl{kaado} &
\Huge か\textbf{\color{magenta}あ}ど \jtl{kaado}\\
\CJKfamily{cjk-horiz-g}
\Huge カ\textbf{\color{magenta}ー}ド \jtl{kaado} &
                                                \\
\end{tabular}
\end{center}
\caption{Kana vowel doubling example}
\label{fig:KanaVowlDoublingExample}
\end{figure}

\bigskip

This character is frequently used and makes \hyperref[sec:Katakana]{katakana}
easier to learn than hiragana. Unlike hiragana, it does not have long vowel
ambiguity.

As mentioned earlier, the orientation of the \hyperref[sec:Katakana]{katakana}
iteration mark changes with the direction of writing. The preceding example
demonstrates this with various writing orientations.

\medskip
\textit{Example:}

\medskip

\begin{figure}[H]
\begin{center}
\begin{tabular}{p{3.5cm}p{3.5cm}p{3.5cm}m{3.5cm}}
horizontally&
\mbox{
\begin{minipage}{3.2cm}
\CJKfamily{cjk-horiz-h}
\Huge カ\textbf{\color{magenta}ー}ド
\CJKfamily{cjk-horiz-g}
\Huge カ\textbf{\color{magenta}ー}ド
\end{minipage}
}
& vertically &
\raisebox{-.5\height}{
\mbox{
\rotatebox{-90}{
\begin{minipage}{3.2cm}
\CJKfamily{cjk-vert-m}
\Huge カ\textbf{\color{magenta}ー}ド
\CJKfamily{cjk-vert-g}
\Huge カ\textbf{\color{magenta}ー}ド
\end{minipage}
}
}
}
\\
\end{tabular}
\end{center}
\caption{Katakana horizontal and vertical vowel doubling example}
\label{fig:KatakanaHirzontalVerticalVowlDoublingExample}
\end{figure}

\medskip

% めったに使われない片仮名 【めったにつかわれないかたかな】
\ifthenelse{\equal{hiragana}{\jtopic}}{%
    \subsection{Seldom Used Hiragana}\jsubsec{めったに使われない平仮名}\label{subsec:SeldomlyUsedHiragana}

\ifor{voice!new writing}{声}{こえ}{Stimme!neue Schreibweise}
\ifor{voice!old writing}{声}{こゑ}{Stimme!alte Schreibweise}

All \textbf{hiragana} mentioned in the \hyperref[sec:Gojuonzu]{gojūonzu (50
sound table)} are used. There is no obsolete character in the table unlike the
\textbf{katakana} \jtl{wo} \jquotesingleja{ヲ}. However sometimes one can find
a \textbf{gojūonzu} with the additional characters: \jquotesingleja{ゐ} wi,
pronounced as \jtl{i}, and \jquotesingleja{ゑ} we, pronounced \jtl{e}, in the
\jtl{wa} row. This characters are old characters and normally not used. It is
safe to skip learning this characters. Words that used to have
\jquotesingleja{ゐ} had it replaced with \jquotesingleja{い} \jtl{i} and words
that used to have \jquotesingleja{ゑ} had it replaced with \jquotesingleja{え}
\jtl{e}. Examples: old \jquotesingleja{ゐる} is now written
\jquotesingleja{いる} \jtl{iru} ({居る} - to be somewhere), old
\jquotesingleja{こゑ} is now written as \jquotesingleja{こえ} \jtl{koe} ({声} -
voice).

}{}
% めったに使われない平仮名 【めったにつかわれないひらがな】
\ifthenelse{\equal{katakana}{\jtopic}}{%
    \subsection{Seldom Used Katakana}\jsubsec{めったに使われない片仮名}\label{subsec:SeldomlyUsedKatakana}

Because all particles are written in \textbf{hiragana} the particle
\textit{wo}, pronounced as \jtl{o}, is written with the hiragana
\jquotesingleja{を}. The \textbf{katakana} counterpart \textit{wo},
\jquotesingleja{ヲ}, also pronounced \jtl{o}, is included in the
\hyperref[sec:Gojuonzu]{gojūonzu}. However it is not used in modern texts and
it is not a particle in modern texts. It was used in old texts , such as
telegrams, as particle. Consequently, learners can typically omit the katakana
\jquotesingleja{ヲ} from their initial studies.

\label{sec:Wi}
\label{sec:We}
\ifor{we}{ヱ}{エ}{we}
\ifor{wi}{ヰ}{イ}{wi}

In some cases the \textbf{gojūonzu} contain archaic chrarcters:
\jquotesingleja{ヰ} wi, pronounced as \jtl{i}, and \jquotesingleja{ヱ} we,
pronounced \jtl{e}, in the \jtl{wa} row. These characters are old characters
and not used in contemporary Japanese. Therefore, it is generally unnecessary
for learners to focus on these characters. Historically, words containing
\jquotesingleja{ヰ} have had it replaced with \jquotesingleja{イ} (\jtl{i}),
and those with \jquotesingleja{ヱ} have transitioned to \jquotesingleja{エ}
(\jtl{e}).

}{}


