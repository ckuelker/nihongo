% +---------------------------------------------------------------------------+
% | frontispiece                                                              |
% +---------------------------------------------------------------------------+
\begingroup
\vspace*{100pt}

\begin{flushleft}
\fontsize{12pt}{12pt}\selectfont\sffamily
\textbf{Other Resources}
\vspace*{18pt}

\fontsize{12pt}{12pt}\selectfont\sffamily
\begin{tabular}{ l l }
\textbf{\sffamily Latest PDF:} &\texttt{\href{https://github.com/ckuelker/nihongo/tree/master/pub/}{https://github.com/ckuelker/nihongo/tree/master/pub}}\\
\textbf{\sffamily Source code:} &\texttt{\href{https://github.com/ckuelker/nihongo/}{https://github.com/ckuelker/nihongo}}\\
\textbf{\sffamily Web page:} &\texttt{\href{https://christian.kuelker.info/nihongo/}{https://christian.kuelker.info/nihongo}}\\
\end{tabular}

\vspace*{24pt}
\fontsize{12pt}{12pt}\selectfont\sffamily
\textbf{A Short History}\\
\vspace*{18pt}

\newcommand{\jfontsizenine}{\fontsize{9pt}{9pt}\selectfont\sffamily}

{

\fontsize{9pt}{9pt}\selectfont\sffamily

The initial versions \texttt{v0.1-v0.8} of this book have been written between
2000 and ⁠2006 by \textbf{Christian Külker} with the Japanese title
\textbf{日本語を書こう!} (German: \textit{\jquotedouble{Lasst uns Japanisch
schreiben!}}) and have been released in printed form under the
\textit{\jquotedouble{GNU-FDL version 1.2 or any later version published by the
Free Software Foundation with the invariant \textit{\jquotesingle{Back Cover
Text}} section}}.  The book was developed as reference and training material
for the language course at the VHS Halle (Ravensberg) in Germany starting in
the year 2000.\medskip

In 2013 Christian Külker translated the content to English and changed the
title to \textbf{日本語の書き方:片仮名} (English: \textbf{The Japanese Script:
Katakana}). The content was modified towards a self study approach and was
released as \texttt{v0.9} to the public under the same license in printed and
PDF form via \url{http://christian.kuelker.info} as well as the source code via
\url{https://github.com/ckuelker/nihongo}.\medskip

The content was updated in 2020 for the versions \texttt{v1.0-v1.2} and the
build system changed to be used under Debian 10 Buster. Some graphics and fonts
changed. \medskip

In the beginning of 2022 the build system was migrated to Debian 11 Bullseye.
The content was updated, partly rewritten and merged with parts from the German
release of the Hiragana book \texttt{v1.2} from 2014
\textbf{日本語の書き方:ひらがな} (German: \textit{\jquotedouble{Die japanische
Schrift}}) translated to English.  The release \texttt{v1.2} in form of PDF and
source code replaced the \textit{\jquotedouble{Back-Cover Text}} section  with
the section \textit{\jquotedouble{A Short History}}.  The URL changed to
\url{https://christian.kuelker.info/nihongo/}.

}
\medskip

\begin{center}
\footnotesize
\begin{tabular}{lll}
\textbf{Version}&\textbf{Year}&\textbf{Title}\\
v1.2&2022&Nihongo 2 - Japanese Script 日本語の書き方 Katakana 片仮名\\
v1.1&2020&Nihongo 2 - Japanese Script 日本語の書き方 Katakana 片仮名\\
v1.0&2020&Nihongo 2 - Japanese Script 日本語の書き方 Katakana 片仮名\\
v0.9&2013&Nihongo 2 - Japanese Script 日本語の書き方 Katakana 片仮名\\
v0.1-v0.8&2000-2006&日本語を書こう! Lasst uns Japanisch schreiben!\\
\end{tabular}
\end{center}

\fontsize{12pt}{12pt}\selectfont\sffamily
\textbf{Changes}

{
  \jfontsizenine

  \begin{description}

    \jfontsizenine
    \setlength{\itemsep}{0pt}%
    \setlength{\parskip}{0pt}%

    \item[\jfontsizenine\texttt{v1.2}]\jfontsizenine

      2022: New pages are frontispiece, halftitle, title, colophon. Updated
      cover page. Content added from the hiragana book \texttt{v1.2}. The
      \textit{\jquotedouble{Back-Cover Text}} replaced with
      \textit{\jquotedouble{A Short History}}. A pronunciation chapter added.
      The technical term list improved and layout changed. More information
      added about punctuation, pitch, and others. Various paragraphs rewritten
      or improved. A list of figures compiled and a list of tables  inculded
      after the table of contents. The build system now supports Debian 11
      Bullseye and partly older versions. The state machine of the build system
      has now more variables and commands driven from a book configuration.

    \item[\jfontsizenine\texttt{v1.1}]\jfontsizenine

      2020: Fixed typos, space, grammar and small layout changes made. The
      invariant section clause of the GNU-FDL dropped. The license changed to
      \textit{\jquotedouble{GNU-FDL version 1.2 or any later version published
                      by the Free Software Foundation with \textbf{no}
      invariant section}}.

    \item[\jfontsizenine\texttt{v1.0}]\jfontsizenine

      2020: The source code changed to compile under Debian 10 Buster. Fonts
      changed in the appendix.

    \item[\jfontsizenine\texttt{v0.9}]\jfontsizenine

      2013: Initial publicly released version as Katakana only book. The title
      changed to \textbf{日本語の書き方:片仮名} (English: \textit{The Japanese
      Script - Katakana}) and modified to a self study approach.

    \item[\jfontsizenine\texttt{v0.1–v1.8}]\jfontsizenine

      2000–2006: The book developed as reference and training material with the
      title \textbf{日本語を書こう!} (German: \textit{Lasst uns Japanisch
      schreiben!}). Changes from feedback made frequently.  Published at least
      2003, 2004 and 2006 under the \textit{\jquotedouble{GNU-FDL version 1.2
                      or any later version published by the Free Software
      Foundation with the invariant \textit{\jquotesingle{Back Cover Text}}
      section}}.

  \end{description}

}

\end{flushleft}
\endgroup
\clearpage
\newpage

