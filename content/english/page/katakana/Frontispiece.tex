% +---------------------------------------------------------------------------+
% | frontispiece                                                              |
% +---------------------------------------------------------------------------+
\begingroup
\vspace*{100pt}

\begin{flushleft}
\fontsize{12pt}{12pt}\selectfont\sffamily
\textbf{Other Ressources}
\vspace*{18pt}

\fontsize{12pt}{12pt}\selectfont\sffamily
\begin{tabular}{ l l }
\textbf{\sffamily Latest PDF:} &\texttt{\href{https://github.com/ckuelker/nihongo/tree/master/pub/}{https://github.com/ckuelker/nihongo/tree/master/pub}}\\
\textbf{\sffamily Source code:} &\texttt{\href{https://github.com/ckuelker/nihongo/}{https://github.com/ckuelker/nihongo}}\\
\textbf{\sffamily Web page:} &\texttt{\href{https://christian.kuelker.info/nihongo/}{https://christian.kuelker.info/nihongo}}\\
\end{tabular}

\vspace*{24pt}
\fontsize{12pt}{12pt}\selectfont\sffamily
\textbf{A Short History}\\
\vspace*{18pt}

\newcommand{\jfontsizenine}{\fontsize{9pt}{9pt}\selectfont\sffamily}

{
\fontsize{9pt}{9pt}\selectfont\sffamily
The initial versions \texttt{v0.1-v0.8} of this book have been written between
2000 and ⁠2006 by \textbf{Christian Külker} with the title
\textbf{日本語を書こう!} (German: \textit{Lasst uns Japanisch schreiben!}) and
have been released under the “GNU-FDL version 1.2 or any later version
published by the Free Software Foundation with the invariant \textit{Back Cover
Text} section”.  It was developed as reference and training book for the
language course at the VHS Halle (Ravensberg) in Germany starting in the year
2000.\medskip

In 2013 Christian Külker changed the title to \textbf{日本語の書き方:片仮名}
in Japanese and \textbf{The Japanese Script: Katakana} in English and
translated also the content to English as well. He modified the content towards
a self study approach that was released as \texttt{v0.9} to the public under
the same license as PDF under \url{http://christian.kuelker.info} as well as in
source under \url{https://github.com/ckuelker/nihongo}\medskip

The versions \texttt{v1.0-v1.2} updated the content and changed the build
system as well as some graphics to compile under Debian 10 Buster. Some fonts
have been changed.\medskip

Beginning in 2022 the build system was migrated to Debian 11 Bullseye and
merged with additional parts of the Hiragana book v1.2 from 2014
\textbf{日本語の書き方:ひらがな} (German: Die japanische Schrift). The result
translated to English was released in 2022 as \texttt{v1.2}. The release
dropped the invariant section and replaced it with this section \textit{A Short
History}. Consequently the license changed to “GNU-FDL version 1.2 or any later
version published by the Free Software Foundation with \textbf{no} invariant
section”. The URL changed to \url{https://christian.kuelker.info/nihongo/}

}
\medskip

\begin{center}
\footnotesize
\begin{tabular}{lll}
\textbf{Version}&\textbf{Year}&\textbf{Title}\\
v1.2&2022&Nihongo 2 - Japanese Script 日本語の書き方 Katakana 片仮名\\
v1.1&2020&Nihongo 2 - Japanese Script 日本語の書き方 Katakana 片仮名\\
v1.0&2020&Nihongo 2 - Japanese Script 日本語の書き方 Katakana 片仮名\\
v0.9&2013&Nihongo 2 - Japanese Script 日本語の書き方 Katakana 片仮名\\
v0.1-v0.8&2000-2006&日本語を書こう! Lasst uns Japanisch schreiben!\\
\end{tabular}
\end{center}

\fontsize{12pt}{12pt}\selectfont\sffamily
\textbf{Changes}

{\jfontsizenine
\begin{description}\jfontsizenine
        \setlength{\itemsep}{0pt}%
        \setlength{\parskip}{0pt}%

        \item[\jfontsizenine\texttt{v1.2}]\jfontsizenine 2022: Add frontispiece, halftitle, title
                and update title page. Adding content from the hiragana book.

        \item[\jfontsizenine\texttt{v1.1}]\jfontsizenine 2020: The invariant section clause of the
                GNU-FDL was dropped.  Typos, space, grammar and small layout
                changes have been made.

        \item[\jfontsizenine\texttt{v1.0}]\jfontsizenine 2020: The source code was changed to
                compile under Debian 10 Buster. Some fonts have been changed in
                the appendix.

        \item[\jfontsizenine\texttt{v0.9}]\jfontsizenine 2013:  Initial
                publicly released version as Katakana only book.  The title was
                changed to \textbf{日本語の書き方:片仮名} (English:
                \textit{The Japanese Script - Katakana}) and adopted to a self
                study approach.

        \item[\jfontsizenine\texttt{v0.1–v1.8}] 2000–2006: Published internally
                as \textbf{日本語を書こう!} (German: \textit{Lasst uns
                Japanisch schreiben!}). It was developed as reference and
                training book for the language course at the VHS Halle
                (Ravensberg) in Germany starting year 2000. It was published
                2003, 2004 and 2006 under the GNU FDL.

\end{description}
}
\end{flushleft}
\endgroup
\clearpage
\newpage

