% +---------------------------------------------------------------------------+
% | content/english/para/Hiragana.tex                                         |
% |                                                                           |
% | Brief paragraph about hiragana suited for the introduction                |
% |                                                                           |
% | Version: 0.1.0                                                            |
% |                                                                           |
% | Changes:                                                                  |
% |                                                                           |
% | 0.1.0 2022-09-14 Christian Külker <c@c8i.org>                             |
% |     - Initial versioned release (fixes in time, and others)               |
% |                                                                           |
% +---------------------------------------------------------------------------+

\ifor{kanji}{漢字}{かんじ}{Kanji}
\ifor{hiragana}{平仮名}{ひらがな}{Hiragana}
\ifor{katakana}{片仮名}{かたかな}{Katakana}
\ifor{okurigana}{送り仮名}{おくりがな}{Okurigana}

Approximately in the 9th century the \lhiragana{} script was developed by
simplifying Chinese characters used for pronunciation. The number of
contemporary \hyperref[sec:Hiragana]{hiragana} where reduced and today 46 are
in use. It is a \hyperref[sec:Mora]{morae} alphabet which is mostly constructed
out of syllables. In the modern Japanese language \textbf{hiragana} is used for
\hyperref[sec:Okurigana]{okurigana} like verb endings, other endings as well as
for phonetic transcription and for all other words which can or should not be
written with \hyperref[sec:Kanji]{kanji}, except words which are written in
\hyperref[sec:Katakana]{katakana}. A simple rule of thumb: if it is not known
whether the word should be written in kanji or katakana, it should probably be
written in \textbf{hiragana}.
