% +---------------------------------------------------------------------------+
% | content/english/para/Kanji.tex                                            |
% |                                                                           |
% | Brief paragraph about kanji suited for the introduction                   |
% |                                                                           |
% | Version: 0.1.1                                                            |
% |                                                                           |
% | Changes:                                                                  |
% |                                                                           |
% | 0.1.1 2024-03-11 Christian Külker <c@c8i.org>                             |
% |     - Explanations: language structure, economical choice                 |
% |     - Change time from 8th to 5th century                                 |
% |     - Grammatical fixes                                                   |
% | 0.1.0 2022-09-16 Christian Külker <c@c8i.org>                             |
% |     - Initial versioned release (fixes in time, and others)               |
% |                                                                           |
% +---------------------------------------------------------------------------+

\ifor{kanji}{漢字}{かんじ}{Kanji}

In approximately the 5th century CE, during the introduction of Buddhism,
roughly 1600 years ago, the first endeavors were undertaken to display the
Japanese language with the only known alphabet in the region, the Chinese
writing system. While the Japanese language was hardly suited for the writing
system (as the  Chinese language was structural different, had no verb endings
or other word changes for example) it was an economical choice (in contrast to
develop a proprietary writing system) since the Chinese characters were well
developed at that time and introduced many new ideas in lexis. The
\jquotesingle{borrowing} of Chinese characters was not a one shot operation it
took centuries and several attempts. This long winded process led to the fact
that some characters were imported more than once from China from different
times, different regions, different dialects and Chinese by itself changed over
the centuries. And because of this one Chinese character can have more than
one pronunciation in the Japanese language. We only can hope that this will
consolidate over the next centuries.

Today this imported characters, but also some self created characters are known
as \ivoc{kanji}{漢字}{かんじ}{Kanji} in Japan. The word \textbf{kanji} is
pronounced \textit{Hanzi} in Chinese and is referencing the character
\jquotesingleja{字} from the Han \jquotesingleja{漢} period of China. Even
though today all Chinese based characters (and even some self invented) are
referenced nowadays as \textbf{kanji}, it does not strictly mean that they are
only from the Han period.

A standard Japanese text does usually contain \textbf{kanji} among other
scripts. To master the Japanese language beyond a certain level and to
overcome the problem of personal illiteracy in Japan it is highly encouraged to
learn at least 600 to 800 \textbf{kanji} characters. To become a fully literate
member of the Japanese society 2000 to 2300 \textbf{kanji} should be learned as
a rough estimate.

Today \textbf{kanji} in the written Japanese language are used for
substantives/ nouns, verbs (stems), adjectives, adjective stems and names.

