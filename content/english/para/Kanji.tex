
\ifor{kanji}{漢字}{かんじ}{Kanji}

1300 years ago the first endeavours where undertaken to display the Japanese
language with the only known alphabet in the region, the Chinese writing
system. While the Japanese language were hardly suited for the writing system
it was an economical choice since the Chinese characters where well developed
at that time and introduced many new ideas in lexis. The 'borrowing' of Chinese
characters was not a one shot operation it took centuries and more than one
attempt. This long winded process led to the fact that some characters where
imported more than once from China from different times and different regions.
And because of this one Chinese character can have more than one pronunciation
in the Japanese language.  We only can hope that this will consolidate over the
next centuries.

Today this imported characters are known as \ivoc{kanji}{漢字}{かんじ}{Kanji}
in Japan.  The word \textbf{kanji} is pronounced \textit{Hanzi} in Chinese and
is referencing the character ({字}) from the Han ({漢}) period of China. Even
though today all Chinese based characters (and even some self invented) are
referenced nowadays as \textbf{kanji}, it does not strictly mean that they are
only from the Han period.

A standard Japanese text do usually contain \textbf{kanji} among other scripts.
To master the Japanese language above a certain level and to overcome the
problem of personal illiteracy (in Japan) it is highly encouraged to learn at
least 600 to 800 \textbf{kanji} characters. To become a fully literate member
of the Japanese society 2000 to 2300 \textbf{kanji} should be learned as a
rough estimate.

Today \textbf{kanji} in the written Japanese language are used for
substantives/ nouns, verbs (stems), adjectives, adjective stems and names.
