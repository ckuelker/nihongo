% +---------------------------------------------------------------------------+
% | content/english/para/Katakana.tex                                         |
% |                                                                           |
% | Brief paragraph about katakana suited for the introduction                |
% |                                                                           |
% | Version: 0.1.0                                                            |
% |                                                                           |
% | Changes:                                                                  |
% |                                                                           |
% | 0.1.0 2022-09-15 Christian Külker <c@c8i.org>                             |
% |     - Initial versioned release (fixes in time, and others)               |
% |                                                                           |
% +---------------------------------------------------------------------------+

\ifor{hiragana}{平仮名}{ひらがな}{Hiragana}
\ifor{katakana}{片仮名}{かたかな}{Katakana}
\ifor{manga}{漫画}{まんが}{manga, Comic}

At roughly the same time as \hyperref[sec:Hiragana]{hiragana}, also
\ivoc{katakana}{片仮名}{かたかな}{Katakana} letters were invented by
simplifying Chinese characters used for pronunciation. However the look and
feel of \textbf{katakana} is more 'square' not so 'rounded' as hiragana.  The
word katakana \jquotesingleja{片仮名} means \jquotedouble{fragmentary kana}.
Katakana characters are derived from \textbf{components}, so called radical's,
of kanji constructed from multiple components.

\textbf{Katakana} is used today for writing words of foreign origin (gairaigo)
and for emphasizing (in commercials or \hyperref[sec:Manga]{manga} for
example), scientific terms, minerals as well as words in the fauna or flora and
onomatopoeia (words that mimic sound). In its emphasizing function it resembled
English text written in \textit{italics}. Some Japanese companies are written
in katakana.

