% ---------------------------------------------------------------------------
\section{Dakuten}\jsec{濁点}
%
\label{sec:Dakuten}
\label{sec:Tenten}
% INDEX
%\ithree{diacritic sign}{濁点}{Diakritisches Zeichen}% Nigorierungszeichen
%\ithree{diacritic sign}{だくてん}{Diakritisches Zeichen}%Nigorierungszeichen
\ithree{diacritic sign, colloquial}{点々}{Diakritisches Zeichen, umgangsspr.}% Nigorierungszeichen
\ithree{diacritic sign, colloquial}{てんてん}{Diakritisches Zeichen, umgangsspr.}% Nigorierungszeichen
\ithree{umlaut}{ウムラウト}{Umlaut}
\ithree{syllable}{音節}{Silbe}
\ithree{kana}{仮名}{Kana}

\ifor{diacritic sign}{濁点}{だくてん}{diakritisches Zeichen}% Nigorierungszeichen

\newcommand{\ldakuten}{\ivoc{dakuten}{濁点}{だくてん}{diakritisches Zeichen}}
\newcommand{\lkana}{\ivoc{kana}{仮名}{かな}{Kana}}
\newcommand{\ltenten}{\ivoc{tenten}{点々}{てんてん}{tenten}}

The \ldakuten{} is written with two strokes \jquotesingleja{゙} and can be
attached not to all but certain \hyperref[sec:Kana]{kana} letters to mark them.
As a consequence the pronunciation of the kana letter changes in a similar way
as a German umlaut. The dakuten is a  diacritic sign and referenced colloquial
as \ltenten{}. For other dakuten, please see \textit{\nameref{sec:Iteration}}
on page \pageref{sec:Iteration}.

\begin{table}[H]
  \begin{center}
    \begin{tabular}{lllll}
      \textbf{Dakuten:}&\textbf{Kana without}&\textbf{Kana with}&\textbf{Reading without}&\textbf{Reading with}\\
      \jHiragana:      &か                    &が                & \jtl{ka}               & \jtl{ga}            \\
      \jKatakana:      &カ                   &ガ                & \jtl{ka}               & \jtl{ga}            \\
    \end{tabular}
  \end{center}
  \caption{Examples for dakuten}
  \label{tab:ExamplesForDakuten}
\end{table}

