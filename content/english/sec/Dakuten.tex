% ---------------------------------------------------------------------------
\section{Dakuten}\jsec{濁点} \label{sec:Dakuten}
%\ithree{diacritic sign}{濁点}{Diakritisches Zeichen}% Nigorierungszeichen
%\ithree{diacritic sign}{だくてん}{Diakritisches Zeichen}%Nigorierungszeichen
\ithree{diacritic sign, colloquial}{点々}{Diakritisches Zeichen, umgangsspr.}% Nigorierungszeichen
\ithree{diacritic sign, colloquial}{てんてん}{Diakritisches Zeichen, umgangsspr.}% Nigorierungszeichen
\ithree{Umlaut}{ウムラウト}{Umlaut}
\ithree{syllable}{音節}{Silbe}
\ithree{kana}{仮名}{Kana}

\ifor{diacritic sign}{濁点}{だくてん}{diakritisches Zeichen}% Nigorierungszeichen

\newcommand{\ldakuten}{\ivoc{dakuten}{濁点}{だくてん}{diakritisches Zeichen}}
\newcommand{\lkana}{\ivoc{kana}{仮名}{かな}{Kana}}
\newcommand{\ltenten}{\ivoc{tenten}{点々}{てんてん}{tenten}}

The \ldakuten{} is a diacritic sign.  It is similar to the German umlaut.  The
\textbf{dakuten} is referenced colloquial as \ltenten{}. It is used in \lkana{}
\hyperref[sec:Syllable]{syllabaries} to mark a consonant to be pronounced
voiced. Two strokes {「゙」} are used near the \hyperref[sec:Kana]{kana} letter
to mark it. For other \textbf{dakuten}, please see \nameref{sec:Iteration} on
page \pageref{sec:Iteration}.
