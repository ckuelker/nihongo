% ---------------------------------------------------------------------------
\section{Diphthong}\jsec{二重母音} \label{sec:Diphthong}
\ithree{diphthong}{二重母音}{Diphthong}
\ithree{diphthong}{にじゅうぼいん}{Diphthong}
\ithree{syllable}{音節}{Silbe}
\ija{\jquotesingleja{アエ}}
\ija{\jquotesingleja{アイ}}
\ija{\jquotesingleja{アウ}}
\ija{\jquotesingleja{アオ}}
\ija{\jquotesingleja{ウエ}}
\ija{\jquotesingleja{ウイ}}
\ija{\jquotesingleja{オエ}}
\ija{\jquotesingleja{オイ}}
\ija{\jquotesingleja{オウ}}
\ifor{diphthong}{二重母音}{にじゅうぼいん}{Diphthong}
\ifor{diphthong}{二重母音}{にじゅうぼいん}{Doppellaut}
\ifor{diphthong}{二重母音}{にじゅうぼいん}{Zweilaut}

A \ivoc{diphthong}{二重母音}{にじゅうぼいん}{Diphthong} is a sound that is
constructed from two different sounds that glide into each other while
pronouncing and form a \hyperref[sec:Syllable]{syllable}. A \textbf{diphthong}
is made out of vocals.  Examples for a \textbf{diphthong} in Japanese are {姪}
\jtl{me.i} and {甥} \jtl{o.i}.  Also  \jquotesingleja{アエ},
\jquotesingleja{アイ}, \jquotesingleja{アウ},
\jquotesingleja{アオ}、\jquotesingleja{ウエ}, \jquotesingleja{ウイ},
\jquotesingleja{オエ}, \jquotesingleja{オイ} or \jquotesingleja{オウ} are
likely to appear as a \textbf{diphthong} in normal conversation in Japanese.
However, they becomes vowel connections when it is pronounced slowly and it is
treated as two vowels in the consciousness of the Japanese speaker.
