% ---------------------------------------------------------------------------
\section{Furigana}\jsec{振り仮名} \label{sec:Furigana}\label{sec:Rubi}
\label{sec:Yomigana}
\ifor{furigana}{振り仮名}{ふりがな}{Furigana}
\ifor{kanji}{漢字}{かんじ}{Kanji}
\ifor{katakana}{片仮名}{かたかな}{Katakana}
\ifor{rōmaji}{ローマ字}{ろーまじ}{Rōmaji}

\newcommand{\lfurigana}{\ivoc{furigana}{振り仮名}{ふりがな}{Furigana}}

The Japanese language has the pronunciation hint build in on demand. It is
called \lfurigana{} and it is an aid for reading \hyperref[sec:Kanji]{kanji}.
Mostly \textbf{furigana} are \hyperref[sec:Kana]{kana}, so basically
\hyperref[sec:Hiragana]{hiragana} or \hyperref[sec:Katakana]{katakana}.
Japanese \textbf{furigana} are written next to the character (mostly
\hyperref[sec:Kanji]{kanji}) which reading can not be expected to be known,
mostly as annotative glosses. At first unknown or difficult
\hyperref[sec:Kanji]{kanji} are candidates for \textbf{furigana} but also in
books for children some if not all \hyperref[sec:Kanji]{kanji} have
\textbf{furigana}. But even in books, for learning English for example,
\textbf{furigana} can be found next to words written in
\hyperref[sec:Romaji]{rōmaji}.

\ifor{hiragana}{平仮名}{ひらがな}{Hiragana}
\ifor{space character}{空白文字}{くうはく・もじ}{Leerzeichen}

When text is written horizontally \textbf{furigana} are written mostly above
the referenced character. In vertically written text \textbf{furigana} are
written on the right site next to the character. \textbf{Good}
\textbf{furigana} tries to place the reading distinguishable to each character
separately. So the first example (kanji+hiragana) is \textbf{not} good. While
the second (kanji+hiragana) is a \textbf{good} usage of \textbf{furigana}. As a
matter of fact \textbf{furigana} is one rare case of using the
\hyperref[sec:SpaceCharacter]{space character}.

\begin{table}[H]
\begin{center}
\begin{tabular}{rl}
 \normalsize over:&\Huge \ruby{東京}{とうきょう} 
 \ruby{東}{とう}\ruby{京}{きょう} 
 \ruby{東}{トー}\ruby{京}{キョー} 
 \ruby{東}{tō}\ruby{京}{kyō} \\
 \normalsize behind:& \Huge 東京(とうきょう)  東京【とうきょう】\\
 \end{tabular}
\end{center}
\caption{Different types of furigana}
\label{tab:DifferentTypesOfFurigana}
\end{table}

\begin{tabular}{ll}
\raisebox{10\height}{
 \framebox[20mm][r]{
 \rotatebox{-90}{
  \begin{minipage}{2.0cm}
\setCJKfamilyfont{cjk-vert}[Script=CJK,RawFeature=vertical]{IPAPMincho}
\renewcommand{\rubysep}{-0.5ex}
  \CJKfamily{cjk-vert}
   \Huge \ruby{東}{とう}\ruby{京}{ きょう}
  \end{minipage}
 }
}
}
&\begin{minipage}{14cm}
Vertically written Tōkyō, as it also can be seen on many signs.\smallskip

\newcommand{\lrubi}{\ivoc{rubi, ruby}{ルビ}{るび}{Rubi}}
\newcommand{\lyomigana}{\ivoc{yomigana}{読み仮名}{よみがな}{Yomigana}}

Other names for \textbf{furigana} are \lrubi{} or \lyomigana{}. The name Ruby
(Japanese {ルビ} \jtl{rubi}) is also known to be a annotation system that can
be used in \LaTeX{} or HTML. Also in China, Taiwan and Korea \textbf{rubi} are
common.

\end{minipage} \\
\end{tabular}
\medskip

\begin{tabular}{ll}
\begin{minipage}{13cm}

A common example for using \textbf{furigana} for adults would be to rename
(better re-read) single words to give them a specific connotation. In science
fictions some astronaut could use the Japanese word {ふるさと} \jtl{furusato} with
the meaning of "my hometown" to refer to the planet Earth {( = {地球}
{【ちきゅう】})}.

\end{minipage}&
\hspace{2em}\begin{minipage}{3cm}
\Huge \ruby{地球}{ふるさと} 
\end{minipage}\\
\end{tabular}\medskip

\begin{tabular}{ll}
\begin{minipage}{13cm}

Or to make it more fancy and international example (maybe also with the
connotation that Japan has no space in the future): Here {アース} refers to
'earth', but {地球} is better understandable by the Japanese audience, if the
Japanese \hyperref[sec:Kanji]{kanji} and the \textbf{furigana} is seen
together.

\end{minipage}&
\hspace{2em}\begin{minipage}{3cm}
\Huge\ruby{地球}{アース}
\end{minipage}\\
\end{tabular}


