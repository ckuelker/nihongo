\section{Gojūonzu}\jsec{五十音図}
% [o] LABEL
\label{sec:Gojuonzu}
% [o] INDEX DESTINATION (DEF)
\ifor{gojūonzu}{五十音図}{ごじゅうおんず}{50@50 Laute Tafel}
% [o] INDEX TARGET
\ifor{kana}{仮名}{かな}{Kana}
\ifor{Iroha}{伊呂波}{いろは}{Iroha}

\newcommand{\lgojuonzu}{\ivoc{gojūonzu}{五十音図}{ごじゅうおんず}{Gojūonzu}}

Traditionally two ways exist to order Japanese \hyperref[sec:Kana]{kana}
characters. One of it is the \lgojuonzu{} (50 sound table), which is used more
often in modern times while the \hyperref[sec:Iroha]{iroha}\footnote{A poem
        with all \hyperref[sec:Kana]{kana} letters to remember easily. However
it is not standard Japanese anymore why it would be difficult to suggest to
learn.} was more popular in the older times.

The \textbf{gojuonzu} is a grid of 10 x 5 squares partly filled with
\hyperref[sec:Kana]{kana}, hiragana or katakana. The roman letters are not part
of the \textbf{gojuonzu} and are added for the convenience of the learner.

% +---------------------------------------------------------------------------+
% | content/tab/Gojuonzu.tex                                                  |
% |                                                                           |
% | 50 sound table in hiragana or katakana                                    |
% |                                                                           |
% | Version: 0.1.0                                                            |
% |                                                                           |
% | Changes:                                                                  |
% |                                                                           |
% | 0.1.0 2020-07-10 Christian Külker <c@c8i.org>                             |
% |     - Initial release                                                     |
% |                                                                           |
% +---------------------------------------------------------------------------+
\ifthenelse{\equal{hiragana}{\jtopic}}{%
% あいうえお
% かきくけこ
% さしすせそ
% たちつてと
% なにぬねの
% はひふへほ
% まみむめも
% やゆよ
% らりるれろ
% わを
% ん
\ien{hiragana gojūonzu}
\ien{hiragana}
\ien{gojūon}
\ija{平仮名五十音図}
\ifor{gojūonzu}{五十音図}{ごじゅうおんず}{50 Laute Tafel}
\bigskip
\begin{center}
%\Huge
\Padding
%\begin{tabular}{m{1.0cm}||m{1.0cm}|m{1.0cm}|m{1.0cm}|m{1.0cm}|m{1.0cm}|}
\begin{tabular}{r||c|c|c|c|c|}
          & \textbf{a}& \textbf{i}& \textbf{u}& \textbf{e}& \textbf{o}\\ \hline \hline
\textbf{-}&あ&い&う&え&お\\\hline
\textbf{k}&か&き&く&け&こ\\\hline
\textbf{s}&さ&し&す&せ&そ\\\hline
\textbf{t}&た&ち&つ&て&と\\\hline
\textbf{n}&な&に&ぬ&ね&ノ\\\hline
\textbf{h}&は&ひ&ふ&へ&ほ\\\hline
\textbf{m}&ま&み&む&め&も\\\hline
\textbf{y}&や&  &ゆ&  &よ\\\hline
\textbf{r}&ら&り&る&れ&ろ\\\hline
\textbf{w}&わ&  &  &  &を\\\hline
\textbf{*}&ん&  &  &  &  \\\hline
\end{tabular}
\end{center}
}{}
\ifthenelse{\equal{katakana}{\jtopic}}{%
% アイウエオ
% カキクケコ
% サシスセソ
% タチツテト
% ナニヌネノ
% ハヒフヘホ
% マミムメモ
% ヤユヨ
% ラリルレロ
% ワヲ
% ン
\ien{katakana!gojūonzu}
\ien{katakana}
\ien{gojūon}
\ija{片仮名五十音図}
\ifor{gojūonzu}{五十音図}{ごじゅうおんず}{50 Laute Tafel}
\bigskip
\begin{center}
%\Huge
\Padding
%\begin{tabular}{m{1.0cm}||m{1.0cm}|m{1.0cm}|m{1.0cm}|m{1.0cm}|m{1.0cm}|}
\begin{tabular}{r||c|c|c|c|c|}
             & \textbf{a}& \textbf{i}& \textbf{u}& \textbf{e}& \textbf{o}\\ \hline \hline
\textbf{-}&ア&イ&ウ&エ&オ\\\hline
\textbf{k}&カ&キ&ク&ケ&コ\\\hline
\textbf{s}&サ&シ&ス&セ&ソ\\\hline
\textbf{t}&タ&チ&ツ&テ&ト\\\hline
\textbf{n}&ナ&ニ&ヌ&ネ&ノ\\\hline
\textbf{h}&ハ&ヒ&フ&ヘ&ホ\\\hline
\textbf{m}&マ&ミ&ム&メ&モ\\\hline
\textbf{y}&ヤ&  &ユ&  &ヨ\\\hline
\textbf{r}&ラ&リ&ル&レ&ロ\\\hline
\textbf{w}&ワ&  &  &  &ヲ\\\hline
\textbf{*}&ン&  &  &  &  \\\hline
\end{tabular}
\end{center}
}{}


The later adopted \jtl{n} was added as one square or in the above example as the
11th line. Even though there are less than 50 letters and more than 50 squares
out of historical reason the name used is still \textbf{gojuonzu}.

For more explanations please read the chapter
\nameref{chap:TheWayToWriteKatakana} and look at the various examples of the
\textbf{gojuonzu} in the appendix starting with \nameref{chap:KatakanaTables}
on page \pageref{chap:KatakanaTables} up to page
\pageref{sec:KatakanaMikachanPB}.

