% ---------------------------------------------------------------------------
\section{Handakuten}\jsec{半濁点} \label{sec:Handakuten} \label{sec:Maru}

\ifor{handakuten}{半濁点}{はんだくてん}{Handakuten}
\ifor{dakuten}{濁点}{だくてん}{Dakuten}
\ien{circle}
\ija{丸}
\ija{まる}
\ide{Kreis}
\ithree{.@゚}{\jquotesingleja{゚}}{.@゚}% removed quotes (single quotes OK, double quotes NG)
\ien{\jtl{h}} \ide{\jtl{h}}
\ien{\jtl{p}} \ide{\jtl{p}}
\ien{pronunciation shift} \ide{Ausprache Verschiebung}
\newcommand{\lhandakuten}{\ivoc{handakuten}{半濁点}{はんだくてん}{Handakuten}}
\newcommand{\lmaru}{\ivoc{maru}{丸}{まる}{Maru}}

In Japanese two different markers for \hyperref[sec:Kana]{kana} are used: The
\ldakuten{} (see \hyperref[sec:Dakuten]{dakuten}) and the \lhandakuten{}. The
latter has the marker of a little circle \jquotesingleja{゚} and is therefore
colloquially described as \lmaru{} and indicates when the pronunciation shifts
from \jtl{h} to \jtl{p}.

