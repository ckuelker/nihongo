% +---------------------------------------------------------------------------+
% | Hentaigana.tex                                                            |
% |                                                                           |
% | Describe the usage of Hentaigana in the Japanese language.                |
% |                                                                           |
% | Version: 0.1.2                                                            |
% |                                                                           |
% | Changes:                                                                  |
% |                                                                           |
% | 0.1.2 2024-03-29 Christian Külker <c@c8i.org>                             |
% |     - Fix typos, improve formatting                                       |
% | 0.1.1 2020-07-10 Christian Külker <c@c8i.org>                             |
% |     - The previous version claims that Hentaigana have not been included  |
% |       in Unicode. While this was true at the time of writing this changed |
% |       with Unicode 10 in 2017. This version reflects the new development. |
% | 0.1.0 2014-09-14 Christian Külker <c@c8i.org>                             |
% |     - Initial release                                                     |
% |                                                                           |
% +---------------------------------------------------------------------------+
% WARNING:
%   - This file contains Unicode characters that are NOT displayed on all
%     computer systems
%   - Also known Unicode editors, like vim, are likely to show misalignment
%     where actually there is none, use gvim, pluma or the like instead.
%   - Some Roman characters are not displayed in lines having a Hentaigana
%   - Some Roman characters are displayed double even if there is only one
%     in lines having a Hentaigana


\section{Hentaigana}\jsec{同音異語}
% [o] LABEL
\label{sec:Hentaigana}
\label{sec:FamilyRegister}
% [o] INDEX DESTINATION (DEF)
\ifor{hentaigana}{変体仮名}{へんたいがな}{Hentaigana}
% [o] INDEX TARGET
\ifor{kana}{仮名}{かな}{Kana}

\newcommand{\lhentaigana}{\ivoc{hentaigana}{変体仮名}{へんたいがな}{Hentaigana}}
\newcommand{\lkoseki}{\ivoc{family register}{戸籍}{こせき}{Familien Register}}

The word \lhentaigana{} is pronounced \jtl{hen・tai・gana} and refers historically
to \hyperref[sec:Kana]{kana} that are used seldom today. They were used until
before 1900 and declared as obsolete\footnote{The word \jtl{hentai} means just
variant.} in the 1900 language reform. Rather than an addition to
\hyperref[sec:Kana]{kana}, hentaigana representing alternative forms
to existing \hyperref[sec:Kana]{kana}. The usage before 1900 were not formalized and every
writer decided which set to use. It was even common to use two or more
different hentaigana (and standard \hyperref[sec:Kana]{kana}) with the
same pronunciation in the same document by the same author.

Until 1947 hentaigana were used for names. In contemporary Japan the usage of
 hentaigana is reduced to traditional decorative elements on shop signs for 
kisoba example. A few marginal uses remain such as: the word "chopsticks" 
\jhatsuon{おてもと} \jtl{otemoto} usally written in hiragana {おてもと} or as
{お手元} can be written in hentaigana on some chopsticks like {御手茂登}, where as
{茂} and {登} are hentaigana. An other example are the names in the Japanese 
family registry \lkoseki{}.

% otemoto: http://trendzatugaku.com/life/post-7344/
% otemoto: http://komonjo.rokumeibunko.com/nyumon/otemoto.html

% Kisoba: 生ki so ha (ha+dakuten) 
% ki = 生
% so = 𛁛 https://de.wikipedia.org/wiki/Datei:TRON_9-8356.gif   (derived from 楚)
% ha = 𛂦 https://de.wikipedia.org/wiki/Datei:Hiragana_HA_01.svg (derived from 者)
% ihttps://en.wiktionary.org/wiki/%F0%9B%82%A6
% This is not even visible in pluma: It can also be added dakuten or handakuten, becoming 𛂦゙ or 𛂦゚, to represent ba or pa respectively. 
% https://japanese.stackexchange.com/questions/29868/is-there-any-way-to-input-hentaigana-into-text-documents
% https://ja.wikipedia.org/wiki/%E3%81%AF_(%E8%80%85%E3%81%AE%E5%A4%89%E4%BD%93%E4%BB%AE%E5%90%8D)
% https://www.sljfaq.org/afaq/hentaigana.html

\begin{figure}[H]
\begin{center}
\Huge
生𛁛𛂦゙ \jhatsuon{きそば} \jtl{kisoba} % Attention: invisible chars in this line
\end{center}
\caption{Kisoba written in hentaigana}
\label{fig:KisobaWrittenInHentaigana}
\end{figure}

\bigskip

\textbf{Examples of \textit{hentaigana}:}

\begin{table}[H]
\begin{center}
\JapaneseFontN
\begin{tabular}{lcccl}
\textbf{UCS}&\textbf{Hentaigana}&\textbf{Pronunciation}&\textbf{Derived From}&\textbf{Note} \\\hline
1B001       & 𛀁                  & \jtl{ye}              & 江                  & Simple \\
1B002       & 𛀂                  & \jtl{a}               & 安                  & Similar to あ\\
...         & ...              & ...                   & ...                 & \\
1B009       & 𛀉                  & \jtl{i}               & 移                  & Complex \\
...         & ...              & ...                   & ...                 &  \\
1B01A       & 𛀚                  & \jtl{ka}              & 可                  & Unexpected pronunciation \\
...         & ...              & ...                   & ...                 & \\
\end{tabular}
\JapaneseDefault
\end{center}
\caption{Examples of hentaigana}
\label{tab:ExamplesOfHentaigana}

\end{table}

Due to Japanese proposals from 2015
\footnote{See
        \href{https://www.unicode.org/L2/L2015/15316-hentaigana-58_438.pdf}{『変体仮名のこれまでとこれから—情報交換のための標準化』}
(The past, present, and future of \textbf{hentaigana}: Standardization for
information processing) by TAKADA Tomokazu (高田智和) et al. and
\href{https://www.unicode.org/L2/L2015/15318-hentaigana.pdf}{About the
inclusion of standardized code points for Hentaigana by YADA Tsutomu (矢田勉)}
}, in 2017 \textbf{hentaigana} became available in Unicode (version 10).
However the usage on computers in 2020, 2021 and 2022 is still difficult. Until
Japanese computer text input methods (like \texttt{Mozc}, \texttt{Anthy}, ...)
support \textbf{hentaigana}, entering this characters on a computer is quite
cumbersome. In \texttt{vim} for example: enter insert mode, press
\texttt{<CTRL+V>+U} and then the hexadecimal UCS number. Of course to actually
see the character a correct font has to be installed. For instance the
font\footnote{See
\href{https://en.wikipedia.org/wiki/Help:Multilingual_support\#Hentaigana}{Wikipedia Help}
for more fonts or the Wikipedia page on
\href{https://en.wikipedia.org/wiki/Hentaigana}{Hentaigana} } \texttt{HanaMinA
Regular} (Hanazono Mincho) can be used to display \textbf{hentaigana}. Be aware that
different fonts with different degree of support are used in one desktop
session. The font used to display this PDF can be different from the font to
display the operating system desktop windows which again can be different from
the font used in the terminal. No font is known to support all characters.

