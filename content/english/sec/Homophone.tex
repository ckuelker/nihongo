\section{Homophone}\jsec{同音異語}
% [o] LABEL
\label{sec:Homophone}
% [o] INDEX DESTINATION (DEF)
\ifor{homophone}{同音異語}{どうおん・いご}{Homofon}
% [o] INDEX TARGET
\ifor{kanji}{漢字}{かんじ}{Kanji}

% Used:
% - 同音異語     どうおん・いご   1. homophone
% Others:
% - 同音異義語   どうおんいぎご  1. homophone; homonym 2. Homonym [ling.]
% - 同音異字     どうおんいじ    1. homophony 2 Homophone
% - 同音語       どうおんご      1. homophone
% - 同訓         どうくん        1. kun homophone

\newcommand{\lhomophone}{\ivoc{homophone}{同音異語}{どうおん・いご}{Homofon}}

The linguistic term \lhomophone{} references the fact that some words in a
language are pronounced equal but posses' a different meaning. The spelling of
a \textbf{homophone} may be equal or different.

\begin{center}\begin{tabular}{lllll}
\textbf{Language}&\textbf{word 1}&\textbf{meaning 1}&\textbf{word 2}&\textbf{meaning 2}\\\hline
German (same writing)      &Fliege&the insect  &Fliege &the bow tie \\
German (different writing) &aß    &ate (to eat)&Aas    &carrion     \\
English (same writing)     &does  &to do       &does   &plural of doe\\
English (different writing)&eight &8           &ate    &to eat       \\
\end{tabular}\end{center}

In general the meaning of a \textbf{homophone} can be deducted from the
context.  The is especially true if the spelling is different and if the
\textbf{homophone} occurs while reading. It is more difficult, but generally in
most cases possible, to deduct the meaning also in the spoken language.

Homophones are rare in European languages like English or German. In Japanese
homophones are extraordinarily often. One reason\footnote{Except the one that
people accept it and may even like it do nothing to reduce them.} is the mass
import of Chinese words centuries ago by 'neglecting' the pronunciation. While
some Chinese words can be distinguished by pitch, they become a true
\textbf{homophone} by flattening all pitches to only two.

To give an extreme case, the following 22 \hyperref[sec:Kanji]{kanji} words
(two \hyperref[sec:Kanji]{kanji} each) are all pronounced \jtl{kikō}.

\begin{center}
{機構} {紀行} {稀覯} {騎行} {貴校} {奇功} {貴公} {起稿} {奇行} {機巧} {寄港}\\
{帰校} {気功} {寄稿} {機甲} {帰航} {奇効} {季候} {気孔} {起工} {気候} {帰港}
\end{center}

Even though they sound the same, in written language they can be
differentiated.

% TODO: what if the meaning is equal? Are they still homophones?
