\section{Iroha}\jsec{伊呂波}
% [o] LABEL
\label{sec:Iroha}
% [o] INDEX DESTINATION (DEF)
\ifor{Iroha}{伊呂波}{いろは}{Iroha}
% [o] INDEX TARGET
\ifor{gojūonzu}{五十音図}{ごじゅうおんず}{50@50 Laute Tafel}
\ifor{hiragana}{平仮名}{ひらがな}{Hiragana}

\newcommand{\liroha}{\ivoc{Iroha}{伊呂波}{いろは}{Iroha}}

The word \liroha{} stands for \jtl{iroha uta} (iroha song) and is a Japanese poem
of the Heian era that contains all \hyperref[sec:Kana]{kana} characters. It was
used to order and memorize \hyperref[sec:Kana]{kana}. In contrast to today it
also contains more or less unused letters, like \jtl{we} or \jtl{wi} and it do not
contain the newer \jtl{n}. Usual the poem is written in
\hyperref[sec:Hiragana]{hiragana} from top to down.

\begin{center}
%\raisebox{10\height}{
%\framebox[20mm][r]{
\rotatebox{-90}{
\begin{minipage}{2.0cm}
    \setCJKfamilyfont{cjk-vert}[Script=CJK,RawFeature=vertical]{IPAPMincho}
    \renewcommand{\rubysep}{-0.5ex}
    \CJKfamily{cjk-vert}
いろはにほへとちりぬるをわかよたれそつねならむうゐのおくやまけふこえてあさきゆめみしゑひもせす
\end{minipage}
}
%}
%}
\end{center}

In this book the modern \hyperref[sec:Gojuonzu]{gojūonzu} is used.

