% ---------------------------------------------------------------------------
% - Iteration Marks
%     - Double a Vowel
%     - Double a Character
%     - Double two (or more) Characters
% ---------------------------------------------------------------------------
\section{Iteration Marks}\jsec{繰り返し記号}
% [o] LABEL
\label{sec:Iteration}
\label{sec:IterationMarks}
\label{sec:Choon}
\label{sec:RepetitionMark}
\label{sec:RepetitionMarkForKanjiAndKana}
\label{sec:Kunojiten}
% [o] INDEX
% {Katakana iteration marks}{反復記号}{はんぷくきごう}{Katakana Wiederholungszeichen}
\ien{iteration marks}\ide{Wiederholungszeichen}
\ifor{kanji}{漢字}{かんじ}{Kanji}
\ifor{katakana}{片仮名}{かたかな}{Katakana}

\newcommand{\lkanji}{\ivoc{kanji}{漢字}{かんじ}{Kanji}}
\newcommand{\literationmark}{\ivoc{iteration mark}{繰り返し記号}{くりかえしきごう}{Wiederholungszeichen}}

An \literationmark{} is a symbol indicating a repeated character. As usual
\textbf{iteration marks} can be found with \hyperref[sec:Kanji]{kanji} as well
as with \hyperref[sec:Katakana]{katakana}, seldom with
\hyperref[sec:Hiragana]{hiragana}. This section will introduce one
\hyperref[sec:Kanji]{kanji} and three \hyperref[sec:Katakana]{katakana}
\textbf{iteration marks} of which one can also be used for
\hyperref[sec:Hiragana]{hiragana}.


\subsection{Double a Vowel}\jsubsec{長音}\label{subsec:Choon}
\ifor{chōon}{長音}{ちょうおん}{Chōon}
\ien{double vowel}\ide{Konsonantenverdopplung}

\newcommand{\lchoon}{\ivoc{chōon}{長音}{ちょうおん}{Chōon}}

The \lchoon{} doubles the previous vowel. Please read the section
\nameref{subsec:DoublingVowels} on page \pageref{subsec:DoublingVowels} for
details.


\subsection{Double a Character}\jsubsec{踊り字}
%\ien{dancing mark}\ide{Tanzzeichen}
\ifor{repetition mark for kanji and kana}{踊り字}{おどりじ}{Wiederholungszeichen für Kanji und Kana}
\ifor{repetition mark}{重ね字}{かさねじ}{Wiederholungszeichen}
\ifor{repetition mark}{繰り返し記号}{くりかえしきごう}{Wiederholungszeichen}
% {Katakana iteration marks}{反復記号}{はんぷくきごう}{Katakana Wiederholungszeichen}

\newcommand{\lodoriji}{\ivoc{dancing mark}{踊り字}{おどりじ}{tanzendes Symbol}}
\newcommand{\lkasaneji}{\ivoc{repetion mark}{重ね字}{かさねじ}{Wiederholungszeichen}}
\newcommand{\lkurikaeshikigou}{\ivoc{repetion mark}{繰り返し記号}{くりかえしきごう}{Wiederholungszeichen}}

Some general names do exist for \textbf{iteration marks} in the Japanese language:
\lodoriji{}, \lkasaneji{} or \lkurikaeshikigou{} as "repetition symbols".

% music 𝄇 ?  {反復記号} {【はんぷくきごう】}

\ifor{dakuten}{濁点}{だくてん}{Dakuten}
\ifor{mora}{モーラ}{もーら}{Mora}

The \textbf{iteration mark} that can repeat any
\hyperref[sec:Katakana]{katakana} is \jquotesingleja{ヽ} and its \ldakuten{} form is
\jquotesingleja{ヾ}. This can only be found in rare\footnote{In the past
        \textbf{iteration marks} where widely used in old texts and may be used
in personal writing still today.} cases. For example the personal name Misuzu
{【みすゞ】} might contain this character and therefore the
\hyperref[sec:Katakana]{katakana} transcription as well. And since the
difference between the second last and the last \hyperref[sec:Mora]{mora} is
only a change in pronunciation the \textbf{dakuten} is added.


\subsection{Double two (or more) Characters}\jsubsec{くの字点}
\ifor{kunojiten}{くの字点}{くのじてん}{Kunojiten}
\ien{double multiple character}\ide{viele Zeichen verdoppeln}
\ifor{kanji}{漢字}{かんじ}{Kanji}
\ifor{okurigana}{送り仮名}{おくりがな}{Okuriagana}

\newcommand{\lkunojiten}{\ivoc{kunojiten}{くの字点}{くのじてん}{Kunojiten}}

In vertical writing exist another \textbf{iteration mark}, the \lkunojiten{}
which consist out of two characters \jquotesingleja{〳}+\jquotesingleja{〵} and
the \textbf{dakuten} form is \jquotesingleja{〴}+\jquotesingleja{〵}. It can
double two or more characters. As for the iteration mark above this is seldom
used.

\begin{center}
\setCJKfamilyfont{cjk-vert}[Script=CJK,RawFeature=vertical]{IPAPMincho}
\renewcommand{\rubysep}{-0.5ex}
%\raisebox{-.5\height}{
%\fbox{
\rotatebox{-90}{
\begin{minipage}{3.0cm} \CJKfamily{cjk-vert}
\Huge \ruby{所々}{ところ〴〵}
\end{minipage}
%}
%}
}
\end{center}

The \textbf{kunojiten} is the same for \hyperref[sec:Hiragana]{hiragana} and
\hyperref[sec:Katakana]{katakana}. The above example shows that the change of
sound {所々} {【ところどころ】} (Engl.: here and there) do not apply to the
\hyperref[sec:Kanji]{kanji} iteration mark \jquotesingleja{々}.

\begin{center}
\setCJKfamilyfont{cjk-vert}[Script=CJK,RawFeature=vertical]{IPAPMincho}
\renewcommand{\rubysep}{-0.5ex}
%\raisebox{-.5\height}{
%\fbox{
\rotatebox{-90}{
\begin{minipage}{3.0cm} \CJKfamily{cjk-vert}
\Huge \ruby{色々}{イロ/\}
\end{minipage}
%}
%}
}
\end{center}

If the \jquotesingleja{〳}+\jquotesingleja{〵}  is not available sometimes a
Japanese full wide slash and backslash is used.
\jquotesingleja{/}+\jquotesingleja{\}

If \hyperref[sec:Okurigana]{okurigana} is present no \textbf{iteration mark}
should be used. For example  {休み休み} {【やすみやすみ】} (Engl.: with a lot
of breaks).

The \textbf{kunojiten} character as such can be doubled by itself.

%\newcolumntype{V}{>{\centering\arraybackslash} m{.4\linewidth} }


\begin{center}
\setCJKfamilyfont{cjk-vert}[Script=CJK,RawFeature=vertical]{IPAPMincho}
\renewcommand{\rubysep}{-0.5ex}
%\fbox{
\begin{tabular}{ccc}
\rotatebox{-90}{
\begin{minipage}{4.5cm} \CJKfamily{cjk-vert}
\LARGE {トントントン}
\end{minipage}
}&
%\includegraphics[scale=1]{../share/katakana/4ar.pdf}
&
\rotatebox{-90}{
\begin{minipage}{4.5cm} \CJKfamily{cjk-vert}
\LARGE  {トン〳〵〳〵}
\end{minipage}
}\\
\end{tabular}
%}
\end{center}

\Link \href{https://ja.wikipedia.org/wiki/%E8%B8%8A%E3%82%8A%E5%AD%97}{https://ja.wikipedia.org/wiki/踊り字}



