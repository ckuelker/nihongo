% ---------------------------------------------------------------------------
\section{Kana}\jsec{仮名}
% [o] LABEL
\label{sec:Kana}
% [o] INDEX
\ifor{kana}{仮名}{かな}{Kana}
\ifor{mora}{モーラ}{もーら}{Mora}
\ifor{kanji}{漢字}{かんじ}{Kanji}
\ifor{okurigana}{送り仮名}{おくりがな}{Okurigana}
\ifor{hentaigana}{変体仮名}{へんたいがな}{Hentaigana}
\ifor{hiragana}{平仮名}{ひらがな}{Hiragana}
\ifor{katakana}{片仮名}{かたかな}{Katakana}
\ifor{man'yōgana}{万葉仮名}{まんようがな}{Man'yōgana}

%\newcommand{\lkana}{\ivoc{kana}{仮名}{かな}{Kana}}

The Japanese script category \lkana{} is a subordinate concept of Japanese
\hyperref[sec:Mora]{mōra} scripts ending with \textit{-kana} or \textit{-gana}.
Some Japanese scripts like \hyperref[sec:Kanji]{kanji} or
\hyperref[sec:Romaji]{rōmaji}, that are ending in \textit{-ji}, are excluded
and \textbf{kana} is often used in contrast to \hyperref[sec:Kanji]{kanji} and
\hyperref[sec:Romaji]{rōmaji} as these are not \hyperref[sec:Mora]{mōra} based
scripts. The concept of \textbf{kana} is also used sometimes to contrast
\hyperref[sec:Kanji]{kanji}, because \hyperref[sec:Kanji]{kanji} possess a
meaning while \textbf{kana} exhibit none.

It is easy to see that in contemporary Japanese scripts like
\hyperref[sec:Hiragana]{hiragana} and \hyperref[sec:Katakana]{katakana} are in
fact \textbf{kana} scripts. However some script categories, like
\hyperref[sec:Okurigana]{okuriagana} or \hyperref[sec:Furigana]{furigana}, are
scripts usually written in \hyperref[sec:Hiragana]{hiragana}, sometimes
\hyperref[sec:Katakana]{katakana}, and are therefore not \textbf{kana} in the
sense of a distinct script. \hyperref[sec:Okurigana]{Okuriagana} or
\hyperref[sec:Furigana]{furigana} are \textbf{kana} used for a certain purpose
and point to a specific function. \hyperref[sec:Furigana]{Furigana} are for
example \textbf{kana} that are used to add endings to words.

Other \textbf{kana} like \hyperref[sec:Hentaigana]{hentaigana} are obsolete and
depreciated versions of \hyperref[sec:Hiragana]{hiragana}. Historically there
have been more than one \hyperref[sec:Hiragana]{hiragana} for one
\hyperref[sec:Mora]{mōra} that where stylistic variants or distinct
alternatives.

And finally \hyperref[sec:Manyogana]{man'yōgana} are Chinese characters that
have been used as phonetic characters around mid 7th century. The name suggest
that this \textbf{kana} are \textbf{kana} in relation to \textit{man'yō} which
points to an old text, the \textit{Man'yōshū}. This name is somewhat misleading
since this Chinese characters where not only used in the \textit{Man'yōshū} in
this fashion as phonetic characters. The same characters have been used over a
long time and the number of characters weren't been constant and also this
characters had a dual use as phonetic characters and ordinary
\hyperref[sec:Kanji]{kanji}.

