% +---------------------------------------------------------------------------+
% | content/english/sec/Katakana.tex                                          |
% |                                                                           |
% | A secyion about katakana suited for the additional information            |
% |                                                                           |
% | Version: 0.1.0                                                            |
% |                                                                           |
% | Changes:                                                                  |
% |                                                                           |
% | 0.1.0 2022-09-17 Christian Külker <c@c8i.org>                             |
% |     - Initial versioned release (fixes in time, and others)               |
% |                                                                           |
% +---------------------------------------------------------------------------+

\section{Katakana}\label{sec:Katakana}\jsec{片仮名}

\ifor{hiragana}{平仮名}{ひらがな}{Hiragana}
\ifor{katakana}{片仮名}{かたかな}{Katakana}

At the beginning of the the Heian period (794-1185/1192) the capital was moved
to Kyoto and the relation to China declined. It was also in this time when
Manyogana (isd in Kanbun) are used and katakana shaped from kanji components
used for pronunciation. From certain manyogana, that were well known,
components where dropped and the remaining part became katakana. For example
from the Chinese character \jquotesingleja{加}, meaning \jquotesingle{to add}
which is pronunced \jtl{ka} the \jquotesingleja{口} component was dropped and
the remainig \jtl{カ} became the kataka character that is pronunced \jtl{ka}.



\begin{figure}[H]
\begin{center}
% https://upload.wikimedia.org/wikipedia/commons/0/0c/Katakana_origine.svg
\includesvg{../share/ei/Katakana_origine}
\caption{Katakana Origin}
\label{fig:KatakanaOrigin} % Label after caption: https://www.overleaf.com/learn/latex/Referencing_Figures
\end{center}
\end{figure}

