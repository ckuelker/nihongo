\section{Phonetic Character}\jsec{表音文字}
% [o] LABEL
\label{sec:PhoneticCharacter}
% [o] INDEX
\ifor{phonetic character}{表音文字}{ひょうおんもじ}{phonetisches Zeichen}
\ifor{man'yōgana}{万葉仮名}{まんようがな}{Man'yōgana}
\ifor{kana}{仮名}{かな}{Kana}

In this document the term \ivoc{phonetic
character}{表音文字}{【ひょうおんもじ】}{phonetisches Zeichen} refers
genetically to a Chinese characters reading and the usage of this character
just for this purpose and \textit{not} for its meaning. This common set
expression has been used in avoidance of the term
\hyperref[sec:Manyogana]{man'yōgana}. See the section \nameref{sec:Manyogana}
on page \pageref{sec:Manyogana} to understand the critique.

The \textbf{phonetic character} has to be distinguished also from the
linguistic term \textit{phonogram} that describes a written character which
represents a \textit{phonem} (speech sound) such as the Latin alphabet or the
Japanese \hyperref[sec:Kana]{kana}.

