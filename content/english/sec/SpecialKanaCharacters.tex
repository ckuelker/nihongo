% +---------------------------------------------------------------------------+
% | SpecialKanaCharacters.tex                                                 |
% |                                                                           |
% | Collect special hiragana and katakana characters                          |
% |                                                                           |
% | Version: 0.1.2                                                            |
% |                                                                           |
% | Changes:                                                                  |
% |                                                                           |
% | 0.1.2 2024-04-02 Christian Külker <c@c8i.org>                             |
% |     - Fix doubling hiragana (oo vs. ou)                                   |
% | 0.1.1 2024-03-29 Christian Külker <c@c8i.org>                             |
% |     - Fix typos                                                           |
% | 0.1.0 2022-08-02 Christian Külker <c@c8i.org>                             |
% |     - Merge katakana and hiragana                                         |
% |                                                                           |
% +---------------------------------------------------------------------------+

\ifthenelse{\equal{hiragana}{\jtopic}}{%
\section{Special Hiragana Characters}\jsec{特別ひらがな}
% LABEL
\label{sec:SpecialHiraganaCharacters}
\label{sec:SpecialKanaCharacters}
% INDEX
\ifor{special hiragana characters}{特別ひらがな}{とくべつひらがな}{Spezielle Hiragana Zeichen}
}{}
\ifthenelse{\equal{katakana}{\jtopic}}{%
\section{Special Katakana Characters}\jsec{特別カタカナ}
% LABEL
\label{sec:SpecialKatakanaCharacters}
\label{sec:SpecialKanaCharacters}
% INDEX
\ifor{special katakana characters}{特別カタカナ}{とくべつかたかな}{Spezielle Katakana Zeichen}
}{}
% INDEX
\ifor{hiragana}{平仮名}{ひらがな}{Hiragana}
\ifor{gojūonzu}{五十音図}{ごじゅうおんず}{50@50 Laute Tafel}

As mentioned before both \textbf{kana} syllables are almost the same, except
the shape. This is especially true for the \hyperref[sec:Gojuonzu]{gojūonzu
(50 sound table)}. This section will show the special characters, some are
different from the ordinary \jtopic{} set.

%TODO check if point changes orientation and alignment in case of changing
%writing direction.

\ifthenelse{\equal{hiragana}{\jtopic}}{%
\subsection{Doubling Vowels in Hiragana}\jsubsec{ひらがなでの倍増母音}
}{}
\ifthenelse{\equal{katakana}{\jtopic}}{%
\subsection{Doubling Vowels in Katakana}\jsubsec{カタカナでの倍増母音}
}{}
% [o] LABEL
\label{subsec:DoublingVowelsIn\jscript}
\label{subsec:DoublingVowels}
\label{sec:DoublingVowelsIn\jscript}
\label{sec:DoublingVowels}
% [o] INDEX
\ifor{doubling vowels}{倍増母音}{ばいぞうぼいん}{Vokalverdopplung}
\ifor{repetition mark}{繰り返し記号}{くりかえしきごう}{Wiederholungszeichen}
\ifor{hiragana}{平仮名}{ひらがな}{Hiragana}
\ifor{katakana}{片仮名}{かたかな}{Katakana}

% ひらがなでの倍増母音 【ばいぞうぼいん】
% カタカナでの倍増母音 【ばいぞうぼいん】


\ifthenelse{\equal{hiragana}{\jtopic}}{%

The usual way to double a vowel in \textbf{hiragana} is to write that hiragana
again. In the hiragana \jquotesingleja{お} character can be doubled with either
\jquotesingleja{う} or \jquotesingleja{お} . So \jtl{ō} becomes either
\jquotesingleja{おう} or \jquotesingleja{おお}. There is no rule to it. This
has to be rememberd. However in most cases \jquotesingleja{お} is doubled as
\jquotesingleja{おう}.

}{}
\ifthenelse{\equal{katakana}{\jtopic}}{%

Special \hyperref[sec:Katakana]{katakana} characters do also exists. The most
important character is \ivoc{chōon}{長音}{ちょうおん}{Chōon} the plain
iteration character \jquotesingleja{ー}, written as a stroke. It is one of the
very few which changes orientation according the writing orientation. When
writing katakana from left to right the iteration character is horizontal,
while writing katakana from up to down it is vertical. The function of this
character is to double the previous mora.  This is also different from
\hyperref[sec:Hiragana]{hiragana}. (For doubling also other katakana character,
refer to section \nameref{sec:Iteration} on page \pageref{sec:Iteration}.)

\bigskip

\CharacterExplanation{k-iteration-s}{In standard gothic fonts the
\hyperref[sec:Katakana]{katakana} iteration character is just a straight line
and it is not possible to understand in which direction it has to written. }

\bigskip

\CharacterExplanation{k-iteration-sm}{However if it is written with a different
font or with a brush it is clearly visible that in horizontal writing it is
written from left to right.}

}{}
\bigskip

%\definecolor{orange}{rgb}{1,0.5,0}
%\definecolor{mygreen}{rgb}{.2,1,.2}

\setCJKfamilyfont{cjk-horiz-m}[Script=CJK,RawFeature=horizontal]{IPAMincho}
\setCJKfamilyfont{cjk-horiz-g}[Script=CJK,RawFeature=horizontal]{IPAPGothic}
%\setCJKfamilyfont{cjk-vert}[Script=CJK,RawFeature=vertical]{Kozuka Gothic Pro M}
\setCJKfamilyfont{cjk-vert-m}[Script=CJK,RawFeature=vertical]{IPAMincho}
\setCJKfamilyfont{cjk-vert-g}[Script=CJK,RawFeature=vertical]{IPAPGothic}

\bigskip
\textit{Example:}

\bigskip

\begin{figure}[H]
\begin{center}
\begin{tabular}{p{7cm}p{7cm}}
Katakana:&Hiragana:\\
\CJKfamily{cjk-horiz-h}
\Huge カ\textbf{\color{magenta}ー}ド \jtl{kaado} &
\Huge か\textbf{\color{magenta}あ}ど \jtl{kaado}\\
\CJKfamily{cjk-horiz-g}
\Huge カ\textbf{\color{magenta}ー}ド \jtl{kaado} &
                                                \\
\end{tabular}
\end{center}
\caption{Kana vowel doubling example}
\label{fig:KanaVowlDoublingExample}
\end{figure}

\bigskip

This character is very often used and makes \hyperref[sec:Hiragana]{hiragana}
for this easier then hiragana. The long vowel ambiguity do not exist.

As mentioned above the orientation of the \hyperref[sec:Katakana]{katakana}
iteration character changes with the direction of writing. The above example
with different writing orientation.

\medskip
\textit{Example:}

\medskip


\begin{figure}[H]
\begin{center}
\begin{tabular}{p{3.5cm}p{3.5cm}p{3.5cm}m{3.5cm}}
horizontally&
\mbox{
\begin{minipage}{3.2cm}
\CJKfamily{cjk-horiz-h}
\Huge カ\textbf{\color{magenta}ー}ド
\CJKfamily{cjk-horiz-g}
\Huge カ\textbf{\color{magenta}ー}ド
\end{minipage}
}
& vertically &
\raisebox{-.5\height}{
\mbox{
\rotatebox{-90}{
\begin{minipage}{3.2cm}
\CJKfamily{cjk-vert-m}
\Huge カ\textbf{\color{magenta}ー}ド
\CJKfamily{cjk-vert-g}
\Huge カ\textbf{\color{magenta}ー}ド
\end{minipage}
}
}
}
\\
\end{tabular}
\end{center}
\caption{Katakana horizontal and vertical vowel doubling example}
\label{fig:KatakanaHirzontalVerticalVowlDoublingExample}
\end{figure}

\medskip

% めったに使われない片仮名 【めったにつかわれないかたかな】
\ifthenelse{\equal{hiragana}{\jtopic}}{%
    \subsection{Seldom Used Hiragana}\jsubsec{めったに使われない平仮名}\label{subsec:SeldomlyUsedHiragana}

\ifor{voice!new writing}{声}{こえ}{Stimme!neue Schreibweise}
\ifor{voice!old writing}{声}{こゑ}{Stimme!alte Schreibweise}

All \textbf{hiragana} mentioned in the \hyperref[sec:Gojuonzu]{gojūonzu (50
sound table)} are used. There is no obsolete character in the table unlike the
\textbf{katakana} \jtl{wo} \jquotesingleja{ヲ}. However sometimes one can find
a \textbf{gojūonzu} with the additional characters: \jquotesingleja{ゐ} wi,
pronounced as \jtl{i}, and \jquotesingleja{ゑ} we, pronounced \jtl{e}, in the
\jtl{wa} row. This characters are old characters and normally not used. It is
safe to skip learning this characters.  Words that used to have
\jquotesingleja{ゐ} had it replaced with \jquotesingleja{い} \jtl{i} and words
that used to have \jquotesingleja{ゑ} had it replaced with \jquotesingleja{え}
\jtl{e}. Examples: old \jquotesingleja{ゐる} is now written
\jquotesingleja{いる} \jtl{iru} ({居る} - to be somewhere), old
\jquotesingleja{こゑ} is now written as \jquotesingleja{こえ} \jtl{koe} ({声} -
voice).

}{}
% めったに使われない平仮名 【めったにつかわれないひらがな】
\ifthenelse{\equal{katakana}{\jtopic}}{%
    \subsection{Seldom Used Katakana}\jsubsec{めったに使われない片仮名}\label{subsec:SeldomlyUsedKatakana}

Since all particles are written in \textbf{hiragana} the particle \textit{wo},
pronounced as \jtl{o}, is written with the hiragana \jquotesingleja{を}. The
equivalent \textbf{katakana} wo \jquotesingleja{ヲ}, also pronounced \jtl{o},
is part of the \hyperref[sec:Gojuonzu]{gojūonzu}, however it is not used in
modern texts and it is not a particle. It was used in old texts (like
telegrams). Therefore the katakana \jquotesingleja{ヲ} can be skipped to learn.

\label{sec:Wi}
\label{sec:We}
\ifor{we}{ヱ}{エ}{we}
\ifor{wi}{ヰ}{イ}{wi}

In some cases the \textbf{gojūonzu} contain additional chrarcters:
\jquotesingleja{ヰ} wi, pronounced as \jtl{i}, and \jquotesingleja{ヱ} we,
pronounced \jtl{e}, in the \jtl{wa} row. This characters are old characters and
normally not used. It is safe to skip learning this characters. Words that used
to have \jquotesingleja{ヰ} had it replaced with \jquotesingleja{イ} \jtl{i}
and words that used to have \jquotesingleja{ヱ} had it replaced with
\jquotesingleja{エ} \jtl{e}.

}{}



