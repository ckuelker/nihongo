%----------------------------------------------------------------------------
\section{Katakana \jtl{a} Row}\jsec{片仮名ア行}\label{sec:KatakanaARow}

\Krow{arow}{a}{i}{u}{e}{o}

\label{letter:a}\KLETTER{a} The \jkatakana{} \jquotesingleja{ア} derives from
the \hyperref[sec:Radical]{radical} of the
\hyperref[sec:PhoneticCharacter]{phonetic character} \jquotesingleja{阿}.
Together with the \jkatakana{} \jquotesingleja{フ} \jtl{fu} the smaller sized
\jquotesingleja{ァ} gives the combination \jquotesingleja{ファ} and is
pronounced as such \jtl{fa}.

\label{letter:i}\KLETTER{i} The \jkatakana{} \jquotesingleja{イ} derives from
the \hyperref[sec:PhoneticCharacter]{phonetic characters} \jquotesingleja{伊}
left element (\hyperref[sec:Radical]{radical}). A smaller version
\jquotesingleja{ィ} is used in combinations with other letters and represents a
\hyperref[sec:Diphthong]{diphthong}.

\label{letter:u}\KLETTER{u} The \jkatakana{} \jquotesingleja{ウ} derives from
the \hyperref[sec:PhoneticCharacter]{phonetic character} \jquotesingleja{宇}. A
smaller version \jquotesingleja{ゥ} is used in combinations with other letters
and represents a \hyperref[sec:Diphthong]{diphthong} and is written as "w".
Even though the combination \jquotesingleja{トゥ} \jtl{tu} exist, it is
relatively new and many words do not use it. In this cases \jquotesingleja{ツ}
\jtl{tsu} is used. \jquotesingleja{ウ} can take \hyperref[sec:Dakuten]{dakuten}
to form \jquotesingleja{ヴ} \jtl{vu}, which is relatively new and can replace
\jquotesingleja{ブ} \jtl{bu}.

% UFuWaSimilarity
\subsection{\jtl{u}, \jtl{fu} and \jtl{wa} Similarity} \label{subsec:UFuWaSimilarity}

The Katakana characters \jquotesingleja{ウ}, \jquotesingleja{フ} and
\jquotesingleja{ワ} can be easily distinguished. All three have a different
stroke count. However the shape is similar. Therefore they can be mistaken.
Especially when they have no context.

\bigskip

\begin{figure}[H]
\begin{center}
\begin{tabular}{|c|c|c|}\hline
\KLETTER{u}&\KLETTER{fu}&\KLETTER{wa}\\\hline
\end{tabular}
\end{center}
\caption{\jtl{u}, \jtl{fu} and \jtl{wa} similarity}
\label{fig:UuFuAndWaSimilarity}
\end{figure}




\newpage

\label{letter:e}\KLETTER{e} The \jkatakana{} \jquotesingleja{エ} derives from
the \hyperref[sec:PhoneticCharacter]{phonetic characters} \jquotesingleja{江}
right element (\hyperref[sec:Radical]{radical}). A smaller version
\jquotesingleja{ェ} is used in combinations with other letters and express
\hyperref[sec:Mora]{morae} of foreign origin. For example \jquotesingleja{ヴェ}
is pronounced \jtl{ve}.

\label{letter:o}\KLETTER{o} The \jkatakana{} \jquotesingleja{オ} derives from
the \hyperref[sec:PhoneticCharacter]{phonetic character} \jquotesingleja{於}. A
smaller version \jquotesingleja{ォ} is used in combinations with other letters
and express' \hyperref[sec:Mora]{morae} of foreign origin. For example
\jquotesingleja{フォ } is pronounced \jtl{fo}.

\newpage

% ---------------------------------------------------------------------------
\subsection{\jtl{a}}\jsubsec{\jquotesingleja{ア}} \label{sec:KatakanaA}

\KatakanaHeader{a}{ The Katakana \jquotesingleja{ア} is written with two
        strokes. The first stroke starts horizontal. The second stroke is a
        curve with can be attached to the first stroke in hand writing, but not
        at the horizontal part - at the end of the first line.}
        \KatakanaTraining{a}

% ---------------------------------------------------------------------------
\subsection{\jtl{i}}\jsubsec{\jquotesingleja{イ}} \label{sec:KatakanaI}

\KatakanaHeader{i}{ The Katakana \jquotesingleja{イ} is written with one
stroke. The first stroke is a curve from upper right to lower left. The second
stroke is a vertical line attached to the first at the top.}
\KatakanaTraining{i}

% ---------------------------------------------------------------------------
\subsection{\jtl{u}}\jsubsec{\jquotesingleja{ウ}} \label{sec:KatakanaU}

\KatakanaHeader{u}{The Katakana \jquotesingleja{ウ} is written with three
strokes. The first stroke a small vertical line. The second a small vertical
line again and the third line a horizontal line connection the two others.}
\KatakanaTraining{u}

% ---------------------------------------------------------------------------
\subsection{\jtl{e}}\jsubsec{\jquotesingleja{エ}} \label{sec:KatakanaE}

\KatakanaHeader{e}{The Katakana \jquotesingleja{エ} is written with three
strokes. It is very geometrically consisting only out of horizontal and
vertical lines connected together.} \KatakanaTraining{e}

% ---------------------------------------------------------------------------
\subsection{\jtl{o}}\jsubsec{\jquotesingleja{オ}} \label{sec:KatakanaO}

\KatakanaHeader{o}{The Katakana \jquotesingleja{オ} is written with three
strokes. The first line is horizontal and together with the second stroke it
constructs a perfect crossing. The third stroke beginning lies at the center of
the crossing.} \KatakanaTraining{o}

% ---------------------------------------------------------------------------
\subsection{\jtl{a} Row Training}\jsubsec{片仮名ア行練習}
\Padding
\begin{longtable}[c]{p{2cm}p{2cm}p{3cm}p{6cm}p{2cm}}
\textit{Katakana}&\textit{Rōmaji}&\textit{Original}&\textit{Remark}&Origin\\\hline
ウエア&wuea&ware&          &English\\
エア  &ea  &air &          &English\\
エイ  &ei  &A   &the letter&English\\
\end{longtable}

\KanaSimpleTraining{Katakana to Rōmaji}{
\Transcribe{1.}{ウエア}{}{wear, ware}
\Transcribe{2.}{エア}{}{air}
\Transcribe{3.}{エイ}{}{A (the letter)}
\Transcribe{4.}{アイ}{}{I (the letter)}
\Transcribe{5.}{オウ}{}{O (the letter)}
%\Transcribe{6.}{イア}{}{ear}
}

\KanaSimpleTraining{Rōmaji to Katakana}{
\Transcribe{1.}{ea}{}{air}
\Transcribe{2.}{ai}{}{I (the letter)}
\Transcribe{3.}{ou}{}{O (the letter)}
\Transcribe{4.}{ei}{}{A (the letter)}
\Transcribe{5.}{uea}{}{wear, ware}
%\Transcribe{6.}{ia}{}{ear}
}

\newpage
\Padding
\begin{longtable}[c]{p{2cm}p{2cm}p{3cm}p{6cm}p{2cm}}
\textit{Katakana}&\textit{Rōmaji}&\textit{Original}&\textit{Remark}&Origin\\\hline
アイ  &ai  &I   &the letter&English\\
オウ  &ou  &O   &the letter&English\\
イア  &ia  &ear &          &English\\
\end{longtable}

\KanaSimpleTraining{English to Rōmaji}{
\Transcribe{1.}{ear}{}{}
\Transcribe{2.}{I (the letter)}{}{}
\Transcribe{3.}{air}{}{}
\Transcribe{4.}{O (the letter)}{}{}
\Transcribe{5.}{wear, ware}{}{}
%\Transcribe{6.}{A (the letter)}{}{}
}

\KanaSimpleTraining{English to Katakana}{
\Transcribe{1.}{I (the letter)}{}{}
\Transcribe{2.}{O (the letter)}{}{}
\Transcribe{3.}{air}{}{}
\Transcribe{4.}{ear}{}{}
\Transcribe{5.}{wear, ware}{}{}
%\Transcribe{6.}{A (the letter)}{}{}
}
\newpage
