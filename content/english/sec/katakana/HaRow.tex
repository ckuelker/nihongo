% ---------------------------------------------------------------------------
\section{ Katakana \jtl{ha} Row}\jsec{片仮名ハ行}\label{sec:KatakanaHaRow}

\Krow{harow}{ha}{hi}{fu}{he}{ho}

\label{letter:ha}\KLETTER{ha} The \textbf{katakana} \jquotesingleja{ハ} is
pronounced \jtl{ha} and derives from the
\hyperref[sec:PhoneticCharacter]{phonetic character} \jquotesingleja{八 }.  A
\hyperref[sec:Dakuten]{dakuten} version exists and pronounced as \jtl{ba}.

\label{letter:hi}\KLETTER{hi} The \textbf{katakana} \jquotesingleja{ヒ} derives
from the \hyperref[sec:PhoneticCharacter]{phonetic characters}
\jquotesingleja{比} reight \hyperref{sec:Radical}{radical}. It is pronounced as
\jtl{hi}. A \hyperref[sec:Dakuten]{dakuten} version exists and pronounced as
\jtl{bi}.

\label{letter:fu}\KLETTER{fu} The \textbf{katakana} \jquotesingleja{フ} derives
from the \hyperref[sec:PhoneticCharacter]{phonetic characters} upper left part
of \jquotesingleja{不 }. It is pronounced as \jtl{fu}. A
\hyperref[sec:Dakuten]{dakuten} version exists and pronounced as \jtl{bu}.

\label{letter:he}\KLETTER{he} The \textbf{katakana} \jquotesingleja{ヘ} derives
from the \hyperref[sec:PhoneticCharacter]{phonetic characters} right
\hyperref{sec:Radical}{radical} of \jquotesingleja{部}. It is pronounced as
\jtl{he}. A \hyperref[sec:Dakuten]{dakuten} version exists and pronounced as
\jtl{be}.

\Warn{Warning}{The Katakana \jquotesingleja{ヘ} is the same character as the
        \hyperref[sec:Hiragana]{hiragana} \jquotesingleja{へ}. In some
        documents they can be distinguished because the font is different.
        However in genral they are the same. }

\label{letter:ho}\KLETTER{ho} The \textbf{katakana} \jquotesingleja{ホ} derives
from the \hyperref[sec:PhoneticCharacter]{phonetic characters} lower right part
of \jquotesingleja{保} wich by itself is the \hyperref[sec:Radical]{radical}
and \hyperref[sec:Kanji]{kanji} of tree. It is pronounced as \jtl{ho}. A
\hyperref[sec:Dakuten]{dakuten} version exists and pronounced as \jtl{bo}.

% UFuWaSimilarity
\subsection{|u|, |fu| and |wa| Similarity} \label{subsec:UFuWaSimilarity}

The Katakana characters {「ウ」}, {「フ」} and {「ワ」} can be easily
distinguished. All three have a different stroke count. However the shape is
similar. Therefore they can be mistaken. Especially when they have no context. 

\bigskip

\begin{center}
\begin{tabular}{|c|c|c|}\hline
\KLETTER{u}&\KLETTER{fu}&\KLETTER{wa}\\\hline
\end{tabular}
\end{center}




\newpage

% ハヒフヘホ
% ---------------------------------------------------------------------------
\subsection{\jtl{ha}}\jsubsec{\jquotesingleja{ハ}} \label{sec:KatakanaHa}

\KatakanaHeader{ha}{ The Katakana \jquotesingleja{ハ} is written with two
strokes. Non of them is striaght.} \KatakanaTraining{ha}

% ---------------------------------------------------------------------------
\subsection{\jtl{hi}}\jsubsec{\jquotesingleja{ヒ}} \label{sec:KatakanaHi}

\KatakanaHeader{hi}{ The Katakana \jquotesingleja{ヒ} is written with two
strokes. One stroke from right to left. The other stroke from up to down and
then a curve.  The difficulty of this character is to hit the first stroke with
the second. } \KatakanaTraining{hi}

% ---------------------------------------------------------------------------
\subsection{\jtl{fu}}\jsubsec{\jquotesingleja{フ}} \label{sec:KatakanaFu}

\KatakanaHeader{fu}{The pronunciation of Katakana \jquotesingleja{フ} is
\textbf{not} \jtl{hu} it is \jtl{fu} and it is written with only one stroke. }
\KatakanaTraining{fu}

% ---------------------------------------------------------------------------
\subsection{\jtl{he}}\jsubsec{\jquotesingleja{ヘ}} \label{sec:KatakanaHe}

\KatakanaHeader{he}{Katakana \jquotesingleja{ヘ} is written with one stroke
from left to right. This is the same character as
\hyperref[sec:Hiragana]{hiragana} \jtl{he}.} \KatakanaTraining{he}

% ---------------------------------------------------------------------------
\subsection{\jtl{ho}}\jsubsec{\jquotesingleja{ホ}} \label{sec:KatakanaHo}

\KatakanaHeader{ho}{

The Katakana \jquotesingleja{ホ} character reminds at the Kanji for tree and is
also written in the same order and with the same amount of stroke. However the
left and right 'root' is not connected to the base. In cursive writing the
character is written with a hook-stroke as the second stroke. This is abstract
available even in the bold form where the second stroke has a small curve at
the end.

}
\KatakanaTraining{ho}

% ---------------------------------------------------------------------------
\subsection{\jtl{ha} Row Training}\jsubsec{片仮名ハ行練習}\label{sec:HaRowTraining}
\Padding
\begin{longtable}[c]{p{3cm}p{2cm}p{3cm}p{5cm}p{2cm}}
\textit{Katakana}&\textit{Rōmaji}&\textit{Original}&\textit{Remark}&\textit{Origin}\\\hline
ホットケーキ&hottokēki&hotcake    &a pancake                         &English\\
コーヒー    &kōhī     &koffie     &珈琲  coffee                      &Dutch\\
ソフト      &sofuto   &soft(ware) &                                  &English \\
\end{longtable}

\KanaSimpleTraining{Katakana to Rōmaji}{
\Transcribe{1.}{ホットケーキ}{}{hotcake}
\Transcribe{2.}{コーヒー}{}{coffee}
\Transcribe{3.}{ソフト}{}{soft(ware)}
\Transcribe{4.}{ハイタッチ}{}{high five}
\Transcribe{5.}{ハウス}{}{house}
%\Transcribe{6.}{ハイネック}{}{high neck}
}

\KanaSimpleTraining{Rōmaji to Katakana}{
\Transcribe{1.}{kōhī}{}{coffee}
\Transcribe{2.}{hottokēki}{}{hotcake}
\Transcribe{3.}{haitacchi}{}{high five}
\Transcribe{4.}{sofuto}{}{soft(ware)}
\Transcribe{5.}{hainekku}{}{high neck}
%\Transcribe{6.}{hausu}{}{house}
}

\newpage
\Padding
\begin{longtable}[c]{p{2cm}p{2cm}p{3cm}p{6cm}p{2cm}}
\textit{Katakana}&\textit{Rōmaji}&\textit{Original}&\textit{Remark}&\textit{Origin}\\\hline
ハイタッチ  &haitacchi&high touch &high five                         &English\\
ハウス      &hausu    &Haus, house&                                  &English, German\\
ハイネック  &hainekku &high neck  &turtle neck style sweater or shirt&English\\
\end{longtable}
\KanaSimpleTraining{English to Rōmaji}{
\Transcribe{1.}{coffee}{}{}
\Transcribe{2.}{hotcake}{}{}
\Transcribe{3.}{high five}{}{}
\Transcribe{4.}{software}{}{}
\Transcribe{5.}{high neck}{}{}
%\Transcribe{6.}{house}{}{}
}

\KanaSimpleTraining{English to Katakana}{
\Transcribe{1.}{hotcake}{}{}
\Transcribe{2.}{high five}{}{}
\Transcribe{3.}{coffee}{}{}
\Transcribe{4.}{high neck}{}{}
\Transcribe{5.}{house}{}{}
%\Transcribe{6.}{software}{}{}
}

\newpage
