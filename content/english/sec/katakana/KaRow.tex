% ---------------------------------------------------------------------------
\section{Katakana \jtl{ka} Row}\jsec{片仮名カ行}\label{sec:KatakanaKaRow}

\Krow{karow}{ka}{ki}{ku}{ke}{ko}

\label{letter:ka}\KLETTER{ka} The  \textbf{katakana} \jquotesingleja{カ} is
pronounced \jtl{ka} and  derives from the
\hyperref[sec:PhoneticCharacter]{phonetic character}s \jquotesingleja{加} left
\hyperref[sec:Radical]{radical}.  A \hyperref[sec:Dakuten]{dakuten} version
exists and pronounced as \jtl{ga}.

%\hyperref[sec:Handakuten]{handakuten} does not exist in daily Japanese.
% \jquotesingleja{一ヵ所} {【いちかしょ】} (one place)
% \jquotesingleja{一ヶ所} {【いちかしょ】} (one place).
% 十ヵ条(十ヶ条)

\Note{Note}{

A smaller version \jquotesingleja{ヵ} is rare but used in combinations with
number particles.  For example in \jquotesingleja{一ヵ月} {【いっかげつ】} (one
month) and others.  This cases can also be written \jquotesingleja{一ヶ月}
{【いっかげつ】} (one month). Please see also \nameref{sec:KatakanaKe}. \Link
\href{https://ja.wikipedia.org/wiki/\%E3\%83\%B5}{ヵ}

}

\label{letter:ki}\KLETTER{ki} The \textbf{katakana} \jquotesingleja{キ} derives
from the \hyperref[sec:PhoneticCharacter]{phonetic character}s middle part of
either \jquotesingleja{機} or \jquotesingleja{幾}.  It is pronounced as
\jtl{ki}.  A \hyperref[sec:Dakuten]{dakuten} version exists and pronounced as
\jtl{gi}.


\label{letter:ku}\KLETTER{ku} The \textbf{katakana} \jquotesingleja{ク} derives
from the \hyperref[sec:PhoneticCharacter]{phonetic character}s left upper part
of \jquotesingleja{久}.  It is pronounced as \jtl{ku}.  A
\hyperref[sec:Dakuten]{dakuten} version exists and pronounced as \jtl{gu}.  A
smaller version exists, but is used for the Ainu Language.

\label{letter:ke}\KLETTER{ke} The \textbf{katakana} \jquotesingleja{ケ} derives
from the \hyperref[sec:PhoneticCharacter]{phonetic character}s upper and left
part of \jquotesingleja{介}.  It is pronounced as \jtl{ke}.  A
\hyperref[sec:Dakuten]{dakuten} version exists and pronounced as \jtl{ge}.  The
smaller version \jquotesingleja{ヶ} is explained in the following note.

\newpage

\Note{Note}{

A smaller version \jquotesingleja{ヶ} is rare but used in combinations with
number particles.  For example in \jquotesingleja{一ヶ月} {【いっかげつ】} (one
month) and others.  This cases can also be written \jquotesingleja{一ヵ月}
{【いっかげつ】} (one month). There are cases where only \jquotesingleja{ヶ}
can be written {七ヶ宿} {【シチカシュク】} (Place at the south west border of
the prefecture Miyagi).  In other rare cases this character can be pronounced
different \jquotesingleja{関ヶ原} {【せきがはら】} (Place at the south border
of the Gifu prefecture, known by the battle at 1600.). Please see also
\nameref{sec:KatakanaKa}. \Link
\href{https://ja.wikipedia.org/wiki/\%E3\%83\%B5}{ヵ}

}

\label{letter:ko}\KLETTER{ko} The \textbf{katakana} \jquotesingleja{コ} derives
from the \hyperref[sec:PhoneticCharacter]{phonetic character}s upper part of
\jquotesingleja{己}.  It is pronounced as \jtl{ko}.  A
\hyperref[sec:Dakuten]{dakuten} version exists and pronounced as \jtl{go}.



\newpage

% ---------------------------------------------------------------------------
\subsection{\jtl{ka}}\jsubsec{\jquotesingleja{カ}} \label{sec:KatakanaKa}

\KatakanaHeader{ka}{ \jtl{ka} is written with 2 strokes. Basically the same way
as the Hiragana \jquotesingleja{か} it looks like a squarish version, but
without the last stroke. The hook at the second stroke is less significant or
important.  } \KatakanaTraining{ka}

% ---------------------------------------------------------------------------
\subsection{\jtl{ki}}\jsubsec{\jquotesingleja{キ}} \label{sec:KatakanaKi}

\KatakanaHeader{ki}{ The shape alignment of the \jquotesingleja{キ} character
is not straight towards its environment. However the junctions are more or less
90 degrees.  } \KatakanaTraining{ki}

% ---------------------------------------------------------------------------
\subsection{\jtl{ku}}\jsubsec{\jquotesingleja{ク}} \label{sec:KatakanaKu}
% ---------------------------------------------------------------------------

\KatakanaHeader{ku}{ The first stroke is similar the stroke of
\jquotesingleja{ケ} is a curve. While the second stroke start aligned and
straight. } \KatakanaTraining{ku}

% ---------------------------------------------------------------------------
\subsection{\jtl{ke}}\jsubsec{\jquotesingleja{ケ}} \label{sec:KatakanaKe}

\KatakanaHeader{ke}{ The \jquotesingleja{ケ} is written with 3 strokes and the
first stroke is similar to the \jquotesingleja{ク}. The second stroke is
aligned and straight.  While the last stroke is a curve.  }
\KatakanaTraining{ke}

% ---------------------------------------------------------------------------
\subsection{\jtl{ko}}\jsubsec{\jquotesingleja{コ}} \label{sec:KatakanaKo}

\KatakanaHeader{ko}{ This character is almost a geometric figure composed out
of two strokes. However unless in European languages this are only 2 strokes
and not 3. The first stroke is the longest one and done similar with all
{漢字}. } \KatakanaTraining{ko}

% ---------------------------------------------------------------------------
\subsection{\jtl{ka} Row Training}\jsubsec{片仮名カ行練習}

\Padding
\begin{longtable}[c]{p{2cm}p{1.5cm}p{1.5cm}p{3cm}p{7cm}}
\textit{Katakana}&\textit{Rōmaji}&\textit{Original}&\textit{Remark}&Origin\\\hline
カキ  &kaki &kaki &柿 persimon&Japanese\\
ケア  &kea  &care &          &English\\
ケイ  &kei  &K    &the letter&English\\
\end{longtable}

\KanaSimpleTraining{Katakana to Rōmaji}{
\Transcribe{1.}{カキ}{}{persimmon}
\Transcribe{2.}{ココア}{}{cocoa}
\Transcribe{3.}{ケア}{}{care}
\Transcribe{4.}{コア}{}{core}
\Transcribe{5.}{ケーキ}{}{cake}
%\Transcribe{6.}{ケイ}{}{K (the letter)}
}

\KanaSimpleTraining{Rōmaji to Katakana}{
\Transcribe{1.}{kokoa}{}{cocoa}
\Transcribe{2.}{k$\overline{\mbox{e}}$ki}{}{cake}
\Transcribe{3.}{kea}{}{care}
\Transcribe{4.}{koa}{}{core}
\Transcribe{5.}{kaki}{}{persimmon}
%\Transcribe{6.}{kei}{}{K (the letter)}
}

\newpage

\Padding
%\begin{longtable}[c]{p{2cm}p{2cm}p{3cm}p{6cm}p{2cm}}
\begin{longtable}[c]{p{2cm}p{2.0cm}p{3.5cm}p{4cm}p{2.5cm}}
\textit{Katakana}&\textit{Rōmaji}&\textit{Original}&\textit{Remark}&Origin\\\hline
コア  &koa  &core &          &English\\
ココア&kokoa&cocoa& hot chocolate &English, from metathesis of Spanish cacao, from Nahuatl cacahuatl\\
ケーキ&kēki &cake &          &English\\
\end{longtable}


\KanaSimpleTraining{English to Rōmaji}{
\Transcribe{1.}{persimon}{}{}
\Transcribe{2.}{cocoa}{}{}
\Transcribe{3.}{care}{}{}
\Transcribe{4.}{core}{}{}
\Transcribe{5.}{K (the letter)}{}{}
%\Transcribe{6.}{cake}{}{}
}

\KanaSimpleTraining{English to Katakana}{
\Transcribe{1.}{cocoa}{}{}
\Transcribe{2.}{cake}{}{}
\Transcribe{3.}{care}{}{}
\Transcribe{4.}{persimon}{}{}
\Transcribe{5.}{K (the letter)}{}{}
%\Transcribe{6.}{core}{}{}
}

\newpage
