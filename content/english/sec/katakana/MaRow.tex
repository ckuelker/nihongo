% ---------------------------------------------------------------------------
\section{Katakana \jtl{ma} Row}\jsec{片仮名マ行}\label{sec:KatakanaMaRow}

\Krow{marow}{ma}{mi}{mu}{me}{mo}

\label{letter:ma}\KLETTER{ma} The  \textbf{katakana} \jquotesingleja{マ} is
pronounced \jtl{ma} and derives from the
\hyperref[sec:PhoneticCharacter]{phonetic character}s \jquotesingleja{末} upper
two parallel horizontal strokes.  A \hyperref[sec:Dakuten]{dakuten} or
\hyperref[sec:Handakuten]{handakuten} version do not exist.

\label{letter:mi}\KLETTER{mi} The  \textbf{katakana} \jquotesingleja{ミ} is
pronounced \jtl{mi} and derives from the
\hyperref[sec:PhoneticCharacter]{phonetic character} \jquotesingleja{三}.  A
\hyperref[sec:Dakuten]{dakuten} or \hyperref[sec:Handakuten]{handakuten}
version do not exist.

\label{letter:mu}\KLETTER{mu} The  \textbf{katakana} \jquotesingleja{ム} is
pronounced \jtl{mu} and derives from the
\hyperref[sec:PhoneticCharacter]{phonetic character}s \jquotesingleja{牟 }
upper part.  A \hyperref[sec:Dakuten]{dakuten} or
\hyperref[sec:Handakuten]{handakuten} version do not exist.

\newpage

\label{letter:me}\KLETTER{me} The  \textbf{katakana} \jquotesingleja{メ} is
pronounced \jtl{me} and derives from the
\hyperref[sec:PhoneticCharacter]{phonetic character}s \jquotesingleja{女}
ilower right part.  A \hyperref[sec:Dakuten]{dakuten} or
\hyperref[sec:Handakuten]{handakuten} version do not exist.

\Note{Note}{%

The characters \hyperref[letter:no]{\jquotesingleja{ノ}},
\hyperref[letter:me]{\jquotesingleja{メ}} and
\hyperref[letter:nu]{\jquotesingleja{ヌ}} are similar and it is easy to make a
mistake. To distinguish \jquotesingleja{メ} it is important to make all strokes
long enough.

}%


\label{letter:mo}\KLETTER{mo} The  \textbf{katakana} \jquotesingleja{モ} is
pronounced \jtl{mo} and derives from the
\hyperref[sec:PhoneticCharacter]{phonetic character}s \jquotesingleja{毛} lower
part exluding the first stroke.  A \hyperref[sec:Dakuten]{dakuten} or
\hyperref[sec:Handakuten]{handakuten} version do not exist.

\newpage

\subsection{\jtl{ma}}\jsubsec{\jquotesingleja{マ}} \label{sec:KatakanaMa}

\KatakanaHeader{ma}{ Katakana \jtl{ma} is written with three strokes.}
\KatakanaTraining{ma}

\subsection{\jtl{mi}}\jsubsec{\jquotesingleja{ミ}} \label{sec:KatakanaMi}

\KatakanaHeader{mi}{ Katakana \jtl{mi} is written with three strokes.}
\KatakanaTraining{mi}

\subsection{\jtl{mu}}\jsubsec{\jquotesingleja{ム}} \label{sec:KatakanaMu}

\KatakanaHeader{mu}{ Katakana \jtl{mu} is written with three strokes.}
\KatakanaTraining{mu}

\subsection{\jtl{me}}\jsubsec{\jquotesingleja{メ}} \label{sec:KatakanaMe}

\KatakanaHeader{me}{ Katakana \jtl{me} is written with three strokes.}
\KatakanaTraining{me}

\subsection{\jtl{mo}}\jsubsec{\jquotesingleja{モ}} \label{sec:KatakanaMo}

\KatakanaHeader{mo}{ Katakana \jtl{mo} is written with three strokes.}
\KatakanaTraining{mo}

\subsection{\jtl{ma} Row Training}\jsubsec{片仮名マ行練習}
\Padding
\begin{longtable}[c]{p{2cm}p{1.5cm}p{2.5cm}p{3cm}p{5cm}}
\textit{Katakana}&\textit{Rōmaji}&\textit{Original}&\textit{Remark}&Origin\\\hline
テーマ  &tēma    &Thema                 &theme                 &German\\
ママ    &mama    &mamá                  &mom                   &Spanish\\
ホーム  &hōmu    &(plat)form            &railway platform      &English\\
\end{longtable}

%シーエム        shīemu    C.M. (Commercial Message)       television commercial   English
%アニメ          anime     anima(tion)                  animated cartoons or films English

\KanaSimpleTraining{Katakana to Rōmaji}{
\Transcribe{1.}{テーマ}{}{theme}
\Transcribe{2.}{ママ}{}{mom}
\Transcribe{3.}{ホーム}{}{railway platform }
\Transcribe{4.}{アメフト}{}{American football}
\Transcribe{5.}{ハモる}{}{to harmonize (singing)}
%\Transcribe{6.}{マスコミ}{}{mass media}
}

\KanaSimpleTraining{Rōmaji to Katakana}{
\Transcribe{1.}{mama}{}{mom}
\Transcribe{2.}{tēma}{}{theme}
\Transcribe{3.}{amefuto}{}{American football}
\Transcribe{4.}{masukomi}{}{mass media}
\Transcribe{5.}{hōmu}{}{railway platform }
%\Transcribe{6.}{hamoru}{}{to harmonize (singing)}
}

\newpage
\Padding
\begin{longtable}[c]{p{2cm}p{2cm}p{4cm}p{4cm}p{3cm}}
\textit{Katakana}&\textit{Rōmaji}&\textit{Original}&\textit{Remark}&Origin\\\hline
アメフト&amefuto &Ame(rican) foot(ball) &American football     &English\\
ハモる  &hamoru  &harmo(ny) + -ru       &to harmonize (singing)&English, Japanese\\
マスコミ&masukomi&mass communication    &mass media            &English\\
\end{longtable}
\KanaSimpleTraining{English to Rōmaji}{
\Transcribe{1.}{theme}{}{}
\Transcribe{2.}{American football}{}{}
\Transcribe{2.}{mom}{}{}
\Transcribe{3.}{to harmonize (singing)}{}{}
\Transcribe{4.}{railway platform }{}{}
%\Transcribe{5.}{mass media}{}{}
}

\KanaSimpleTraining{English to Katakana}{
\Transcribe{1.}{American football}{}{}
\Transcribe{2.}{mom}{}{}
\Transcribe{3.}{railway platform }{}{}
\Transcribe{4.}{theme}{}{}
\Transcribe{5.}{mass media}{}{}
%\Transcribe{6.}{to harmonize (singing)}{}{}
}

\newpage
