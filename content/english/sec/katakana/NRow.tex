% ---------------------------------------------------------------------------
\section{Katakana \jtl{n} Row}\jsec{片仮名ン行}\label{sec:KatakanaNrow}

\Krow{nrow}{n}{s}{s}{s}{s}

\KLETTER{n} The  \textbf{katakana} \jquotesingleja{ン} is pronounced  \jtl{n}
and  derives from the \hyperref[sec:PhoneticCharacter]{phonetic character}s
\jquotesingleja{尓} upper part.  A \hyperref[sec:Dakuten]{dakuten} or
\hyperref[sec:Handakuten]{handakuten} version do not exist.

The \hyperref[sec:Kana]{kana} \jquotesingleja{ン}  is the only Japanese
character which do not end\footnote{%

In some cases the ending of other \hyperref[sec:Kana]{kana} like
\jquotesingleja{す} in the word {です} for example is not pronounced.

}%

in a vowel. The \hyperref[sec:Kana]{kana} \jquotesingleja{む} or
\jquotesingleja{ム} with the sound \jtl{mu} was originally\footnote{%

The character \jquotesingleja{ん} was an exceptional character (Hentaigana)
used for \jtl{n} and \jtl{mu} and was declared obsolete in 1900.

}%

used for the \jtl{n} sound and become an official character in 1900.

The \jquotesingleja{ん} character is the only Japanese letter which can not be
used\footnote{An exception are the Ryukyu languages. For example \jtl{nnsu} as
ンース (Ryukyu: miso) } to started a word. However it is possible to start
foreign words with the \jquotesingleja{ン} character. For example Ngorongoro as
ンゴロンゴロ.

In some computer systems {(漢字片仮名変換)} {【かんじかたかなへんかん】} it
is needed to press 'nn' (2x 'n') to get a single \jquotesingleja{ん} or
\jquotesingleja{ン}.

On the other hand, see the following table for notation of 'n' and 'nn':
\bigskip

\Note{Note}{
\begin{center}
\begin{tabular}{ccc}
\textit{Rōmaji}&\textit{Hiragana}&\textit{Katakana}\\
n     &ん      &ン\\
nn    &んん    &ンン\\
nh    &んー    &ンー\\
\end{tabular}
\end{center}
}

%\Note{Note}{Please see section \nameref{subsec:SoRiNAmbiguity} for the explanation
%how to write and distinguish \jtl{so}, \jtl{n} and \jtl{ri}.
%}

% SoRiNAmbiguity
\subsection{|so|, |ri| and |n| Ambiguity} \label{subsec:SoRiNAmbiguity}

The Katakana characters {「ソ」}, {「リ」} and {「ン」} can be difficult to
distinguish. All three are made out of only 2 strokes. And especially |so| and
|n| can be hard to tell. In a sentence of course the context can help a lot.
But what are the rules for this characters to write properly and distinguish?

\bigskip

\begin{center}
\begin{tabular}{|c|c|c|}\hline
\KLETTER{so}&\KLETTER{n}&\KLETTER{ri}\\\hline
\end{tabular}
\end{center}

\CharacterExplanation{soexplanation}{ To write the letter |so| it is important
to align both lines \textbf{horizontally} (red line) and to \textbf{non-align}
the ends (blue line) vertically. In this way it is possible to distinguish |so|
from |n|, but not from |ri|. To also distinguish it from |ri| you have to write
the first stroke not horizontally nor vertically (green line).  }

\CharacterExplanation{nexplanation}{ To write the letter |n| it is important to
a align both lines \textbf{vertically} (red line) and to \textbf{non-align} the
ends (blue line). In this way it is possible to distinguish |n| from |so|. If
both lines are aligned there should not be a problem to distinguish it from
|ri|. Usually the angle of the green line is different, but only a small
indicator. }

\CharacterExplanation{riexplanation}{ To write the letter |ri| it is important
to \textbf{align} both of the line beginnings \textbf{horizontally} (red line)
and to make sure that both lines are \textbf{parallel} (green lines). There
should be \textbf{no alignment} on the left side (blue line)
\textbf{vertically}. The difference between |so| and |ri| is that |ri| need to
start with two \textbf{parallel} lines wile |so| should not. Please compare the
left green line for an explanation.  }



\newpage

\subsection{\jtl{n}}\jsubsec{\jquotesingleja{ン}} \label{sec:KatakanaN}
%
\KatakanaHeader{n}{ Katakana \jtl{n} is written with two strokes.}
\KatakanaTraining{n}

%\subsection{\jtl{n} Row Training}\jsubsec{片仮名ン行練習}
%
%\KanaSimpleTraining{Katakana to Rōmaji}{
%\Transcribe{1.}{ココア}{}{cocoa}
%}
%
%\KanaSimpleTraining{Rōmaji to Katakana}{
%\Transcribe{1.}{kokoa}{}{cocoa}
%}
%
%\newpage
%\KanaSimpleTraining{English to Rōmaji}{
%\Transcribe{1.}{persimmon}{}{}
%}

%\KanaSimpleTraining{English to Katakana}{
%\Transcribe{1.}{cocoa}{}{}
%}

%\newpage
