% ---------------------------------------------------------------------------
\section{Katakana \jtl{na} Row}\jsec{片仮名ナ行}\label{sec:KatakanaNaRow}

\Krow{narow}{na}{ni}{nu}{ne}{no}

\label{letter:na}\KLETTER{na} The  \textbf{katakana} \jquotesingleja{ナ} is
pronounced \jtl{na} and  derives from the
\hyperref[sec:PhoneticCharacter]{phonetic character}s \jquotesingleja{奈} upper
left corner part.  A \hyperref[sec:Dakuten]{dakuten} version  or
\hyperref[sec:Handakuten]{handakuten} do not exist.

\label{letter:ni}\KLETTER{ni} The  \textbf{katakana} \jquotesingleja{ニ} is
pronounced \jtl{ni} and  derives from the
\hyperref[sec:PhoneticCharacter]{phonetic character}s \jquotesingleja{奈} upper
right part.  A \hyperref[sec:Dakuten]{dakuten} version  or
\hyperref[sec:Handakuten]{handakuten} do not exist.

\label{letter:nu}\KLETTER{nu} The  \textbf{katakana} \jquotesingleja{ヌ} is
pronounced \jtl{nu} and  derives from the
\hyperref[sec:PhoneticCharacter]{phonetic character}s \jquotesingleja{奴} right
part.  A \hyperref[sec:Dakuten]{dakuten} version  or
\hyperref[sec:Handakuten]{handakuten} do not exist.

\Note{Note}{%

The characters \hyperref[letter:no]{\jquotesingleja{ノ}},
\hyperref[letter:me]{\jquotesingleja{メ}} and
\hyperref[letter:nu]{\jquotesingleja{ヌ}} are similar and it is easy to make a
mistake. To distinguish \jquotesingleja{メ} it is important to make all strokes
long enough.

}%


\newpage

\label{letter:ne}\KLETTER{ne} The  \textbf{katakana} \jquotesingleja{ネ} is
pronounced \jtl{ne} and  derives from the
\hyperref[sec:PhoneticCharacter]{phonetic character}s \jquotesingleja{祢} upper
left  part.  A \hyperref[sec:Dakuten]{dakuten} version  or
\hyperref[sec:Handakuten]{handakuten} do not exist.

\label{letter:no}\KLETTER{no} The  \textbf{katakana} \jquotesingleja{ノ} is
pronounced \jtl{no} and  derives from the
\hyperref[sec:PhoneticCharacter]{phonetic character}s \jquotesingleja{乃} upper
left part.  A \hyperref[sec:Dakuten]{dakuten} version  or
\hyperref[sec:Handakuten]{handakuten} do not exist.

\Note{Note}{%

The characters \hyperref[letter:no]{\jquotesingleja{ノ}},
\hyperref[letter:me]{\jquotesingleja{メ}} and
\hyperref[letter:nu]{\jquotesingleja{ヌ}} are similar and it is easy to make a
mistake. To distinguish \jquotesingleja{メ} it is important to make all strokes
long enough.

}%

\newpage

% ---------------------------------------------------------------------------
\subsection{\jtl{na}}\jsubsec{\jquotesingleja{ナ}} \label{sec:KatakanaNa}

\KatakanaHeader{na}{ Katakana \jtl{na} is written with two strokes.} \KatakanaTraining{na}

% ---------------------------------------------------------------------------
\subsection{\jtl{ni}}\jsubsec{\jquotesingleja{ニ}} \label{sec:KatakanaNi}

\KatakanaHeader{ni}{ Katakana \jtl{ni} is written with two strokes.} \KatakanaTraining{ni}

% ---------------------------------------------------------------------------
\subsection{\jtl{nu}}\jsubsec{\jquotesingleja{ヌ}} \label{sec:KatakanaNu}

\KatakanaHeader{nu}{Katakana \jtl{nu} is written with two strokes.} \KatakanaTraining{nu}

% ---------------------------------------------------------------------------
\subsection{\jtl{ne}}\jsubsec{\jquotesingleja{ネ}} \label{sec:KatakanaNe}

\KatakanaHeader{ne}{Katakana \jtl{ne} is written with three strokes.} \KatakanaTraining{ne}

% ---------------------------------------------------------------------------
\subsection{\jtl{no}}\jsubsec{\jquotesingleja{ノ}} \label{sec:KatakanaNo}

\KatakanaHeader{no}{Katakana \jtl{no} is written with one stroke.} \KatakanaTraining{no}

% ---------------------------------------------------------------------------
\subsection{ \jtl{na} Row Training}\jsubsec{片仮名ナ行練習}
\Padding
\begin{longtable}[c]{p{2cm}p{2cm}p{3cm}p{6cm}p{2cm}}
\textit{Katakana}&\textit{Rōmaji}&\textit{Original}&\textit{Remark}&\textit{Origin}\\\hline
ナース  &nāsu &nurse      &                                        &English\\
ネット  &netto&net(work)  &                                        &English\\
アニス &anisu&anise      &pimpinella anisum                       &\\
\end{longtable}

\KanaSimpleTraining{Katakana to Rōmaji}{
\Transcribe{1.}{ネット}{}{net(work)}
\Transcribe{2.}{ナース}{}{nurse}
\Transcribe{3.}{アニス}{}{anise}
\Transcribe{4.}{ニート}{}{NEET}
\Transcribe{5.}{ナイター}{}{night + -er}
%\Transcribe{6.}{ノート}{}{note}
}

\KanaSimpleTraining{Rōmaji to Katakana}{
\Transcribe{1.}{nōto}{}{note}
\Transcribe{2.}{netto}{}{net(work)}
\Transcribe{3.}{anisu}{}{anise}
\Transcribe{4.}{nāsu}{}{nurse}
\Transcribe{5.}{naitā}{}{night + -er}
%\Transcribe{4.}{ニート}{}{NEET}
}

\newpage
\Padding
\begin{longtable}[c]{p{2cm}p{2cm}p{3cm}p{6cm}p{2cm}}
\textit{Katakana}&\textit{Rōmaji}&\textit{Original}&\textit{Remark}&\textit{Origin}\\\hline
ニート  &nīto &NEET       &Not in Education, Employment or Training&English\\
ナイター&naitā&night + -er&     a night game                       &English\\
ノート  &nōto &note       & note, notebook                         &English\\
\end{longtable}
\KanaSimpleTraining{English to Rōmaji}{
\Transcribe{1.}{anise}{}{}
\Transcribe{2.}{net(work)}{}{}
\Transcribe{3.}{note}{}{}
\Transcribe{4.}{nurse}{}{}
\Transcribe{5.}{NEET}{}{}
%\Transcribe{5.}{night + -er}{}{}
}

\KanaSimpleTraining{English to Katakana}{
\Transcribe{1.}{nurse}{}{}
\Transcribe{2.}{note}{}{}
\Transcribe{3.}{net(work)}{}{}
\Transcribe{4.}{anise}{}{}
\Transcribe{5.}{night + -er}{}{}
%\Transcribe{4.}{NEET}{}{}
}

\newpage
