% ---------------------------------------------------------------------------
\section{Katakana \jtl{ra} row}\jsec{片仮名ラ行}\label{sec:KatakanaRaRow}

\Krow{rarow}{ra}{ri}{ru}{re}{ro}

\KLETTER{ra} The  \textbf{katakana} \jquotesingleja{ラ} is pronounced  \jtl{ra}
(flapped 'r') and  derives from the \hyperref[sec:PhoneticCharacter]{phonetic
character}s \jquotesingleja{良} upper right corner part.  A
\hyperref[sec:Dakuten]{dakuten}  or \hyperref[sec:Handakuten]{handakuten}
version  do not exist.

\Note{Note}{

The sound of the Japanese /r/ is  neither a central nor a lateral flap, but may
vary between the two. To an English speaker, its pronunciation varies between a
flapped 'd' (as in American English buddy) and a flapped 'l'.
\href{https://en.wikipedia.org/wiki/Japanese_phonology}{(Wikipedia Japanese
Phonology)}.

}

\KLETTER{ri} The  \textbf{katakana} \jquotesingleja{リ} is pronounced  \jtl{ri}
(flapped 'r') and  derives from the \hyperref[sec:PhoneticCharacter]{phonetic
character}s \jquotesingleja{利}  right site part.  A
\hyperref[sec:Dakuten]{dakuten}  or \hyperref[sec:Handakuten]{handakuten}
version  do not exist.

%\Note{Note}{Please see section \nameref{subsec:SoRiNAmbiguity} for the explanation
%how to write and distinguish \jtl{so}, \jtl{n} and \jtl{ri}.
%}

\KLETTER{ru} The  \textbf{katakana} \jquotesingleja{ル} is pronounced  \jtl{ru}
(flapped 'r') and  derives from the \hyperref[sec:PhoneticCharacter]{phonetic
character}s \jquotesingleja{流} lower left corner part.  A
\hyperref[sec:Dakuten]{dakuten}  or \hyperref[sec:Handakuten]{handakuten}
version  do not exist.

\KLETTER{re} The  \textbf{katakana} \jquotesingleja{レ} is pronounced  \jtl{re}
(flapped 'r') and  derives from the \hyperref[sec:PhoneticCharacter]{phonetic
character}s \jquotesingleja{礼} upper right site part.  A
\hyperref[sec:Dakuten]{dakuten}  or \hyperref[sec:Handakuten]{handakuten}
version  do not exist.

\KLETTER{ro} The  \textbf{katakana} \jquotesingleja{ロ} is pronounced  \jtl{ro}
(flapped 'r') and  derives from the \hyperref[sec:PhoneticCharacter]{phonetic
character}s \jquotesingleja{呂} upper part.  A \hyperref[sec:Dakuten]{dakuten}
or \hyperref[sec:Handakuten]{handakuten} version  do not exist.

% SoRiNAmbiguity
\subsection{\jtl{so}, \jtl{ri} and \jtl{n} Ambiguity} \label{subsec:SoRiNAmbiguity}

The Katakana characters \jquotesingleja{ソ}, \jquotesingleja{リ} and
\jquotesingleja{ン} can be difficult to distinguish. All three are made out of
only 2 strokes. And especially \jtl{so} and \jtl{n} can be hard to tell. In a
sentence of course the context can help a lot.  But what are the rules for this
characters to write properly and distinguish?

\bigskip

\begin{figure}[H]
\begin{center}
\begin{tabular}{|c|c|c|}\hline
\KLETTER{so}&\KLETTER{n}&\KLETTER{ri}\\\hline
\end{tabular}
\end{center}
\caption{\jtl{so}, \jtl{ri} and \jtl{n} ambiguity}
\label{fig:SoRiAndNAmbiguity}
\end{figure}

\CharacterExplanation{soexplanation}{

To write the letter \jtl{so} it is important to align both lines
\textbf{horizontally} (red line) and to \textbf{non-align} the ends (blue line)
vertically. In this way it is possible to distinguish \jtl{so} from \jtl{n},
but not from \jtl{ri}. To also distinguish it from \jtl{ri} you have to write
the first stroke not horizontally nor vertically (green line).

}

\CharacterExplanation{nexplanation}{

To write the letter \jtl{n} it is important to a align both lines
\textbf{vertically} (red line) and to \textbf{non-align} the ends (blue line).
In this way it is possible to distinguish \jtl{n} from \jtl{so}. If both lines
are aligned there should not be a problem to distinguish it from \jtl{ri}.
Usually the angle of the green line is different, but only a small indicator.

}

\CharacterExplanation{riexplanation}{

To write the letter \jtl{ri} it is important to \textbf{align} both of the line
beginnings \textbf{horizontally} (red line) and to make sure that both lines
are \textbf{parallel} (green lines). There should be \textbf{no alignment} on
the left side (blue line) \textbf{vertically}. The difference between \jtl{so}
and \jtl{ri} is that \jtl{ri} need to start with two \textbf{parallel} lines
wile \jtl{so} should not. Please compare the left green line for an
explanation.

}



\newpage

% ラリルレロ
\subsection{\jtl{ra}}\jsubsec{\jquotesingleja{ラ}} \label{sec:KatakanaRa}

\KatakanaHeader{ra}{ Katakana \jtl{ra} is written with two strokes.} \KatakanaTraining{ra}

\subsection{\jtl{ri}}\jsubsec{\jquotesingleja{リ}} \label{sec:KatakanaRi}

\KatakanaHeader{ri}{ Katakana \jtl{ri} is written with two strokes.} \KatakanaTraining{ri}

\subsection{\jtl{ru}}\jsubsec{\jquotesingleja{ル}} \label{sec:KatakanaRu}

\KatakanaHeader{ru}{ Katakana \jtl{ru} is written with two strokes.} \KatakanaTraining{ru}

\subsection{\jtl{re}}\jsubsec{\jquotesingleja{レ}} \label{sec:KatakanaRe}

\KatakanaHeader{re}{ Katakana \jtl{re} is written with one stroke.} \KatakanaTraining{re}

\subsection{\jtl{ro}}\jsubsec{\jquotesingleja{ロ}} \label{sec:KatakanaRa}

\KatakanaHeader{ro}{ Katakana \jtl{ro} is written with three strokes.} \KatakanaTraining{ro}

\subsection{\jtl{ra} Row Training}\jsubsec{片仮名ラ行練習}
\Padding
\begin{longtable}[c]{p{2cm}p{1.5cm}p{2.5cm}p{3cm}p{6cm}}
\textit{Katakana}&\textit{Rōmaji}&\textit{Original}&\textit{Remark}&Origin\\\hline
ヒステリー  &hisuterī  &Hysterie      &hysteria               &German\\
メール      &mēru      &e-mail        &electronic mail        &English\\
イラスト    &irasuto   &illust(ration)&illustration           &English\\
\end{longtable}
\Padding

%ロスタイム      rosutaimu       loss time       added time, additional time     English


\KanaSimpleTraining{Katakana to Rōmaji}{
\Transcribe{1.}{ヒステリー}{}{hysteria}
\Transcribe{2.}{メール}{}{e-mail}
\Transcribe{3.}{イラスト}{}{illustration}
\Transcribe{4.}{プレイガイド}{}{play guide}
\Transcribe{5.}{ノイローゼ}{}{neurosis}
\Transcribe{6.}{アロエ}{}{aloe}
}

\KanaSimpleTraining{Rōmaji to Katakana}{
\Transcribe{1.}{mēru}{}{e-mail}
%\Transcribe{2.}{irasuto}{}{illustration}
\Transcribe{3.}{hisuterī}{}{hysteria}
\Transcribe{4.}{noirōze}{}{neurosis}
\Transcribe{5.}{pureigaido}{}{play guide}
\Transcribe{6.}{aroe}{}{aloe}
}

\newpage
\begin{longtable}[c]{p{2.5cm}p{2.5cm}p{2.5cm}p{5.5cm}p{2cm}}
\textit{Katakana}&\textit{Rōmaji}&\textit{Original}&\textit{Remark}&Origin\\\hline
プレイガイド&pureigaido&play + guide  &(theater) ticket agency&English\\
ノイローゼ  &noirōze   &Neurose       &neurosis               &German\\
アロエ      &aroe      &Aloë          &aloe                   &Dutch\\
\end{longtable}
\KanaSimpleTraining{English to Rōmaji}{
%\Transcribe{1.}{e-mail}{}{}
\Transcribe{2.}{play guide}{}{}
\Transcribe{3.}{hysteria}{}{}
\Transcribe{4.}{neurosis}{}{}
\Transcribe{5.}{illustration}{}{}
\Transcribe{6.}{aloe}{}{}
}

\KanaSimpleTraining{English to Katakana}{
\Transcribe{1.}{illustration}{}{}
%\Transcribe{2.}{play guide}{}{}
\Transcribe{3.}{aloe}{}{}
\Transcribe{4.}{neurosis}{}{}
\Transcribe{5.}{hysteria}{}{}
\Transcribe{6.}{mēru}{}{e-mail}
}

\newpage
