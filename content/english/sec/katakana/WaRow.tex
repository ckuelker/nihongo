% ---------------------------------------------------------------------------
\section{Katakana \jtl{wa} Row}\jsec{片仮名ワ行}\label{sec:KatakanaWaRow}

\Krow{warow}{wa}{s}{s}{s}{wo}

\label{letter:wa}\KLETTER{wa} The \textbf{katakana} \jquotesingleja{ワ} is
pronounced \jtl{wa} and derives from the
\hyperref[sec:PhoneticCharacter]{phonetic characters} \jquotesingleja{和} right
site part. A \hyperref[sec:Dakuten]{dakuten} do, a
\hyperref[sec:Handakuten]{handakuten} do not exist. The
\hyperref[sec:Dakuten]{dakuten} version \jquotesingleja{ヷ} was introduced
recently and is not well understood especially by older people who would read
it as \jtl{ba}.

\begin{table}[H]
\begin{center}\begin{tabular}{lll}
\textit{Rōmaji}&\textit{Katakana}&\textit{Alternatives}\\
\jtl{wa}           &ワ               &\\
\jtl{va}           &ヷ               &\small ヴァ、ヴぁ、う゛ぁ\\
\jtl{wā}           &ワー             &\\
\jtl{vā}           &ヷー             &\small ヴァア、ヴぁア、う゛ぁあ\\
\end{tabular}\end{center}
\caption{\jtl{wa} or \jtl{va} with alternatives}
\label{tab:WaOrVaWithAlternatives}
\end{table}


\label{letter:wo}\KLETTER{wo} The \textbf{katakana} \jquotesingleja{ヲ} is
pronounced \jtl{wo} and derives from the
\hyperref[sec:PhoneticCharacter]{phonetic characters} \jquotesingleja{乎}. A
\hyperref[sec:Dakuten]{dakuten} do and a \hyperref[sec:Handakuten]{handakuten}
do not exist.

\begin{table}[H]
\begin{center}\begin{tabular}{lll}
\textit{Rōmaji}&\textit{Katakana}&\textit{Alternatives}\\
\jtl{wo}           &ヲ               &\\
\jtl{vo}           &ヺ               &seldomly used, more often: ヴォ\\
\end{tabular}\end{center}
\caption{\jtl{wo} or \jtl{vo} with alternatives}
\label{tab:WoOrVoWithAlternatives}
\end{table}

\Note{Note}{%

It is safe to skip learning this character. See
\nameref{subsec:SeldomlyUsedKatakana} on page
\pageref{subsec:SeldomlyUsedKatakana} for a detailed description.

}

\newpage
% UFuWaSimilarity
\subsection{\jtl{u}, \jtl{fu} and \jtl{wa} Similarity} \label{subsec:UFuWaSimilarity}

The Katakana characters \jquotesingleja{ウ}, \jquotesingleja{フ} and
\jquotesingleja{ワ} can be easily distinguished. All three have a different
stroke count. However the shape is similar. Therefore they can be mistaken.
Especially when they have no context.

\bigskip

\begin{figure}[H]
\begin{center}
\begin{tabular}{|c|c|c|}\hline
\KLETTER{u}&\KLETTER{fu}&\KLETTER{wa}\\\hline
\end{tabular}
\end{center}
\caption{\jtl{u}, \jtl{fu} and \jtl{wa} similarity}
\label{fig:UuFuAndWaSimilarity}
\end{figure}




% VaAmbiguity
\subsection{\jtl{va}  Ambiguity} \label{subsec:VaAmbiguity}

The \jtl{va} ambiguity is mostly not a letter difficulty, it is a sound
representation problem. The Rōmaji \jtl{va} can be written in many different
ways and that is true for some other characters of the \jtl{wa} row too. The
lack of standardization and consistency make it hard to guess how one should
write a certain word with this sound.

\bigskip

\begin{table}[H]
\begin{center}
\begin{tabular}{p{4cm}p{5cm}p{5cm}p{1cm}}
%\KLETTER{so}&\KLETTER{n}&\KLETTER{ri}\\\hline
\ifthenelse{\equal{hiragana}{\jtopic}}{%
\textit{Hiragana}     &ゔぁ            &わ゙                &ば           \\\hline%
\textit{Romaji}       &\texttt{va}     &\texttt{va}       &\texttt{ba}  \\%\hline
\textit{Input}        &\texttt{va}     &                  &\texttt{ba}  \\%\hline
}{}
\ifthenelse{\equal{katakana}{\jtopic}}{%
\textit{Karakana}     &\textbf{ヴァ}   &ヷ                &バ           \\%
\textit{Romaji}       &\texttt{va}     &\texttt{va}       &\texttt{ba}  \\%\hline
\textit{Input}        &\texttt{va,ba}  &                  &\texttt{ba}  \\%\hline
}{}
\textit{Pronunciation}&usually \jtl{va}     &usually \jtl{va}     &\jtl{ba}\\%\hline
                      &some people \jtl{ba} &older people \jtl{ba}&             \\%\hline
                      &older people \jtl{ba}&                 &             \\%\hline
\end{tabular}
\end{center}
\caption{\jtl{va} ambiguity}
\label{tab:VaAmbiguity}
\end{table}

\footnotesize
\begin{verbatim}
Unicode:
ヷ U+30F7, &#12535 KATAKANA LETTER VA; Composition: ワ [U+30EF] + ◌゙ [U+3099]
ヴ U+30F4, &#12532 KATAKANA LETTER VU; Composition: ウ [U+30A6] + ◌゙ [U+3099]
ァ U+30A1, &#12449 KATAKANA LETTER SMALL;
\end{verbatim}



% ヴィオロン    = violin
% ヴァイオリン  = violin
% バイオリン    = violin

\newpage

%ワヲ
\subsection{\jtl{wa}}\jsubsec{\jquotesingleja{ワ}}\label{sec:KatakanaWa}

\KatakanaHeader{wa}{ Katakana \jtl{wa} is written with two strokes.}
\KatakanaTraining{wa}

\subsection{\jtl{wo}}\jsubsec{\jquotesingleja{ヲ}}\label{sec:KatakanaWo}

\KatakanaHeader{wo}{ Katakana \jtl{wo} is written with two strokes. }
\KatakanaTraining{wo}

\subsection{\jtl{wa} Row Training}\jsubsec{片仮名ワ行練習}

\Padding
\begin{longtable}[c]{p{3cm}p{2.5cm}p{3.5cm}p{5cm}p{2cm}}
\textit{Katakana}&\textit{Rōmaji}&\textit{Original}&\textit{Remark}&Origin\\\hline
ホワイトデー&howaitodē &White + Day       &White Day, March 14th &English\\
ワープロ    &wāpuro    &wor(d) pro(cessor)&word processor        &English\\
\end{longtable}
\KanaSimpleTraining{Katakana to Rōmaji}{
\Transcribe{1.}{ホワイトデー}{}{White + Day}
\Transcribe{2.}{ワープロ}{}{word processor}
\Transcribe{3.}{ワイシャツ}{}{dress shirt}
\Transcribe{4.}{ヷ}{}{}
\Transcribe{5.}{ヴァルヴ}{}{valve}
}

\KanaSimpleTraining{Rōmaji to Katakana}{
\Transcribe{1.}{wāpuro}{}{word processor}
\Transcribe{2.}{howaitodē}{}{White + Day}
\Transcribe{3.}{va}{}{}
\Transcribe{4.}{waishatsu}{}{dress shirt}
\Transcribe{5.}{varuvu}{}{valve}
}

\newpage
\Padding
\begin{longtable}[c]{p{2cm}p{2.5cm}p{4.5cm}p{3cm}p{3cm}}
\textit{Katakana}&\textit{Rōmaji}&\textit{Original}&\textit{Remark}&Origin\\\hline
ワイシャツ  &waishatsu &Y shirt (from "white shirt")&dress shirt           &English\\
ヷ          &va        &                            &different writing     &\\
ヴァルヴ    &varuvu    &valve                       &                      &English\\
\end{longtable}

\KanaSimpleTraining{English to Rōmaji}{
\Transcribe{1.}{dress shirt}{}{}
\Transcribe{2.}{White + Day}{}{}
\Transcribe{3.}{valve}{}{}
\Transcribe{4.}{word processor}{}{}
\Transcribe{5.}{va}{}{}
}

\KanaSimpleTraining{English to Katakana}{
\Transcribe{1.}{valve}{}{}
\Transcribe{2.}{White + Day}{}{}
\Transcribe{3.}{dress shirt}{}{}
\Transcribe{4.}{word processor}{}{}
\Transcribe{5.}{va}{}{}
}

\newpage
