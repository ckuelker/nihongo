% ---------------------------------------------------------------------------
\section{Katakana \jtl{ya} Row}\jsec{片仮名ヤ行}\label{sec:KatakanaYaRow}

\Krow{yarow}{ya}{s}{yu}{s}{yo}

\KLETTER{ya} The  \textbf{katakana} \jquotesingleja{ヤ} is pronounced  \jtl{ya}
and  derives from the \hyperref[sec:PhoneticCharacter]{phonetic character}s
\jquotesingleja{也} upper left part.  A \hyperref[sec:Dakuten]{dakuten} or
\hyperref[sec:Handakuten]{handakuten} version do not exist.

\KLETTER{yu} The  \textbf{katakana} \jquotesingleja{ユ} is pronounced  \jtl{yu}
and  derives from the \hyperref[sec:PhoneticCharacter]{phonetic character}s
\jquotesingleja{由 } lower middle part.  A \hyperref[sec:Dakuten]{dakuten} or
\hyperref[sec:Handakuten]{handakuten} version do not exist.

\KLETTER{yo} The  \textbf{katakana} \jquotesingleja{ヨ} is pronounced  \jtl{yo}
and  derives from the \hyperref[sec:PhoneticCharacter]{phonetic character}s
\jquotesingleja{與} upper right part.  A \hyperref[sec:Dakuten]{dakuten} or
\hyperref[sec:Handakuten]{handakuten} version do not exist.

\newpage
\subsection{Yōon}\jsubsec{拗音}

All characters from the \jquotesingleja{ヤ} row can be used in it's smaller form to crate
combined phonetics Yōon ({拗音} {【ようおん】}).

\begin{center} \Large
\begin{tabular}{llll}
      &ャ  &ュ  &ョ  \\
k - キ&キャ&キュ&キョ\\
s - シ&シャ&シュ&ショ\\
c - チ&チャ&チュ&チョ\\
n - ニ&ニャ&ニュ&ニョ\\
h - ヒ&ヒャ&ヒュ&ヒョ\\
m - ミ&ミャ&ミュ&ミョ\\
r - リ&リャ&リュ&リョ\\
\end{tabular}

Dakuten

\begin{tabular}{llll}
g - ギ&ギャ&ギュ&ギョ\\
j - ジ&ジャ&ジュ&ジョ\\
b - ビ&ビャ&ビュ&ビョ\\
\end{tabular}

Handakuten

\begin{tabular}{llll}
p - ピ&ピャ&ピュ&ピョ\\
\end{tabular}
\end{center}


\newpage

%ヤユヨ
\subsection{\jtl{sa}}\jsubsec{\jquotesingleja{ヤ}}\label{sec:KatakanaYa}

\KatakanaHeader{ya}{ Katakana \jtl{ya} is written with two strokes.} \KatakanaTraining{ya}

\subsection{\jtl{yu}}\jsubsec{\jquotesingleja{ユ}}\label{sec:KatakanaYu}

\KatakanaHeader{yu}{ Katakana \jtl{yu} is written with two strokes.} \KatakanaTraining{yu}

\subsection{\jtl{yo}}\jsubsec{\jquotesingleja{ヨ}}\label{sec:KatakanaYo}

\KatakanaHeader{yo}{Katakana \jtl{yo} is written with three strokes.} \KatakanaTraining{yo}

\subsection{\jtl{ya} Row Training}\jsubsec{片仮名ヤ行練習}
\Padding
\begin{longtable}[c]{p{2cm}p{1.5cm}p{2.5cm}p{3cm}p{6cm}}
\textit{Katakana}&\textit{Rōmaji}&\textit{Original}&\textit{Remark}&Origin\\\hline
イヤー              &iyā          &ear, year    &                         &English\\
ユーザー            &yūzā         &user         &                         &English\\
ヨード              &yōdo         &Jod          &iodine                   &German\\
\end{longtable}

\KanaSimpleTraining{Katakana to Rōmaji}{
\Transcribe{1.}{イヤー}{}{ear, year}
\Transcribe{2.}{ユーザー}{}{user}
\Transcribe{3.}{ヨード}{}{iodine}
\Transcribe{4.}{ユニットバス}{}{unit bath}
\Transcribe{5.}{ヨット}{}{sailboat}
\Transcribe{6.}{ニュー・イヤーズ・イブ}{}{new years eve}
}

\KanaSimpleTraining{Rōmaji to Katakana}{
\Transcribe{1.}{yūzā}{}{user}
\Transcribe{2.}{iyā}{}{ear, year}
\Transcribe{3.}{yunittobasu}{}{unit bath}
\Transcribe{4.}{yōdo}{}{iodine}
\Transcribe{5.}{nyū iyāzu ibu}{}{new years eve}
%\Transcribe{6.}{yotto}{}{sailboat}
}

\newpage
\Padding
\begin{longtable}[c]{p{3.9cm}p{2.4cm}p{2.6cm}p{4.8cm}p{1.3cm}}
\textit{Katakana}&\textit{Rōmaji}&\textit{Original}&\textit{Remark}&Origin\\\hline
ユニットバス        &yunittobasu  &unit bath    &prefabricated module bath&English\\
ヨット              &yotto        &yacht        &sailboat                 &English\\
ニュー・イヤーズ・イブ&nyū iyāzu ibu&new years eve&                         &English\\
\end{longtable}

\KanaSimpleTraining{English to Rōmaji}{
\Transcribe{3.}{unit bath}{}{}
\Transcribe{4.}{iodine}{}{}
\Transcribe{5.}{new years eve}{}{}
\Transcribe{1.}{user}{}{}
\Transcribe{6.}{sailboat}{}{}
%\Transcribe{2.}{ear, year}{}{}
}

\KanaSimpleTraining{English to Katakana}{
\Transcribe{1.}{sailboat}{}{}
\Transcribe{2.}{iodine}{}{}
\Transcribe{3.}{new years eve}{}{}
\Transcribe{4.}{unit bath}{}{}
\Transcribe{5.}{user}{}{}
%\Transcribe{2.}{ear, year}{}{}
}

\newpage
