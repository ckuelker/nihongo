% UFuWaSimilarity
\subsection{\jtl{u}, \jtl{fu} and \jtl{wa} Similarity} \label{subsec:UFuWaSimilarity}

The Katakana characters \jquotesingleja{ウ}, \jquotesingleja{フ} and
\jquotesingleja{ワ} can be easily distinguished. All three have a different
stroke count. However the shape is similar. Therefore they can be mistaken.
Especially when they have no context.

\bigskip

\begin{figure}[H]
\begin{center}
\begin{tabular}{|c|c|c|}\hline
\KLETTER{u}&\KLETTER{fu}&\KLETTER{wa}\\\hline
\end{tabular}
\end{center}
\caption{\jtl{u}, \jtl{fu} and \jtl{wa} similarity}
\label{fig:UuFuAndWaSimilarity}
\end{figure}


\begin{tikzpicture}
    % Draw the Katakana character ウ
    \node[scale=5] at (-0.5,0) {\textcolor{black}{ウ}};
    \node[scale=5] at (2,0) {\textcolor{black}{→}};
    %\node[scale=5, rotate=55 ] at (10,0) {\textcolor{black}{ウ}};
    %\node[scale=5, rotate=145 ] at (5,0) {\textcolor{black}{ウ}};
    \node[scale=5, rotate=205 ] at (4.5,0) {\textcolor{black}{ウ}};
    \node[scale=5] at (7,0) {\textcolor{black}{→}};
    \draw[thick, dashed, red] (3.5,-0.8) -- (5.5,-0.8);
    % Overlay a sideways blue 'U'
    %\node[scale=5, rotate=145, text=blue] at (0,0) {U};
    \node[scale=5, text=orange] at (9.5,0) {U};
\end{tikzpicture}

ウ (U): This character looks like a 'U' turned sideways. Imagine it as a 'U'
that's been tipped over. The small dash on the top can be seen as a little
support to keep the 'U' from falling down.

\begin{tikzpicture}
    % Draw the Katakana character フ
    \node[scale=5] at (-0.5,0) {\textcolor{black}{フ}};

    % Overlay the flag pole
    \draw[ultra thick, orange] (-1,0.8) -- (-1,-2); % flag pole
\end{tikzpicture}

フ (Fu): Think of this character as a flag blowing in the wind. The two lines
are the edges of the flag, fluttering to the right. The little dash on the left
is the flagpole. This image can help you remember the sound 'Fu' as it sort of
mimics the sound of a flag flapping in the wind.

% Idea 1 - a bit too complex
%\begin{tikzpicture}
%    % Draw the Katakana character ワ
%    \node[scale=5] at (-0.5,0) {\textcolor{black}{ワ}};
%    \node[scale=5] at (1.5,0) {\textcolor{black}{→}};
%    \node[scale=5, rotate=270] at (3.5,0) {\textcolor{black}{ワ}};
%    \node[scale=5] at (5.5,0) {\textcolor{black}{→}};
%    \node[scale=5, rotate=270] at (7.5,0) {\textcolor{black}{ワ}};
%    \node[scale=5, rotate=270, yscale=-1] at (9.5,0) {\textcolor{black}{ワ}};
%    \node[scale=5] at (11.5,0) {\textcolor{black}{→}};
%    % todo: x=x-0.5
%    \node[scale=5, rotate=270] at (14,0) {\textcolor{black}{ワ}};
%    \node[scale=5, rotate=270, yscale=-1] at (15.5,0) {\textcolor{black}{ワ}};
%    \draw[ultra thick, blue] (13.2,0.3) to[bend right] (16.3,0.3); % smile
%
%    % Overlay the smiley face
%    \draw[ultra thick, red] (-0.5,-0.5) to[bend left] (0.5,-0.5); % smile
%    \draw[ultra thick, blue] (-0.5,-0.5) to[bend right] (0.5,-0.5); % smile
%
%    \draw[ultra thick, red] (0,0) -- (0,0.5); % nose
%\end{tikzpicture}

% Idea 2
\hspace*{10mm}
%\begin{minipage}{\textwidth}
\begin{minipage}{\dimexpr\textwidth-10mm\relax}
\begin{tikzpicture}
    \node[scale=5] at (3.4,0) {\textcolor{black}{ワ}};
    \draw[ultra thick, orange] (1.0,0.7) to[bend right=90] (4.0,0.7); % smile
    \draw[ultra thick, orange] (1.0,0.7) arc (180:360:1.5); % half-circle arc
    \draw[ultra thick, orange] (2.9,0) -- (2.9,0.5); % nose
\end{tikzpicture}
\end{minipage}

ワ (Wa): This character can be seen as a simplified smiley face. The two
strokes combine to make a sort of smile (with the longer stroke being the mouth
and the shorter one, a nose). It's like a happy face saying "Wa!" in delight.

