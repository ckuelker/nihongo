% VaAmbiguity
\subsection{\jtl{va}  Ambiguity} \label{subsec:VaAmbiguity}

The \jtl{va} sound ambiguity is mostly not a letter difficulty, it is a sound
representation problem, as this sound does not naturally occur in traditional
Japanese phonetics. Modern Japanese has adapted by using different methods to
approximate this sound, especially for loanwords and foreign names. The Rōmaji
\jtl{va} can be written in many different ways and that is true for some other
characters of the \jtl{wa} row too. The lack of standardization and consistency
make it hard to guess how one should write a certain word with this sound.

\bigskip

\begin{table}[H]
\begin{center}
\begin{tabular}{p{4cm}p{5cm}p{5cm}p{1cm}}
%\KLETTER{so}&\KLETTER{n}&\KLETTER{ri}\\\hline
\ifthenelse{\equal{hiragana}{\jtopic}}{%
\textit{Hiragana}     &ゔぁ                 &わ゙                       &ば           \\\hline%
\textit{Romaji}       &\texttt{va}          &\texttt{va}              &\texttt{ba}  \\%\hline
\textit{Input}        &\texttt{va}          &                         &\texttt{ba}  \\%\hline
}{}
\ifthenelse{\equal{katakana}{\jtopic}}{%
\textit{Karakana}     &\textbf{ヴァ}        &ヷ                        &バ           \\%
\textit{Romaji}       &\texttt{va}          &\texttt{va}               &\texttt{ba}  \\%\hline
\textit{Input}        &\texttt{va,ba}       &                          &\texttt{ba}  \\%\hline
}{}
\textit{Pronunciation}&usually \jtl{va}     &usually \jtl{va}          &\jtl{ba}     \\%\hline
                      &some people \jtl{ba} &only older people \jtl{ba}&             \\%\hline
                      &older people \jtl{ba}&                          &             \\%\hline
\end{tabular}
\end{center}
\caption{\jtl{va} ambiguity}
\label{tab:VaAmbiguity}
\end{table}

\begin{enumerate}

    \item \textbf{ Using ヴァ (Vu + small a = U with dakuten + small a):}

     ヴァイオリン (Vu + a + i + o + ri + n) - This represents the word
     "violin." The ヴァ combination is used to create the \jtl{va} sound.

    \item \textbf{Using ヷ (Wa with dakuten):}

    This form is much less common and is primarily used in older texts or
    stylistic writings. It's not typically used in standard modern Japanese.

    \item \textbf{Using バ (Ba = ha with dakuten):}

    バニラ (Ba + ni + ra) - This is used for the word "vanilla." In this case,
    the "va" sound is approximated using the \jtl{ba} sound, which is closer
    to the traditional Japanese phonetic system.

\end{enumerate}

\footnotesize
\begin{verbatim}
Unicode:
ヷ U+30F7, &#12535 KATAKANA LETTER VA; Composition: ワ [U+30EF] + ◌゙ [U+3099]
ヴ U+30F4, &#12532 KATAKANA LETTER VU; Composition: ウ [U+30A6] + ◌゙ [U+3099]
ァ U+30A1, &#12449 KATAKANA LETTER SMALL;
\end{verbatim}
