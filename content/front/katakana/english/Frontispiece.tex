    \begin{center}
        \textbf{Back-Cover Text:}
        \begin{tabular}{|l|}\hline
            \begin{minipage}{140mm}\medskip\footnotesize

                The original version of this book was written by
                \textbf{Christian Külker} and was copyrighted 2000-2006,
                2013-2014 and 2020 under the GNU-FDL version 1.2 or any later
                version published by the Free Software Foundation with only the
                back cover text as invariant section. From 2020 on it is
                published under GNU-FDL version 1.2 or any later version
                published by the Free Software Foundation with \textbf{no}
                invariant section.\medskip

                Original PDF:
                \href{https://github.com/ckuelker/nihongo/tree/master/pub/}{https://github.com/ckuelker/nihongo/tree/master/pub}

                Source Code:
                \href{https://github.com/ckuelker/nihongo/}{https://github.com/ckuelker/nihongo}

                Web site:
                \href{https://christian.kuelker.info/nihongo/}{https://christian.kuelker.info/nihongo}

                \flushright  Christian Külker, Bielefeld, \jdate, \texttt{v-\jversion}

                \medskip

            \end{minipage}\\ \hline
        \end{tabular}
    \end{center}
    \bigskip

\footnotesize

\textbf{Changes:}

\texttt{v1.2 2022}: Content was updated, errors corrected, typos and formats
improved. URLs changed from http to https. Changed index from \texttt{multind}
to \texttt{imakeindex}. Tested under Debian 11 Bullseye with Texlive-2022.

\texttt{v1.1 2020}: The invariant section clause of the GNU-FDL was dropped.
Typos, space, grammar and small layout changes have been made.

\texttt{v1.0 2020}: The source code was changed to compile under Debian 10
Buster. Some fonts have been changed in the appendix.

\texttt{v0.9 2014}: Initial publicly released version as Katakana only book.
The title was changed to \textbf{日本語の書き方:片仮名} (English: \textit{The
Japanese Script - Katakana}) and adopted to a self study approach.

\texttt{v0.1 - v0.8}: Published internally as \textbf{日本語を書こう!}
(German: \textit{Lasst uns Japanisch schreiben!}). It was developed as
reference and training book for the language course at the VHS Halle
(Ravensberg) in Germany starting year 2000. It was published 2003, 2004 and
2006 under the GNU FDL.


%    In case you have the original book, please write to
%    \href{mailto:christian.kuelker@cipworx.org}{christian.kuelker@cipworx.org}
%    for hints, suggestions, thanks, complains.

