\section{Gojūonzu}\jsec{五十音図} 
% [o] LABEL
\label{sec:Gojuonzu}
% [o] INDEX DESTINATION (DEF)
\ifor{Gojūonzu}{五十音図}{ごじゅうおんず}{50@50 Laute Tafel}
% [o] INDEX TARGET
\ifor{Kana}{仮名}{かな}{Kana}
\ifor{Iroha}{伊呂波}{いろは}{Iroha}

Traditionally two ways exist to order Japanese characters. One of it is the
\textbf{Gojūonzu} (50 sound table) -  {五十音図}   {【ごじゅうおんず}】, which
is used more often in modern times while the
\hyperref[sec:Iroha]{Iroha}\footnote{A poem with all \hyperref[sec:Kana]{Kana}
letters to remember easily. However it is not standard Japanese anymore why it
would be difficult to suggest to learn. } was more popular in the older times.

The \textbf{Gojūonzu} is a grid of 10 x 5 squares partly filled with
\hyperref[sec:Kana]{Kana}. The roman letter are not part of the
\textbf{Gojūonzu} and are added for the convenience of the learner. 

% アイウエオ
% カキクケコ
% サシスセソ
% タチツテト
% ナニヌネノ
% ハヒフヘホ
% マミムメモ
% ヤユヨ
% ラリルレロ
% ワヲ
% ン
\index{Katakana Gojūonzu}
\index{Katakana}
\index{Gojūon}
\index{片仮名五十音図}
\bigskip
\begin{center}
%\Huge
\Padding
%\begin{tabular}{m{1.0cm}||m{1.0cm}|m{1.0cm}|m{1.0cm}|m{1.0cm}|m{1.0cm}|}
\begin{tabular}{r||c|c|c|c|c|}
             & \textbf{a}& \textbf{i}& \textbf{u}& \textbf{e}& \textbf{o}\\ \hline \hline
\textbf{-}&ア&イ&ウ&エ&オ\\\hline 
\textbf{k}&カ&キ&ク&ケ&コ\\\hline 
\textbf{s}&サ&シ&ス&セ&ソ\\\hline 
\textbf{t}&タ&チ&ツ&テ&ト\\\hline 
\textbf{n}&ナ&ニ&ヌ&ネ&ノ\\\hline 
\textbf{h}&ハ&ヒ&フ&ヘ&ホ\\\hline 
\textbf{m}&マ&ミ&ム&メ&モ\\\hline 
\textbf{y}&ヤ&  &ユ&  &ヨ\\\hline 
\textbf{r}&ラ&リ&ル&レ&ロ\\\hline 
\textbf{w}&ワ&  &  &  &ヲ\\\hline 
\textbf{*}&ン&  &  &  &  \\\hline 
\end{tabular}
\end{center}



The later adopted /n/ was added as one square or in the above example as the
11th line.  Even though there less the 50 letters and more the 50 squares out
of historical reason the name is still \textbf{Gojūonzu}.

For more explanation please read the chapter
\nameref{chap:TheWayToWriteKatakana} and look at the various examples of the
\textbf{Gojūonzu} in the appendix starting with \nameref{chap:KatakanaTables}
on page \pageref{chap:KatakanaTables} up to \pageref{sec:KatakanaMikachanPB}.

