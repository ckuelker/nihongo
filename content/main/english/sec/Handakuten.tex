% ---------------------------------------------------------------------------
\section{Handakuten}\jsec{半濁点} \label{sec:Handakuten}
\ifor{Handakuten}{半濁点}{はんだくてん}{Handakuten}
\ifor{Dakuten}{濁点}{だくてん}{Dakuten}
\ien{circle}
\ija{丸}
\ija{まる}
\ide{Kreis}
\ithree{.@゚}{「゚」}{.@゚}% removed quotes (single quotes OK, double quotes NG)
\ien{|h|} \ide{|h|}
\ien{|p|} \ide{|p|}
\ien{pronunciation shift} \ide{Ausprache Verschiebung}

In Japanese two different {濁点} {【だくてん】} are used. The {濁点}  and  the
{半濁点} {【はんだくてん】} has the marker of a little circle {「゚」} and is
therefore colloquially described as {丸} {【まる】} and indicates when the
pronunciation shifts from |h| to |p|.

