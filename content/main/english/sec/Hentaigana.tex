% +---------------------------------------------------------------------------+
% | Hentaigana.tex                                                            |
% |                                                                           |
% | Describe the usage of Hentaigana in the Japanese language.                |
% |                                                                           |
% | Version: 0.1.1                                                            |
% |                                                                           |
% | Changes:                                                                  |
% |                                                                           |
% | 0.1.1 2020-07-10 Christian Külker <c@c8i.org>                             |
% |     - The previous version claims that Hentaigana have not been included  |
% |       in Unicode. While this was true at the time of writing this changed |
% |       with Unicode 10 in 2017. This version reflects the new development. |
% |                                                                           |
% | 0.1.0 2014-09-14 Christian Külker <c@c8i.org>                             |
% |     - Initial release                                                     |
% |                                                                           |
% +---------------------------------------------------------------------------+
% WARNING:
%   - This file contains Unicode characters that are NOT displayed on all
%     computer systems
%   - Also known Unicode editors, like vim, are likely to show misalignment
%     where actually there is none, use gvim, pluma or the like instead.
%   - Some Roman characters are not displayed in lines having a Hentaigana
%   - Some Roman characters are displayed double even if there is only one
%     in lines having a Hentaigana


\section{Hentaigana}\jsec{同音異語}
% [o] LABEL
\label{sec:Hentaigana}
% [o] INDEX DESTINATION (DEF)
\ifor{Hentaigana}{変体仮名}{へんたいがな}{Hentaigana}
% [o] INDEX TARGET
\ifor{Kana}{仮名}{かな}{Kana}

\textit{Hentaigana} (変体仮名【へんたいがな】, pronounced |hentaiɡana|) are
historical \hyperref[sec:Kana]{Kana} that are used seldom today. They were used
until before 1900 and declared as obsolete\footnote{The word /hentai/ means
just variant} in the 1900 language reform. Rather than an addition to
\hyperref[sec:Kana]{Kana}, \textit{Hentaigana} representing alternative forms
to existing \hyperref[sec:Kana]{Kana}. The usage were not formalized and every
writer decides which set to use. It was even common to use two or more
different \textit{Hentaigana} (and standard \hyperref[sec:Kana]{Kana}) with
the same pronunciation in the same document by the same author.

Until 1947 \textit{Hentaigana} were used for names. In contemporary Japan the
usage of \textit{Hentaigana} is reduced to traditional decorative elements on
shop signs for example. A few marginal uses remain such as: the word /otemoto/
is written in \textit{Hentaigana} on some chopsticks or the names in the
Japanese family registry (戸籍 koseki).

\bigskip

\textbf{Examples of \textit{Hentaigana}:}

\begin{center}
\JapaneseFontN
\begin{tabular}{lcccl}
\textbf{UCS}&\textbf{Hentaigana}&\textbf{Pronunciation}&\textbf{Derived From}&\textbf{Note} \\\hline
1B001& 𛀁 & |ye|& 江  & Simple \\
1B002& 𛀂 & |a| & 安  & Similar \\
...  &... &...  &...  & \\
1B009& 𛀉   & |i|  & 移  & Complex \\
...  &... &...  &...  & \\
1B01A& 𛀚 &|ka| & 可  & Unexpected pronunciation \\
...  &... &...  &...  & \\
\end{tabular}
\JapaneseDefault
\end{center}

Due to Japanese proposals from 2015\footnote{See
\href{https://www.unicode.org/L2/L2015/15316-hentaigana-58_438.pdf}{『変体仮名のこれまでとこれから—情報交換のための標準化』
(The past, present, and future of Hentaigana: Standardization for information
processing) by TAKADA Tomokazu (高田智和) et al.} and
\href{https://www.unicode.org/L2/L2015/15318-hentaigana.pdf}{About the inclusion
of standardized codepoints for Hentaigana by YADA Tsutomu (矢田勉)} }.
\textit{Hentaigana} became available in Unicode (version 10) in 2017. However
the usage on computers in 2020 is still difficult. Until Japanese computer text
input methods (like \texttt{Mozc}, \texttt{Anthy}, ...) support
\textit{Hentaigana}, entering this characters on a computer is quite
cumbersome. In \texttt{vim} for example: enter insert mode, press
\texttt{<CTRL+V>+U} and then the hexadecimal UCS number. For instance the
font\footnote{See
\href{https://en.wikipedia.org/wiki/Help:Multilingual_support\#Hentaigana}{Wikipedia
Help} for more fonts or the Wikipedia page on
\href{https://en.wikipedia.org/wiki/Hentaigana}{Hentaigana} } \texttt{HanaMinA
Regular} (Hanazono Mincho) can be used to display \textit{Hentaigana}.
