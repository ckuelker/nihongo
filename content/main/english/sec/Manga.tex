% ---------------------------------------------------------------------------
\section{Manga}\jsec{漫画} 
% [o] LABEL
\label{sec:Manga}
% [o] INDEX
\ifor{manga}{漫画}{まんが}{manga, Comic}

The Japanese version of comics is called \textbf{Manga} ({漫画} {【まんが】})
created on Japan or by Japanese authors. Some people say that \textbf{Manga} is
different from comics and deserve a name by its own. When \textbf{Manga} is
used in this book, then to distinguish it from comics in that sense that it is
written in Japanese posses a dynamic writing style\footnote{Conforming to a
style developed in Japan in the late 19th century.}  that is sometimes
challanging to its audience.

The term \textbf{Manga} for just Japanese comics is more used outside Japan. In
Japan all comics are referenced as \textbf{Manga} as well as with {コミック}
/komikku/ for all kinds of comics. The word \textbf{Manga} itself can be
translated as "whimsical drawings" or "impromptu sketches." and is used since
the late 18th century.


