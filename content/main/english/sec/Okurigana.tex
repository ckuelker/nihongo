% ---------------------------------------------------------------------------
\section{Okurigana}\jsec{送り仮名}
% [o] LABEL
\label{sec:Okurigana}
\label{sec:Nokurigana}
% [o] INDEX
\ifor{okurigana}{送り仮名}{おくりがな}{Okurigana}
\ifor{katakana}{片仮名}{かたかな}{Katakana}
\ifor{kana}{仮名}{かな}{Kana}
\ifor{hiragana}{平仮名}{ひらがな}{Hiragana}
\ifor{kanji}{漢字}{かんじ}{Kanji}
\ifor{nokurigana}{ノくり仮名}{のくりがな}{Nokurigana}

The term \ivoc{okurigana}{送り仮名}{おくりがな}{Okurigana} in Japanese is
\textit{not} a script by its own as the name \hyperref[sec:Kana]{kana} suggest.
\textbf{Okurigana} are \hyperref[sec:Kana]{kana} but either
\hyperref[sec:Hiragana]{hiragana} or \hyperref[sec:Katakana]{katakana} that are
used to write the ending of words in most cases verbs. More precise
\textbf{okuriagna} are suffixes of \hyperref[sec:Kanji]{kanji}. After 1945 only
\hyperref[sec:Hiragana]{hiragana} are used to write \textbf{okurigana} while
before \hyperref[sec:Katakana]{katakana} was used.

\textbf{Okurigana} are the mandatory compromise using static Chinese letters to
write the Japanese language. Next to make \hyperref[sec:Kanji]{kanji} flexible
the other function is to mark the beginning are ending of words in sentences.

\textbf{Okurigana} have two purposes. (1) conjugate (a) verbs and (b)
adjectives. With very few exceptions\footnote{ {皮肉る} {【ひにくる】},
{牛耳る}  {【ぎゅうじる】} and {退治る} {【たいじる】}.}  \textbf{okurigana}
will only inflect \hyperref[sec:Kanji]{kanji} as \textit{kun'yomi}.  (2) Change
the meaning or reading of a \hyperref[sec:Kanji]{kanji} by different
\textbf{okurigana}.

\textit{Example: Okuriagana change the meaning (tense):}

\medskip
\begin{tabular}{ll}
\hspace{2cm}(1) {見る} {【みる】} & see \\
\hspace{2cm}(2) {見た} {【みた】} & saw \\
\end{tabular}

\medskip
In the above example the \textbf{okurigana} of (1) is {「る」} and the
\textbf{okurigana} of (2) is {「た」}.

\textit{Example: Okuriagana change the reading:}

\medskip
\begin{tabular}{ll}
\hspace{2cm}(1) {下さる} {【くださる】} & to give \\
\hspace{2cm}(2) {下りる} {【おりる】} &  to get off (a train for example)/ to descend \\
\hspace{2cm}(3) {下がる} {【さがる】} &  to dangle (intransitive)\\
\end{tabular}

\medskip
So in many cases the \textbf{okurigana} directly after the
\hyperref[sec:Kanji]{kanji} changes the meaning.

\textit{Example: Okuriagana change the meaning (transitivity) :}

\medskip
\begin{tabular}{ll}
\hspace{2cm}(1) {下がる} {【さがる】} &  to dangle (intransitive)\\
\hspace{2cm}(2){下げる} {【さげる】} &  to let off (transitive)\\
\end{tabular}

\medskip
As in the above case many Japanese verbs come in transitive and intransitive
pairs. The reading of the \hyperref[sec:Kanji]{kanji} is often shared.

\subsection*{Okurigana in the Middle}

\textbf{Okurigana} can also be found in the middle of Japanese words.

\textit{Example:}

\begin{center}\begin{tabular}{ll}
(1) {繰り返し} {【くりかえし】} &  to repeat\\
\end{tabular}\end{center}

\subsection*{Invisible Okuriagna - ノくり仮名}

The term \ivoc{nokurigana}{ノくり仮名}{のくりがな}{Nokurigana} was inspired by
the site \texttt{https://kanjidamage.com} but the writing was changed from
\hyperref[sec:Romaji]{rōmaji} to katakana+okurigana+kanji (The
        \hyperref[sec:Katakana]{katakana} {「ノ」} derives from the English
        'no', and the word as such is a violation of the Japanese
        \textbf{okurigana}\footnote{Because \hyperref[sec:Katakana]{katakana}
                do not have \textbf{okurigana}. But also in case there would be
no violation the /o/ of /okuri/ would be vilify to a honorific prefix and then
to be ripped out by the 'no' in a very non polite way.} which describes a
violation of \textbf{okurigana}) Of course the term  is not official, but quite
funny in this case, that basically one should be very angry with the fact that
there are some Japanese words witch do have \textbf{okurigana} but are not
written (but of course pronounced!). The not so funny part with those words is
that if one knows the reading of the \hyperref[sec:Kanji]{kanji} it is
impossible to look them up in a dictionary. So lets strike back and spread the
word of the \textbf{nokurigana} - {ノくり仮名}.

\begin{center}\begin{tabular}{ll}
(1) {取引} = {取り引き} {【と(り)ひ(き)】} &  Transaction\\
(2) {受付} = {受け付け} {【う(け)つ(け)】} &  Reception\\
\end{tabular}\end{center}


\Link \href{https://kanjidamage.com/tags/43}{https://kanjidamage.com/tags/43}








