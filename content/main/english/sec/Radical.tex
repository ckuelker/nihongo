% ---------------------------------------------------------------------------
\section{Radical}\jsec{部首}
% [o] LABEL
\label{sec:Radical}
% [o]  INDEX
\ifor{radical}{部首}{ぶしゅ}{Radikal}
\ifor{kanji}{漢字}{かんじ}{Kanji}

A \ivoc{radical}{部首}{ぶしゅ}{Radikal} is a root particle or character of a
Sino-Japanese character aka \\hyperref[sec:Kanji]{kanji}. It is the most
significant part of a Sino-Japanese character. The concept was developed in
China for Chinese characters and is today known under the same name {部首}
(pinyin: bùshǒu).

There is no general definition what a \textbf{radical} is or how many are
existing and it can vary a lot. The author of a dictionary has the power to
defined what a \textbf{radical} is and how much there will be in that
dictionary.

In more traditional Chinese or Japanese dictionaries a number of 214 or 244
\textbf{radicals} is quite common. However some modern approaches like the
\Link \href{https://www.hadamitzky.de/english/works_books.htm#KD}{\textit{The
Kanji Dictionary} of Marc Spahn and Wolfgang Hadamitzky from 1996} a totally
different number of 79 can be found.

\Note{Note}{\footnotesize Before buying a \hyperref[sec:Kanji]{kanji}
dictionary, make sure that the \textbf{radical} system used suits your taste.
Sometimes it can be observed that Japanese dictionaries are stricter in the
definition of a \textbf{radical} because a given \hyperref[sec:Kanji]{kanji}
can only be retrieved via exactly \textit{one} \textbf{radical}. While in many
Chinese dictionaries \textit{every} \textbf{radical} of a Chinese character can
be used to find it. The Japanese approach is of course good in terms of
systematic and didactic for learners, however it can take significant longer to
look up a character by \textbf{radical} in Japanese.}

