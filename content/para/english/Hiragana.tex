
\ifor{kanji}{漢字}{かんじ}{Kanji}
\ifor{hiragana}{平仮名}{ひらがな}{Hiragana}
\ifor{katakana}{片仮名}{かたかな}{Katakana}
\ifor{okurigana}{送り仮名}{おくりがな}{Okurigana}

Approx. in the 9th century the \lhiragana{} script was developed by simplifying
Chinese characters used for pronunciation. The number of contemporary
\textbf{hiragana} where reduced and today 46 are in use. It is a
\hyperref[sec:Mora]{morae} alphabet which is mostly constructed out of
syllables. In the modern Japanese language \textbf{hiragana} is used for
\hyperref[sec:Okurigana]{okurigana} like verb endings, other endings as well as
for phonetic transcription and for all other words which can or should not be
written with \hyperref[sec:Kanji]{kanji}, except words which are written in
\hyperref[sec:Katakana]{katakana}. A simple rule of thumb: if it is not known
whether the word should be written in \hyperref[sec:Kanji]{kanji} or
\hyperref[sec:Katakana]{katakana}, it should probably be written in
\textbf{hiragana}.
