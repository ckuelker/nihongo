
% ---------------------------------------------------------------------------

\phantomsection
% [o] LABEL
\label{para:Conventions}
% [o] INDEX TARGET
\ifor{kanji}{漢字}{かんじ}{Kanji}
\ifor{hiragana}{平仮名}{ひらがな}{Hiragana}
\ifor{katakana}{片仮名}{かたかな}{Katakana}
\ifor{rōmaji}{ローマ字}{ろーまじ}{Rōmaji}
\ifor{Hepburn System}{ヘボン式}{へぼんしき}{Hepburn System}

\bigskip

\textbf{Conventions Used in this Book}

% --- \jtl
% Transliteration: ⟨...⟩
% Math mode
%\newcommand{\jtl}[1]{$\langle#1\rangle$}% Japanese Transliteration IPA
% DejaVuSans mode
\newcommand{\jtl}[1]{\fontspec{DejaVuSans}⟨#1⟩\fontspec{FreeSans}}% Japanese Transliteration IPA
% --- \jphonemic
\newcommand{\jphonemic}[1]{/#1/}% Japanese phonemic /.../
% --- \jphonetic
\newcommand{\jphonetic}[1]{[#1]}% Japanese phonetic [...]

\medskip

\begin{itemize}

    \item[\Link]

        External hyperlinks are marked with a blue arrow.

        \medskip \textit{Example:} \medskip

        \begin{center}

            Please look at the download page for this document, if there is a
            new version\\ \Link
            \href{https://christian.kuelker.info/nihongo}{https://christian.kuelker.info/nihongo}

        \end{center}

    \item[{【}\ldots{】}]

        The reading of Japanese characters (\hyperref[sec:Kanji]{kanji}) are
        \textbf{not} given in the section or chapter heading but as soon as
        possible.  If the reading is given it will be given in
        \hyperref[sec:Hiragana]{hiragana} script. To mark this reading it will
        start with a Japanese bracket {【} and end with a Japanese bracket
        {】}.

        \medskip \textit{Example:} \medskip

        \begin{center} \Large Kanji {漢字} {【かんじ】} \end{center}

        \medskip

    \item[\jtl{\ldots}]

         If readings (transliteration) of Japanese are also given in
         \hyperref[sec:Romaji]{rōmaji} according to the
         \hyperref[sec:Hepburn]{Hepburn system}, this is indicated by an
         \textit{angle bracket} '\fontspec{DejaVuSans}⟨\fontspec{FreeSans}' at
         the beginning of the reading (transliteration) and an \textit{angle
         bracket} at the end '\fontspec{DejaVuSans}⟩\fontspec{FreeSans}'. This
         follow the International Phonetic Alphabet (IPA) rules of \Link
         \href{https://en.wikipedia.org/wiki/International_Phonetic_Alphabet#Brackets_and_transcription_delimiters}{brackets
         and transcription delimiters}.

        \medskip \textit{Example:} \medskip 

        \begin{center} First hiragana letter \Large {あ} {\jtl{a}} \end{center}
        \begin{center} First katakana letter \Large {ア} {\jtl{a}} \end{center}

%    \item[\jphonemic{\ldots}]
%
%         If the (phonemic) pronunciation of Japanese is also given in the
%         International Phonetic Alphabet (IPA), this is indicated by a
%         \textit{slash} '/' at the beginning of the reading and a
%         \textit{slash} at the end '/'. This indicates an abstract phonemic
%         notation by expressing only features that are distinctive.
%
%    \item[\jphonetic{\ldots}]
%
%        If the (phonetic) prounciation of Japanese is given, it includes
%        usually details beyond the phonemic pronunciation and is according to
%        \Link
%        \href{https://en.wikipedia.org/wiki/International_Phonetic_Alphabet#Brackets_and_transcription_delimiters}{IPA}
%        indicated by \textit{square brackets}. 
%
%        \medskip \textit{Example:} \medskip
%
%        The \textit{kana} letter 「ん」 or 「ン」 is phonetically pronounced differently, depending on the
%        letter context of the word.
%
%        \begin{tabular}{ll}
%        \jphonetic{n} &before n, t, d, r, ts, z, ch and j\\
%        \jphonetic{m} &before m, p and b\\
%        \jphonetic{ŋ} &before k and g\\
%        \jphonetic{ɴ} &at the end of utterances\\
%        \jphonetic{ũ͍} &before vowels, palatal approximants (y), consonants h, f, s, sh and w\\
%        \jphonetic{ĩ} &after the vowel i if another vowel, palatal approximant or consonant f, s, sh, h or w follows\\
%        \end{tabular}
%        {\tiny\Link\textit{Source: \url{https://en.wikipedia.org/wiki/N_(kana)}}}

\end{itemize}
