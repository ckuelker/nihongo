% ---------------------------------------------------------------------------
\section{Handakuten}\jsec{半濁点} \label{sec:Handakuten}
\ifor{Handakuten}{半濁点}{はんだくてん}{Handakuten}
\ifor{Dakuten}{濁点}{だくてん}{Dakuten}
\ien{circle}
\ija{丸}
\ija{まる}
\ide{Kreis}
\ithree{.@゚}{「゚」}{.@゚}% removed quotes (single quotes OK, double quotes NG)
\ien{|h|} \ide{|h|}
\ien{|p|} \ide{|p|}
\ien{pronunciation shift} \ide{Ausprache Verschiebung}
\newcommand{\lhandakuten}{\ivoc{handakuten}{半濁点}{はんだくてん}{Handakuten}}
\newcommand{\lmaru}{\ivoc{maru}{丸}{まる}{Maru}}

In Japanese two different markers for \hyperref[sec:Kana]{kana} are used: The
\ldakuten{} (see \hyperref[sec:Dakuten]{dakuten}) and the \lhandakuten{}. The
latter has the marker of a little circle {「゚」} and is therefore colloquially
described as \lmaru{} and indicates when the pronunciation shifts from |h| to
|p|.

