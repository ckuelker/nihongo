% VaAmbiguity
\subsection{\jtl{va}  Ambiguity} \label{subsec:VaAmbiguity}

The \jtl{va} ambiguity is mostly not a letter difficulty, it is a sound
representation problem. The Rōmaji \jtl{va} can be written in many different ways
and that is true for some other characters of the \jtl{wa} row too. The lack of
standardization and consistency make it hard to guess how one should write a
certain word with this sound.

\bigskip

\begin{table}[H]
\begin{center}
\begin{tabular}{p{4cm}p{5cm}p{5cm}p{1cm}}
%\KLETTER{so}&\KLETTER{n}&\KLETTER{ri}\\\hline
\ifthenelse{\equal{hiragana}{\jtopic}}{%
\textit{Hiragana}     &ゔぁ            &わ゙                &ば           \\\hline%
\textit{Romaji}       &\texttt{va}     &\texttt{va}       &\texttt{ba}  \\%\hline
\textit{Input}        &\texttt{va}     &                  &\texttt{ba}  \\%\hline
}{}
\ifthenelse{\equal{katakana}{\jtopic}}{%
\textit{Karakana}     &\textbf{ヴァ}   &ヷ                &バ           \\%
\textit{Romaji}       &\texttt{va}     &\texttt{va}       &\texttt{ba}  \\%\hline
\textit{Input}        &\texttt{va,ba}  &                  &\texttt{ba}  \\%\hline
}{}
\textit{Pronunciation}&usually \jtl{va}     &usually \jtl{va}     &\jtl{ba}\\%\hline
                      &some people \jtl{ba} &older people \jtl{ba}&             \\%\hline
                      &older people \jtl{ba}&                 &             \\%\hline
\end{tabular}
\end{center}
\caption{\jtl{va} Ambiguity}
\label{tab:VaAmbiguity}
\end{table}

\footnotesize
\begin{verbatim}
Unicode:
ヷ U+30F7, &#12535 KATAKANA LETTER VA; Composition: ワ [U+30EF] + ◌゙ [U+3099]
ヴ U+30F4, &#12532 KATAKANA LETTER VU; Composition: ウ [U+30A6] + ◌゙ [U+3099]
ァ U+30A1, &#12449 KATAKANA LETTER SMALL;
\end{verbatim}

