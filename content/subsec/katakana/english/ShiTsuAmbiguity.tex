% ShiTsuAmbiguity
\subsection{|shi| and |tsu| Ambiguity} \label{subsec:ShiTsuAmbiguity}

% シ
% ツ

The Katakana characters {「シ」} and {「ツ」} are difficult to distinguish.
Both are made out of 3 strokes and even the length are equal. In a sentence of
course the context can help a lot. But what are the rules for this characters
to write properly and distinguish?

\bigskip

\begin{figure}[H]
\begin{center}
\begin{tabular}{|c|c|}\hline
\KLETTER{shi}&\KLETTER{tsu}\\\hline
\end{tabular}
\end{center}
\caption{|shi| and |tsu| Ambiguity}
\label{fig:ShiAndTsuAmbiguity}
\end{figure}

\CharacterExplanation{shiexplanation}{ To write the letter |shi| it is
important to align three lines \textbf{vertically} (red line) and to
\textbf{non-align} the ends (blue line). In this way it is possible to
distinguish |shi| from |tsu|. Of course also the angle of the frist two lines
are different, but in hadwriting this is difficult to match. As a rule of thumb
make the third line double as long as the first two but short enough to not
align it at the end. }

\CharacterExplanation{tsuexplanation}{ To write the letter |tsu| it is
important to align all tree lines \textbf{horizontally} (red line) and to
\textbf{non-align} the ends (blue line). In this way it is possible to
distinguish |tsu| from |shi|. Of course also the angle of the first two lines
are different, but in handwriting this is difficult to match. As a rule of
thumb make the third line double as long as the first two but short enough to
not align it at the end. }

