\chapter*{Introduction}\jchap{前書き}
\ithree{Introduction}{前書き}{Einleitung}

\ithree{Hiragana}{平仮名}{Hiragana}
The book is the second volume of \textit{The Japanese Script} with the focus to
teach \textit{Katakana}. As a matter of fact it is assumed that the read
already mastered volume one \textit{The Japanese Script - Hiragana} before
continue learning \textit{Katakana}. 

\ithree{self learning}{独修}{Selbstudium}
Being able to read and write Japanese is a core skill when learning Japanese.
And \textit{Katakana} is one of the two very basic scripts of Japanese to be
learned.  This book is written with the aim to help in that, based on self
experience as a learner of \textit{Katakana} as well as from teaching
experience and with feedback of many students. This is the first edition as a
book for self leaning approach.  So please report suggestions or problems. 

The first chapter will introduce the Japanese writing system and different
alphabets. I you are already familiar with it, you can safely skip this
chapter. In any case all terms are explained in the last chapter.

The second chapter starts with the introduction of writing and reading single
\textit{Katakana} letters. The chapter ends with special \textit{Katakana}
letters. It is advised to read this chapter before starting the training. 

The third chapter goes right into action by offering row based training
sessions for each character as well as simple training for writing some
Japanese \textit{Katakana} words.

The last chapter provides an alphabetically ordered glossary about the most
important key words.

The appendix contain tables of all important \textit{Katakana} and tables of
\textit{Katakana} with different fonts to understand the margin (how match can
be diverted from the standard and the character is still recognized) of the
characters. The second part are two tables with Latin letters to memorize the
pronunciation. The last part of the appendix are three indices: in English and
German for the learner and in Japanese for the teachers. 




