\chapter*{Introduction}\jchap{前書き}
\ithree{Introduction}{前書き}{Einleitung}

\ithree{Hiragana}{平仮名}{Hiragana}
The book is the second volume of \textit{The Japanese Script} with the focus to
teach \textit{Katakana}. As a matter of fact it is assumed that the reader
already mastered volume one \textit{The Japanese Script - Hiragana} before
continue learning \textit{Katakana}. 

\ithree{self learning}{独修}{Selbstudium}
Being able to read and write Japanese is a core skill when learning Japanese.
And \textit{Katakana} is one of the two very basic scripts of Japanese to be
learned.  This book is written with the aim to help in that, based on self
experience as a learner of \textit{Katakana} as well as from teaching
experience and with feedback of many students. This is the first edition as a
book for self leaning approach. So please report suggestions or problems. 

The \hyperref[chap:JapaneseWritingSystem]{first chapter}
(\nameref{chap:JapaneseWritingSystem}) will introduce the Japanese writing
system and different alphabets. If you are already familiar with it, you can
safely skip this chapter. In any case all terms are explained in the
\hyperref[chap:Terminology]{last chapter}.

The \hyperref[TheWayToWriteKatakana]{second chapter}
(\nameref{TheWayToWriteKatakana})starts with the introduction of writing and
reading single \textit{Katakana} letters. The chapter ends with special
\textit{Katakana} letters. It is advised to read this chapter before starting
the training. 

The \hyperref[chap:KatakanaTraining]{third chapter}
(\nameref{chap:KatakanaTraining}) goes right into action by offering row based
training sessions for each character as well as simple training for writing
some Japanese \textit{Katakana} words.

The \hyperref[chap:Terminology]{last chapter} (\nameref{chap:Terminology})
provides an alphabetically ordered glossary about the most important key words.
It is recommended to read one article at a time to deepen the understanding of
the Japanese language in general and the way of writing Japanese in particular.
The order do not matter. However it is not mandatory to read this chapter to
learn\textit{ Katakana}. 

The appendix contain tables of all important \textit{Katakana} and
\textit{Katakana} written in different fonts in the
\nameref{chap:KatakanaTables} part. Even though this is not explicit mentioned
in the following chapters it is important to have a look at this tables from
time to time when learning \textit{Katakana} to understand the margin (how
match can be diverted from the standard and the character is still recognized)
of the character to learn. The second part included
\hyperref[chap:RomajiTables]{two tables with Latin letters} to memorize the
pronunciation. In the forth part a list of used
\hyperref[chap:JapaneseTechnicalTerms]{main technical terms in Japanese} can be
found with references to the text where they are explained.  The last part of
the appendix offer three indices: in \hyperref[chap:EnglishIndex]{English}
(Index) and \hyperref[chap:GermanIndex]{German} (Fachbegriffe) for the learner
and in \hyperref[chap:JapaneseIndex]{Japanese} (索引) for the teachers. 




