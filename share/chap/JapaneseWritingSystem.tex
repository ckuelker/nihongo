% ===========================================================================
\chapter{Japanese Writing System - 日本語の書き方}\label{chap:JapaneseWritingSystem}

From the perspective of an European the Japanese script (how Japanese is
written) looks strange and difficult at the first sight and many people mistake
Japanese for Chinese\footnote{In German language the word "Fachchinesisch"
(TODO) for example is synonym of something that is not understandable. The
perception to understand Japanese is almost the same.} writing.  For Japanese
the Japanese script is just ordinary. On the other side the writing system of
an European language is also not easy to a Japanese. Most Japanese will not
notice the whole difficulty because they are introduced to English at an early
age and school English is just a subset of every day written English. The
difficulties starts where Japanese are exposed to every day written English or
any other European language with all it different graphical representations.

Most Europeans believe that they are using only one writing script. At a closer
look that is wrong.

\bigskip Example of 4 different representation of the reading "a":

\begin{center}
\begin{tabular}{|l|l|l|l|}
\textbf{Character}&\textbf{Alphabet}&\textbf{Reading}&\textbf{Remark}\\\hline
\textit{a}     &  Italic        & a & printed script, small letter ``a'' \\ 
\texttt{a}     &  Typewriter    & a & printed script, small letter ``a'' \\ 
A              &  Serif         & a & printed script, capital letter ``a'' \\ 
$\mathfrak{A}$ & Fraktur\footnote{Some of this writing scripts where used 
actively in the beginning of last century, while is is more common to only 
read them now.}& a & Fraktur, capital letter ``a''  \\ 
\end{tabular}
\end{center}

Some of this writing scripts where used active in the beginning of last
century, while is is more common to only read them now. 

For an European adult\footnote{European children have to learn that
"\textit{a}" is the same as "\texttt{a}". And even adults have difficulties to
read "$\mathfrak{A}$" out of context as "A".}  the "kinship" of the above
graphic elements is obvious. However it is a cultural achievement to associate
them to each other and it is by no means obvious from a foreign (or learner's)
perspective. 

In a similar way the equality of {「あ」} and {「ア」} is obvious for a
Japanese, but not for an European. When got used to it, it will become not
strange or difficult any more.

As in European text also in Japanese text a number of different scripts can be
found. Next to the known scripts in Europe\footnote{German for example:
Fraktur, Latin, special characters like umlauts or eszett, Indian numbers}
there are two Japanese alphabets\footnote{ From a scientific point of view it
can be argued that the Roman a-z or A-Z is an \textit{alphabet} but the
Japanese\hyperref[sec:Hiragana]{Hiragana}and\hyperref[sec:Katakana]{Katakana}are
not. On the other side we could try to argue that a-z and A-Z are to different
\textit{alphabets}, because the graphical representation of a sound is
different (and alpha is Greek letter anyway). If we follow this argumentation
we might state that\hyperref[sec:Hiragana]{Hiragana}is small writing while
Katakana is capital writing. However both ways of argumentations have its short
comings. By using the word \textit{alphabet}
for\hyperref[sec:Hiragana]{Hiragana}as well as for "Typewriter" above two goals
are in the focus of mind. First, the word \textit{alphabet} is a generic term
for a common set of letters that is understood by everyone and second by using
an average term for European and Japanese language the similarities should been
stressed and not the (of course) existing differences. The friction by using
the word \textit{alphabet} for "Typewriter" for example is well understood and
intended. } \hyperref[sec:Hiragana]{Hiragana} and
\hyperref[sec:Katakana]{Katakana}, both are referenced as \textit{Kana} and the
letters derived from Chinese Characters called \hyperref[sec:Kanji]{Kanji}.

Example:

\begin{center}
\begin{tabular}{|l|l|l|l|}
\textbf{Character}&\textbf{Alphabet}&\textbf{Reading}&\textbf{Remark}\\\hline
あ& Hiragana & a & no meaning, just the letter  ``a'' in Hiragana \\
ァ& Katakana & a & no meaning, just the letter ``a'' in Katakana \\
阿& Kanji    & a & { angle, to please, part of roof, hill, Africa}\\
\end{tabular}
\end{center}

Japanese can be written in two directions. First, old fashioned from up to down
- vertically with columns from right to left. And second, modern (as in
  English) from left to right - horizontally with rows from up to down. Within
  this four alphabets are used: Roman-Arab letters (our letters),
  \hyperref[sec:Kanji]{Kanji} (Chinese derived letters)
  \hyperref[sec:Hiragana]{Hiragana} (Newer Japanese characters) and
  \hyperref[sec:Katakana]{Katakana} (also newer Japanese characters).  This
  mixture of alphabets is named \textit{Kanji-Kana-Majiri-Bun}
  (Kanji-Kana-Mixed-Text). The most common are \hyperref[sec:Kanji]{Kanji} and
  \hyperref[sec:Hiragana]{Hiragana}. Each of the scripts are introduced in the
  following sections.

\section*{\textit{Kanji}} 

1300 years ago the first endeavours where undertaken to display the Japanese
language with the only known alphabet in the region, the Chinese writing
system. While the Japanese language where hardly suited for the writing system
it was an  economical choice since the Chinese characters where well developed
at that time and introduced many new ideas in lexic. The 'borrowing' of Chinese
characters was not a one shot operation it took centuries and more then one
attempt. This long winded process led to the fact that some Characters where
imported more then once from China from different times and different regions.
And because of this one Chinese character can have more then one pronunciation.
We hope that this will consolidate over the next centuries.  Today this
imported characters are known as \hyperref[sec:Kanji]{Kanji} in Japan.
\hyperref[sec:Kanji]{Kanji} is written \textit{Hanzi} in Chinese and
referencing the character from the Han period of China. Even though today all
Chinese based Characters (and even some self invented) are referenced nowadays
as Kanji, it does not strictly mean that they only from the Han period.

A standard Japanese text do contain Kanji. To master the Japanese language over
a certain level and to be over come the problem of personal analphabetism in
Japan it is highly encouraged to learn at least 600 to 800 characters. To
become fully literate member of the Japanese society 2000 to 2300 Kanji should
be learned. 

Today  Kanji in written Japanese language are used for substantives/ nouns,
verbs, adjectives and names. 

\section*{\textit{Hiragana}}

Approx. in the 9th century the \hyperref[sec:Hiragana]{Hiragana} script was
developed by simplifying Chinese characters used for pronunciation. The number
of contemporary \hyperref[sec:Hiragana]{Hiragana} where reduced and today 46
are used.  \hyperref[sec:Hiragana]{Hiragana} is a morae alphabet which is
mostly constructed out of syllables. In modern Japanese language
\hyperref[sec:Hiragana]{Hiragana} is used for  verb endings, other endings,
phonetic transcription and for all other words which can, should not be written
with Kanji, except words which are written in Katakana. In simple words: if it
is not known weather the word should be written in Kanji or Katakan write in
\hyperref[sec:Hiragana]{Hiragana}.

\section*{\textit{Katakana}}

At the same time as \hyperref[sec:Hiragana]{Hiragana}, also
\hyperref[sec:Katakana]{Katakana} letters where invented by simplifying the
same Chinese characters used for pronounciation.  However the look and feel of
Katakana is more 'square' not so 'rounded' as
\hyperref[sec:Hiragana]{Hiragana}.

\hyperref[sec:Katakana]{Katakana} is used today for writing words of foreign
origin and for emphasizing (in commercials or magna) as well as word in the
fauna or flora. 

\section*{Roman/ Latin/ Arab Characters}

The western characters are mainly used for writing numbers in the horizontal
writing. Also for abrehaviations capital and small letters are used. Sometimes
they are modified. For example the measurement of distance in the metric entity
"km" occupies to places in western scripts "k" + "m" while it only hold one
place in Japanese. Citation of foreign books are also done in western letters
an can pop up without warning the middle of the text.
