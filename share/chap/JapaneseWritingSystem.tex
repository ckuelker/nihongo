% ===========================================================================
\chapter{Japanese Writing System - 日本語の書き方}\label{chap:JapaneseWritingSystem}

From the perspective of an European the Japanese script (how Japanese is
written) looks strange and difficult at the first sight and many people mistake
Japanese for Chinese\footnote{In German language the word "Fachchinesisch"
(Lit.: profession Chinese, Engl.: gobbledygook, Amer.: gobbledegook) for
example is synonym of something that is not understandable. The perception to
understand Japanese is almost the same.} writing. For Japanese the Japanese
script is just ordinary. On the other side the writing system of an European
language is also not easy to a Japanese. Most Japanese will not notice the
whole difficulty because they are introduced to English at an early age and
school English is just a subset of every day written English. The difficulties
starts where Japanese are exposed to every day written English or any other
European language with all it different graphical representations.

Most Europeans believe that they are using only one writing script. At a closer
look that is wrong.

\bigskip Example of 4 different representation of the reading "a":

\begin{center}
\begin{tabular}{|l|l|l|l|}
\textbf{Character}&\textbf{Alphabet}&\textbf{Reading}&\textbf{Remark}\\\hline
\textit{a}     &  Italic        & a & printed script, small letter ``a'' \\ 
\texttt{a}     &  Typewriter    & a & printed script, small letter ``a'' \\ 
A              &  Serif         & a & printed script, capital letter ``a'' \\ 
$\mathfrak{A}$ & Fraktur\footnote{Some of this writing scripts where used 
actively in the beginning of last century, while is is more common to only 
read them now.}& a & Fraktur, capital letter ``a''  \\ 
\end{tabular}
\end{center}

Some of this writing scripts where used active in the beginning of last
century, while is is more common to only read them now. 

For an European adult\footnote{European children have to learn that
"\textit{a}" is the same as "\texttt{a}". And even adults have difficulties to
read "$\mathfrak{A}$" out of context as "A".}  the "kinship" of the above
graphic elements is obvious. However it is a cultural achievement to associate
them to each other and it is by no means obvious from a foreign (or learner's)
perspective. 

In a similar way the equality of {「あ」} and {「ア」} is obvious for a
Japanese, but not for an European. When got used to it, it will become not
strange or difficult any more.

As in European text also in Japanese text a number of different scripts can be
found. Next to the known scripts in Europe\footnote{German for example:
Fraktur, Latin, special characters like umlauts or eszett (the German
symbol for a voiceless "s" after a long vowel (such as in "großer Mann") or a
diphthong (such as in "weißer Hai"). ('ß')), Indian numbers} there are two
Japanese alphabets
%\footnote{ From a scientific point of view it can be argued
%that the Roman a-z or A-Z is an \textit{alphabet} but the
%Japanese \hyperref[sec:Hiragana]{Hiragana} and \hyperref[sec:Katakana]{Katakana}
%are not. On the other side we could try to argue that a-z and A-Z are to
%different \textit{alphabets}, because the graphical representation of a sound
%is different (and alpha is Greek letter anyway). If we follow this
%argumentation we might state that \hyperref[sec:Hiragana]{Hiragana} is small
%writing while Katakana is capital writing. However both ways of argumentations
%have its short comings. By using the word \textit{alphabet}
%for \hyperref[sec:Hiragana]{Hiragana} as well as for "Typewriter" above two goals
%are in the focus of mind. First, the word \textit{alphabet} is a generic term
%for a common set of letters that is understood by everyone and second by using
%an average term for European and Japanese language the similarities should been
%stressed and not the (of course) existing differences. The friction by using
%the word \textit{alphabet} for "Typewriter" for example is well understood and
%intended. } 
 \hyperref[sec:Hiragana]{Hiragana} and
\hyperref[sec:Katakana]{Katakana}, both are referenced as \textit{Kana} and
the letters derived from Chinese characters called \hyperref[sec:Kanji]{Kanji}.

Example:

\begin{center}
\begin{tabular}{|l|l|l|l|}
\textbf{Character}&\textbf{Alphabet}&\textbf{Reading}&\textbf{Remark}\\\hline
あ& Hiragana & a & no meaning, just the letter  ``a'' in Hiragana \\
ァ& Katakana & a & no meaning, just the letter ``a'' in Katakana \\
阿& Kanji    & a & { angle, to please, part of roof, hill, Africa}\\
\end{tabular}
\end{center}

Japanese can be written in two directions. First, old fashioned from up to down
- vertically with columns from right to left. And second, modern (as in
  English) from left to right - horizontally with rows from up to down. Within
  this four alphabets are used: Roman-Indian letters (our letters),
  \hyperref[sec:Kanji]{Kanji} (Chinese derived letters)
  \hyperref[sec:Hiragana]{Hiragana} (Newer Japanese characters) and
   \hyperref[sec:Katakana]{Katakana}  (also newer Japanese characters).  This
  mixture of alphabets is named \textit{Kanji-Kana-Majiri-Bun}
  (Kanji-Kana-Mixed-Text). The most common are \hyperref[sec:Kanji]{Kanji} and
  \hyperref[sec:Hiragana]{Hiragana}. Each of the scripts are introduced in the
  following sections.

\section*{\textit{Kanji}} 

\ifor{Kanji}{漢字}{かんじ}{Kanji}

1300 years ago the first endeavours where undertaken to display the Japanese
language with the only known alphabet in the region, the Chinese writing
system. While the Japanese language where hardly suited for the writing system
it was an economical choice since the Chinese characters where well developed
at that time and introduced many new ideas in lexis. The 'borrowing' of Chinese
characters was not a one shot operation it took centuries and more then one
attempt. This long winded process led to the fact that some characters where
imported more then once from China from different times and different regions.
And because of this one Chinese character can have more then one pronunciation.
We hope that this will consolidate over the next centuries. Today this imported
characters are known as \textbf{Kanji} in Japan. \textbf{Kanji} is written
\textit{Hanzi} in Chinese and referencing the character from the Han period of
China. Even though today all Chinese based characters (and even some self
invented) are referenced nowadays as \textbf{Kanji}, it does not strictly mean
that they are only from the Han period.

A standard Japanese text do contain \textbf{Kanji}. To master the Japanese
language over a certain level and to over come the problem of personal
illiteracy (in Japan) it is highly encouraged to learn at least 600 to 800
characters. To become a fully literate member of the Japanese society 2000 to
2300 \textbf{Kanji} should be learned.

Today \textbf{Kanji} in written Japanese language are used for substantives/
nouns, verbs, adjectives and names.


\section*{\textit{Hiragana}}

\ifor{Kanji}{漢字}{かんじ}{Kanji}
\ifor{Hiragana}{平仮名}{ひらがな}{Hiragana}
\ifor{Katakana}{片仮名}{かたかな}{Katakana}
\ifor{Okurigana}{送り仮名}{おくりあがな}{Okurigana}

Approx. in the 9th century the \textbf{Hiragana} script - written in Japanese
as {平仮名} {【ひらがな】} - was developed by simplifying Chinese characters
used for pronunciation. The number of contemporary \textbf{Hiragana} where
reduced and today 46 are used. \textbf{Hiragana} is a
\hyperref[sec:Mora]{morae} alphabet which is mostly constructed out of
syllables. In modern Japanese language \textbf{Hiragana} is used for
\hyperref[sec:Okurigana]{Okurigana} like verb endings, other endings as well as
for phonetic transcription and for all other words which can or should not be
written with \hyperref[sec:Kanji]{Kanji}, except words which are written in
\hyperref[sec:Katakana]{Katakana}. In simple words: if it is not known weather
the word should be written in \hyperref[sec:Kanji]{Kanji} or
\hyperref[sec:Katakana]{Katakana} write in \textbf{Hiragana}.


\section*{\textit{Katakana}}
At the same time as \hyperref[sec:Hiragana]{Hiragana}, also \textbf{Katakana}
letters where invented by simplifying the same Chinese characters used for
pronunciation.  However the look and feel of \textbf{Katakana} is more 'square'
not so 'rounded' as \hyperref[sec:Hiragana]{Hiragana}.

\textbf{Katakana} is used today for writing words of foreign origin and for
emphasizing (in commercials or \hyperref[sec:Manga]{Manga}) as well as word in
the fauna or flora.



\section*{\textit{Roman/ Latin/ Indian-Arabic Characters}}

% ---------------------------------------------------------------------------
\ifor{Rōmaji}{ローマ字}{ろーまじ}{Rōmaji}

In temporary Japan words written in western letters become more popular and
some parts of the written language is already westernized, like (Indian/
Arabic) numbers written in horizontal text almost per default. This western
Latin letters are called \textbf{Rōmaji} and are written in Japanese as
{ローマ字} {【ろおまじ】}, even though some of them are from different origin
like Indian numbers for example.



