% autoatically generated by: extract-ifor
\chapter{List of Japanese Technical Terms}
\label{chap:ListOfJapaneseTechnicalTerms}
\label{sec:JapaneseTechnicalTerms}
\jchap{日本専門用語リスト}
\ifor{Japanese Technical Terms}{日本専門用語}{にほんせんもんようご}{japanische Fachbegriffe}
\normalsize Ordered by Japanese pronunciation (Hiragana).
\footnotesize\Padding
\begin{longtable}[c]{p{.5cm}p{3.5cm}p{4cm}p{3.5cm}p{3.5cm}}
\textbf{\#}&\textbf{Japanese}&\textbf{Hiragana}&\textbf{English}&\textbf{German}\\ \hline
1&送り仮名&おくりがな&\hyperref[sec:Okurigana]{Okurigana}&Okurigana\\
2&踊り字&おどりじ&\hyperref[sec:RepitionMarkForKanjiAndKana]{repition mark for Kanji and Kana}&Wiederholungszeichen für Kanji und Kana\\
3&音節&おんせつ&\hyperref[sec:Syllable]{syllable}&Silbe\\
4&重ね字&かさねじ&\hyperref[sec:RepitionMark]{repition mark}&Wiederholungszeichen\\
5&片仮名&かたかな&\hyperref[sec:Katakana]{Katakana}&Katakana\\
6&仮名&かな&\hyperref[sec:Kana]{Kana}&Kana\\
7&漢字&かんじ&\hyperref[sec:Kanji]{Kanji}&Kanji\\
8&空白文字&くうはく ・ もじ&\hyperref[sec:SpaceCharacter]{space character}&Leerzeichen\\
9&空白文字&くうはく・もじ&\hyperref[sec:SpaceCharacter]{space character}&Leerzeichen\\
10&くの字点&くのじてん&\hyperref[sec:Kunojiten]{Kunojiten}&Kunojiten\\
11&繰り返し記号&くりかえしきごう&\hyperref[sec:RepitionMark]{repition mark}&Wiederholungszeichen\\
12&訓令式ローマ字&くんれいろうまじ&\hyperref[sec:KunreiSystem]{Kunrei System}&Kunrei System\\
13&五十音図&ごじゅうおんず&\hyperref[sec:Gojuonzu]{Gojūonzu}&50 Laute Tafel\\
14&修正ヘボン式ローマ字&しゅうせい・へぼんしき・ろうまじ&\hyperref[sec:NewerHepburnSystem]{newer Hepburn System}&neueres Hepburn System\\
15&専門用語&せんもんようご&\hyperref[sec:Terminology]{Terminology}&Fachbegriffe\\
16&濁点&だくてん&\hyperref[sec:Dakuten]{Dakuten}&Dakuten\\
17&長音&ちょうおん&\hyperref[sec:Choon]{Chōon}&Chōon\\
18&同音異語&どうおん・いご&\hyperref[sec:Homophone]{homophone}&Homophon\\
19&特別カタカナ&とくべつかたかな&\hyperref[sec:SpecialKatakanaCharacters]{Special Katakana Characters}&Spezielle Katakana Zeichen\\
20&日本式ローマ字&にほんしきろうまじ&\hyperref[sec:JapanSystemLatinLetters]{Japan System Latin letters}&Lateinische Buchstaben des Japanischen Systems\\
21&日本専門用語&にほんせんもんようご&\hyperref[sec:JapaneseTechnicalTerms]{Japanese Technical Terms}&japanische Fachbegriffe\\
22&ノくり仮名&のくりがな&\hyperref[sec:Nokurigana]{Nokurigana}&Nokurigana\\
23&倍増母音&ばいぞうぼいん&\hyperref[sec:DoublingVowels]{doubling vowels}&Vokalverdopplung\\
24&半濁点&はんだくてん&\hyperref[sec:Handakuten]{Handakuten}&Handakuten\\
25&筆画&ひっかく&\hyperref[sec:Stroke]{Stroke}&Strich\\
26&筆画の種類&ひっかくのしゅるい&\hyperref[sec:StrokeTypes]{Stroke Types}&Strich Typen\\
27&表音文字&ひょうおんもじ&\hyperref[sec:PhoneticCharacter]{Phonetic Character}&Phonetisches Zeichen\\
28&標準ヘボン式ローマ字&ひょうじゅん・へぼん・ろまあじ&\hyperref[sec:OlderHepburnSystem]{older Hepburn System}&altes Hepburn System\\
29&平仮名&ひらがな&\hyperref[sec:Hiragana]{Hiragana}&Hiragana\\
30&部首&ぶしゅ&\hyperref[sec:Radical]{radical}&Radikal\\
31&振り仮名&ふりがな&\hyperref[sec:Furigana]{Furigana}&Furigana\\
32&ヘボン式&へぼんしき&\hyperref[sec:HepburnSystem]{Hepburn System}&Hepburn System\\
33&変体仮名&へんたいがな&\hyperref[sec:Hentaigana]{Hentaigana}&Hentaigana\\
34&丸&まる&\hyperref[sec:Circle]{circle}&Kreis\\
35&漫画&まんが&\hyperref[sec:Manga]{manga}&Manga\\
36&万葉仮名&まんようがな&\hyperref[sec:Manyogana]{Man'yōgana}&Man'yōgana\\
37&万葉集&まんようしゅう&\hyperref[sec:Manyoshu]{Man'yōshu}&Man'yōshu\\
38&モーラ&もーら&\hyperref[sec:Mora]{mora}&Mora\\
39&読み仮名&よみがな&\hyperref[sec:Yomigana]{Yomigana}&Yomigana\\
40&ルビ&るび&\hyperref[sec:Rubi]{rubi}&Rubi\\
41&ローマ字&ろーまじ&\hyperref[sec:Romaji]{Rōmaji}&Rōmaji\\
\end{longtable}
