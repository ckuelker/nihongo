% autoatically generated by: extract-ifor
\chapter{List of Japanese Technical Terms}
\label{chap:ListOfJapaneseTechnicalTerms}
\label{sec:JapaneseTechnicalTerms}
\jchap{日本専門用語リスト}
\ifor{Japanese technical terms}{日本専門用語}{にほんせんもんようご}{japanische Fachbegriffe}
\normalsize Ordered by Japanese pronunciation (Hiragana).
\footnotesize\Padding
\begin{longtable}[c]{p{.5cm}p{3.5cm}p{4cm}p{3.5cm}p{3.5cm}}
\textbf{\#}&\textbf{Japanese}&\textbf{Hiragana}&\textbf{English}&\textbf{German}\\ \hline
1&伊呂波&いろは&\hyperref[sec:Iroha]{Iroha}&Iroha\\
2&イントネーション&いんとねーしょん&\hyperref[sec:Intonation]{intonation}&Betonung\\
3&送り仮名&おくりがな&\hyperref[sec:Okurigana]{Okurigana}&Okurigana\\
4&踊り字&おどりじ&\hyperref[sec:RepitionMarkForKanjiAndKana]{repition mark for Kanji and Kana}&Wiederholungszeichen für Kanji und Kana\\
5&音節&おんせつ&\hyperref[sec:Syllable]{syllable}&Silbe\\
6&重ね字&かさねじ&\hyperref[sec:RepitionMark]{repition mark}&Wiederholungszeichen\\
7&片仮名&かたかな&\hyperref[sec:Katakana]{Katakana}&Katakana\\
8&仮名&かな&\hyperref[sec:Kana]{Kana}&Kana\\
9&漢字&かんじ&\hyperref[sec:Kanji]{Kanji}&Kanji\\
10&空白文字&くうはく ・ もじ&\hyperref[sec:SpaceCharacter]{space character}&Leerzeichen\\
11&空白文字&くうはく・もじ&\hyperref[sec:SpaceCharacter]{space character}&Leerzeichen\\
12&くの字点&くのじてん&\hyperref[sec:Kunojiten]{Kunojiten}&Kunojiten\\
13&繰り返し記号&くりかえしきごう&\hyperref[sec:RepitionMark]{repition mark}&Wiederholungszeichen\\
14&訓令式ローマ字&くんれいろうまじ&\hyperref[sec:KunreiSystem]{Kunrei system}&Kunrei System\\
15&五十音図&ごじゅうおんず&\hyperref[sec:Gojuonzu]{Gojūonzu}&50 Laute Tafel\\
16&修正ヘボン式ローマ字&しゅうせい・へぼんしき・ろうまじ&\hyperref[sec:NewerHepburnSystem]{newer Hepburn system}&neueres Hepburn System\\
17&専門用語&せんもんようご&\hyperref[sec:Terminology]{terminology}&Fachbegriffe\\
18&濁点&だくてん&\hyperref[sec:Dakuten]{Dakuten}&Dakuten\\
19&長音&ちょうおん&\hyperref[sec:Choon]{Chōon}&Chōon\\
20&同音異語&どうおん・いご&\hyperref[sec:Homophone]{homophone}&Homophon\\
21&特別カタカナ&とくべつかたかな&\hyperref[sec:SpecialKatakanaCharacters]{special Katakana characters}&Spezielle Katakana Zeichen\\
22&日本式ローマ字&にほんしきろうまじ&\hyperref[sec:JapanSystemLatinLetters]{Japan system Latin letters}&Lateinische Buchstaben des Japanischen Systems\\
23&日本専門用語&にほんせんもんようご&\hyperref[sec:JapaneseTechnicalTerms]{Japanese technical terms}&japanische Fachbegriffe\\
24&ノくり仮名&のくりがな&\hyperref[sec:Nokurigana]{Nokurigana}&Nokurigana\\
25&倍増母音&ばいぞうぼいん&\hyperref[sec:DoublingVowels]{doubling vowels}&Vokalverdopplung\\
26&発音&はつおん&\hyperref[sec:Pronuciation]{pronuciation}&Aussprache\\
27&半濁点&はんだくてん&\hyperref[sec:Handakuten]{Handakuten}&Handakuten\\
28&筆画&ひっかく&\hyperref[sec:Stroke]{stroke}&Strich\\
29&筆画の種類&ひっかくのしゅるい&\hyperref[sec:StrokeTypes]{stroke types}&Strich Typen\\
30&表音文字&ひょうおんもじ&\hyperref[sec:PhoneticCharacter]{phonetic character}&phonetisches Zeichen\\
31&標準ヘボン式ローマ字&ひょうじゅん・へぼん・ろまあじ&\hyperref[sec:OlderHepburnSystem]{older Hepburn system}&altes Hepburn System\\
32&平仮名&ひらがな&\hyperref[sec:Hiragana]{Hiragana}&Hiragana\\
33&部首&ぶしゅ&\hyperref[sec:Radical]{radical}&Radikal\\
34&振り仮名&ふりがな&\hyperref[sec:Furigana]{Furigana}&Furigana\\
35&ヘボン式&へぼんしき&\hyperref[sec:HepburnSystem]{Hepburn system}&Hepburn System\\
36&変体仮名&へんたいがな&\hyperref[sec:Hentaigana]{Hentaigana}&Hentaigana\\
37&漫画&まんが&\hyperref[sec:Manga]{manga}&Manga\\
38&万葉仮名&まんようがな&\hyperref[sec:Manyogana]{Man'yōgana}&Man'yōgana\\
39&万葉集&まんようしゅう&\hyperref[sec:Manyoshu]{Man'yōshu}&Man'yōshu\\
40&モーラ&もーら&\hyperref[sec:Mora]{mora}&Mora\\
41&読み仮名&よみがな&\hyperref[sec:Yomigana]{Yomigana}&Yomigana\\
42&ルビ&るび&\hyperref[sec:Rubi]{rubi}&Rubi\\
43&ローマ字&ろーまじ&\hyperref[sec:Romaji]{Rōmaji}&Rōmaji\\
\end{longtable}
