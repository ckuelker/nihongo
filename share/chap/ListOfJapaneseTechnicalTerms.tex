\chapter{List of Japanese Technical Terms}
\label{chap:JapaneseTechnicalTerms}
\jchap{日本専門用語リスト}
\ifor{Japanese Technical Terms}{日本専門用語}{にほんせんもんようご}{japanische Fachbegriffe}
\footnotesize\Padding
\begin{longtable}[c]{p{.5cm}p{3.5cm}p{4cm}p{3.5cm}p{3.5cm}}
\textbf{\#}&\textbf{Japanese}&\textbf{Hiragana}&\textbf{English}&\textbf{German}\\ \hline
1&送り仮名&おくりがな&\hyperref[sec:Okurigana]{Okurigana}&Okuriagana\\
2&踊り字&おどりじ&\hyperref[sec:Dancingmark]{"dancing mark"}&"Tanzzeichen"\\
3&片仮名&かたかな&\hyperref[sec:Katakana]{Katakana}&Katakana\\
4&仮名&かな&\hyperref[sec:Kana]{Kana}&Kana\\
5&漢字&かんじ&\hyperref[sec:Kanji]{Kanji}&Kanji\\
6&くの字点&くのじてん&\hyperref[sec:Doublemultiplecharacter]{double multiple character}&viele Zeichen verdoppeln\\
7&修正ヘボン式ローマ字&しゅうせい・へぼんしき・ろうまじ&\hyperref[sec:NewerHepburnSystem]{newer Hepburn System}&neueres Hepburn System\\
8&濁点&だくてん&\hyperref[sec:Dakuten]{Dakuten}&Dakuten\\
9&長音&ちょうおん&\hyperref[sec:Choon]{Chōon}&Chōon\\
10&日本専門用語&にほんせんもんようご&\hyperref[sec:JapaneseTechnicalTerms]{Japanese Technical Terms}&japanische Fachbegriffe\\
11&半濁点&はんだくてん&\hyperref[sec:Handakuten]{Handakuten}&Handakuten\\
12&反復記号&はんぷくきごう&\hyperref[sec:Chrarcaterrepitionmark]{chrarcater repition mark}&Chracter Wiederholungszeichen\\
13&標準ヘボン式ローマ字&ひょうじゅん・へぼん・ろまあじ&\hyperref[sec:OlderHepburnSystem]{older Hepburn System}&altes Hepburn System\\
14&平仮名&ひらがな&\hyperref[sec:Hiragana]{Hiragana}&Hiragana\\
15&振り仮名&ふりがな&\hyperref[sec:Furigana]{Furigana}&Furigana\\
16&ヘボン式&へぼんしき&\hyperref[sec:HepburnSystem]{Hepburn System}&Hepburn System\\
17&変体仮名&へんたいがな&\hyperref[sec:Hentaigana]{Hentaigana}&Hentaigana\\
18&丸&まる&\hyperref[sec:Circle]{circle}&Kreis\\
19&万葉仮名&まんようがな&\hyperref[sec:Manyogana]{Man'yōgana}&Man'yōgana\\
20&モーラ&もーら&\hyperref[sec:Mora]{Mōra}&Mōra\\
21&読み仮名&よみがな&\hyperref[sec:Yomigana]{Yomigana}&Yomigana\\
22&ルビ&るび&\hyperref[sec:Rubi]{rubi}&Rubi\\
23&ローマ字&ろーまじ&\hyperref[sec:Romaji]{Rōmaji}&Rōmaji\\
\end{longtable}
