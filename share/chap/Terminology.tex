% ===========================================================================
\chapter{Terminology - 専門用語}

The following section (ordered Roman alphabetically) can be used by itself to
understand some key concepts of Japanese language by explaining keywords
{専門用語} {【せんもんようご】}.

% A
% B
% C
% D
% ---------------------------------------------------------------------------
\section{濁点 - Dakuten} \label{sec:Dakuten}

The {濁点} {【だくてん】} is a diacritic sign. Similar to the German Umlaut.
The {濁点} is referenced colloquial as {点々} {【てんてん】}.  It us used to in
{仮名} \hyperref[sec:Syllable]{syllabaries} to mark a consonant to be
pronounced voiced. Two strokes {「゙」} are used near the Katakana letter.  For
other {濁点}, please see \nameref{sec:Iteration}.

    % label sec:Dakuten
% ---------------------------------------------------------------------------
\section{Diphthong - 二重母音} \label{sec:Diphthong}

A diphthong {二重母音} {【にじゅうぼいん】} is a sound that is constructed from
two different sounds that glide into each other while pronouncing and form a
syllable. A diphthong is made out of vocals. Examples for a diphthong in
Japanese are {姪} |me.i| and {甥} |o.i|. Also  {「アエ」}, {「アイ」},
{「アウ」}, {「アオ、{「ウエ」}, {「ウイ」}, {「オエ」}, {「オイ」} or
{「オウ」} are likely to appear as a diphthong in normal conversation in
Japanese.  However, they becomes vowel connections when it is pronounced slowly
and it is treated as two vowels in the consciousness of the Japanese speaker.
  % label sec:Diphthong
% E
% F
% ---------------------------------------------------------------------------
\section{Furigana - 振り仮名} \label{sec:Furigana}


The Japanese \textbf{Furigana} - written in Japanese {振り仮名} {【ふりがな】}
- is an aid for reading \hyperref[sec:Kanji]{Kanji}. \textbf{Furigana} are
\hyperref[sec:Kana]{Kana}, so basically \hyperref[sec:Hiragana]{Hiragana} or
\hyperref[sec:Katakana]{Katakana}. \textbf{Furigana} are written next to the
character (mostly \hyperref[sec:Kanji]{Kanji}) which reading can not be
expected to be know mostly as annotative glosses. At first unknown or difficult
\hyperref[sec:Kanji]{Kanji} are candidates for \textbf{Furigana} but also in
books for Children some if not all \hyperref[sec:Kanji]{Kanji} have
\textbf{Furigana}. But even in books for learning English for example
\textbf{Furigana} can be found next to words written in
\hyperref[sec:Romaji]{Rōmaji}.

Text written horizontally \textbf{Furigana} are written mostly above the
referenced character. In vertically written text \textbf{Furigana} is written
on the right site next to the character. Good \textbf{Furigana} tries to place
the reading distinguishable to each character separately. So the
first example (Kanji+Hiragana) is not good. While the second (Kanji+Hiragana)
is a good usage of \textbf{Furigana}. 

\begin{center}
\begin{tabular}{rl}
 \normalsize over:&\Huge \ruby{東京}{とうきょう} 
 \ruby{東}{とう}\ruby{京}{きょう} 
 \ruby{東}{トー}\ruby{京}{キョー} 
 \ruby{東}{tō}\ruby{京}{kyō} \\
 \normalsize behind:& \Huge 東京(とうきょう)  東京【とうきょう】\\
 \end{tabular}
\end{center}

\begin{tabular}{ll}
\raisebox{10\height}{
 \framebox[20mm][r]{
 \rotatebox{-90}{
  \begin{minipage}{2.0cm} 
\setCJKfamilyfont{cjk-vert}[Script=CJK,RawFeature=vertical]{IPAPMincho}
\renewcommand{\rubysep}{-0.5ex}
  \CJKfamily{cjk-vert}
   \Huge \ruby{東}{とう}\ruby{京}{ きょう}
  \end{minipage}
 }
}
}
&\begin{minipage}{14cm}
Vertically written Tōkyō as it also can be seen on many signs.\smallskip

Other names for \textbf{Furigana} are Ruby/Rubi or Yomigana {読み仮名}
{【よみがな】}.  Ruby (Japanese {ルビ} /rubi/) is also a annotation system that
can be used in \LaTeX or HTML. Rubi are  also common in China, Taiwan and
Korea. \end{minipage}
\\
\end{tabular}
\bigskip

\begin{tabular}{ll}
\begin{minipage}{14cm}

A common example for using \textbf{Furigana} for adults would be to rename
(better re-read) single words to give them a specific connotation.  In science
fictions some astronaut could use the Japanese word {ふるさと} /furusato/  with
the meaning of "my hometown" to refer to the planet Earth ( =
{地球}{【ちきゅう】}). Or to make it more fancy and international (may be also
with connotation that Japan has no space in the future):

\end{minipage}&
\begin{minipage}{2cm}
\Huge \ruby{地球}{ふるさと} 
\end{minipage}\\
\end{tabular}

\begin{tabular}{lp{2cm}}
\begin{minipage}{14cm}
Here {アース} refers to 'earth', but {地球} is better understandable by the
Japanese audience.
\end{minipage}
&
\mbox{\Huge\ruby{地球}{アース} }
\\
\end{tabular}






   % label sec:Furigana
% G
% H
% ---------------------------------------------------------------------------
\section{Handakuten}\jsec{半濁点} \label{sec:Handakuten}
\ifor{Handakuten}{半濁点}{はんだくてん}{Handakuten}
\ifor{Dakuten}{濁点}{だくてん}{Dakuten}
\ifor{circle}{丸}{まる}{Kreis}
\ithree{"゚"}{「゚」}{"゚"}
\ien{|h|} \ide{|h|}
\ien{|p|} \ide{|p|}
\ien{pronunciation shift} \ide{Ausprache Verschiebung}

In Japanese two different {濁点} {【だくてん】} are used. The {濁点}  and  the
{半濁点} {【はんだくてん】} has the marker of a little circle {「゚」} and is
therefore colloquially described as {丸} {【まる】} and indicates when the
pronunciation shifts from |h| to |p|.

 % label sec:Handakuten
% ---------------------------------------------------------------------------
\section{Hepburn System}\jsec{ヘボン式}
%[o] LABEL
\label{sec:Hepburn}
\label{sec:HepburnSystem}
\label{sec:OlderHepburnSystem}
\label{sec:NewerHepburnSystem}
% [o] INDEX
\ifor{Hepburn system}{ヘボン式}{へぼんしき}{Hepburn System}
\ifor{older Hepburn system}{標準ヘボン式ローマ字}{ひょうじゅん・へぼん・ろまあじ}{altes Hepburn System}
\ifor{newer Hepburn system}{修正ヘボン式ローマ字}{しゅうせい・へぼんしき・ろうまじ}{neueres Hepburn System}
\ithree{James Curtis Hepburn}{James Curtis Hepburn}{James Curtis Hepburn}

\begin{tabular}{lr}
\begin{minipage}{10.5cm}

The { ヘボン式} {【へぼんしき】} is one of the two most important transcription
systems for Japanese written \hyperref[sec:Mora]{morae} based language. The
{ヘボン式} is most used system worldwide and in Japan.

The word {ヘボン} (hebon) is an old writing of the name \textbf{Hepburn}, a US
American physician, translator, educator and lay Christian missionary, who used
it his first Japanese English Dictionary (3rd ed.) in 1867.

There are manly two different variants. The older {標準ヘボン式ローマ字}
{【ひょうじゅん・へぼん・ろまあじ】} variant, which is used for signs at train
stations. And the new variant the {修正ヘボン式ローマ字}
{【しゅうせい・へぼんしき・ろうまじ】} which is used as a revised system since
1954 in Kenkyusha dictionaries. Most western scientists are using this system.
This system is also used in this book.

\Link \href{http://en.wikipedia.org/wiki/James_Curtis_Hepburn}{Hepburn}

\end{minipage}
&
\raisebox{-.47\height}{
\includegraphics[scale=0.5,trim= 00 00 00 00]{../share/ei/James_Curtis_Hepburn.jpg}}
\\
\end{tabular}


    % label sec:Hepburn
% ---------------------------------------------------------------------------
\section{Hiragana}\jsec{平仮名} 
% [o] LABEL
\label{sec:Hiragana}
% [o] INDEX
\ifor{Hiragana}{平仮名}{ひらがな}{Hiragana}

Approx. in the 9th century the \textbf{Hiragana} script - written in Japanese
as {平仮名} {【ひらがな】} - was developed by simplifying Chinese characters
used for pronunciation. The number of contemporary \textbf{Hiragana} where
reduced and today 46 are used.  \textbf{Hiragana} is a
\hyperref[sec:Mora]{morae} alphabet which is mostly constructed out of
syllables. In modern Japanese language \textbf{Hiragana} is used for
\hyperref[sec:Okurigana]{Okurigana} like  verb endings, other endings as well
as for phonetic transcription and for all other words which can or should not
be written with \hyperref[sec:Kanji]{Kanji}, except words which are written in
\hyperref[sec:Katakana]{Katakana}. In simple words: if it is not known weather
the word should be written in \hyperref[sec:Kanji]{Kanji} or
\hyperref[sec:Katakana]{Katakana} write in \textbf{Hiragana}.

   % label sec:Hiragana
% I
% ---------------------------------------------------------------------------
\section{Katakana Iteration Marks - ??? } \label{sec:Iteration}

As with {漢字} {【かんじ】} also {片仮名} has a iteration mark.  「ヽ」 and its
{濁点} {【だくてん】} form {「ヾ」}. This can only be
found in rare cases. For example the personal name Misuzu 【みすゞ】might
contain this character. And since the difference between the second last
and the last mora is only a change in pronunciation the {濁点} is added.

In vertical writing exist another iteration marker {くの字点} {【くのじてん】}
which consist out of two characters {「〳」+「〵」} and the {濁点} form
is {「〴」+「〵」}

  % label sec:Iteration
% J
% K
% ---------------------------------------------------------------------------
\section{Kana - 仮名} \label{sec:Kana}

TODO
       % label sec:Kana
% ---------------------------------------------------------------------------
\section{Kanji}\jsec{漢字} \label{sec:Kanji}
%[o] INDEX
\ifor{Kanji}{漢字}{かんじ}{Kanji}

1300 years ago the first endeavours where undertaken to display the Japanese
language with the only known alphabet in the region, the Chinese writing
system. While the Japanese language where hardly suited for the writing system
it was an  economical choice since the Chinese characters where well developed
at that time and introduced many new ideas in lexis. The 'borrowing' of Chinese
characters was not a one shot operation it took centuries and more then one
attempt. This long winded process led to the fact that some characters where
imported more then once from China from different times and different regions.
And because of this one Chinese character can have more then one pronunciation.
We hope that this will consolidate over the next centuries.  Today this
imported characters are known as \textbf{Kanji} in Japan.  \textbf{Kanji} is
written \textit{Hanzi} in Chinese and referencing the character from the Han
period of China. Even though today all Chinese based characters (and even some
self invented) are referenced nowadays as \textbf{Kanji}, it does not strictly
mean that they only from the Han period.

A standard Japanese text do contain \textbf{Kanji}. To master the Japanese
language over a certain level and to be over come the problem of personal
illiteracy in Japan it is highly encouraged to learn at least 600 to 800
characters. To become fully literate member of the Japanese society 2000 to
2300 \textbf{Kanji} should be learned.

Today  \textbf{Kanji} in written Japanese language are used for substantives/
nouns, verbs, adjectives and names.


      % label sec:Kanji
\section{Kunrei System - 訓令式ローマ字} \label{sec:Kunrei}

The modern Kunrei System {訓令式ローマ字} {【くんれいろうまじ】}  is the
official writing system of Japan. It was confirmed in 1994 by the Cabinet and
is available as ISO 3602:1989. The Kunrei System predecessor was introduced
1985 by Dr. Aikitsu Tanakadate as {日本式ローマ字} {【にほんしきろうまじ】} and
tries a more systematical approach to map Hiragana and Katakana to equal Roman
letters. The {五十音図} {【ごじゅうおんず】} in the {訓令式ローマ字} is as
follows:

\Info{訓令式ローマ字 - Kunrei System}{
\begin{center}
\begin{tabular}{|c|c|c|c|c|}\hline
   a & i& u& e& o\\\hline
   ka&ki&ku&ke&ko\\\hline
   sa&si&su&se&so\\\hline
   ta&ti&tu&te&to\\\hline
   na&ni&nu&ne&no\\\hline
   ha&hi&hu&he&ho\\\hline
   ma&mi&mu&me&mo\\\hline
   ya&  &yu&  &yo\\\hline
   ra&ri&ru&re&ro\\\hline
   wa&  &  &  & o\\\hline
     &  &  &  & n\\\hline
\end{tabular}
\end{center}
}

Even tough the system is official, many entities, like the train system, are
not using it. The use the Hepburn System.

The {訓令式ローマ字} is not part of this book. Please see \nameref{sec:Hepburn}
on page \pageref{sec:Hepburn} for the system in use.
     % label sec:Kunrei
% L
% M
% ---------------------------------------------------------------------------
\section{Manga - マンが} \label{sec:Manga
}

TODO
      % label sec:Manga
% ---------------------------------------------------------------------------
\section{Man'yōgana}\jsec{万葉仮名} \label{sec:Manyogana}

The development of distinct Japanese writing begun 600 AD by writers and
scholars reducing some Chinese characters to its bare phonetic value. The
meaning of this characters where ignored. Around 760 a collection of Japanese
poetry was published, the \Link
\href{http://en.wikipedia.org/wiki/Man%27y%C5%8Dsh%C5%AB}{万葉集
【まんようしゅう】}, in which Chinese characters where uses as phonetic
letters. In regard to {万葉集} {【まんようしゅう】} the characters are named
{万葉仮名} {【まんようがな】}

The origin of the \textbf{Man'yōgana} script in poetry and art lead to some
problems in the understanding for the reader. Since the usage of phonetic
Chinese characters where mixed with regular Chinese characters and the
reasoning about which character to use was more form and shape aesthetic then
pragmatic, the meaning was difficult to grasp.

However the royal household or other scholars did not see a necessity to change
the status quo, because the high aim was to write poetry and other texts in
Chinese and \textbf{Man'yōgana} was considered appropriate only for notes,
diaries and love letters.

\Note{Note}{\footnotesize By the end of the 8th Century 970
\hyperref[sec:Kanji]{{漢字} {【かんじ】}} where used to pronounce the 90
\hyperref[sec:Mora]{morae}. This directly shows that there was no bijective map
between sound and character. For |ka| for example the following
\textbf{Man'yōgana} can be used {「可」}, {「何」}, {「加」}, {「架」},
{「香」}, {「蚊」}, {「迦」}. }

The number of \textbf{Man'yōgana} from which \hyperref[sec:Katakana]{Katakana}
likely derived is smaller.  

\Hint{Man'yōgana used for creation of {片仮名} {【かたかな】}}{
\begin{center}
\begin{tabular}{|c||c|c|c|c|c|}\hline
 & a& i  & u  & e& o\\\hline\hline
-&阿&伊  &宇  &江&於\\\hline
k&加&機幾&久  &介&己\\\hline
s&散&之  &須  &世&曽\\\hline
t&多&千  &州川&天&止\\\hline
n&奈&仁  &奴  &祢&乃\\\hline
h&八&比  &不  &部&保\\\hline
m&末&三  &牟  &女&毛\\\hline
y&也&    &由  &  &與\\\hline
r&良&利  &流  &礼&呂\\\hline
w&和&井  &    &恵&乎\\\hline
*&尓&    &    &  &  \\\hline
\end{tabular}
\end{center}
}

The scientific term \textbf{Man'yōgana} is used by Western and Japanese
scientists. However it is not without critique. The term \textbf{Man'yōgana}
might lead to the illusion that it was a defined set of characters in use for
transcribing Chinese or writing Japanese texts or the second illusion that one
sound is represented by only  one \textbf{Man'yōgana}. Both is not true. First,
all Chinese Characters could in principle be used as \textbf{Man'yōgana} (and
therefore the term is basically useless). Actually the reason to chose one
character was sometimes just because out of aesthetic reasons, the shape or
some additional meaning. And second, normally many different
\textbf{Man'yōgana} (Chinese characters) where used for the same pronunciation
in the same text.  Making it efficient or easy was not the target of the
scholars using this kind of \hyperref[sec:PhoneticCharacter]{phonetic
characters} at that time.


\Link \href{http://en.wikipedia.org/wiki/Manyogana}{Man'yōgana}
\Link \href{http://en.wikipedia.org/wiki/Man%27y%C5%8Dsh%C5%AB}{万葉集}

  % label sec:Manyogana
% ---------------------------------------------------------------------------
\section{Mora - モーラ} \label{sec:Mora}

The concept of \textbf{mora}  (plural morae or moras; often symbolized μ) is
used in the science of linguistics. It describes a joint unit in pronunciation
(phonology) that constructs a syllable. The definition of a \textbf{mora} can
vary.  In Japanese the detection of \textbf{morae} is comparably simple. The
world {「チョコレート」} for example consist out of the following 5
\textbf{morae} {「チョ」},{「コ」},{「レ」},{「ー」} and {「ト」} while it
consist only out of four \hyperref[sec:Syllable]{syllables}
{(\hyperref[sec:Syllable]{音節} 【おんせつ】)} {「チョ」},{「コ」},{「レー」}
and {「ト」}.

       % label sec:Mora
% N
% O
% ---------------------------------------------------------------------------
\section{Okurigana}\jsec{送り仮名}
% [o] LABEL
\label{sec:Okurigana}
\label{sec:Nokurigana}
% [o] INDEX
\ifor{Okurigana}{送り仮名}{おくりあがな}{Okurigana}
\ifor{Katakana}{片仮名}{かたかな}{Katakana}
\ifor{Kana}{仮名}{かな}{Kana}
\ifor{Hiragana}{平仮名}{ひらがな}{Hiragana}
\ifor{Kanji}{漢字}{かんじ}{Kanji}
\ifor{Nokurigana}{ノくり仮名}{のくりがな}{Nokurigana}

The term \textbf{Okurigana} is written {送り仮名}{【おくりがな】} in Japanese,
but it is \textit{not} a script by its own as the name
\hyperref[sec:Kana]{Kana} suggest.  \textbf{Okurigana} are
\hyperref[sec:Kana]{Kana} but either \hyperref[sec:Hiragana]{Hiragana} or
\hyperref[sec:Katakana]{Katakana} that are used to write the ending of words in
most cases verbs. More precise \textbf{Okuriagna} are suffixes of
\hyperref[sec:Kanji]{Kanji}. After 1945 only
\hyperref[sec:Hiragana]{Hiragana} are used to write \textbf{Okurigana} while
before \hyperref[sec:Katakana]{Katakana} was used. 

\textbf{Okurigana} are the mandatory compromise using static Chinese letters to
write the Japanese language. Next to make \hyperref[sec:Kanji]{Kanji} flexible
the other function is to mark the beginning are ending of words in sentences. 

\textbf{Okurigana} have two purposes. (1) conjugate (a) verbs and (b)
adjectives. With very few exceptions\footnote{ {皮肉る} {【ひにくる】},
{牛耳る}  {【ぎゅうじる】} and {退治る} {【たいじる】}.}  Okuriagna will only
inflect \hyperref[sec:Kanji]{Kanji} as Kun'yomi.  (2) Change the meaning or
reading of a \hyperref[sec:Kanji]{Kanji} by different \textbf{Okurigana}.

\textit{Example: Okuriagana change the meaning (tense):}

\begin{center}\begin{tabular}{ll}
(1) {見る} {【みる】} & see \\
(2) {見た} {【みた】} & saw \\
\end{tabular}\end{center}

In the above example the \textbf{Okurigana} of (1) is {「る」} and the
\textbf{Okurigana} of (2) is {「た」}.

\textit{Example: Okuriagana change the reading:}

\begin{center}\begin{tabular}{ll}
(1) {下さる} {【くださる】} & to give \\
(2) {下りる} {【おりる】} &  to get off (a train for example)/ to descend \\
(3) {下がる} {【さがる】} &  to dangle (intransitive)\\
\end{tabular}\end{center}

So in many cases the \textbf{Okurigana} directly after the
\hyperref[sec:Kanji]{Kanji} changes the meaning.

\textit{Example: Okuriagana change the meaning (transitivity) :}

\begin{center}\begin{tabular}{ll}
(1) {下がる} {【さがる】} &  to dangle (intransitive)\\
(2){下げる} {【さげる】} &  to let off (transitive)\\
\end{tabular}\end{center}

As in the above case many Japanese verbs come in transitive and intransitive
pairs.  The reading of the \hyperref[sec:Kanji]{Kanji} is often shared. 

\subsection*{Okurigana in the Middle}

\textbf{Okurigana} can also be found in the middle of Japanese words.

\textit{Example:}

\begin{center}\begin{tabular}{ll}
(1) {繰り返し} {【くりかえし】} &  to repeat\\
\end{tabular}\end{center}

\subsection*{Invisible Okuriagna - ノくり仮名}

The term {ノくり仮名} {【のくりがな】} was inspired by the site
\texttt{http://kanjidamage.com} but the writing was changed from Rōmaji to
Katakana+Okurigana+Kanji (The \hyperref[sec:Katakana]{Katakana} {「ノ」}
derives (of course) from the English 'no', and the word as such is a violation
of the Japanese \textbf{Okurigana}\footnote{Because
\hyperref[sec:Katakana]{Katakana} do not have \textbf{Okurigana}. But also in
case there would be no violation the /o/ of /okuri/ would be vilify to a
honorific prefix and then to be ripped out by the 'no' in a very non polite
way.} which describes a violation of \textbf{Okurigana}) Of course the term  is
not official, but quite funny in this case, that basically one should be very
angry with the fact that there are some Japanese words witch do have
\textbf{Okurigana} but are not written (but of course pronounced!).  The not so
funny part with those words is that if one knows the reading of the
\hyperref[sec:Kanji]{Kanji} it is impossible to look them up in a dictionary.
So lets strike back and spread the word of the ノくり仮名.

\begin{center}\begin{tabular}{ll}
(1) {取引} {取り引き} {【と(り)ひ(き)】} &  Transaction\\
(2) {受付} {受け付け} {【う(け)つ(け)】} &  Reception\\
\end{tabular}\end{center}


\Link \href{http://kanjidamage.com/tags/43}{http://kanjidamage.com/tags/43}








  % label sec:Okurigana
% P
\section{Phonetic Character - } \label{sec:PhoneticCharacter}

In this document the term \textbf{Phonetic Character} refers genetically to a
Chinese characters reading and the usage of this character just for this
purpose and \textit{not} for its meaning. This common set expression has been
used in avoidance of the term \hyperref[sec:Manyogana]{Manyogana}. See the
section \nameref{sec:Manyogana} on page \pageref{sec:Manyogana} to understand
the critique.

 % label sec:PoneticCharacter
% Q
% R
% ---------------------------------------------------------------------------
\section{部首- Radical} \label{sec:Radical}

A radical {部首} {【ぶしゅ】} is a root particle or character of a Sino-Japanese 
character {漢字} {【かんじ】}. It is the most significant part of a Sino-Japanese
character. The concept was developed in China for Chinese characters and is
today known under the same name {部首} (pinyin: bùshǒu).

There is no general definition what a radical is or how many are existing and it 
can vary a lot. The author of a dictionary has the power to defined what a radical
is and how much there will be in that dictionary.

In more traditional Chinese or Japanese dictionaries a number of 214 or 244
radicals is quite common. However some modern approaches like the
\href{http://www.hadamitzky.de/english/works_books.htm#KD}{\textit{The Kanji
Dictionary} of Marc Spahn and Wolfgang Hadamitzky from 1996} a totally
different number of 79 can be found.

\Note{Note}{\footnotesize Before buying a {漢字} dictionary, make sure that the
radical system used suits your taste. Sometimes it can be observed that
Japanese dictionaries are stricter in the definition of a radical because a
given {漢字} can only be retrieved via exactly \textbf{one} radical. While in
many Chinese dictionaries \textbf{every} radical of a Chinese character can be
used to find it. The Japanese approach is of course good in terms of systematic
and didactic for learners, however it can take significant longer to look up a
character by radical.  }

    % label sec:Radical
% ---------------------------------------------------------------------------
\section{Rōmaji  - ローマ字} \label{sec:Romaji}


% ---------------------------------------------------------------------------
\ifor{Rōmaji}{ローマ字}{ろーまじ}{Rōmaji}

In temporary Japan words written in western letters become more popular and
some parts of the written language is already westernized, like (Indian/
Arabic) numbers written in horizontal text almost per default. This western
Latin letters are called \textbf{Rōmaji} and are written in Japanese as
{ローマ字} {【ろおまじ】}, even though some of them are from different origin
like Indian numbers for example.



The western characters are mainly used for writing numbers in the horizontal
writing. Also for abbreviations capital and small letters are used. Sometimes
they are modified. For example the measurement of distance in the metric entity
"km" occupies to places in western scripts "k" + "m" while it only hold one
place in Japanese {「㎞」} or even one place in
\hyperref[sec:Katakana]{Katakana}  {「㌔」}. While the latter is ambiguous to
us, because colloquial kilogram is referenced as only "kilo".

\Note{One Space Rōmaji}{\begin{center}\small
\begin{tabular}{ll}
\textit{Western Multiple Space Letters}&\textit{One Space Rōmaji}\\
mg&㎎\\
mm&㎜\\
kg&㎏\\
cm&㎝\\
km&㎞\\
qm&㎡\\
qcc&㏄\\
\end{tabular}
\end{center}
}

There are other shapes of Rōmaji for numbers or letters:

%\fontspec{IPAPMincho}
\fontspec{IPAPGothic}

\begin{center}
\begin{tabular}{ll}
Roman       &ⅠⅡⅢⅣⅤⅥⅦⅧⅨⅩⅪⅫ...\\
Blac circle & ❶❷❸❹❺❻❼❽❾❿...\\
Withe circle &①②③④⑤⑥⑦⑧⑨⑩...\\
Withe double circle & ⓵⓶⓷⓸⓹⓺⓻⓼⓽⓾...\\
Letters             &ⓐⓑⓒ...\\
\end{tabular}
\end{center}
\newpage
\fontspec{FreeSans}

In a number of incidents in typography multiple Katakana are condensed
into one space, where normally only one Katakana would exist. In some cases the
direction of writing is even diagonal. This part of exception are not part of
this document and should be viewed under the peculiar aesthetic of Japanese
printing.

\Note{One Space Katakana}{\begin{center}\small
\begin{tabular}{ll}
\textit{Western Meaning}&\textit{One Space Katakana}\\
&㌃\\
calorie&㌍\\
kilo&㌔\\
gram&㌘\\
centi-&㌢\\
cent&㌣\\
\$&㌦\\
t&㌧\\
\%&㌫\\
ha&㌶\\
pages&㌻\\
milli-&㍉\\
mbar (millibar)&㍊\\
m (meter)&㍍\\
l (liter)&㍑\\
&㍗\\
\end{tabular}
\end{center}
}

Citation of foreign books are also done in western letters an can pop up
without warning the middle of the text.
     % label sec:Romaji
% S
% ---------------------------------------------------------------------------
\section{Syllable}\jsec{音節}
% [o] LABEL
\label{sec:Syllable}
% [o] INDEX
\ifor{syllable}{音節}{おんせつ}{Silbe} % DESTINATION
\ifor{mora}{モーラ}{もーら}{Mora}      % TARGET

A \textbf{syllable} {音節}  {【おんせつ】}  is a phonetic building block for
words. It influences the rhythm of a spoken language. In Western languages a
\textbf{syllable} is made out of one or more letters. In Japanese it is often
one character (of \hyperref[sec:Kana]{Kana}), but not always. For a better
understanding of the Japanese it is important to understand the concept of
\hyperref[sec:Mora]{mora}.

\Link \href{http://en.wikipedia.org/wiki/Syllable}{Syllable}
\Link \href{http://ja.wikipedia.org/wiki/%E9%9F%B3%E7%AF%80}{音節}

   % label sec:Syllable
% T
% U
% V
% W
% X
% Y
% Z

