% ===========================================================================
\chapter{Terminology - 専門用語}

The following section (ordered Roman alphabetically) can be used by itself to
understand some key concepts of Japanese language. 

% A
% B
% C
% D
% ---------------------------------------------------------------------------
\section{濁点 - Dakuten} \label{sec:Dakuten}

The {濁点} {【だくてん】} is a diacritic sign. Similar to the German Umlaut.
The {濁点} is referenced colloquial as {点々} {【てんてん】}.  It us used to in
{仮名} \hyperref[sec:Syllable]{syllabaries} to mark a consonant to be
pronounced voiced. Two strokes {「゙」} are used near the Katakana letter.  For
other {濁点}, please see \nameref{sec:Iteration}.

    % label sec:Dakuten
% ---------------------------------------------------------------------------
\section{Diphthong - 二重母音} \label{sec:Diphthong}

A diphthong {二重母音} {【にじゅうぼいん】} is a sound that is constructed from
two different sounds that glide into each other while pronouncing and form a
syllable. A diphthong is made out of vocals. Examples for a diphthong in
Japanese are {姪} |me.i| and {甥} |o.i|. Also  {「アエ」}, {「アイ」},
{「アウ」}, {「アオ、{「ウエ」}, {「ウイ」}, {「オエ」}, {「オイ」} or
{「オウ」} are likely to appear as a diphthong in normal conversation in
Japanese.  However, they becomes vowel connections when it is pronounced slowly
and it is treated as two vowels in the consciousness of the Japanese speaker.
  % label sec:Diphthong
% E
% F
% G
% H
% ---------------------------------------------------------------------------
\section{Handakuten}\jsec{半濁点} \label{sec:Handakuten}
\ifor{Handakuten}{半濁点}{はんだくてん}{Handakuten}
\ifor{Dakuten}{濁点}{だくてん}{Dakuten}
\ifor{circle}{丸}{まる}{Kreis}
\ithree{"゚"}{「゚」}{"゚"}
\ien{|h|} \ide{|h|}
\ien{|p|} \ide{|p|}
\ien{pronunciation shift} \ide{Ausprache Verschiebung}

In Japanese two different {濁点} {【だくてん】} are used. The {濁点}  and  the
{半濁点} {【はんだくてん】} has the marker of a little circle {「゚」} and is
therefore colloquially described as {丸} {【まる】} and indicates when the
pronunciation shifts from |h| to |p|.

 % label sec:Handakuten
% ---------------------------------------------------------------------------
\section{Hepburn System}\jsec{ヘボン式}
%[o] LABEL
\label{sec:Hepburn}
\label{sec:HepburnSystem}
\label{sec:OlderHepburnSystem}
\label{sec:NewerHepburnSystem}
% [o] INDEX
\ifor{Hepburn system}{ヘボン式}{へぼんしき}{Hepburn System}
\ifor{older Hepburn system}{標準ヘボン式ローマ字}{ひょうじゅん・へぼん・ろまあじ}{altes Hepburn System}
\ifor{newer Hepburn system}{修正ヘボン式ローマ字}{しゅうせい・へぼんしき・ろうまじ}{neueres Hepburn System}
\ithree{James Curtis Hepburn}{James Curtis Hepburn}{James Curtis Hepburn}

\begin{tabular}{lr}
\begin{minipage}{10.5cm}

The { ヘボン式} {【へぼんしき】} is one of the two most important transcription
systems for Japanese written \hyperref[sec:Mora]{morae} based language. The
{ヘボン式} is most used system worldwide and in Japan.

The word {ヘボン} (hebon) is an old writing of the name \textbf{Hepburn}, a US
American physician, translator, educator and lay Christian missionary, who used
it his first Japanese English Dictionary (3rd ed.) in 1867.

There are manly two different variants. The older {標準ヘボン式ローマ字}
{【ひょうじゅん・へぼん・ろまあじ】} variant, which is used for signs at train
stations. And the new variant the {修正ヘボン式ローマ字}
{【しゅうせい・へぼんしき・ろうまじ】} which is used as a revised system since
1954 in Kenkyusha dictionaries. Most western scientists are using this system.
This system is also used in this book.

\Link \href{http://en.wikipedia.org/wiki/James_Curtis_Hepburn}{Hepburn}

\end{minipage}
&
\raisebox{-.47\height}{
\includegraphics[scale=0.5,trim= 00 00 00 00]{../share/ei/James_Curtis_Hepburn.jpg}}
\\
\end{tabular}


    % label sec:Hepburn
% I
% ---------------------------------------------------------------------------
\section{Katakana Iteration Marks - ??? } \label{sec:Iteration}

As with {漢字} {【かんじ】} also {片仮名} has a iteration mark.  「ヽ」 and its
{濁点} {【だくてん】} form {「ヾ」}. This can only be
found in rare cases. For example the personal name Misuzu 【みすゞ】might
contain this character. And since the difference between the second last
and the last mora is only a change in pronunciation the {濁点} is added.

In vertical writing exist another iteration marker {くの字点} {【くのじてん】}
which consist out of two characters {「〳」+「〵」} and the {濁点} form
is {「〴」+「〵」}

  % label sec:Iteration
% J
% K
% ---------------------------------------------------------------------------
\section{Kanji}\jsec{漢字} \label{sec:Kanji}
%[o] INDEX
\ifor{Kanji}{漢字}{かんじ}{Kanji}

1300 years ago the first endeavours where undertaken to display the Japanese
language with the only known alphabet in the region, the Chinese writing
system. While the Japanese language where hardly suited for the writing system
it was an  economical choice since the Chinese characters where well developed
at that time and introduced many new ideas in lexis. The 'borrowing' of Chinese
characters was not a one shot operation it took centuries and more then one
attempt. This long winded process led to the fact that some characters where
imported more then once from China from different times and different regions.
And because of this one Chinese character can have more then one pronunciation.
We hope that this will consolidate over the next centuries.  Today this
imported characters are known as \textbf{Kanji} in Japan.  \textbf{Kanji} is
written \textit{Hanzi} in Chinese and referencing the character from the Han
period of China. Even though today all Chinese based characters (and even some
self invented) are referenced nowadays as \textbf{Kanji}, it does not strictly
mean that they only from the Han period.

A standard Japanese text do contain \textbf{Kanji}. To master the Japanese
language over a certain level and to be over come the problem of personal
illiteracy in Japan it is highly encouraged to learn at least 600 to 800
characters. To become fully literate member of the Japanese society 2000 to
2300 \textbf{Kanji} should be learned.

Today  \textbf{Kanji} in written Japanese language are used for substantives/
nouns, verbs, adjectives and names.


      % label sec:Kanji
\section{Kunrei System - 訓令式ローマ字} \label{sec:Kunrei}

The modern Kunrei System {訓令式ローマ字} {【くんれいろうまじ】}  is the
official writing system of Japan. It was confirmed in 1994 by the Cabinet and
is available as ISO 3602:1989. The Kunrei System predecessor was introduced
1985 by Dr. Aikitsu Tanakadate as {日本式ローマ字} {【にほんしきろうまじ】} and
tries a more systematical approach to map Hiragana and Katakana to equal Roman
letters. The {五十音図} {【ごじゅうおんず】} in the {訓令式ローマ字} is as
follows:

\Info{訓令式ローマ字 - Kunrei System}{
\begin{center}
\begin{tabular}{|c|c|c|c|c|}\hline
   a & i& u& e& o\\\hline
   ka&ki&ku&ke&ko\\\hline
   sa&si&su&se&so\\\hline
   ta&ti&tu&te&to\\\hline
   na&ni&nu&ne&no\\\hline
   ha&hi&hu&he&ho\\\hline
   ma&mi&mu&me&mo\\\hline
   ya&  &yu&  &yo\\\hline
   ra&ri&ru&re&ro\\\hline
   wa&  &  &  & o\\\hline
     &  &  &  & n\\\hline
\end{tabular}
\end{center}
}

Even tough the system is official, many entities, like the train system, are
not using it. The use the Hepburn System.

The {訓令式ローマ字} is not part of this book. Please see \nameref{sec:Hepburn}
on page \pageref{sec:Hepburn} for the system in use.
     % label sec:Kunrei
% L
% M
% ---------------------------------------------------------------------------
\section{Man'yōgana}\jsec{万葉仮名} \label{sec:Manyogana}

The development of distinct Japanese writing begun 600 AD by writers and
scholars reducing some Chinese characters to its bare phonetic value. The
meaning of this characters where ignored. Around 760 a collection of Japanese
poetry was published, the \Link
\href{http://en.wikipedia.org/wiki/Man%27y%C5%8Dsh%C5%AB}{万葉集
【まんようしゅう】}, in which Chinese characters where uses as phonetic
letters. In regard to {万葉集} {【まんようしゅう】} the characters are named
{万葉仮名} {【まんようがな】}

The origin of the \textbf{Man'yōgana} script in poetry and art lead to some
problems in the understanding for the reader. Since the usage of phonetic
Chinese characters where mixed with regular Chinese characters and the
reasoning about which character to use was more form and shape aesthetic then
pragmatic, the meaning was difficult to grasp.

However the royal household or other scholars did not see a necessity to change
the status quo, because the high aim was to write poetry and other texts in
Chinese and \textbf{Man'yōgana} was considered appropriate only for notes,
diaries and love letters.

\Note{Note}{\footnotesize By the end of the 8th Century 970
\hyperref[sec:Kanji]{{漢字} {【かんじ】}} where used to pronounce the 90
\hyperref[sec:Mora]{morae}. This directly shows that there was no bijective map
between sound and character. For |ka| for example the following
\textbf{Man'yōgana} can be used {「可」}, {「何」}, {「加」}, {「架」},
{「香」}, {「蚊」}, {「迦」}. }

The number of \textbf{Man'yōgana} from which \hyperref[sec:Katakana]{Katakana}
likely derived is smaller.  

\Hint{Man'yōgana used for creation of {片仮名} {【かたかな】}}{
\begin{center}
\begin{tabular}{|c||c|c|c|c|c|}\hline
 & a& i  & u  & e& o\\\hline\hline
-&阿&伊  &宇  &江&於\\\hline
k&加&機幾&久  &介&己\\\hline
s&散&之  &須  &世&曽\\\hline
t&多&千  &州川&天&止\\\hline
n&奈&仁  &奴  &祢&乃\\\hline
h&八&比  &不  &部&保\\\hline
m&末&三  &牟  &女&毛\\\hline
y&也&    &由  &  &與\\\hline
r&良&利  &流  &礼&呂\\\hline
w&和&井  &    &恵&乎\\\hline
*&尓&    &    &  &  \\\hline
\end{tabular}
\end{center}
}

The scientific term \textbf{Man'yōgana} is used by Western and Japanese
scientists. However it is not without critique. The term \textbf{Man'yōgana}
might lead to the illusion that it was a defined set of characters in use for
transcribing Chinese or writing Japanese texts or the second illusion that one
sound is represented by only  one \textbf{Man'yōgana}. Both is not true. First,
all Chinese Characters could in principle be used as \textbf{Man'yōgana} (and
therefore the term is basically useless). Actually the reason to chose one
character was sometimes just because out of aesthetic reasons, the shape or
some additional meaning. And second, normally many different
\textbf{Man'yōgana} (Chinese characters) where used for the same pronunciation
in the same text.  Making it efficient or easy was not the target of the
scholars using this kind of \hyperref[sec:PhoneticCharacter]{phonetic
characters} at that time.


\Link \href{http://en.wikipedia.org/wiki/Manyogana}{Man'yōgana}
\Link \href{http://en.wikipedia.org/wiki/Man%27y%C5%8Dsh%C5%AB}{万葉集}

  % label sec:Manyogana
% ---------------------------------------------------------------------------
\section{Mora - モーラ} \label{sec:Mora}

The concept of \textbf{mora}  (plural morae or moras; often symbolized μ) is
used in the science of linguistics. It describes a joint unit in pronunciation
(phonology) that constructs a syllable. The definition of a \textbf{mora} can
vary.  In Japanese the detection of \textbf{morae} is comparably simple. The
world {「チョコレート」} for example consist out of the following 5
\textbf{morae} {「チョ」},{「コ」},{「レ」},{「ー」} and {「ト」} while it
consist only out of four \hyperref[sec:Syllable]{syllables}
{(\hyperref[sec:Syllable]{音節} 【おんせつ】)} {「チョ」},{「コ」},{「レー」}
and {「ト」}.

       % label sec:Mora
% N
% O
% P
\section{Phonetic Character - } \label{sec:PhoneticCharacter}

In this document the term \textbf{Phonetic Character} refers genetically to a
Chinese characters reading and the usage of this character just for this
purpose and \textit{not} for its meaning. This common set expression has been
used in avoidance of the term \hyperref[sec:Manyogana]{Manyogana}. See the
section \nameref{sec:Manyogana} on page \pageref{sec:Manyogana} to understand
the critique.

 % label sec:PoneticCharacter
% Q
% R
% ---------------------------------------------------------------------------
\section{部首- Radical} \label{sec:Radical}

A radical {部首} {【ぶしゅ】} is a root particle or character of a Sino-Japanese 
character {漢字} {【かんじ】}. It is the most significant part of a Sino-Japanese
character. The concept was developed in China for Chinese characters and is
today known under the same name {部首} (pinyin: bùshǒu).

There is no general definition what a radical is or how many are existing and it 
can vary a lot. The author of a dictionary has the power to defined what a radical
is and how much there will be in that dictionary.

In more traditional Chinese or Japanese dictionaries a number of 214 or 244
radicals is quite common. However some modern approaches like the
\href{http://www.hadamitzky.de/english/works_books.htm#KD}{\textit{The Kanji
Dictionary} of Marc Spahn and Wolfgang Hadamitzky from 1996} a totally
different number of 79 can be found.

\Note{Note}{\footnotesize Before buying a {漢字} dictionary, make sure that the
radical system used suits your taste. Sometimes it can be observed that
Japanese dictionaries are stricter in the definition of a radical because a
given {漢字} can only be retrieved via exactly \textbf{one} radical. While in
many Chinese dictionaries \textbf{every} radical of a Chinese character can be
used to find it. The Japanese approach is of course good in terms of systematic
and didactic for learners, however it can take significant longer to look up a
character by radical.  }

    % label sec:Radical
% S
% ---------------------------------------------------------------------------
\section{Syllable}\jsec{音節}
% [o] LABEL
\label{sec:Syllable}
% [o] INDEX
\ifor{syllable}{音節}{おんせつ}{Silbe} % DESTINATION
\ifor{mora}{モーラ}{もーら}{Mora}      % TARGET

A \textbf{syllable} {音節}  {【おんせつ】}  is a phonetic building block for
words. It influences the rhythm of a spoken language. In Western languages a
\textbf{syllable} is made out of one or more letters. In Japanese it is often
one character (of \hyperref[sec:Kana]{Kana}), but not always. For a better
understanding of the Japanese it is important to understand the concept of
\hyperref[sec:Mora]{mora}.

\Link \href{http://en.wikipedia.org/wiki/Syllable}{Syllable}
\Link \href{http://ja.wikipedia.org/wiki/%E9%9F%B3%E7%AF%80}{音節}

   % label sec:Syllable
% T
% U
% V
% W
% X
% Y
% Z

