Approx. in the 9th century the \textbf{Hiragana} script - written in Japanese
as {平仮名} {【ひらがな】} - was developed by simplifying Chinese characters
used for pronunciation. The number of contemporary \textbf{Hiragana} where
reduced and today 46 are used.  \textbf{Hiragana} is a
\hyperref[sec:Mora]{morae} alphabet which is mostly constructed out of
syllables. In modern Japanese language \textbf{Hiragana} is used for
\hyperref[sec:Okurigana]{Okurigana} like  verb endings, other endings as well
as for phonetic transcription and for all other words which can or should not
be written with \hyperref[sec:Kanji]{Kanji}, except words which are written in
\hyperref[sec:Katakana]{Katakana}. In simple words: if it is not known weather
the word should be written in \hyperref[sec:Kanji]{Kanji} or
\hyperref[sec:Katakana]{Katakana} write in \textbf{Hiragana}.
