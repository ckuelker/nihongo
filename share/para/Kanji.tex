
\ifor{Kanji}{漢字}{かんじ}{Kanji}

1300 years ago the first endeavours where undertaken to display the Japanese
language with the only known alphabet in the region, the Chinese writing
system. While the Japanese language were hardly suited for the writing system
it was an economical choice since the Chinese characters where well developed
at that time and introduced many new ideas in lexis. The 'borrowing' of Chinese
characters was not a one shot operation it took centuries and more than one
attempt. This long winded process led to the fact that some characters where
imported more than once from China from different times and different regions.
And because of this one Chinese character can have more than one pronunciation.
We hope that this will consolidate over the next centuries. Today this imported
characters are known as \textbf{Kanji} in Japan. \textbf{Kanji} is written
\textit{Hanzi} in Chinese and referencing the character from the Han period of
China. Even though today all Chinese based characters (and even some self
invented) are referenced nowadays as \textbf{Kanji}, it does not strictly mean
that they are only from the Han period.

A standard Japanese text do contain \textbf{Kanji}. To master the Japanese
language over a certain level and to overcome the problem of personal
illiteracy (in Japan) it is highly encouraged to learn at least 600 to 800
characters. To become a fully literate member of the Japanese society 2000 to
2300 \textbf{Kanji} should be learned.

Today \textbf{Kanji} in written Japanese language are used for substantives/
nouns, verbs, adjectives and names.
