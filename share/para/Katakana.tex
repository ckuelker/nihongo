
\ifor{Hiragana}{平仮名}{ひらがな}{Hiragana}
\ifor{Katakana}{片仮名}{かたかな}{Katakana}
\ifor{Manga}{漫画}{まんが}{manga, Comic}

At the same time as \hyperref[sec:Hiragana]{Hiragana}, also \textbf{Katakana}
letters where invented by simplifying the same Chinese characters used for
pronunciation. However the look and feel of \textbf{Katakana} is more 'square'
not so 'rounded' as \hyperref[sec:Hiragana]{Hiragana}.

\textbf{Katakana} is used today for writing words of foreign origin and for
emphasizing (in commercials or \hyperref[sec:Manga]{Manga}) as well as words in
the fauna or flora.

