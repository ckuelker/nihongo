\bigskip

\textbf{Conventions Used in this Document}\label{para:Conventions}

\bigskip

(1) The reading of Japanese characters (\hyperref[sec:Kanji]{Kanji}) are
\textbf{not} given in the section or chapter heading but as soon as possible.
If the reading is given it will be given in \hyperref[sec:Hiragana]{Hiragana}
script. To mark this reading it will start with a Japanese bracket {【} and end
with a Japanese bracket {】}.

\bigskip
\textit{Example:}

\bigskip
\begin{center}
     Kani {漢字} {【かんじ】}
\end{center}

\bigskip

(2) If readings of Japanese are also given in \hyperref[sec:Romaji]{Rōmaji}
according to the \hyperref[sec:Hepburn]{Hepburn system}, this is indicated by a
slash '/' at the beginning of the reading and a slash at the end.

\bigskip
\textit{Example:}

\bigskip
\begin{center}
     First Katakana letter {ア} {/a/}
\end{center}

\bigskip

(3) External hyper links are marked with an blue arrow.

\bigskip
\textit{Example:}

\bigskip
\begin{center}
     Please look at the download page for this document, if there is a new
     version\\ \Link
     \href{http://christian.kuelker.info/nihongo}{http://christian.kuelker.info/nihongo}
\end{center}

