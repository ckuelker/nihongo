
% ---------------------------------------------------------------------------

\phantomsection
% [o] LABEL
\label{para:Conventions}
% [o] INDEX TARGET
\ifor{Kanji}{漢字}{かんじ}{Kanji}
\ifor{Hiragana}{平仮名}{ひらがな}{Hiragana}
\ifor{Rōmaji}{ローマ字}{ろーまじ}{Rōmaji}
\ifor{Hepburn System}{ヘボン式}{へぼんしき}{Hepburn System}

\medskip

\textbf{Conventions Used in this Document}

\medskip

(1) The reading of Japanese characters (\hyperref[sec:Kanji]{Kanji}) are
\textbf{not} given in the section or chapter heading but as soon as possible.
If the reading is given it will be given in \hyperref[sec:Hiragana]{Hiragana}
script. To mark this reading it will start with a Japanese bracket {【} and end
with a Japanese bracket {】}.

\medskip
\textit{Example:}

\medskip
\begin{center} \Large Kanji {漢字} {【かんじ】} \end{center}
\medskip

(2) If readings of Japanese are also given in \hyperref[sec:Romaji]{Rōmaji}
according to the \hyperref[sec:Hepburn]{Hepburn system}, this is indicated by a
slash '/' at the beginning of the reading and a slash at the end.

\medskip
\textit{Example:}

\medskip
\begin{center} First Katakana letter \Large {ア} {/a/} \end{center}

\medskip

(3) External hyperlinks are marked with an blue arrow.

\medskip
\textit{Example:}

\medskip

\begin{center}

Please look at the download page for this document, if there is a new version\\
\Link
\href{http://christian.kuelker.info/nihongo}{http://christian.kuelker.info/nihongo}

\end{center}

