% ===========================================================================
% PREAMBLE STANDARD (English)
% ===========================================================================
\usepackage{fontspec}%           provides font selecting commands
\usepackage{xunicode}%           provides unicode character macros
\usepackage{xltxtra} %           provides some fixes/extras
%\usepackage[utf8x]{inputenc} %  problems with xetex?
%\usepackage[utf8]{inputenc} %   problems with xetex?
%\usepackage[T1]{fontenc} %      problems with xetex and UF8?
\usepackage{tabularx}%           order matters
\usepackage{tikz}%               order matters
%\usepackage{pgf-pie} %          order matters
\usepackage{wallpaper} %         order matters
\usepackage{tabu} %              make thick lines
\usepackage{colortbl} %          make color lines
\usepackage[]{graphicx}
\usepackage{xcolor}
\usepackage{hyperref}%           configured later!
\usepackage{fullpage,longtable}% better tables
\usepackage{hhline}%             better tables
\setlength{\parindent}{0cm}
\setlength{\arrayrulewidth}{2pt}

% ===========================================================================
% SET FONTS 
%\usepackage{pifont}
%\usepackage{helvet}
\usepackage{latexsym}
\usepackage{amssymb,amsmath}

%\setmainfont{Ume Mincho S3}    % Serif      2 OK
%\setmainfont{AR PL UMing HK}   % Typewriter 1 GOOD
%\setmainfont{Sawarabi Gothic}   % SanSerif   1 GOOD
%\setmainfont{WenQuanYi Zen Hei}% SanSerif   2 OK (wrong g?)
%\setmainfont{Komatuna P}       % SanSerif   3 MAMA (but strong!)

% Should be LMRoman12 though
%\setmainfont[Mapping=tex-text]{LMRoman10}
%\setsansfont{AR PL UMing HK}
%\setmonofont{AR PL UMing HK}

%\setmainfont{IPAPMincho}
%\setmonofont{IPAPGothic}
%\setsansfont{IPAPMincho}
%\setmainfont{IPAPMincho}
%\setmonofont{IPAMincho}
%\setsansfont{IPAPMincho}

%\fontspec[
%ItalicFont={IPAGothic},
%BoldFont={IPAGothic},
%ItalicFeatures={FakeSlant},
%BoldFeatures={FakeBold=1.5},
%]{IPAPMincho}

%BoldFont = URW Palladio L, 
%ItalicFont = URW Palladio L,
%BoldItalicFont = URW Palladio L,
%SlantedFont = URW Palladio L,
%BoldSlantedFont = URW Palladio L,
%SmallCapsFont = URW Palladio L,
%UprightFont = *-Roman
%\setmainfont{YOzFontE}   % SanSerif   1 GOOD
%\setmonofont{AR PL UMing HK}

% ------  2014-09-06 -----
%\defaultfontfeatures{Ligatures=TeX}
%\setmainfont[SlantedFont={LMRoman10},
%             SlantedFeatures={FakeSlant=0.2}]{LMRoman10}
%\setsansfont[SlantedFont={LMRoman10},
%             SlantedFeatures={FakeSlant=0.2}]{LMRoman10}
\addtokomafont{footnote}{\footnotesize\sffamily}

\usepackage{array}% for vertical alignment in tabular with m{SIZE}
\usepackage{xeCJK}
\usepackage{ruby}
\renewcommand{\rubysep}{0.25ex}
\setCJKmainfont{IPAPMincho}
\setCJKfamilyfont{JapaneseDejima}{Dejima}
\setCJKfamilyfont{JapaneseYOzFont}{YOzFont90} %hand
\setCJKfamilyfont{JapaneseYOzFontA}{YOzFontA}
\setCJKfamilyfont{JapaneseYOzFontC}{YOzFontC90}
\setCJKfamilyfont{JapaneseYOzFontE}{YOzFontE90}
\setCJKfamilyfont{JapaneseYOzFontF}{YOzFontF90}
\setCJKfamilyfont{JapaneseYOzFontM}{YOzFontN90}
\setCJKfamilyfont{JapaneseYOzFontP}{YOzFontP90}
\setCJKfamilyfont{JapaneseIPAPGothic}{IPAPGothic}
\setCJKfamilyfont{JapaneseDefault}{IPAPMincho}
\setCJKfamilyfont{JapaneseMikachanPB}{mikachan-PB}
\newcommand\JapaneseYOzFont{\CJKfamily{JapaneseYOzFont}\CJKnospace}
\newcommand\JapaneseYOzFontA{\CJKfamily{JapaneseYOzFontA}\CJKnospace}
\newcommand\JapaneseYOzFontC{\CJKfamily{JapaneseYOzFontC}\CJKnospace}
\newcommand\JapaneseYOzFontE{\CJKfamily{JapaneseYOzFontE}\CJKnospace}
\newcommand\JapaneseYOzFontF{\CJKfamily{JapaneseYOzFontF}\CJKnospace}
\newcommand\JapaneseYOzFontM{\CJKfamily{JapaneseYOzFontM}\CJKnospace}
\newcommand\JapaneseYOzFontP{\CJKfamily{JapaneseYOzFontP}\CJKnospace}
\newcommand\JapaneseIPAPGothic{\CJKfamily{JapaneseIPAPGothic}\CJKnospace}
\newcommand\JapaneseIPAPMincho{\CJKfamily{JapaneseIPAPMincho}\CJKnospace}
\newcommand\JapaneseGothic{\CJKfamily{JapaneseIPAPGothic}\CJKnospace}
\newcommand\JapaneseDejima{\CJKfamily{JapaneseDejima}\CJKnospace}
\newcommand\JapaneseDefault{\CJKfamily{JapaneseDefault}\CJKnospace}
\newcommand\JapaneseMikachanPB{\CJKfamily{JapaneseMikachanPB}\CJKnospace}

%\setCJKsansfont[]{}
%\setCJKmonofont[...]{...}

% Should be LMRoman12 though
%\newcommand{\Circ}[1]{\fontspec{Sawarabi Gothic}#1\setmainfont{LMRoman10}}

% ===========================================================================
% WRITE JAPANESE TEXT VERTICALLY
% derived from jltxdoc.cls and plext.dtx
%\usepackage{plext}
\def\tsample#1{%
%  \hbox to\linewidth\bgroup\vrule width.1pt\hss
  \hbox to 150mm \bgroup\vrule width.1pt\hss
    \vbox\bgroup\hrule height.1pt
      \vskip.5\baselineskip
%      \vbox to\linewidth\bgroup\tate\hsize=#1\relax\vss}
      \vbox to 150mm \bgroup\tate\hsize=#1\relax\vss}
\def\endtsample{%
      \vss\egroup
      \vskip.5\baselineskip
    \hrule height.1pt\egroup
  \hss\vrule width.1pt\egroup}

% ---------------------------------------------------------------------------
% ADD THIS PACKAGES INCASE latex AND NOT platex is used:
% \documentclass[titlepage,12pt,a4paper,german]{book}
% PAKETE
% Deutsche Umlaute etc.
% \usepackage{2up}
% \usepackage{natbib}
% \bibpunct[;]{(}{)}{;}{a}{,}{,}
% Sprache (nach natbib)
% \usepackage{babel}
% pakete aus deutscher distribution
% \usepackage{2up}


% ===========================================================================
% Better hypphenation - more space
\sloppy
\hyphenation{
bei-spiels-wei-se
chi-nesi-sche 
chi-nesi-sch-en
ein-ge-se-tzt
Ge-sell-schafts-ge-spra-che
gram-mati-schen
Ja-pa-ni-sch 
Kanji-Kana-Majiri-Bun 
Schrei-bung 
Schrift-sys-tem
Schrift-zei-chen 
}


% ===========================================================================
% PAGELAYOUT
\usepackage{geometry}
\geometry{
  top=1in,            % <-- you want to adjust this 0.5
  inner=.8in,
  outer=.6in,
  bottom=1.5in,
  footskip=5ex,   
  headheight=5ex,       % <-- and this 3
  headsep=5ex,          % <-- and this 2
}
% better head lines
\usepackage{fancyhdr}
\pagestyle{fancy}
%\fancyhead{30pt}
\setlength\headheight{25.5pt}


% ===========================================================================
% NEW DEFINITIONS FOR SYSTEM COMMANDS
\renewcommand{\chaptermark}[1]{\markboth{\textsf{\scriptsize \thechapter.\ #1\normalsize}}{}}
\renewcommand{\sectionmark}[1]{\markright{\textsf{\scriptsize\thesection.\ #1\normalsize}}{}}
 
%\renewcommand{\chaptermark}[1]{%
%\markboth{\textsf{\scriptsize \thechapter.%
%\ \chaptername:\ #1\normalsize}}{}}
%\renewcommand{\sectionmark}[1]{%
%\markright{\textsf{\scriptsize\thesection.\ #1\normalsize}}{}}

% ===========================================================================
% SETTING OWN VALUES
\setcounter{secnumdepth}{3}
\setcounter{tocdepth}{3} 
%\setlength{\baselineskip}{0pt}
%\setlength{\parskip}{\smallskipamount}
%\setlength{\parindent}{0pt}
 
% ===========================================================================
% BETTER PLACING of fig and tab ENVIRONMENTS
\usepackage{flafter}

% ===========================================================================
% LINKS
\providecolor{myblue}{rgb}{0,0.49995,1}
\hypersetup{colorlinks, 
          citecolor=myblue,
          filecolor=myblue,
          linkcolor=myblue,
          urlcolor=myblue}
% ===========================================================================
% FOR BOXES
\usepackage{microtype}
\usepackage[framemethod=TikZ]{mdframed}
\usepackage{tcolorbox}
\usepackage[tikz]{bclogo}
\usepackage{lipsum}

% ===========================================================================
% REVISION DATE VERSION
\include{VERSION}
\include{DATE}
\include{revision}

% ===========================================================================
% INDEX
\usepackage{makeidx}
\makeindex
