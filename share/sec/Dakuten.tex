% ---------------------------------------------------------------------------
\section{Dakuten}\jsec{濁点} \label{sec:Dakuten}
\ithree{diacritic sign}{濁点}{Diakritisches Zeichen}
\ithree{diacritic sign}{だくてん}{Nigorierungszeichen}
\ithree{Umlaut}{ウムラウト}{Umlaut}
\ithree{syllabe}{音節}{Silbe}

The \textbf{Dakuten} - Japanese {濁点} {【だくてん】} - is a diacritic sign.
Similar to the German Umlaut.  The {濁点} is referenced colloquial as {点々}
{【てんてん】}.  It us used to in {仮名} \hyperref[sec:Syllable]{syllabaries}
to mark a consonant to be pronounced voiced. Two strokes {「゙」} are used near
the Katakana letter.  For other {濁点}, please see \nameref{sec:Iteration} on
page \pageref{sec:Iteration}.

