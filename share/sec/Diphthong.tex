% ---------------------------------------------------------------------------
\section{Diphthong}\jsec{二重母音} \label{sec:Diphthong}
\ithree{diphthong}{二重母音}{Diphthong}
\ithree{diphthong}{にじゅうぼいん}{Diphthong}
\ithree{syllable}{音節}{Silbe}
\ija{「アエ」}
\ija{「アイ」}
\ija{「アウ」}
\ija{「アオ」}
\ija{「ウエ」}
\ija{「ウイ」}
\ija{「オエ」}
\ija{「オイ」}
\ija{「オウ」}

A \textbf{diphthong} {二重母音} {【にじゅうぼいん】} is a sound that is
constructed from two different sounds that glide into each other while
pronouncing and form a \hyperref[sec:Syllable]{syllable}. A \textbf{diphthong}
is made out of vocals.  Examples for a \textbf{diphthong} in Japanese are {姪}
|me.i| and {甥} |o.i|.  Also  {「アエ」}, {「アイ」}, {「アウ」},
{「アオ」}、{「ウエ」}, {「ウイ」}, {「オエ」}, {「オイ」} or {「オウ」} are
likely to appear as a \textbf{diphthong} in normal conversation in Japanese.
However, they becomes vowel connections when it is pronounced slowly and it is
treated as two vowels in the consciousness of the Japanese speaker.
