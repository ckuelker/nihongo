% ---------------------------------------------------------------------------
\section{二重母音 - Diphthong} \label{sec:Diphthong}

A diphthong {二重母音} {【にじゅうぼいん】} is a sound that is constructed from
two different sounds that glide into each other while pronouncing and form a
syllable. A diphthong is made out of vocals. Examples for a diphthong in
Japanese are {姪} |me.i| and {甥} |o.i|. Also  {「アエ」}, {「アイ」},
{「アウ」}, {「アオ、{「ウエ」}, {「ウイ」}, {「オエ」}, {「オイ」} or
{「オウ」} are likely to appear as a diphthong in normal conversation in
Japanese.  However, they becomes vowel connections when it is pronounced slowly
and it is treated as two vowels in the consciousness of the Japanese speaker.
