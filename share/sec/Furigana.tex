% ---------------------------------------------------------------------------
\section{Furigana - 振り仮名} \label{sec:Furigana}


The Japanese \textbf{Furigana} - written in Japanese {振り仮名} {【ふりがな】}
- is an aid for reading \hyperref[sec:Kanji]{Kanji}. \textbf{Furigana} are
\hyperref[sec:Kana]{Kana}, so basically \hyperref[sec:Hiragana]{Hiragana} or
\hyperref[sec:Katakana]{Katakana}. \textbf{Furigana} are written next to the
character (mostly \hyperref[sec:Kanji]{Kanji}) which reading can not be
expected to be know mostly as annotative glosses. At first unknown or difficult
\hyperref[sec:Kanji]{Kanji} are candidates for \textbf{Furigana} but also in
books for Children some if not all \hyperref[sec:Kanji]{Kanji} have
\textbf{Furigana}. But even in books for learning English for example
\textbf{Furigana} can be found next to words written in
\hyperref[sec:Romaji]{Rōmaji}.

Text written horizontally \textbf{Furigana} are written mostly above the
referenced character. In vertically written text \textbf{Furigana} is written
on the right site next to the character. Good \textbf{Furigana} tries to place
the reading distinguishable to each character separately. So the
first example (Kanji+Hiragana) is not good. While the second (Kanji+Hiragana)
is a good usage of \textbf{Furigana}. 

\begin{center}
\begin{tabular}{rl}
 \normalsize over:&\Huge \ruby{東京}{とうきょう} 
 \ruby{東}{とう}\ruby{京}{きょう} 
 \ruby{東}{トー}\ruby{京}{キョー} 
 \ruby{東}{tō}\ruby{京}{kyō} \\
 \normalsize behind:& \Huge 東京(とうきょう)  東京【とうきょう】\\
 \end{tabular}
\end{center}

\begin{tabular}{ll}
\raisebox{10\height}{
 \framebox[20mm][r]{
 \rotatebox{-90}{
  \begin{minipage}{2.0cm} 
\setCJKfamilyfont{cjk-vert}[Script=CJK,RawFeature=vertical]{IPAPMincho}
\renewcommand{\rubysep}{-0.5ex}
  \CJKfamily{cjk-vert}
   \Huge \ruby{東}{とう}\ruby{京}{ きょう}
  \end{minipage}
 }
}
}
&\begin{minipage}{14cm}
Vertically written Tōkyō as it also can be seen on many signs.\smallskip

Other names for \textbf{Furigana} are Ruby/Rubi or Yomigana {読み仮名}
{【よみがな】}.  Ruby (Japanese {ルビ} /rubi/) is also a annotation system that
can be used in \LaTeX or HTML. Rubi are  also common in China, Taiwan and
Korea. \end{minipage}
\\
\end{tabular}
\bigskip

\begin{tabular}{ll}
\begin{minipage}{14cm}

A common example for using \textbf{Furigana} for adults would be to rename
(better re-read) single words to give them a specific connotation.  In science
fictions some astronaut could use the Japanese word {ふるさと} /furusato/  with
the meaning of "my hometown" to refer to the planet Earth ( =
{地球}{【ちきゅう】}). Or to make it more fancy and international (may be also
with connotation that Japan has no space in the future):

\end{minipage}&
\begin{minipage}{2cm}
\Huge \ruby{地球}{ふるさと} 
\end{minipage}\\
\end{tabular}

\begin{tabular}{lp{2cm}}
\begin{minipage}{14cm}
Here {アース} refers to 'earth', but {地球} is better understandable by the
Japanese audience.
\end{minipage}
&
\mbox{\Huge\ruby{地球}{アース} }
\\
\end{tabular}






