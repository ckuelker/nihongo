% ---------------------------------------------------------------------------
\section{Furigana - 振り仮名} \label{sec:Furigana}


The Japanese \textit{Furigana} - written in Japanese {振り仮名} {【ふりがな】}
- is an aid for reading \hyperref[sec:Kanji]{Kanji}. \textit{Furigana} are
\hyperref[sec:Kana]{Kana}, so basically \hyperref[sec:Hiragana]{Hiragana} or
\hyperref[sec:Katakana]{Katakana}. \textit{Furigana} are written next to the
character (mostly \hyperref[sec:Kanji]{Kanji}) which reading can not be
expected to be know mostly as annotative glosses. At first unknown or difficult \hyperref[sec:Kanji]{Kanji}
are candidates for \textit{Furigana} but also in books for Children some if not
all \hyperref[sec:Kanji]{Kanji} have \textit{Furigana}. But even in books for
learning English for example \textit{Furigana} can be found next to words
written in \hyperref[sec:Romaji]{Rōmaji}.

Text written horizontally \textit{Furigana} are written mostly above the
referenced character. In vertically written text \textit{Furigana} is written
on the right site next to the character. Good \textit{Furigana} tries to place
the reading distinguishable to each character separately. So the
first example (Kanji+Hiragana) is not good. While the second (Kanji+Hiragana)
is a good usage of \textit{Furigana}. 

\begin{center}
\begin{tabular}{rl}
 \normalsize over:&\Huge \ruby{東京}{とうきょう}
 \ruby{東}{とう}\ruby{京}{きょう}
 \ruby{東}{トー}\ruby{京}{キョー}
 \ruby{東}{tō}\ruby{京}{kyō} \\
 \normalsize behind:& \Huge 東京(とうきょう) 東京【とうきょう】\\
 \end{tabular}
\end{center}


Vertically written Tōkyō as it also can be seen on many signs.

\begin{center}
\setCJKfamilyfont{cjk-vert}[Script=CJK,RawFeature=vertical]{IPAPMincho}
\renewcommand{\rubysep}{-0.5ex}
%\raisebox{-.5\height}{
%\fbox{
\rotatebox{-90}{
\begin{minipage}{2.0cm} \CJKfamily{cjk-vert}
\Huge \ruby{東}{とう}\ruby{京}{ きょう}
\end{minipage}
%}
%}
}
\end{center}
\bigskip

Other names for Furigana are ruby/rubi or yomigana {読み仮名} {【よみがな】}.
Ruby (Japanese {ルビ} /rubi/) is also a annotation system that can be used in
\LaTeX or HTML. Rubi are  also common in China, Taiwan and Korea. 

A common example for using Furigana for adults:

\begin{center}
\Huge \ruby{地球}{ふるさと} 
\end{center}

In science fictions some astronaut could use the Japanese word {ふるさと}
/furusato/  with the meaning of "my hometown" to refer to the planet Earth ( =
{地球}{【ちきゅう】}). Or to make it more fancy and international (may be also
with connotation that Japan has no space in the future):

\begin{center}
\Huge \ruby{地球}{アース} 
\end{center}

Here {アース} refers to 'earth', but {地球} is better understandable by the
Japanese audience.

