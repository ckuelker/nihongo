\section{Hentaigana}\jsec{同音異語} 
% [o] LABEL
\label{sec:Hentaigana}
% [o] INDEX DESTINATION (DEF)
\ifor{Hentaigana}{変体仮名}{へんたいがな}{Hentaigana}
% [o] INDEX TARGET
\ifor{Kana}{仮名}{かな}{Kana}

\textit{Hentaigana} are historical \hyperref[sec:Kana]{Kana} that are used
seldom today. The where  used until before 1900 and declared as obsolete
(/hentai/ - variant) in the 1900 language reform.  Rather then an addition to
\hyperref[sec:Kana]{Kana} \textit{Hentaigana} represent alternative forms to
existing \hyperref[sec:Kana]{Kana}. The usage where not formalized and every
writer decides which set to use. It was even common to used two or more
different \textit{Hentaigana} (and standard \hyperref[sec:Kana]{Kana}) with
same pronunciation in the same document. 

In contemporary Japan the usage of \textit{Hentaigana} is reduced to
traditional decorative elements on shop signs for example, but a few marginal
uses remain: the word /otemoto/ is written in \textit{Hentaigana} on some
chopsticks.

Since \textit{Hentaigana} are not available in Unicode it is difficult to
impossible to use them in daily writing on computers. Please refer to Wikipedia
to have a visual impression. \Link
\href{http://en.wikipedia.org/wiki/Hentaigana}{http://en.wikipedia.org/wiki/Hentaigana}



