\section{Iroha}\jsec{伊呂波} 
% [o] LABEL
\label{sec:Iroha}
% [o] INDEX DESTINATION (DEF)
\ifor{Iroha}{伊呂波}{いろは}{Iroha}
% [o] INDEX TARGET
\ifor{Gojūonzu}{五十音図}{ごじゅうおんず}{50@50 Laute Tafel}
\ifor{Hiragana}{平仮名}{ひらがな}{Hiragana}

The word \textit{Iroha} stands for /iroha uta/ (\textit{Iroha} song) and is a
Japanese poem of the Heian era that contains all Kana words.  In contrast to
today it also contains more or less unuses letters, like /we/ or /wi/ and it do
not contain the newer /n/. Usual the poem is written in
\hyperref[sec:Hiragana]{Hiragana} from top to down.

\begin{center}
%\raisebox{10\height}{
%\framebox[20mm][r]{
\rotatebox{-90}{
\begin{minipage}{2.0cm}
    \setCJKfamilyfont{cjk-vert}[Script=CJK,RawFeature=vertical]{IPAPMincho}
    \renewcommand{\rubysep}{-0.5ex}
    \CJKfamily{cjk-vert}
いろはにほへとちりぬるをわかよたれそつねならむうゐのおくやまけふこえてあさきゆめみしゑひもせす
\end{minipage}
}
%}
%}
\end{center}

In this book the modern \hyperref[sec:Gojuonzu]{Gojūonzu} is used.

