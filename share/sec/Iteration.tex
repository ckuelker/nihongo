% ---------------------------------------------------------------------------
\section{Katakana Iteration Marks}\jsec{反復記号 } \label{sec:Iteration}
% [o] INDEX
\ifor{Katakana iteration marks}{反復記号}{はんぷくきごう}{Katakana Wiederholungszeichen}
\ien{iteration marks}\ide{Wiederholungszeichen}
\ifor{Kanji}{漢字}{かんじ}{Kanji}
\ifor{Katakana}{片仮名}{かたかな}{Katakana}

As with \hyperref[sec:Kanji]{Kanji} {漢字} {【かんじ】} also
\hyperref[sec:Katakana]{Katakana} {片仮名} has \textbf{iteration marks}.
However \hyperref[sec:Katakana]{Katakana} has three \textbf{iteration marks}. 

\subsection{Double a Vowel}\jsubsec{長音}\label{subsec:Choon}
\ifor{Chōon}{長音}{ちょうおん}{Chōon}
\ien{double vowel}\ide{Konsonantenverdopplung}


The Chōon {長音} {【ちょうおん】} doubles the previous vowel. Please read the
section \nameref{subsec:DoublingVowel} on page \pageref{subsec:DoublingVowel}.

\subsection{Double a Character}\jsubsec{踊り字}
\ifor{"dancing mark"}{踊り字}{おどりじ}{"Tanzzeichen"}
\ien{repition mark} \ide{Wiederholungszeichen}
\ifor{}{重ね字}{かさねじ}{}
\ifor{}{繰り返し記号}{くりかえしきごう}{}
\ifor{chrarcater repition mark}{反復記号}{はんぷくきごう}{Chracter Wiederholungszeichen}
\ifor{}{}{}{}

Some general names exists for \textbf{iteration marks} in the Japanese
language: {踊り字} {【おどりじ】} the so called  "dancing mark", {重ね字}
{【かさねじ】}, {繰り返し記号} {【くりかえしきごう】}, or {反復記号}
{【はんぷくきごう】} as "repetition symbols".

\ifor{Dakuten}{濁点}{だくてん}{Dakuten}
\ifor{Mōra}{モーラ}{もーら}{Mōra}
The iteration mark that can repeat any \hyperref[sec:Katakana]{Katakana} is
{「ヽ」}  and its {濁点} {【だくてん】} form is {「ヾ」}. This can only be
found in rare\footnote{\textbf{Iteration marks} where wildly used in old texts
and may be used in personal writing.} cases. For example the personal name
Misuzu 【みすゞ】might contain this character and therefore the Katakana
transcription as well. And since the difference between the second last and
the last \hyperref[sec:Mora]{Mora} is only a change in pronunciation the {濁点}
is added.

\subsection{Double two (or more) Characters}\jsubsec{くの字点}
\ifor{double multiple character}{くの字点}{くのじてん}{viele Zeichen verdoppeln}
\ifor{Kanji}{漢字}{かんじ}{Kanji}
\ifor{Okurigana}{送り仮名}{おくらいがな}{Okuriagana}

In vertical writing exist another iteration marker {くの字点} {【くのじてん】}
which consist out of two characters {「〳」+「〵」} and the {濁点} form is
{「〴」+「〵」}. It can double two or more characters. As for the iteration
mark above this is seldom used.

\begin{center}
\setCJKfamilyfont{cjk-vert}[Script=CJK,RawFeature=vertical]{IPAPMincho}
\renewcommand{\rubysep}{-0.5ex}
%\raisebox{-.5\height}{
%\fbox{
\rotatebox{-90}{
\begin{minipage}{3.0cm} \CJKfamily{cjk-vert}
\Huge \ruby{所々}{ところ〴〵}
\end{minipage}
%}
%}
}
\end{center}

The {くの字点} is the same for \hyperref[sec:Hiragana]{Hiragana} and
\hyperref[sec:Katakana]{Katakana}. The above example shows that the change of
sound {所々} {【ところどころ】} (Engl.: here and there) do not apply to the
\hyperref[sec:Kanji]{Kanji} iteration mark {「々」}.

\begin{center}
\setCJKfamilyfont{cjk-vert}[Script=CJK,RawFeature=vertical]{IPAPMincho}
\renewcommand{\rubysep}{-0.5ex}
%\raisebox{-.5\height}{
%\fbox{
\rotatebox{-90}{
\begin{minipage}{3.0cm} \CJKfamily{cjk-vert}
\Huge \ruby{色々}{イロ/\}
\end{minipage}
%}
%}
}
\end{center}

If the {「〳」+「〵」}  is not available sometimes a Japanese full wide slash
and backslash is used. {「/」+「\」} 

If \hyperref[sec:Okurigana]{Okurigana} is present no iteration mark should be
used. For example  {休み休み} {【やすみやすみ】} (Engl.: with a lot of breaks).


The {くの字点} character as such can be doubled by itself.

%\newcolumntype{V}{>{\centering\arraybackslash} m{.4\linewidth} }


\begin{center}
\setCJKfamilyfont{cjk-vert}[Script=CJK,RawFeature=vertical]{IPAPMincho}
\renewcommand{\rubysep}{-0.5ex}
%\fbox{
\begin{tabular}{ccc}
\rotatebox{-90}{
\begin{minipage}{4.5cm} \CJKfamily{cjk-vert}
\LARGE {トントントン} 
\end{minipage}
}& 
%\includegraphics[scale=1]{../share/katakana/4ar.pdf}
&
\rotatebox{-90}{
\begin{minipage}{4.5cm} \CJKfamily{cjk-vert}
\LARGE  {トン〳〵〳〵}
\end{minipage}
}\\
\end{tabular}
%}
\end{center}

\Link \href{http://ja.wikipedia.org/wiki/%E8%B8%8A%E3%82%8A%E5%AD%97}{http://ja.wikipedia.org/wiki/踊り字}



