% ---------------------------------------------------------------------------
\section{Kana}\jsec{仮名} 
% [o] LABEL
\label{sec:Kana}
% [o] INDEX
\ifor{Kana}{仮名}{かな}{Kana}
\ifor{Mōra}{モーラ}{もーら}{Mōra}
\ifor{Kanji}{漢字}{かんじ}{Kanji}

The Japanese category \textbf{Kana} ({仮名} {【かな】}) represents Japanese
\hyperref[sec:Mora]{Mōra} scripts that are part of the Japanese writing
system. \textbf{Kana} is often used in contrast to
\hyperref[sec:Kanji]{Kanji}, because \hyperref[sec:Kanji]{Kanji} also posses
meaning while all \textbf{Kana} have not. 

\ifor{Kana}{仮名}{かな}{Kana}
\ifor{Okurigana}{送り仮名}{おくりがな}{Okurigana}
Contemporary \textbf{Kana} scripts are \hyperref[sec:Hiragana]{Hiragana} and
\hyperref[sec:Katakana]{Katakana}. While other words in Japanese language also
end with the category \textbf{Kana} but do not represents a script, like
\hyperref[sec:Okurigana]{Okuriagana} or \hyperref[sec:Furigana]{Furigana}
which just refers to \hyperref[sec:Hiragana]{Hiragana} or
\hyperref[sec:Katakana]{Katakana} used for certain functions or situations.  

\ifor{Kana}{仮名}{かな}{Kana}
\ifor{Hentaigana}{変体仮名}{へんたいがな}{Hentaigana}
\ifor{Mōra}{モーラ}{もーら}{Mōra}
\ifor{Hiragana}{平仮名}{ひらがな}{Hiragana}
Other \textbf{Kana} like Hentaigana ({変体仮名} {【へんたいがな】}) are
obsolete and depreciated versions of \hyperref[sec:Hiragana]{Hiragana}.
Historically  there have been more then one \hyperref[sec:Hiragana]{Hiragana}
for one \hyperref[sec:Mora]{Mora} that where stylistic variants or distinct
alternatives.

\ifor{Kana}{仮名}{かな}{Kana}
\ifor{Man'yōgana}{万葉仮名}{まんようがな}{Man'yōgana}
And finally \hyperref[sec:Monyogana]{Man'yōgana} are Chinese characters that
are used as phonetic characters around mid 7th century. This name is somewhat
misleading since Chinese characters where not only used in the Man'yōshū in
this fashion as well the characters which where used over a long time and the
number of where not constant.


