\section{片仮名 ア行 - Katakana |a| Row}  \label{sec:KatakanaARow}

\Krow{arow}{a}{i}{u}{e}{o}

\label{letter:a}\KLETTER{a} The 片仮名 {「ア」} derives from the
\hyperref[sec:Manyogana]{万葉仮名} characters {「阿」} left element
(\hyperref[sec:Radical]{radical}).  A smaller version {「ァ」} is used in
combinations with other letters as {「ファ」} and is pronounced as |fa| in
\hyperref[sec:Hepburn]{Hepburn} transcription.

\label{letter:u}\KLETTER{i} The 片仮名 {「イ」} derives from the
\hyperref[sec:Manyogana]{万葉仮名} characters {「伊」} left element
(\hyperref[sec:Radical]{radical}).  A smaller version {「ィ」} is used in
combinations with other letters and represents a
\hyperref[sec:Diphthong]{diphthong}. 

\label{letter:u}\KLETTER{u} The 片仮名 {「ウ」} derives from the
\hyperref[sec:Manyogana]{万葉仮名} character {「宇」}. A smaller version
{「ゥ」} is used in combinations with other letters and represents a
\hyperref[sec:Diphthong]{diphthong} and is written as "w". Even though the
combination {「トゥ」} |tu| exist, it is relatively new and many words do not
use it. In this cases {「ツ 」} |tsu| is used. {「ウ」} can take
\hyperref[sec:Dakuten]{Dakuten} to form {「ヴ」} |vu|, which is relatively new
and can replace {「ブ」} |bu|. 

\Note{Note}{%

Be aware that the characters \hyperref[letter:fu]{「フ」},
\hyperref[letter:wa]{「ワ」}  and \hyperref[letter:u]{「ウ」} look very
similar.  Make sure that you spend extra training on distinguish them. 

}%


\newpage 

\label{letter:e}\KLETTER{e} The 片仮名 {「エ」} derives from the
\hyperref[sec:Manyogana]{万葉仮名} characters {「江」} right element
(\hyperref[sec:Radical]{radical}). A smaller version {「ェ」} is used in
combinations with other letters and express \hyperref[sec:Mora]{morae} of
foreign origin. For example {「ヴェ」} as pronounced |ve|.

\label{letter:o}\KLETTER{o} The 片仮名 {「オ」} derives from the
\hyperref[sec:Manyogana]{万葉仮名} character {「於」}. A smaller version
{「ォ」} is used in combinations with other letters and express
\hyperref[sec:Mora]{morae} of foreign origin. For example {「フォ 」} as
pronounced |fe|.

\newpage


% ---------------------------------------------------------------------------
\subsection{ア - |a|} \label{sec:KatakanaA}

\KatakanaHeader{a}{ The Katakana {「ア」} is written with two strokes. The
first stroke starts horizontal. The second stroke is a curve with can be
attached to the first stroke in hand writing, but not at the horizontal part -
at the end of the first line.} \KatakanaTraining{a}

% ---------------------------------------------------------------------------
\subsection{イ - |i|} \label{sec:KatakanaI}

\KatakanaHeader{i}{ The Katakana {「イ」} is written with one stroke. The first
stroke is a curve from upper right to lower left. The second stroke is a
vertical line attached to the first at the top.} \KatakanaTraining{i}

% ---------------------------------------------------------------------------
\subsection{ウ - |u|} \label{sec:KatakanaU}

\KatakanaHeader{u}{The Katakana {「ウ」} is written with three strokes. The
first stroke a small vertical line. The second a small vertical line again and
the third line a horizontal line connection the two others.}
\KatakanaTraining{u}

% ---------------------------------------------------------------------------
\subsection{エ - |e|} \label{sec:KatakanaE}

\KatakanaHeader{e}{The Katakana {「エ」} is written with three strokes. It is
very geometrically consisting only out of horizontal and vertical lines
connected together.} \KatakanaTraining{e}

% ---------------------------------------------------------------------------
\subsection{オ - |o|} \label{sec:KatakanaO}

\KatakanaHeader{o}{The Katakana {「オ」} is written with three strokes. The
first line is horizontal and together with the second stroke it constructs a
perfect crossing. The third stroke beginning lies at the center of the
crossing.} \KatakanaTraining{o}

\section{片仮名ア行練習 -  |a| Row Training}

\KatakanaSimpleTraining{Katakana to Romaji}{
\Transcribe{1.}{ウエア}{}{wear, ware}
\Transcribe{2.}{エア}{}{air}
\Transcribe{3.}{エイ}{}{A (the letter)}
\Transcribe{4.}{アイ}{}{I (the letter)}
\Transcribe{5.}{オウ}{}{O (the letter)}
\Transcribe{6.}{イア}{}{ear}
}

\KatakanaSimpleTraining{Romaji to Katakana}{
\Transcribe{1.}{ea}{}{air}
\Transcribe{2.}{ai}{}{I (the letter)}
\Transcribe{3.}{ou}{}{O (the letter)}
\Transcribe{4.}{ei}{}{A (the letter)}
\Transcribe{5.}{uea}{}{wear, ware}
\Transcribe{6.}{ia}{}{ear}
}

\newpage

\KatakanaSimpleTraining{English to Romaji}{
\Transcribe{1.}{ear}{}{}
\Transcribe{2.}{I (the letter)}{}{}
\Transcribe{3.}{air}{}{}
\Transcribe{4.}{O (the letter)}{}{}
\Transcribe{5.}{wear, ware}{}{}
\Transcribe{6.}{A (the letter)}{}{}
}

\KatakanaSimpleTraining{English to Katakana}{
\Transcribe{2.}{I (the letter)}{}{}
\Transcribe{3.}{O (the letter)}{}{}
\Transcribe{1.}{air}{}{}
\Transcribe{6.}{ear}{}{}
\Transcribe{5.}{wear, ware}{}{}
\Transcribe{4.}{A (the letter)}{}{}
}
\newpage
