% ---------------------------------------------------------------------------
\section{ Katakana /ha/ Row - 片仮名ハ行}\label{sec:KatakanaHaRow}

\Krow{harow}{ha}{hi}{fu}{he}{ho}

\label{letter:ha}\KLETTER{ha} The  片仮名 {「ハ」} is pronounced  /ha/ and
derives from the \hyperref[sec:PhoneticCharacter]{Phonetic Character} {「八 」}.
A \hyperref[sec:Dakuten]{濁点} version exists and pronounced as /ba/.

\label{letter:hi}\KLETTER{hi} The 片仮名 {「ヒ」} derives from the
\hyperref[sec:PhoneticCharacter]{Phonetic Characters} {「比」} reight
\hyperref{sec:Radical}{radical}.  It is pronounced as /hi/.  A
\hyperref[sec:Dakuten]{濁点} version exists and pronounced as /bi/.

\label{letter:fu}\KLETTER{fu} The 片仮名 {「フ」} derives from the
\hyperref[sec:PhoneticCharacter]{Phonetic Characters} upper left part of {「不
」}.  It is pronounced as /fu/.  A \hyperref[sec:Dakuten]{濁点} version exists
and pronounced as /bu/. 

\label{letter:he}\KLETTER{he} The 片仮名 {「ヘ」} derives from the
\hyperref[sec:PhoneticCharacter]{Phonetic Characters} right
\hyperref{sec:Radical}{radical} of {「部」}.  It is pronounced as /he/.  A
\hyperref[sec:Dakuten]{濁点} version exists and pronounced as /be/.  

\Warn{Warning}{The Katakana {「ヘ」} is the same character as the
\hyperref[sec:Hiragana]{Hiragana} {「へ」}. In some documents they can be
distinguished because the font is different. However in genral they are the
same. }

\label{letter:ho}\KLETTER{ho} The 片仮名 {「ホ」} derives from the
\hyperref[sec:PhoneticCharacter]{Phonetic Characters} lower right part of
{「保」} wich by itself is the \hyperref[sec:Radical]{radical}  and
\hyperref[sec:Kanji]{漢字【かんじ】}  of tree.  It is pronounced as /ho/.  A
\hyperref[sec:Dakuten]{濁点} version exists and pronounced as /bo/.

% UFuWaSimilarity
\subsection{|u|, |fu| and |wa| Similarity} \label{subsec:UFuWaSimilarity}

The Katakana characters {「ウ」}, {「フ」} and {「ワ」} can be easily
distinguished. All three have a different stroke count. However the shape is
similar. Therefore they can be mistaken. Especially when they have no context. 

\bigskip

\begin{center}
\begin{tabular}{|c|c|c|}\hline
\KLETTER{u}&\KLETTER{fu}&\KLETTER{wa}\\\hline
\end{tabular}
\end{center}




\newpage

% ハヒフヘホ
% ---------------------------------------------------------------------------
\subsection{/ha/ - 「ハ」} \label{sec:KatakanaHa}

\KatakanaHeader{ha}{ The Katakana {「ハ」} is written with two strokes. Non of
them is striaght.} \KatakanaTraining{ha}

% ---------------------------------------------------------------------------
\subsection{/hi/ - 「ヒ」} \label{sec:KatakanaHi}

\KatakanaHeader{hi}{ The Katakana {「ヒ」}  is written with two strokes. One
stroke from right to left. The other stroke from up to down and then a curve.
The difficulty of this character is to hit the first stroke with the second.  }
\KatakanaTraining{hi}

% ---------------------------------------------------------------------------
\subsection{/fu/ - 「フ」} \label{sec:KatakanaFu}

\KatakanaHeader{fu}{The pronuciation of Katakana {「フ」} is \textbf{not} /hu/
it is /fu/ and it is written with only one stroke. } \KatakanaTraining{fu}

% ---------------------------------------------------------------------------
\subsection{/he/ - 「ヘ」} \label{sec:KatakanaHe}

\KatakanaHeader{he}{Katakana {「ヘ」} is written with one stroke from left to
right. This is the same character as \hyperref[sec:Hiragana]{Hiragana} /he/.}
\KatakanaTraining{he}

% ---------------------------------------------------------------------------
\subsection{/ho/ - 「ホ」} \label{sec:KatakanaHo}

\KatakanaHeader{ho}{The Kataka {「ホ」} character reminds at the Kanji for tree
and is also written in the same order and with the same amount of stroke.
However the left and righ 'root' is not connected to the base. In cursive
writing the character is written with a hook-stroke as the second stroke. This
is abstract available even in the bold form where the second stroke has a small
curve at the end.} \KatakanaTraining{ho}

% ---------------------------------------------------------------------------
\subsection{/ha/ Row Training - 片仮名ハ行練習}\label{sec:HaRowTraining}
\Padding
\begin{longtable}[c]{p{3cm}p{2cm}p{3cm}p{5cm}p{2cm}}
\textit{Katakana}&\textit{Rōmaji}&\textit{Original}&\textit{Remark}&\textit{Origin}\\\hline
ホットケーキ&hottokēki&hotcake    &a pancake                         &English\\
コーヒー    &kōhī     &koffie     &珈琲  coffee                      &Dutch\\
ソフト      &sofuto   &soft(ware) &                                  &English \\
\end{longtable}

\KatakanaSimpleTraining{Katakana to Rōmaji}{
\Transcribe{1.}{ホットケーキ}{}{hotcake}
\Transcribe{2.}{コーヒー}{}{coffee}
\Transcribe{3.}{ソフト}{}{soft(ware)}
\Transcribe{4.}{ハイタッチ}{}{high five}
\Transcribe{5.}{ハウス}{}{house}
%\Transcribe{6.}{ハイネック}{}{high neck}
}

\KatakanaSimpleTraining{Rōmaji to Katakana}{
\Transcribe{1.}{kōhī}{}{coffee}
\Transcribe{2.}{hottokēki}{}{hotcake}
\Transcribe{3.}{haitacchi}{}{high five}
\Transcribe{4.}{sofuto}{}{soft(ware)}
\Transcribe{5.}{hainekku}{}{high neck}
%\Transcribe{6.}{hausu}{}{house}
}

\newpage
\Padding
\begin{longtable}[c]{p{2cm}p{2cm}p{3cm}p{6cm}p{2cm}}
\textit{Katakana}&\textit{Rōmaji}&\textit{Original}&\textit{Remark}&\textit{Origin}\\\hline
ハイタッチ  &haitacchi&high touch &high five                         &English\\
ハウス      &hausu    &Haus, house&                                  &English, German\\
ハイネック  &hainekku &high neck  &turtle neck style sweater or shirt&English\\
\end{longtable}
\KatakanaSimpleTraining{English to Rōmaji}{
\Transcribe{1.}{coffee}{}{}
\Transcribe{2.}{hotcake}{}{}
\Transcribe{3.}{high five}{}{}
\Transcribe{4.}{software}{}{}
\Transcribe{5.}{high neck}{}{}
%\Transcribe{6.}{house}{}{}
}

\KatakanaSimpleTraining{English to Katakana}{
\Transcribe{1.}{hotcake}{}{}
\Transcribe{2.}{high five}{}{}
\Transcribe{3.}{coffee}{}{}
\Transcribe{4.}{high neck}{}{}
\Transcribe{5.}{house}{}{}
%\Transcribe{6.}{software}{}{}
}

\newpage
