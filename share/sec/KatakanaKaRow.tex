% ---------------------------------------------------------------------------
\section{Katakana /ka/ Row}\jsec{片仮名 カ行}\label{sec:KatakanaKaRow}

\Krow{karow}{ka}{ki}{ku}{ke}{ko}

\label{letter:ka}\KLETTER{ka} The  片仮名 {「カ」} is pronounced  /ka/ and  derives from the
\hyperref[sec:PhoneticCharacter]{Phonetic Character}s {「加」} left
\hyperref[sec:Radical]{radical}.  A \hyperref[sec:Dakuten]{濁点} version exists
and pronounced as /ga/.

%\hyperref[sec:Handakuten]{半濁点} does not exist in daily Japanese.  
% {「一ヵ所」} {【いちかしょ】} (one place)
% {「一ヶ所」} {【いちかしょ】} (one place).
% 十ヵ条(十ヶ条)


\Note{Note}{A smaller version {「ヵ」} is rare but used in combinations with
number particles.  For example in {「一ヵ月」} {【いっかげつ】} (one month) and
others.  This cases can also be written {「一ヶ月」} {【いっかげつ】} (one
month). Please see also \nameref{sec:KatakanaKe}. \Link
\href{https://ja.wikipedia.org/wiki/\%E3\%83\%B5}{ヵ} }

\label{letter:ki}\KLETTER{ki} The 片仮名 {「キ」} derives from the
\hyperref[sec:PhoneticCharacter]{Phonetic Character}s middle part of either {「機」} or
{「幾」}.  It is pronounced as /ki/.  A \hyperref[sec:Dakuten]{濁点} version
exists and pronounced as /gi/.


\label{letter:ku}\KLETTER{ku} The 片仮名 {「ク」} derives from the
\hyperref[sec:PhoneticCharacter]{Phonetic Character}s left upper part of {「久」}.  It
is pronounced as /ku/.  A \hyperref[sec:Dakuten]{濁点} version exists and
pronounced as /gu/.  A smaller version exists, but is used for the Ainu
Language.



\label{letter:ke}\KLETTER{ke} The 片仮名 {「ケ」} derives from the
\hyperref[sec:PhoneticCharacter]{Phonetic Character}s upper and left part of {「介」}.
It is pronounced as /ke/.  A \hyperref[sec:Dakuten]{濁点} version exists and
pronounced as /ge/.  The smaller version {「ヶ」} is explained in the following
note.

\newpage

\Note{Note}{ A smaller version {「ヶ」} is rare but used in combinations with
number particles.  For example in {「一ヶ月」} {【いっかげつ】} (one month) and
others.  This cases can also be written {「一ヵ月」} {【いっかげつ】} (one
month). There are cases where only {「ヶ」} can be written {七ヶ宿}
{【シチカシュク】} (Place at the south west border of the prefecture Miyagi).
In other rare cases this character can be pronounced different {「関ヶ原」}
{【せきがはら】} (Place at the south border of the Gifu prefecture, known by
the battle at 1600.). Please see also \nameref{sec:KatakanaKa}. \Link
\href{https://ja.wikipedia.org/wiki/\%E3\%83\%B5}{ヵ} }

\label{letter:ko}\KLETTER{ko} The 片仮名 {「コ」} derives from the
\hyperref[sec:PhoneticCharacter]{Phonetic Character}s upper part of {「己」}.  It is
pronounced as /ko/.  A \hyperref[sec:Dakuten]{濁点} version exists and
pronounced as /go/.



\newpage

% ---------------------------------------------------------------------------
\subsection{/ka/}\jsubsec{「カ」} \label{sec:KatakanaKa}

\KatakanaHeader{ka}{ /ka/ is written with 2 strokes. Basically the same way as
the Hiragana {「か」} it looks like a squarish version, but without the last
stroke. The hook at the second stroke is less significant or important.  }
\KatakanaTraining{ka}

% ---------------------------------------------------------------------------
\subsection{/ki/}\jsubsec{「キ」} \label{sec:KatakanaKi}

\KatakanaHeader{ki}{ The shape alignment of the 「キ」character is not straight
towards its environment. However the junctions are more or less 90 degrees.  }
\KatakanaTraining{ki}

% ---------------------------------------------------------------------------
\subsection{/ku/}\jsubsec{「ク」} \label{sec:KatakanaKu}
% ---------------------------------------------------------------------------

\KatakanaHeader{ku}{ The first stroke is similar the stroke of {「ケ 」} is a
curve. While the second stroke start aligned and straight. }
\KatakanaTraining{ku}

% ---------------------------------------------------------------------------
\subsection{/ke/}\jsubsec{「ケ」} \label{sec:KatakanaKe}

\KatakanaHeader{ke}{ The {「ケ 」} is written with 3 strokes and the first
stroke is similar to the {「ク」}. The second stroke is aligned and straight.
While the last stroke is a curve.  } \KatakanaTraining{ke}

% ---------------------------------------------------------------------------
\subsection{/ko/}\jsubsec{「コ」} \label{sec:KatakanaKo}

\KatakanaHeader{ko}{ This character is almost a geometric figure composed out
of two strokes. However unless in European languages this are only 2 strokes
and not 3. The first stroke is the longest one and done similar with all
{漢字}. } \KatakanaTraining{ko}

% ---------------------------------------------------------------------------
\subsection{/ka/ Row Training}\jsubsec{片仮名カ行練習}

\Padding
\begin{longtable}[c]{p{2cm}p{1.5cm}p{1.5cm}p{3cm}p{7cm}}
\textit{Katakana}&\textit{Rōmaji}&\textit{Original}&\textit{Remark}&Origin\\\hline
カキ  &kaki &kaki &柿 persimon&Japanese\\
ケア  &kea  &care &          &English\\
ケイ  &kei  &K    &the letter&English\\
\end{longtable}

\KatakanaSimpleTraining{Katakana to Rōmaji}{
\Transcribe{1.}{カキ}{}{persimmon}
\Transcribe{2.}{ココア}{}{cocoa}
\Transcribe{3.}{ケア}{}{care}
\Transcribe{4.}{コア}{}{core}
\Transcribe{5.}{ケーキ}{}{cake}
%\Transcribe{6.}{ケイ}{}{K (the letter)}
}

\KatakanaSimpleTraining{Rōmaji to Katakana}{
\Transcribe{1.}{kokoa}{}{cocoa}
\Transcribe{2.}{k$\overline{\mbox{e}}$ki}{}{cake}
\Transcribe{3.}{kea}{}{care}
\Transcribe{4.}{koa}{}{core}
\Transcribe{5.}{kaki}{}{persimmon}
%\Transcribe{6.}{kei}{}{K (the letter)}
}

\newpage

\Padding
%\begin{longtable}[c]{p{2cm}p{2cm}p{3cm}p{6cm}p{2cm}}
\begin{longtable}[c]{p{2cm}p{2.0cm}p{3.5cm}p{4cm}p{2.5cm}}
\textit{Katakana}&\textit{Rōmaji}&\textit{Original}&\textit{Remark}&Origin\\\hline
コア  &koa  &core &          &English\\
ココア&kokoa&cocoa& hot chocolate &English, from metathesis of Spanish cacao, from Nahuatl cacahuatl\\
ケーキ&kēki &cake &          &English\\
\end{longtable}


\KatakanaSimpleTraining{English to Rōmaji}{
\Transcribe{1.}{persimon}{}{}
\Transcribe{2.}{cocoa}{}{}
\Transcribe{3.}{care}{}{}
\Transcribe{4.}{core}{}{}
\Transcribe{5.}{K (the letter)}{}{}
%\Transcribe{6.}{cake}{}{}
}

\KatakanaSimpleTraining{English to Katakana}{
\Transcribe{1.}{cocoa}{}{}
\Transcribe{2.}{cake}{}{}
\Transcribe{3.}{care}{}{}
\Transcribe{4.}{persimon}{}{}
\Transcribe{5.}{K (the letter)}{}{}
%\Transcribe{6.}{core}{}{}
}

\newpage
