% ---------------------------------------------------------------------------
\section{Pronunciation and Intonation}\jsec{発音とイントネーション}
% [o] LABEL
\label{sec:PronunciationAndIntonation}
\label{sec:Pronuciation}
\label{sec:Intonation}
% [o] INDEX
\ifor{Pronuciation}{}{}{}
\ifor{Intonation}{}{}{}
\ifor{Katakana}{片仮名}{かたかな}{Katakana}
\ifor{Hiragana}{平仮名}{ひらがな}{Hiragana}
\ifor{mora}{モーラ}{もーら}{Mora}
\ifor{Gojūonzu}{五十音図}{ごじゅうおんず}{50 Laute Tafel}

The \textit{pronunciation} of \hyperref[sec:Katakana]{Katakana} is the same as
for \hyperref[sec:Hiragana]{Hiragana}. Therefore every
\hyperref[sec:Syllable]{syllable}, more precise every \hyperref[sec:Mora]{mora}
corresponds to a \hyperref[sec:Katakana]{Katakana} character and is constructed
as 'consonant' + 'vowel' with the exception of |n|. This system of letter for
each \hyperref[sec:Mora]{mora} makes \textit{pronunciation} absolutely clear
with no ambiguities.  However the simplicity of
\hyperref[sec:Katakana]{Katakana} does not mean that \textit{pronunciation} in
Japanese is simple for English speakers as it is for Germans.  The rigid
structure of the fixed \hyperref[sec:Mora]{mora} sound in Japanese creates the
challenge of learning the proper intonation and duration of Japanese
\textit{pronunciation}.

Almost each Japanese word can be chunked into \hyperref[sec:Mora]{morae} of
high and low pitch witch is a crucial aspect of the spoken language. Compared
to Chinese, Japanese luckily have only two pitches: hi and low. Sometimes this
difference can be even important for the lexis. Homophones can have for example
a difference in pitch which make them distinguishable.  The intonation of high
and low pitches is a crucial aspect of the spoken language. One of the biggest
problems for obtaining a natural sounding \textit{pronunciation} is the
incorrect intonation. Many European or American learners speak without paying
attention to the correct pitch. That makes the speech sound non-natural for
Japanese. In some language course try to let the learner memorize the natural
pitch of a word or even teach rules for memorization. While there is clearly a
possibility for linguistic rules, they are hard to remember and master. Also
because they can remember the rules it is still possible to learn the correct
intonation by resorting to language learning techniques used by infants or
small children: mimicking native Japanese speakers. Therefore it is highly
advised to expose oneself to as many Japanese spoken language as possible and
to mimic it. Radio, podcasts, drama and television to name a few. However, it
is not advised to listen too much artificial sources like anime or commercials.

\bigskip
\begin{tabular}{rl}
-&every (yes \textbf{every}) \hyperref[sec:Mora]{mora} is \textit{pronounced} 
  with the same length\\
-&there is no short and long \hyperref[sec:Mora]{mora} or letters\\
-&every \hyperref[sec:Mora]{mora} has a pitch: high or low\\
-&every pitch matters\\
-&the pitch can change  sometimes with its context\\
-&the pitch can change with a dialect - however standard Japanese has well 
  defined pitches\\
\end{tabular}

\bigskip

The \textit{pronunciation} of \hyperref[sec:Katakana]{Katakana} is exactly the
same as for \hyperref[sec:Hiragana]{Hiragana} and most sounds are very close to
the Latin \textit{pronunciation} but in general are \textit{pronounced} a
little shorter without any stress. Only the /ra/ sounds, like in /ra/, /ri/,
/ru/, /re/ and /ro/ have no similarity in European languages. 


The sound of the Japanese /r/ is  neither a central nor a lateral flap, but may
vary between the two. To an English speaker, its pronunciation varies between a
flapped 'd' (as in American English buddy) and a flapped 'l'.
\href{http://en.wikipedia.org/wiki/Japanese_phonology}{(Wikipedia Japanese
Phonology)}.


The following table displays the \textit{pronunciation} in the
\hyperref[sec:Gojuonzu]{Gojūonzu}.

\ien{Rōmaji Gojūonzu}
\ien{Rōmaji}
\ien{Gojūon}
\ija{ローマ字五十音図}
\ija{ローマ字}
\bigskip
\begin{center}
%\LARGE
%\Huge
\Padding
\begin{tabular}{c||c|c|c|c|c|}
&\textbf{a}&\textbf{i}&\textbf{u}&\textbf{e}&\textbf{o}\\\hline\hline
\textbf{-}&a&i&u&e&o\\\hline
\textbf{k}&ka&ki&ku&ke&ko\\\hline
\textbf{s}&sa&shi&su&se&so\\\hline
\textbf{t}&ta&chi&tsu&te&to\\\hline
\textbf{n}&na&ni&nu&ne&no\\\hline
\textbf{h}&ha&hi&fu&he&ho\\\hline
\textbf{m}&ma&mi&mu&me&mo\\\hline
\textbf{y}&ya&&yu&&yo\\\hline
\textbf{r}&ra&ri&ru&re&ro\\\hline
\textbf{w}&wa&&&&o\\\hline
\textbf{{*}}&n&&&&\\\hline
\end{tabular}
\end{center}




