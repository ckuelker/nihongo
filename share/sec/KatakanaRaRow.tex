% ラリルレロ
\section{片仮名  ラ行 - Katakana |ra| row}

\Krow{rarow}{ra}{ri}{ru}{re}{ro}

\KLETTER{ra} The  片仮名 {「ラ」} is pronounced  |ra| (flapped 'r')  and  derives from the
\hyperref[sec:Manyogana]{万葉仮名} characters {「良」} upper right corner part.
A \hyperref[sec:Dakuten]{濁点}  or \hyperref[sec:Handakuten]{半濁点} version  do not exist.

\Note{Note}{The sound of the Japanese |r| is  neither a central nor a lateral
flap, but may vary between the two.  To an English speaker, its pronunciation
varies between a flapped d (as in American English buddy) and a flapped l.
\href{http://en.wikipedia.org/wiki/Japanese_phonology}{(Wikipedia Japanese Phonology)}.}

\KLETTER{ri} The  片仮名 {「リ」} is pronounced  |ri| (flapped 'r')  and  derives from the
\hyperref[sec:Manyogana]{万葉仮名} characters {「利」}  right site part.
A \hyperref[sec:Dakuten]{濁点}  or \hyperref[sec:Handakuten]{半濁点} version  do not exist.

\Note{Note}{Please see section \nameref{subsec:SoRiNAmbiguity} for the explanation
how to write and distinguish |so|, |n| and |ri|.
}

\newpage

\KLETTER{ru} The  片仮名 {「ル」} is pronounced  |ru| (flapped 'r')  and  derives from the
\hyperref[sec:Manyogana]{万葉仮名} characters {「流」} lower left corner part.
A \hyperref[sec:Dakuten]{濁点}  or \hyperref[sec:Handakuten]{半濁点} version  do not exist.


\KLETTER{re} The  片仮名 {「レ」} is pronounced  |re| (flapped 'r')  and  derives from the
\hyperref[sec:Manyogana]{万葉仮名} characters {「礼」} upper right site part.
A \hyperref[sec:Dakuten]{濁点}  or \hyperref[sec:Handakuten]{半濁点} version  do not exist.

\KLETTER{ro} The  片仮名 {「ロ」} is pronounced  |ro| (flapped 'r')  and  derives from the
\hyperref[sec:Manyogana]{万葉仮名} characters {「呂」} upper part.
A \hyperref[sec:Dakuten]{濁点}  or \hyperref[sec:Handakuten]{半濁点} version  do not exist.


\newpage

% ラリルレロ
\subsection{ラ - |ra|} \label{sec:KatakanaRa}

\KatakanaHeader{ra}{ Katakana |ra| is written with two strokes.} \KatakanaTraining{ra}

\subsection{リ - |ri|} \label{sec:KatakanaRi}

\KatakanaHeader{ri}{ Katakana |ri| is written with two strokes.} \KatakanaTraining{ri}

\subsection{ル - |ru|} \label{sec:KatakanaRu}

\KatakanaHeader{ru}{ Katakana |ru| is written with two strokes.} \KatakanaTraining{ru}

\subsection{レ - |re|} \label{sec:KatakanaRe}

\KatakanaHeader{re}{ Katakana |re| is written with one stroke.} \KatakanaTraining{re}

\subsection{ロ - |ro|} \label{sec:KatakanaRa}

\KatakanaHeader{ro}{ Katakana |ro| is written with three strokes.} \KatakanaTraining{ro}

\section{片仮名ラ行練習 -  |ra| Row Training}

\KatakanaSimpleTraining{Katakana to Romaji}{
\Transcribe{1.}{ココア}{}{cocoa}
}

\KatakanaSimpleTraining{Romaji to Katakana}{
\Transcribe{1.}{kokoa}{}{cocoa}
}

\newpage
\KatakanaSimpleTraining{English to Romaji}{
\Transcribe{1.}{persimon}{}{}
}

\KatakanaSimpleTraining{English to Katakana}{
\Transcribe{1.}{cocoa}{}{}
}

\newpage
