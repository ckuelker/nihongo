% ---------------------------------------------------------------------------
\section{Katakana /ra/ row - 片仮名ラ行}\label{sec:KatakanaRaRow}

\Krow{rarow}{ra}{ri}{ru}{re}{ro}

\KLETTER{ra} The  片仮名 {「ラ」} is pronounced  /ra/ (flapped 'r')  and  derives from the
\hyperref[sec:PhoneticCharacter]{Phonetic Character}s {「良」} upper right corner part.
A \hyperref[sec:Dakuten]{濁点}  or \hyperref[sec:Handakuten]{半濁点} version  do not exist.

\Note{Note}{

The sound of the Japanese /r/ is  neither a central nor a lateral flap, but may
vary between the two. To an English speaker, its pronunciation varies between a
flapped 'd' (as in American English buddy) and a flapped 'l'.
\href{http://en.wikipedia.org/wiki/Japanese_phonology}{(Wikipedia Japanese
Phonology)}.

}

\KLETTER{ri} The  片仮名 {「リ」} is pronounced  /ri/ (flapped 'r')  and  derives from the
\hyperref[sec:PhoneticCharacter]{Phonetic Character}s {「利」}  right site part.
A \hyperref[sec:Dakuten]{濁点}  or \hyperref[sec:Handakuten]{半濁点} version  do not exist.

%\Note{Note}{Please see section \nameref{subsec:SoRiNAmbiguity} for the explanation
%how to write and distinguish /so/, /n/ and /ri/.
%}


\KLETTER{ru} The  片仮名 {「ル」} is pronounced  /ru/ (flapped 'r')  and  derives from the
\hyperref[sec:PhoneticCharacter]{Phonetic Character}s {「流」} lower left corner part.
A \hyperref[sec:Dakuten]{濁点}  or \hyperref[sec:Handakuten]{半濁点} version  do not exist.


\KLETTER{re} The  片仮名 {「レ」} is pronounced  /re/ (flapped 'r')  and  derives from the
\hyperref[sec:PhoneticCharacter]{Phonetic Character}s {「礼」} upper right site part.
A \hyperref[sec:Dakuten]{濁点}  or \hyperref[sec:Handakuten]{半濁点} version  do not exist.

\KLETTER{ro} The  片仮名 {「ロ」} is pronounced  /ro/ (flapped 'r')  and  derives from the
\hyperref[sec:PhoneticCharacter]{Phonetic Character}s {「呂」} upper part.
A \hyperref[sec:Dakuten]{濁点}  or \hyperref[sec:Handakuten]{半濁点} version  do not exist.

% SoRiNAmbiguity
\subsection{|so|, |ri| and |n| Ambiguity} \label{subsec:SoRiNAmbiguity}

The Katakana characters {「ソ」}, {「リ」} and {「ン」} can be difficult to
distinguish. All three are made out of only 2 strokes. And especially |so| and
|n| can be hard to tell. In a sentence of course the context can help a lot.
But what are the rules for this characters to write properly and distinguish?

\bigskip

\begin{center}
\begin{tabular}{|c|c|c|}\hline
\KLETTER{so}&\KLETTER{n}&\KLETTER{ri}\\\hline
\end{tabular}
\end{center}

\CharacterExplanation{soexplanation}{ To write the letter |so| it is important
to align both lines \textbf{horizontally} (red line) and to \textbf{non-align}
the ends (blue line) vertically. In this way it is possible to distinguish |so|
from |n|, but not from |ri|. To also distinguish it from |ri| you have to write
the first stroke not horizontally nor vertically (green line).  }

\CharacterExplanation{nexplanation}{ To write the letter |n| it is important to
a align both lines \textbf{vertically} (red line) and to \textbf{non-align} the
ends (blue line). In this way it is possible to distinguish |n| from |so|. If
both lines are aligned there should not be a problem to distinguish it from
|ri|. Usually the angle of the green line is different, but only a small
indicator. }

\CharacterExplanation{riexplanation}{ To write the letter |ri| it is important
to \textbf{align} both of the line beginnings \textbf{horizontally} (red line)
and to make sure that both lines are \textbf{parallel} (green lines). There
should be \textbf{no alignment} on the left side (blue line)
\textbf{vertically}. The difference between |so| and |ri| is that |ri| need to
start with two \textbf{parallel} lines wile |so| should not. Please compare the
left green line for an explanation.  }



\newpage

% ラリルレロ
\subsection{/ra/ - 「ラ」} \label{sec:KatakanaRa}

\KatakanaHeader{ra}{ Katakana /ra/ is written with two strokes.} \KatakanaTraining{ra}

\subsection{/ri/ - 「リ」} \label{sec:KatakanaRi}

\KatakanaHeader{ri}{ Katakana /ri/ is written with two strokes.} \KatakanaTraining{ri}

\subsection{/ru/ - 「ル」} \label{sec:KatakanaRu}

\KatakanaHeader{ru}{ Katakana /ru/ is written with two strokes.} \KatakanaTraining{ru}

\subsection{/re/ - 「レ」} \label{sec:KatakanaRe}

\KatakanaHeader{re}{ Katakana /re/ is written with one stroke.} \KatakanaTraining{re}

\subsection{/ro/ - 「ロ」} \label{sec:KatakanaRa}

\KatakanaHeader{ro}{ Katakana /ro/ is written with three strokes.} \KatakanaTraining{ro}

%\subsection{/ra/ Row Training - 片仮名ラ行練習}

%\KatakanaSimpleTraining{Katakana to Rōmaji}{
%\Transcribe{1.}{ココア}{}{cocoa}
%}

%\KatakanaSimpleTraining{Rōmaji to Katakana}{
%\Transcribe{1.}{kokoa}{}{cocoa}
%}

%\newpage
%\KatakanaSimpleTraining{English to Rōmaji}{
%\Transcribe{1.}{persimon}{}{}
%}

%\KatakanaSimpleTraining{English to Katakana}{
%\Transcribe{1.}{cocoa}{}{}
%}

\newpage
