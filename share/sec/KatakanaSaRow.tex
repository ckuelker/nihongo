\section{片仮名  サ行 - Katakana |sa| Row} \label{sec:KatakanaSaRow}

\Krow{sarow}{sa}{shi}{su}{se}{so}

\KLETTER{sa} The  片仮名 {「サ」} is pronounced  |sa| and  derives from the
\hyperref[sec:Manyogana]{万葉仮名} characters {「散」} upper left corner
\hyperref[sec:Radical]{radical}.  A \hyperref[sec:Dakuten]{濁点} version exists
and pronounced as |za|.

\KLETTER{shi} The 片仮名 {「シ」} derives from the
\hyperref[sec:Manyogana]{万葉仮名} character  {「之 」}.  It is pronounced as
|shi|.  A \hyperref[sec:Dakuten]{濁点} version exists and pronounced as |ji|.

\Note{Note}{Please see section \nameref{subsec:ShiTsuAmbiguity} for the
explanation how to write and distinguish |shi| and |tsu|.  }

\KLETTER{su} The 片仮名 {「ス」} derives from the
\hyperref[sec:Manyogana]{万葉仮名} characters right lower part of {「須」}.  It
is pronounced as |su|.  A \hyperref[sec:Dakuten]{濁点} version exists and
pronounced as |zu|. 

\KLETTER{se} The 片仮名 {「セ」} derives from the
\hyperref[sec:Manyogana]{万葉仮名} characters middle left part of {「世」}.
It is pronounced as |se|.  A \hyperref[sec:Dakuten]{濁点} version exists and
pronounced as |ze|.  

\newpage

\KLETTER{so} The 片仮名 {「ソ」} derives from the
\hyperref[sec:Manyogana]{万葉仮名} characters upper right part of {「曽」}.  It is
pronounced as |so|.  A \hyperref[sec:Dakuten]{濁点} version exists and
pronounced as |zo|.

% SoRiNAmbiguity
\subsection{|so|, |ri| and |n| Ambiguity} \label{subsec:SoRiNAmbiguity}

The Katakana characters {「ソ」}, {「リ」} and {「ン」} can be difficult to
distinguish. All three are made out of only 2 strokes. And especially |so| and
|n| can be hard to tell. In a sentence of course the context can help a lot.
But what are the rules for this characters to write properly and distinguish?

\bigskip

\begin{center}
\begin{tabular}{|c|c|c|}\hline
\KLETTER{so}&\KLETTER{n}&\KLETTER{ri}\\\hline
\end{tabular}
\end{center}

\CharacterExplanation{soexplanation}{ To write the letter |so| it is important
to align both lines \textbf{horizontally} (red line) and to \textbf{non-align}
the ends (blue line).  In this way it is possible to distinguish |so| from |n|,
but not from |ri|. To also distinguish it from |ri| you have to write the first
stroke not horizontally nor vertically.  }

\CharacterExplanation{nexplanation}{ To write the letter |n| it is important to
a align both lines \textbf{vertically} (red line) and to \textbf{non-align} the
ends (blue line). In this way it is possible to distinguish |n| from |so|. If
both lines are aligned there should not be a problem to distinguish it from
|ri|.  }

\CharacterExplanation{riexplanation}{ To write the letter |ri| it is important
to a align both lines \textbf{vertically} (red line) and to \textbf{non-align}
the ends (blue line). The difference between |so| and |ri| is that |ri| need to
start with two \textbf{parallel} lines wile |so| does not. Please see green
lines for explanation.  }




\newpage

\subsection{サ - |sa|} \label{sec:KatakanaSa}

\KatakanaHeader{sa}{ Katakana {「サ」} is written with three strokes. All
crossings of strokes are in a 90 degree angle.  The starts of all strokes are
aligned eitehr horizontally or vertically. The last stroke has a curve.}
\KatakanaTraining{sa}

\subsection{シ - |shi|} \label{sec:KatakanaShi}

\KatakanaHeader{shi}{ The Katakana {「シ」} is written with three strokes. All
three strokes are aligned vertically in the beginning. Please see section
\nameref{subsec:ShiTsuAmbiguity}.}

\KatakanaTraining{shi}

\subsection{ス - |su|} \label{sec:KatakanaSu}

\KatakanaHeader{su}{The Katakana {「ス」} is written with two strokes. The first
stroke startes horizontally aligned. The second stroke touches the first stroke
at the beginning.}

\KatakanaTraining{su}

\subsection{セ - |se|} \label{sec:KatakanaSe}

\KatakanaHeader{se}{ The Katakana {「セ」} is written with two strokes. The
crossing has \textbf{no} 90 degree angle. The curve of the second stroke as
almost a 90 deegre angle. } 

\KatakanaTraining{se}

\subsection{ソ - |so|} \label{sec:KatakanaSo}

\KatakanaHeader{so}{ The Katakana {「ソ」} is written with two strokes. The
first stroke is not aligned verticall but it is aligned horizontally withe the
second stroke. Please see section \nameref{subsec:SoRiNAmbiguity}.}

\KatakanaTraining{so}

\section{片仮名サ行練習 -  |sa| Row Training}

\KatakanaSimpleTraining{Katakana to Romaji}{
\Transcribe{1.}{ココア}{}{cocoa}
}

\KatakanaSimpleTraining{Romaji to Katakana}{
\Transcribe{1.}{kokoa}{}{cocoa}
}

\newpage
\KatakanaSimpleTraining{English to Romaji}{
\Transcribe{1.}{persimon}{}{}
}

\KatakanaSimpleTraining{English to Katakana}{
\Transcribe{1.}{cocoa}{}{}
}

\newpage
