%----------------------------------------------------------------------------
\section{Katakana /sa/ Row - 片仮名サ行}\label{sec:KatakanaSaRow}

\Krow{sarow}{sa}{shi}{su}{se}{so}

\label{letter:sa}\KLETTER{sa} The  片仮名 {「サ」} is pronounced  /sa/ and
derives from the \hyperref[sec:PhoneticCharacter]{Phonetic Character}s {「散」}
upper left corner \hyperref[sec:Radical]{radical}.  A
\hyperref[sec:Dakuten]{濁点} version exists and pronounced as /za/.

\label{letter:shi}\KLETTER{shi} The 片仮名 {「シ」} derives from the
\hyperref[sec:PhoneticCharacter]{Phonetic Character}  {「之 」}.  It is
pronounced as /shi/.  A \hyperref[sec:Dakuten]{濁点} version exists and
pronounced as /ji/.

\Note{Note}{Please see section \nameref{subsec:ShiTsuAmbiguity} for the
explanation how to write and distinguish /shi/ and /tsu/.  }

\label{letter:su}\KLETTER{su} The 片仮名 {「ス」} derives from the
\hyperref[sec:PhoneticCharacter]{Phonetic Character}s right lower part of
{「須」}.  It is pronounced as /su/.  A \hyperref[sec:Dakuten]{濁点} version
exists and pronounced as /zu/. 

\label{letter:se}\KLETTER{se} The 片仮名 {「セ」} derives from the
\hyperref[sec:PhoneticCharacter]{Phonetic Character}s middle left part of
{「世」}.  It is pronounced as /se/.  A \hyperref[sec:Dakuten]{濁点} version
exists and pronounced as /ze/.  

\newpage

\label{letter:so}\KLETTER{so} The 片仮名 {「ソ」} derives from the
\hyperref[sec:PhoneticCharacter]{Phonetic Character}s upper right part of
{「曽」}.  It is pronounced as /so/.  A \hyperref[sec:Dakuten]{濁点} version
exists and pronounced as /zo/.

% SoRiNAmbiguity
\subsection{|so|, |ri| and |n| Ambiguity} \label{subsec:SoRiNAmbiguity}

The Katakana characters {「ソ」}, {「リ」} and {「ン」} can be difficult to
distinguish. All three are made out of only 2 strokes. And especially |so| and
|n| can be hard to tell. In a sentence of course the context can help a lot.
But what are the rules for this characters to write properly and distinguish?

\bigskip

\begin{center}
\begin{tabular}{|c|c|c|}\hline
\KLETTER{so}&\KLETTER{n}&\KLETTER{ri}\\\hline
\end{tabular}
\end{center}

\CharacterExplanation{soexplanation}{ To write the letter |so| it is important
to align both lines \textbf{horizontally} (red line) and to \textbf{non-align}
the ends (blue line) vertically. In this way it is possible to distinguish |so|
from |n|, but not from |ri|. To also distinguish it from |ri| you have to write
the first stroke not horizontally nor vertically (green line).  }

\CharacterExplanation{nexplanation}{ To write the letter |n| it is important to
a align both lines \textbf{vertically} (red line) and to \textbf{non-align} the
ends (blue line). In this way it is possible to distinguish |n| from |so|. If
both lines are aligned there should not be a problem to distinguish it from
|ri|. Usually the angle of the green line is different, but only a small
indicator. }

\CharacterExplanation{riexplanation}{ To write the letter |ri| it is important
to \textbf{align} both of the line beginnings \textbf{horizontally} (red line)
and to make sure that both lines are \textbf{parallel} (green lines). There
should be \textbf{no alignment} on the left side (blue line)
\textbf{vertically}. The difference between |so| and |ri| is that |ri| need to
start with two \textbf{parallel} lines wile |so| should not. Please compare the
left green line for an explanation.  }



\newpage
% ---------------------------------------------------------------------------
\subsection{/sa/ - 「サ」}\label{sec:KatakanaSa}

\KatakanaHeader{sa}{ Katakana {「サ」} is written with three strokes. All
crossings of strokes are in a 90 degree angle.  The starts of all strokes are
aligned eitehr horizontally or vertically. The last stroke has a curve.}
\KatakanaTraining{sa}

% ---------------------------------------------------------------------------
\subsection{/shi/ - 「シ」}\label{sec:KatakanaShi}

\KatakanaHeader{shi}{ The Katakana {「シ」} is written with three strokes. All
three strokes are aligned vertically in the beginning. Please see section
\nameref{subsec:ShiTsuAmbiguity}.} \KatakanaTraining{shi}

% ---------------------------------------------------------------------------
\subsection{/su/ - 「ス」}\label{sec:KatakanaSu}

\KatakanaHeader{su}{The Katakana {「ス」} is written with two strokes. The
first stroke startes horizontally aligned. The second stroke touches the first
stroke at the beginning.} \KatakanaTraining{su}

% ---------------------------------------------------------------------------
\subsection{/se/ - 「セ」}\label{sec:KatakanaSe}

\KatakanaHeader{se}{ The Katakana {「セ」} is written with two strokes. The
crossing has \textbf{no} 90 degree angle. The curve of the second stroke as
almost a 90 deegre angle. } \KatakanaTraining{se}

% ---------------------------------------------------------------------------
\subsection{/so/ - 「ソ」}\label{sec:KatakanaSo}

\KatakanaHeader{so}{ The Katakana {「ソ」} is written with two strokes. The
first stroke is not aligned verticall but it is aligned horizontally withe the
second stroke. Please see section \nameref{subsec:SoRiNAmbiguity}.}
\KatakanaTraining{so}

% ---------------------------------------------------------------------------
\subsection{/sa/ Row Training - 片仮名サ行練習}\label{sec:SaRowTraining}
% 3 78 エキス 
%3 357 スカイ 
%3 360 スキー 
%3 3 アイス 
%3 111 ガーゼ 
%3 146 イエス
\Padding
\begin{longtable}[c]{p{2cm}p{2cm}p{3cm}p{6cm}p{2cm}}
\textit{Katakana}&\textit{Rōmaji}&\textit{Original}&\textit{Remark}&\textit{Origin}\\\hline
エキス&ekisu&ex(tract)&extract&Dutch\\
スカイ&sukai&sky&&English\\
スキー&sukī&ski&noun for skiing&English\\
\end{longtable}
\KatakanaSimpleTraining{Katakana to Rōmaji}{
\Transcribe{1.}{エキス}{}{extract}      % ekisu
\Transcribe{2.}{スカイ}{}{sky}          % sukai
\Transcribe{3.}{スキー}{}{ski}          % sukī
\Transcribe{4.}{アイス}{}{ice}          % aisu
\Transcribe{5.}{ガーゼ}{}{gauze}        % gāze
%\Transcribe{6.}{イエス}{}{Jesus}        % iesu
}

\KatakanaSimpleTraining{Rōmaji to Katakana}{
\Transcribe{1.}{sukai}{}{sky}         % sukai
\Transcribe{2.}{ekisu}{}{extract}     % ekisu
\Transcribe{3.}{aisu}{}{ice}          % aisu
\Transcribe{4.}{suki}{}{ski}          % sukī
\Transcribe{5.}{iesu}{}{Jesus}        % iesu
%\Transcribe{6.}{gāze}{}{gauze}        % gāze
}

\newpage

\Padding
\begin{longtable}[c]{p{2cm}p{2cm}p{3cm}p{6cm}p{2cm}}
\textit{Katakana}&\textit{Rōmaji}&\textit{Original}&\textit{Remark}&Origin\\\hline
アイス&aisu&ice&water ice, ice cream&English\\
ガーゼ&gāze&Gaze&gauze&German\\
イエス&iesu&Jesus&Jesus&Portuguese\\
\end{longtable}
\KatakanaSimpleTraining{English to Rōmaji}{
\Transcribe{1.}{extract}{}{}       % ekisu
\Transcribe{2.}{sky}{}{}           % sukai
\Transcribe{3.}{Jesus}{}{}         % iesu
\Transcribe{4.}{gauze}{}{}         % gāze
\Transcribe{5.}{ice}{}{}           % aisu
%\Transcribe{6.}{ski}{}{}           % sukī
}

\KatakanaSimpleTraining{English to Katakana}{
\Transcribe{1.}{sky}{}{}           % sukai
\Transcribe{2.}{gauze}{}{}         % gāze
\Transcribe{3.}{ice}{}{}           % aisu
\Transcribe{4.}{Jesus}{}{}         % iesu
\Transcribe{5.}{extract}{}{}       % ekisu
%\Transcribe{6.}{ski}{}{}           % sukī
}

\newpage
