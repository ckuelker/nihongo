% タチツテト
\section{片仮名  タ行 - Katakana |ta| Row}  \label{sec:KatakanaTaRow}

\Krow{tarow}{ta}{chi}{tsu}{te}{to}

\KLETTER{ta} The  片仮名 {「タ」} is pronounced  |ta| and  derives from the
\hyperref[sec:Manyogana]{万葉仮名} characters {「多 」} upper or lover
\hyperref[sec:Radical]{radical}.  A \hyperref[sec:Dakuten]{濁点} version exists
and pronounced as |da|.

\KLETTER{chi} The 片仮名 {「チ」} derives from the
\hyperref[sec:Manyogana]{万葉仮名} character  {「千」}.  It is pronounced as
|chi|.  A \hyperref[sec:Dakuten]{濁点} version exists and pronounced as |ji|.

\KLETTER{tsu} The 片仮名 {「ツ」} derives from the
\hyperref[sec:Manyogana]{万葉仮名} characters {「州」} or {「川」} .  It is
pronounced as |tsu|.  A \hyperref[sec:Dakuten]{濁点} version exists and
pronounced as |zu|. 

\KLETTER{te} The 片仮名 {「テ」} derives from the
\hyperref[sec:Manyogana]{万葉仮名} characters lower left part of {「天 」}.  It
is pronounced as |te|.  A \hyperref[sec:Dakuten]{濁点} version exists and
pronounced as |de|.  

\newpage

\KLETTER{to} The 片仮名 {「ト」} derives from the
\hyperref[sec:Manyogana]{万葉仮名} characters right part of {「止」}.  It is
pronounced as |to|.  A \hyperref[sec:Dakuten]{濁点} version exists and
pronounced as |do|.

% ShiTsuAmbiguity
\subsection{|shi| and |tsu| Ambiguity} \label{subsec:ShiTsuAmbiguity}

% シ
% ツ

The Katakana characters {「シ」} and  {「ツ」} are difficult to distinguish.
Both are made out of 3 strokes and even the lenght are equal.  In a sentence of
course the context can help a lot.  But what are the rules for this characters
to write properly and distinguish?

\bigskip

\begin{center}
\begin{tabular}{|c|c|}\hline
\KLETTER{shi}&\KLETTER{tsu}\\\hline
\end{tabular}
\end{center}

\CharacterExplanation{shiexplanation}{ To write the letter |shi| it is
important to align three lines \textbf{vertically} (red line) and to
\textbf{non-align} the ends (blue line).  In this way it is possible to
distinguish |shi| from |tsu|.  Of course also the angle of the frist two lines
are different, but in hadwriting this is difficult to match. As a rule of thumb
make the third line double as long as the first two but short enough to not
align it at the end.  }

\CharacterExplanation{tsuexplanation}{ To write the letter |tsu| it is
important to align all tree lines \textbf{horizontally} (red line) and to
\textbf{non-align} the ends (blue line). In this way it is possible to
distinguish |tsu| from |shi|.  Of course also the angle of the frist two lines
are different, but in hadwriting this is difficult to match. As a rule of thumb
make the third line double as long as the first two but short enough to not
align it at the end.  }



 


\newpage

% タチツテト
\subsection{タ - |ta|} \label{sec:KatakanaTa}

\KatakanaHeader{ta}{ Katakana |ta| is written with three strokes. The first
stroke is a small curve. The secosnd stroke starts horizontally attached to the
first stroke. The third stroke ends at the second stroke.}
\KatakanaTraining{ta}

\subsection{チ - |chi|} \label{sec:KatakanaChi}

\KatakanaHeader{chi}{ Katakana |chi| is written with three strokes. The first
stroke is a light curve. The second ihorizontally straight line. The third line
is a curve that joints the first and the second.} \KatakanaTraining{chi}

\subsection{ツ - |tsu|} \label{sec:KatakanaTsu}

\KatakanaHeader{tsu}{Katakana |tsu| is written with three strokes. The first
and second stroke are short. And the beginning of all three strokes is aligned
horizontally. The third stroke is the longest, but the end is not alignd wit
the beginning of the first stroke. } \KatakanaTraining{tsu}

\subsection{テ - |te|} \label{sec:KatakanaTe}

\KatakanaHeader{te}{ Katakana |te| is written with three strokes. The first
stroke is the shortest and horizontally. The second stroke is not aligned
vertically in the beginning, but also perfectly horizontally. The third stroke
is a small curve attached to the middle of the second stroke. }
\KatakanaTraining{te}

\subsection{ト - |to|} \label{sec:KatakanaTo}

\KatakanaHeader{to}{ Katakana |to| is written with 2 strokes. The first stroke
is a vertical line. Attached to this line there is short straight line to the
right. In some hand writings this line is a small curve to the right.}
\KatakanaTraining{to}

\section{片仮名タ行練習 - |ta| Row Training}

Consider the following vocabularies: 

\begin{longtable}{|p{2cm}|p{2cm}|p{3cm}|p{6cm}|p{2cm}|}\hline
ドイツ &doitsu &Duits(land), Deutsch(land) &Germany &Dutch, German\\\hline
オペ &ope &OP (short for Operation) &surgical operation, minor surgery &German\\\hline
タバコ, たばこ or 煙草 &tabako &tabaco &tobacco, cigarette &Portuguese\\\hline

\end{longtable}


\KatakanaSimpleTraining{Katakana to Romaji}{
\Transcribe{1.}{ココア}{}{cocoa}
}

\KatakanaSimpleTraining{Romaji to Katakana}{
\Transcribe{1.}{kokoa}{}{cocoa}
}

\newpage
\KatakanaSimpleTraining{English to Romaji}{
\Transcribe{1.}{persimon}{}{}
}

\KatakanaSimpleTraining{English to Katakana}{
\Transcribe{1.}{cocoa}{}{}
}

\newpage
