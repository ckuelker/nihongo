% ---------------------------------------------------------------------------
\section{Katakana /wa/ Row - 片仮名ワ行}\label{sec:KatakanaWaRow}

\Krow{warow}{wa}{s}{s}{s}{wo}

\label{letter:wa}\KLETTER{wa} The 片仮名 {「ワ」} is pronounced  /wa/ and
derives from the \hyperref[sec:PhoneticCharacter]{Phonetic Character}s {「和」}
right site part.  A \hyperref[sec:Dakuten]{濁点}  or
\hyperref[sec:Handakuten]{半濁点} do not exist.


\begin{center}\begin{tabular}{lll}
\textit{Rōmaji}&\textit{Katakana}&\textit{Alternatives}\\
/wa/&ワ  &\\
/va/&ヷ  &\small ヴァ、ヴぁ、う゛ぁ\\
/wā/&ワー&\\
/vā/&ヷー&\small ヴァア、ヴぁア、う゛ぁあ\\
\end{tabular}\end{center}


\label{letter:wo}\KLETTER{wo} The 片仮名 {「ヲ」} is pronounced  /wo/ and
derives from the \hyperref[sec:PhoneticCharacter]{Phonetic Character}s
{「乎」}.  A \hyperref[sec:Dakuten]{濁点}  or \hyperref[sec:Handakuten]{半濁点}
do not exist.

\begin{center}\begin{tabular}{lll}
\textit{Rōmaji}&\textit{Katakana}&\textit{Alternatives}\\
/wo/&ヲ  &\\
/vo/&ヺ  &seldomly used, more often: ヴォ\\
\end{tabular}\end{center}

\Note{Note}{%

It is safe to skip learning this character. See
\nameref{subsec:SeldomlyUsedKatakana} on page
\pageref{subsec:SeldomlyUsedKatakana}  for a detailed description.

}

\newpage
% UFuWaSimilarity
\subsection{|u|, |fu| and |wa| Similarity} \label{subsec:UFuWaSimilarity}

The Katakana characters {「ウ」}, {「フ」} and {「ワ」} can be easily
distinguished. All three have a different stroke count. However the shape is
similar. Therefore they can be mistaken. Especially when they have no context. 

\bigskip

\begin{center}
\begin{tabular}{|c|c|c|}\hline
\KLETTER{u}&\KLETTER{fu}&\KLETTER{wa}\\\hline
\end{tabular}
\end{center}




% VaAmbiguity
\subsection{|va|  Ambiguity} \label{subsec:VaAmbiguity}

The Rōmaji |va| can be written in many different ways and that is true for some
other characters of the {「ワ」} rowi too. The lack of standardization and
consistency make it hard to guess how one should write a certain word with this
sound.

\bigskip

\begin{center}
\Huge
\begin{tabular}{|c|c|c|}\hline
%\KLETTER{so}&\KLETTER{n}&\KLETTER{ri}\\\hline
ヴァ&ヷ&バ\\\hline
\end{tabular}
\end{center}



% ヴィオロン    = violin
% ヴァイオリン  = violin
% バイオリン    = violin

\newpage

%ワヲ
\subsection{/wa/ - 「ワ」}\label{sec:KatakanaWa}

\KatakanaHeader{wa}{ Katakana /wa/ is written with two strokes.}
\KatakanaTraining{wa}

\subsection{/wo/ - 「ヲ」}\label{sec:KatakanaWo}

\KatakanaHeader{wo}{ Katakana /wo/ is written with two strokes. }
\KatakanaTraining{wo}

\subsection{/wa/ Row Training - 片仮名ワ行練習}

\Padding
\begin{longtable}[c]{p{3cm}p{2.5cm}p{3.5cm}p{5cm}p{2cm}}
\textit{Katakana}&\textit{Rōmaji}&\textit{Original}&\textit{Remark}&Origin\\\hline
ホワイトデー&howaitodē &White + Day       &White Day, March 14th &English\\
ワープロ    &wāpuro    &wor(d) pro(cessor)&word processor        &English\\
\end{longtable}
\KatakanaSimpleTraining{Katakana to Rōmaji}{
\Transcribe{1.}{ホワイトデー}{}{White + Day}
\Transcribe{2.}{ワープロ}{}{word processor}
\Transcribe{3.}{ワイシャツ}{}{dress shirt}
\Transcribe{4.}{ヷ}{}{}
\Transcribe{5.}{ヴァルヴ}{}{valve}
}

\KatakanaSimpleTraining{Rōmaji to Katakana}{
\Transcribe{1.}{wāpuro}{}{word processor}
\Transcribe{2.}{howaitodē}{}{White + Day}
\Transcribe{3.}{va}{}{}
\Transcribe{4.}{waishatsu}{}{dress shirt}
\Transcribe{5.}{varuvu}{}{valve}
}

\newpage
\Padding
\begin{longtable}[c]{p{2cm}p{2.5cm}p{4.5cm}p{3cm}p{3cm}}
\textit{Katakana}&\textit{Rōmaji}&\textit{Original}&\textit{Remark}&Origin\\\hline
ワイシャツ  &waishatsu &Y shirt (from "white shirt")&dress shirt           &English\\
ヷ          &va        &                            &different writing     &\\ 
ヴァルヴ    & varuvu   &valve                       &                      &English\\
\end{longtable}

\KatakanaSimpleTraining{English to Rōmaji}{
\Transcribe{1.}{dress shirt}{}{}
\Transcribe{2.}{White + Day}{}{}
\Transcribe{3.}{valve}{}{}
\Transcribe{4.}{word processor}{}{}
\Transcribe{5.}{va}{}{}
}

\KatakanaSimpleTraining{English to Katakana}{
\Transcribe{1.}{valve}{}{}
\Transcribe{2.}{White + Day}{}{}
\Transcribe{3.}{dress shirt}{}{}
\Transcribe{4.}{word processor}{}{}
\Transcribe{5.}{va}{}{}
}

\newpage
