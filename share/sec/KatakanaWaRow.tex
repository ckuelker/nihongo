% ---------------------------------------------------------------------------
\section{Katakana /wa/ Row - 片仮名ワ行}\label{sec:KatakanaWaRow}

\Krow{warow}{wa}{s}{s}{s}{wo}

\label{letter:wa}\KLETTER{wa} The 片仮名 {「ワ」} is pronounced  /wa/ and
derives from the \hyperref[sec:PhoneticCharacter]{Phonetic Character}s {「和」}
right site part.  A \hyperref[sec:Dakuten]{濁点}  or
\hyperref[sec:Handakuten]{半濁点} do not exist.


\begin{center}\begin{tabular}{lll}
\textit{Rōmaji}&\textit{Katakana}&\textit{Alternatives}\\
/wa/&ワ  &\\
/va/&ヷ  &\small ヴァ、ヴぁ、う゛ぁ\\
/wā/&ワー&\\
/vā/&ヷー&\small ヴァア、ヴぁア、う゛ぁあ\\
\end{tabular}\end{center}


\label{letter:wo}\KLETTER{wo} The 片仮名 {「ヲ」} is pronounced  /wo/ and
derives from the \hyperref[sec:PhoneticCharacter]{Phonetic Character}s {「乎」}.
A \hyperref[sec:Dakuten]{濁点}  or \hyperref[sec:Handakuten]{半濁点} do not
exist.

\begin{center}\begin{tabular}{lll}
\textit{Rōmaji}&\textit{Katakana}&\textit{Alternatives}\\
/wo/&ヲ  &\\
/vo/&ヺ  &seldomly used, more often: ヴォ\\
\end{tabular}\end{center}

\Note{Note}{%

It is safe to skip learning this character. See
\nameref{subsec:SeldomlyUsedKatakana} on page
\pageref{subsec:SeldomlyUsedKatakana}  for a detailed description.

}

\newpage
% UFuWaSimilarity
\subsection{|u|, |fu| and |wa| Similarity} \label{subsec:UFuWaSimilarity}

The Katakana characters {「ウ」}, {「フ」} and {「ワ」} can be easily
distinguished. All three have a different stroke count. However the shape is
similar. Therefore they can be mistaken. Especially when they have no context. 

\bigskip

\begin{center}
\begin{tabular}{|c|c|c|}\hline
\KLETTER{u}&\KLETTER{fu}&\KLETTER{wa}\\\hline
\end{tabular}
\end{center}




\newpage

%ワヲ
\subsection{/wa/ - 「ワ」}\label{sec:KatakanaWa}

\KatakanaHeader{wa}{ Katakana /wa/ is written with two strokes.}
\KatakanaTraining{wa}

\subsection{/wo/ - 「ヲ」}\label{sec:KatakanaWo}

\KatakanaHeader{wo}{ Katakana /wo/ is written with two strokes. }
\KatakanaTraining{wo}

%\subsection{/wa/ Row Training - 片仮名ワ行練習}
%
%\KatakanaSimpleTraining{Katakana to Rōmaji}{
%\Transcribe{1.}{ココア}{}{cocoa}
%}

%\KatakanaSimpleTraining{Rōmaji to Katakana}{
%\Transcribe{1.}{kokoa}{}{cocoa}
%}

%\newpage
%\KatakanaSimpleTraining{English to Rōmaji}{
%\Transcribe{1.}{persimon}{}{}
%}

%\KatakanaSimpleTraining{English to Katakana}{
%\Transcribe{1.}{cocoa}{}{}
%}

\newpage
