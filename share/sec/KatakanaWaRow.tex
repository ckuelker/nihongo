% ---------------------------------------------------------------------------
\section{Katakana /wa/ Row - 片仮名ワ行}\label{sec:KatakanaWaRow}

\Krow{warow}{wa}{s}{s}{s}{wo}

\label{letter:wa}\KLETTER{wa} The 片仮名 {「ワ」} is pronounced  /wa/ and  derives from the
\hyperref[sec:PhoneticCharacter]{Phonetic Caracter}s {「和」} right site part.  A
\hyperref[sec:Dakuten]{濁点}  or \hyperref[sec:Handakuten]{半濁点} do not
exist.

\newpage

\label{letter:wo}\KLETTER{wo} The 片仮名 {「ヲ」} is pronounced  /wo/ and  derives from the
\hyperref[sec:PhoneticCharacter]{Phonetic Caracter}s {「乎」}.  A
\hyperref[sec:Dakuten]{濁点}  or \hyperref[sec:Handakuten]{半濁点} do not
exist.

\Note{Note}{%

It is safe to skip learning this character. See
\nameref{subsec:SeldomlyUsedKatakana} on page
\pageref{subsec:SeldomlyUsedKatakana}  for a detailed description.

}

\newpage

%ワヲ
\subsection{/wa/ - 「ワ」}\label{sec:KatakanaWa}

\KatakanaHeader{wa}{ Katakana /wa/ is written with two strokes.}
\KatakanaTraining{wa}

\subsection{/wo/ - 「ヲ」}\label{sec:KatakanaWo}

\KatakanaHeader{wo}{ Katakana /wo/ is written with two strokes. }
\KatakanaTraining{wo}

%\subsection{/wa/ Row Training - 片仮名ワ行練習}
%
%\KatakanaSimpleTraining{Katakana to Romaji}{
%\Transcribe{1.}{ココア}{}{cocoa}
%}

%\KatakanaSimpleTraining{Romaji to Katakana}{
%\Transcribe{1.}{kokoa}{}{cocoa}
%}

%\newpage
%\KatakanaSimpleTraining{English to Romaji}{
%\Transcribe{1.}{persimon}{}{}
%}

%\KatakanaSimpleTraining{English to Katakana}{
%\Transcribe{1.}{cocoa}{}{}
%}

\newpage
