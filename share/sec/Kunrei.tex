\section{訓令式ローマ字 - Kunrei System} \label{sec:Kunrei}

The modern Kunrei System {訓令式ローマ字} {【くんれいろうまじ】}  is the
official writing system of Japan. It was confirmed in 1994 by the Cabinet and
is available as ISO 3602:1989. The Kunrei System predecessor was introduced
1985 by Dr. Aikitsu Tanakadate as {日本式ローマ字} {【にほんしきろうまじ】} and
tries a more systematical approach to map Hiragana and Katakana to equal Roman
letters. The {五十音図} {【ごじゅうおんず】} in the {訓令式ローマ字} is as
follows:

\Info{訓令式ローマ字 - Kunrei System}{
\begin{center}
\begin{tabular}{|c|c|c|c|c|}\hline
   a & i& u& e& o\\\hline
   ka&ki&ku&ke&ko\\\hline
   sa&si&su&se&so\\\hline
   ta&ti&tu&te&to\\\hline
   na&ni&nu&ne&no\\\hline
   ha&hi&hu&he&ho\\\hline
   ma&mi&mu&me&mo\\\hline
   ya&  &yu&  &yo\\\hline
   ra&ri&ru&re&ro\\\hline
   wa&  &  &  & o\\\hline
     &  &  &  & n\\\hline
\end{tabular}
\end{center}
}

Even tough the system is official, many entities, like the train system, are
not using it. The use the Hepburn System.

The {訓令式ローマ字} is not part of this book. Please see \nameref{sec:Hepburn}
on page \pageref{sec:Hepburn} for the system in use.
