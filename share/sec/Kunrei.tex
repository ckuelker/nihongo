\section{Kunrei System}\jsec{訓令式ローマ字} \label{sec:Kunrei}

\begin{tabular}{lr}
\begin{minipage}{9.6cm}

The modern \textbf{Kunrei} System {訓令式ローマ字} {【くんれいろうまじ】}  is
the official writing system of Japan. It was confirmed in 1994 by the Cabinet
and is available as ISO 3602:1989. The \textbf{Kunrei} System predecessor was
introduced 1985 by Dr. Aikitsu Tanakadatei ({田中舘愛橘}) as {日本式ローマ字}
{【にほんしきろうまじ】} (Nihon-/Nipponshikiromaji) and tries a more
systematical approach to map \hyperref[sec:Hiragana]{Hiragana} and
\hyperref[sec:Katakana]{Katakana} to equal Roman letters. The {五十音図}
{【ごじゅうおんず】} in the {訓令式ローマ字} is as follows:

\Link \href{http://en.wikipedia.org/wiki/Tanakadate_Aikitsu}{Tanakadate}

\end{minipage}
&
\raisebox{-.5\height}{
\includegraphics[scale=0.25,trim= 00 00 00 00]{../share/ei/Aikitsu_Tanakadate_r.jpg}}
\\
\end{tabular}



\Info{訓令式ローマ字 - Kunrei System}{
\begin{center}
\begin{tabular}{|c|c|c|c|c|}\hline
   a & i& u& e& o\\\hline
   ka&ki&ku&ke&ko\\\hline
   sa&si&su&se&so\\\hline
   ta&ti&tu&te&to\\\hline
   na&ni&nu&ne&no\\\hline
   ha&hi&hu&he&ho\\\hline
   ma&mi&mu&me&mo\\\hline
   ya&  &yu&  &yo\\\hline
   ra&ri&ru&re&ro\\\hline
   wa&  &  &  & o\\\hline
     &  &  &  & n\\\hline
\end{tabular}
\end{center}
}

Even tough the system is official, many entities (like the train system) are
not using it. They use the Hepburn System.

The {訓令式ローマ字} is not part of this book. Please see \nameref{sec:Hepburn}
(on page \pageref{sec:Hepburn}) for the system in use.
