% ---------------------------------------------------------------------------
\section{Man'yōgana}\jsec{万葉仮名}
% [o] LABEL
\label{sec:Manyogana}
\label{sec:Manyoshu}
% [o] INDEX
\ifor{Man'yōgana}{万葉仮名}{まんようがな}{Man'yōgana}
\ifor{Man'yōshu}{万葉集}{まんようしゅう}{Man'yōshu}
\ifor{mora}{モーラ}{もーら}{Mora}
\ifor{Kanji}{漢字}{かんじ}{Kanji}
\ifor{Katakana}{片仮名}{かたかな}{Katakana}

The development of distinct Japanese writing begun 600 AD by writers and
scholars reducing some Chinese characters to its bare phonetic value. The
meaning of this characters where ignored. Around 760 a collection of Japanese
poetry was published, the \Link
\href{https://en.wikipedia.org/wiki/Man%27y%C5%8Dsh%C5%AB}{万葉集
【まんようしゅう】}, in which Chinese characters where uses as phonetic
letters. In regard to \textit{Man'yōshu} {万葉集} {【まんようしゅう】} the
characters are named {万葉仮名} {【まんようがな】}

The origin of the \textbf{Man'yōgana} script in poetry and art lead to some
problems in the understanding for the reader. Since the usage of phonetic
Chinese characters where mixed with regular Chinese characters and the
reasoning about which character to use was more form and shape aesthetic then
pragmatic, the meaning was difficult to grasp.

However the royal household or other scholars did not see a necessity to change
the status quo, because the high aim was to write poetry and other texts in
Chinese and \textbf{Man'yōgana} was considered appropriate only for notes,
diaries and love letters.

\Note{Note}{\footnotesize By the end of the 8th Century 970
\hyperref[sec:Kanji]{{漢字} {【かんじ】}} where used to pronounce the 90
\hyperref[sec:Mora]{morae}. This directly shows that there was no bijective map
between sound and character. For |ka| for example the following
\textbf{Man'yōgana} can be used {「可」}, {「何」}, {「加」}, {「架」},
{「香」}, {「蚊」}, {「迦」}. }

The number of \textbf{Man'yōgana} from which \hyperref[sec:Katakana]{Katakana}
likely derived is smaller.

\Hint{Man'yōgana used for creation of {片仮名} {【かたかな】}}{
\begin{center}
\begin{tabular}{|c||c|c|c|c|c|}\hline
 & a& i  & u  & e& o\\\hline\hline
-&阿&伊  &宇  &江&於\\\hline
k&加&機幾&久  &介&己\\\hline
s&散&之  &須  &世&曽\\\hline
t&多&千  &州川&天&止\\\hline
n&奈&仁  &奴  &祢&乃\\\hline
h&八&比  &不  &部&保\\\hline
m&末&三  &牟  &女&毛\\\hline
y&也&    &由  &  &與\\\hline
r&良&利  &流  &礼&呂\\\hline
w&和&井  &    &恵&乎\\\hline
*&尓&    &    &  &  \\\hline
\end{tabular}
\end{center}
}

The scientific term \textbf{Man'yōgana} is used by Western and Japanese
scientists. However it is not without critique. The term \textbf{Man'yōgana}
might lead to the illusion that it was a defined set of characters in use for
transcribing Chinese or writing Japanese texts or the second illusion that one
sound is represented by only  one \textbf{Man'yōgana}. Both is not true. First,
all Chinese Characters could in principle be used as \textbf{Man'yōgana} (and
therefore the term is basically useless). Actually the reason to chose one
character was sometimes just because out of aesthetic reasons, the shape or
some additional meaning. And second, normally many different
\textbf{Man'yōgana} (Chinese characters) where used for the same pronunciation
in the same text.  Making it efficient or easy was not the target of the
scholars using this kind of \hyperref[sec:PhoneticCharacter]{phonetic
characters} at that time.


\Link \href{https://en.wikipedia.org/wiki/Manyogana}{Man'yōgana}
\Link \href{https://en.wikipedia.org/wiki/Man%27y%C5%8Dsh%C5%AB}{万葉集}

