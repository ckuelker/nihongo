% ---------------------------------------------------------------------------
\section{Mora}\jsec{モーラ}
% [o] LABEL
\label{sec:Mora}
% [o] INDEX
\ifor{mora}{モーラ}{もーら}{Mora}
\ifor{syllable}{音節}{おんせつ}{Silbe}
\ien{morae} \ide{Morae}
\ien{moras} \ide{Moras}

The concept of \textbf{mora}  (plural morae or moras; often symbolized μ) is
used in the science of linguistics. It describes a joint unit in pronunciation
(phonology) that constructs a syllable. The definition of a \textbf{mora} can
vary.  In Japanese the detection of \textbf{morae} is comparably simple. The
world {「チョコレート」} for example consist out of the following 5
\textbf{morae} {「チョ」},{「コ」},{「レ」},{「ー」} and {「ト」} while it
consist only out of four \hyperref[sec:Syllable]{syllables}
{(\hyperref[sec:Syllable]{音節} 【おんせつ】)} {「チョ」},{「コ」},{「レー」}
and {「ト」}.

