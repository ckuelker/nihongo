% ---------------------------------------------------------------------------
\section{Okurigana - 送り仮名} \label{sec:Okurigana}

The term Okuriagna is written {送り仮名}{【おくりがな】} in Japanese, but it is
\textit{not} a script by its own as the name \hyperref[sec:Kana]{Kana} suggest.
\textbf{Okurigana} are \hyperref[sec:Kana]{Kana} but either
\hyperref[sec:Hiragana]{Hiragana} or \hyperref[sec:Katakana]{Katakana} that are
used to write the ending of words in most cases verbs. Today only
\hyperref[sec:Hiragagana]{Hiragana} are used to write \textbf{Okurigana} while
before 1945 (TODO) \hyperref[sec:Katakana]{Katakana} was used. 

\textbf{Okurigana} are the mandatory compromise using static Chinese letters to
write the Japanese language. Next to make \hyperref[sec:kanji]{Kanji} flexible
the other function is to mark te beginning are ending of words in sentences. 
