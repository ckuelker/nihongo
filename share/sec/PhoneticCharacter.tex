\section{Phonetic Character - 表音文字} \label{sec:PhoneticCharacter}

In this document the term \textbf{Phonetic Character} ({表音文字}
{【ひょうおんもじ】}) refers genetically to a Chinese characters reading and
the usage of this character just for this purpose and \textit{not} for its
meaning. This common set expression has been used in avoidance of the term
\hyperref[sec:Manyogana]{Manyogana}. See the section \nameref{sec:Manyogana} on
page \pageref{sec:Manyogana} to understand the critique.

A \textbf{Phonetic Character} as to be distinguished from the linguistic term
\textit{phonogram} that describes a written character which represents a phonem
(speech sound) such as Latin alphabet or Japanese \hyperref[sec:Kana]{Kana}. 

