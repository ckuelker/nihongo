\section{Phonetic Character}\jsec{表音文字} \label{sec:PhoneticCharacter}

In this document the term \textbf{Phonetic Character} ({表音文字}
{【ひょうおんもじ】}) refers genetically to a Chinese characters reading and
the usage of this character just for this purpose and \textit{not} for its
meaning. This common set expression has been used in avoidance of the term
\hyperref[sec:Manyogana]{Man'yōgana}. See the section \nameref{sec:Manyogana}
on page \pageref{sec:Manyogana} to understand the critique.

The \textbf{Phonetic Character} has to be distinguished also from the
linguistic term \textit{phonogram} that describes a written character which
represents a \textit{phonem} (speech sound) such as the Latin alphabet or the
Japanese \hyperref[sec:Kana]{Kana}. 

