% ---------------------------------------------------------------------------
\section{Rōmaji  - ローマ字} \label{sec:Romaji}


% ---------------------------------------------------------------------------
\ifor{Rōmaji}{ローマ字}{ろーまじ}{Rōmaji}

In temporary Japan words written in western letters become more popular and
some parts of the written language is already westernized, like (Indian/
Arabic) numbers written in horizontal text almost per default. This western
Latin letters are called \textbf{Rōmaji} and are written in Japanese as
{ローマ字} {【ろおまじ】}, even though some of them are from different origin
like Indian numbers for example.



The western characters are mainly used for writing numbers in the horizontal
writing. Also for abbreviations capital and small letters are used. Sometimes
they are modified. For example the measurement of distance in the metric entity
"km" occupies to places in western scripts "k" + "m" while it only hold one
place in Japanese {「㎞」} or even one place in
\hyperref[sec:Katakana]{Katakana}  {「㌔」}. While the latter is ambiguous to
us, because colloquial kilogram is referenced as only "kilo".

\Note{One Space Rōmaji}{\begin{center}\small
\begin{tabular}{ll}
\textit{Western Multiple Space Letters}&\textit{One Space Rōmaji}\\
mg&㎎\\
mm&㎜\\
kg&㎏\\
cm&㎝\\
km&㎞\\
qm&㎡\\
qcc&㏄\\
\end{tabular}
\end{center}
}

There are other shapes of Rōmaji for numbers or letters:

%\fontspec{IPAPMincho}
\fontspec{IPAPGothic}

\begin{center}
\begin{tabular}{ll}
Roman       &ⅠⅡⅢⅣⅤⅥⅦⅧⅨⅩⅪⅫ...\\
Blac circle & ❶❷❸❹❺❻❼❽❾❿...\\
Withe circle &①②③④⑤⑥⑦⑧⑨⑩...\\
Withe double circle & ⓵⓶⓷⓸⓹⓺⓻⓼⓽⓾...\\
Letters             &ⓐⓑⓒ...\\
\end{tabular}
\end{center}
\newpage
\fontspec{FreeSans}

In a number of incidents in typography multiple Katakana are condensed
into one space, where normally only one Katakana would exist. In some cases the
direction of writing is even diagonal. This part of exception are not part of
this document and should be viewed under the peculiar aesthetic of Japanese
printing.

\Note{One Space Katakana}{\begin{center}\small
\begin{tabular}{ll}
\textit{Western Meaning}&\textit{One Space Katakana}\\
&㌃\\
calorie&㌍\\
kilo&㌔\\
gram&㌘\\
centi-&㌢\\
cent&㌣\\
\$&㌦\\
t&㌧\\
\%&㌫\\
ha&㌶\\
pages&㌻\\
milli-&㍉\\
mbar (millibar)&㍊\\
m (meter)&㍍\\
l (liter)&㍑\\
&㍗\\
\end{tabular}
\end{center}
}

Citation of foreign books are also done in western letters an can pop up
without warning the middle of the text.
