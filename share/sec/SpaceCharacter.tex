\section{Space Character}\jsec{空白文字} 
% [o] LABEL
\label{sec:SpaceCharacter}
% [o] INDEX DESTINATION (DEF)
\ifor{space character}{空白文字}{くうはく ・ もじ}{Leerzeichen}
% [o] INDEX TARGET
\ifor{Katakana}{片仮名}{かたかな}{Katakana}
\ifor{Kanji}{漢字}{かんじ}{Kanji}
\ifor{Hiragana}{平仮名}{ひらがな}{Hiragana}
\ifor{Rōmaji}{ローマ字}{ろーまじ}{Rōmaji}

The \textit{space character} in Western (Latin letter based) languages is used
to separate words. In antique texts a separation of words was \textbf{not}
common and those where difficult to read.  In the 7th century AD the word
separation was introduced. In the beginning of printed books the space wide was
fixed and to archive this the width of the letters where not fixed which
produced an easy to read text body.

The invention of typewriters and computers destroyed this approach of
aesthetically advanced typography. The typewriters had still a fixed (too
large) space width but also fixed letters. While the computer on screen behave
not better as a typewriter in the beginning and in printing, the spaces are
variable and the letters are fixed, the opposite of the elegant book printing
of the 15th century AD.

With the invention of Unicode the \textit{space character} is not longer a
singularity.  The Unicode fonts have now many\footnote{To give an example:
U+2008 Punctuation Space, U+2009 Thin Space, ..., U+FEFF Zero Width No-Break
Space, to just name a few.} \textit{space characters}. 

The Japanese computer fonts do have a \textit{space character}. Traditionally
more then one.  The most important \textit{space character} is the double wide
\textit{space character} which is exactly as wide as a
\hyperref[sec:Kanji]{Kanji} character. And the single wide space character that
is as wide as \hyperref[sec:Romaji]{Rōmaji} or half wide
\hyperref[sec:Hiragana]{Hiragana} or \hyperref[sec:Katakana]{Katakana}. 

However even though there is a \textit{space character} nowadays in Japanese
fonts it is \textbf{not} used to separate words from each other. Because of
this the word border can only be detected by heuristics and changes in scripts,
for example: \hyperref[sec:Katakana]{Katakana} to \hyperref[sec:Kanji]{Kanji},
\hyperref[sec:Hiragana]{Hiragana} to \hyperref[sec:Kanji]{Kanji},
\hyperref[sec:Katakana]{Katakana} to \hyperref[sec:Hiragana]{Hiragana} and so
on. Detecting words is a major task in learning Japanese.

The \textit{space character} in Japan is used to indent text to mark
paragraphs. To separate functional entities in the text like author from
heading. 

As a matter of fact the \textit{space character} in modern Japanese plays a
very unimportant role. 

This was not always so. In old Japanese there where an additional usage of
\textit{space characters} as {闕字} {【けつじ】} to leave space in front of
names of important persons or verbs to honor them.

\smallskip
\textbf{Example:}

\begin{center}\Padding\begin{tabular}{p{4cm}p{3cm}p{4cm}}
 {「  上様」} & {【うえさま】}      & Mister Ue \\
 {「登  城」} &  {【とう じょう】} & registered castle\\
\end{tabular}\end{center}
\smallskip

However this usage was abandoned in the Meiji era. 

\Link \href{http://ja.wikipedia.org/wiki/%E9%97%95%E5%AD%97}{http://ja.wikipedia.org/wiki/闕字}

