\section{Special Katakana Characters}\jsec{特別カタカナ}
% [o] LABEL
\label{sec:SpecialKatakanaCharacters}
% [o] INDEX
\ifor{special Katakana characters}{特別カタカナ}{とくべつかたかな}{Spezielle Katakana Zeichen}
\ifor{Katakana}{片仮名}{かたかな}{Katakana}
\ifor{Gojūonzu}{五十音図}{ごじゅうおんず}{50 Laute Tafel}


As mentioned before \hyperref[sec:Katakana]{Katakana} is almost like Hiragana.
This is true for the \hyperref[sec:Gojuonzu]{Gojūonzu (50 sound table)  {五十音図}
{【ごじゅうおんず】}} This section will show the special characters, some are
different from the Hiragana set.

Special in some sense are characters used for punctuation, like {「。」},
{「!」} and {「?」}.  These are similar to the western counterparts but
differ a little bit. While it is obvious for the small circle {「。」}, also
{「!」} and {「?」} differ from the western equivalent in that sense that
they are \textbf{centerd} and occupy more (white) space. This characters among
other characters are used equally among Hiragana,
\hyperref[sec:Katakana]{Katakana} and Kanji. Therefore this section will not
further mention them.

%TODO check if point changes orientation and alignment in case of changing
%writing direction. 

\subsection{Doubling Vowels in Katakana}\jsubsec{カタカナでの倍増母音}
% [o] LABEL
\label{subsec:DoublingVowelsInKatakana}
\label{subsec:DoublingVowels}
\label{sec:DoublingVowelsInKatakana}
\label{sec:DoublingVowels}
% [o] INDEX
\ifor{doubling vowels}{倍増母音}{ばいぞうぼいん}{Vokalverdopplung}
\ifor{repition mark}{繰り返し記号}{くりかえしきごう}{Wiederholungszeichen}
\ifor{Katakana}{片仮名}{かたかな}{Katakana}

% カタカナでの倍増母音 【ばいぞうぼいん】

Special \hyperref[sec:Katakana]{Katakana} characters do also exists. The most
important character is Chōon {長音} {【ちょうおん】} the plain iteration
character {「ー」}, written as a stroke. It is one of the very few which
changes orientation according the writing orientation. When writing
\hyperref[sec:Katakana]{Katakana} from left to right the iteration character is
horizontal, while writing \hyperref[sec:Katakana]{Katakana} from up to down it
is vertical. The function of this character is to double the previous mora.
This is also different from Hiragana. (For doubling als other
\hyperref[sec:Katakana]{Katakana} caracter, refere to section
\nameref{sec:Iteration} on page \pageref{sec:Iteration}.)

\bigskip

\CharacterExplanation{k-iteration-s}{In standard gothic fonts the
\hyperref[sec:Katakana]{Katakana} iteration character is just a straight line
and it is not possible to understand in which direction it has to written. }

\bigskip

\CharacterExplanation{k-iteration-sm}{However if it is written with a different
font or with a brush it is clearly visible that in horizontal writing it is
written from left to right.} 

\bigskip


%\definecolor{orange}{rgb}{1,0.5,0}
%\definecolor{mygreen}{rgb}{.2,1,.2}

\bigskip
\textit{Example:}

\bigskip

\begin{center}
\begin{tabular}{p{7cm}p{7cm}}
Katakana:&Hiragana:\\
\Huge カ\textbf{\color{magenta}ー}ド /kaado/ &\Huge か\textbf{\color{magenta}あ}ど /kaado/\\
\end{tabular}
\end{center}

\bigskip

This character is very often used and makes \hyperref[sec:Katakana]{Katakana}
for this easier then Hiragana. The long vowel ambiguity do not exist.

As mentioned above the orientation of the \hyperref[sec:Katakana]{Katakana}
iteration character changes with the direction of writing. The above example
with different writing orientation.

\medskip
\textit{Example:}

\medskip

\begin{center}
\begin{tabular}{p{3.5cm}p{3.5cm}p{3.5cm}m{3.5cm}}
horizontally&
\setCJKfamilyfont{cjk-vert}[Script=CJK,RawFeature=horizontal]{IPAPGothic}
\mbox{
\begin{minipage}{3.2cm}
\Huge カ\textbf{\color{magenta}ー}ド
\end{minipage}
}
& vertically &
%\setCJKfamilyfont{cjk-vert}[Script=CJK,RawFeature=vertical]{Kozuka Gothic Pro M}
\setCJKfamilyfont{cjk-vert}[Script=CJK,RawFeature=vertical]{IPAPGothic}
\raisebox{-.5\height}{
\mbox{
\rotatebox{-90}{
\begin{minipage}{3.2cm} \CJKfamily{cjk-vert}
\Huge カ\textbf{\color{magenta}ー}ド
\end{minipage}
}
}
}
\\
\end{tabular}
\end{center}
\medskip



\subsection{Seldom Used Katakana}\jsubsec{めったに使われない片仮名}\label{subsec:SeldomlyUsedKatakana}

% めったに使われない片仮名 【めったにつかわれないかたかな】

Even though |wo| {「ヲ」}  is part if the standard letters. But since all
particle are written in Hiragana and in this case |wo| is written {「を」} the
learning of {「ヲ」} can be skipped. Unless it is important to read old texts,
like telegrams.


