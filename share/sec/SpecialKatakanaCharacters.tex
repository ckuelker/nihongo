\section{Special Katakana Characters}

As mentioned before Katakana is almost like Hiragana. This is true for the
\hyperref{sec:50SoundTable}{50 sound table  {五十音図} {【ごじゅうおんず】}}
This section will show the special characters, some are different from the
Hiragana set.

Special in some sense are characters used for interpunction, like {。}, {!}
and {?}.  These are similar to the western counterparts but differ a little
bit. While it is obvious for the small cirlce {。}, also {!} and {?} differ
from the western equivalent in that sence that they are \textbf{centerd} and
occupy more (white) space. This characters among other characters are used
equally among Hiragana, Katakana and Kanji. Therefore this section will not
further mention them.

%TODO check if point changes orientation and alignment in case of changing
%writing direction. 

\subsection{Doubling Vowls in Katakana}

Special Kataka characts do also exists. The most important charact is
the plain iteration character {ー}, written as a stroke. It is one of the
very few which changes orientation according the writing orientation. When
writing Katakana from left to right the interation character is horizontal,
while writing Katakana from up to down it is vertical. The function of
this character is to double the pervious mora. This is also different
from Hiragana.

%\definecolor{orange}{rgb}{1,0.5,0}
%\definecolor{mygreen}{rgb}{.2,1,.2}

\bigskip
\textit{Example:}

\bigskip
\begin{tabular}{ll}
Katakana:&Hiragana:\\
Kaado カ\textbf{\color{magenta}ー}ド&Kaado か\textbf{\color{magenta}あ}ど\\
\end{tabular}

\bigskip
This caracter is very often used and makes Katakana for this easier then
Hiragana. The long vowl abiguity do not exist.

\subsection{Seldomly Used Katakana}\label{subsec:SeldomlyUsedKatakana}

Even though |wo| {「ヲ」}  is part if the standard letters. But since all
particle are written in Hiragana and in this case |wo| is written {「を」} the
learning of {「ヲ」} can be skipped. Unless it is important to read old texts,
like telegrams.


