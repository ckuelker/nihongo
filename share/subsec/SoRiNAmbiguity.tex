% SoRiNAmbiguity
\subsection{|so|, |ri| and |n| Ambiguity} \label{subsec:SoRiNAmbiguity}

The Katakana characters {「ソ」}, {「リ」} and {「ン」} can be difficult to
distinguish. All three are made out of only 2 strokes. And especially |so| and
|n| can be hard to tell. In a sentence of course the context can help a lot.
But what are the rules for this characters to write properly and distinguish?

\bigskip

\begin{center}
\begin{tabular}{|c|c|c|}\hline
\KLETTER{so}&\KLETTER{n}&\KLETTER{ri}\\\hline
\end{tabular}
\end{center}

\CharacterExplanation{soexplanation}{ To write the letter |so| it is important
to align both lines \textbf{horizontally} (red line) and to \textbf{non-align}
the ends (blue line).  In this way it is possible to distinguish |so| from |n|,
but not from |ri|. To also distinguish it from |ri| you have to write the first
stroke not horizontally nor vertically.  }

\CharacterExplanation{nexplanation}{ To write the letter |n| it is important to
a align both lines \textbf{vertically} (red line) and to \textbf{non-align} the
ends (blue line). In this way it is possible to distinguish |n| from |so|. If
both lines are aligned there should not be a problem to distinguish it from
|ri|.  }

\CharacterExplanation{riexplanation}{ To write the letter |ri| it is important
to a align both lines \textbf{vertically} (red line) and to \textbf{non-align}
the ends (blue line). The difference between |so| and |ri| is that |ri| need to
start with two \textbf{parallel} lines wile |so| does not. Please see green
lines for explanation.  }

